\section{Lecture 4 (March 17th)}
{\bf 1.3}\hspace{2ex}Equipartition of Energy
\\
\begin{thm}
(Equipartition Theorem) At temperature $T$, the average energy of any quadratic degree of freedom is $kT/2$. If a system contains $N$ molecules, each with $f$ degrees of freedom and there are no other (non-quadratic) temperature-dependent forms of energy, then its total thermal energy is
\[U_{thermal}=N\cdot f\cdot \dfrac{1}{2}kT\]
\end{thm}
\vspace{2ex}
\begin{ex}
3-dimensional translational motion (monotomic gas) ($f=3$)
\[\dfrac{1}{2}mv^2_{x}+\dfrac{1}{2}mv^2_{y}+\dfrac{1}{2}mv^2_{z}\]
\end{ex}
\vspace{2ex}
\begin{ex}
For a diatomic molecule $(f=5)$ or vibrating molecule $(f=2)$, we have
\[U=\dfrac{1}{2}mv_{x}^2+\dfrac{1}{2}mv_{y}^2+\dfrac{1}{2}mv_{z}^2+\dfrac{1}{2}I_{x}\omega_{x} ^2+\dfrac{1}{2}I_{y}\omega_{y} ^2\hspace{5ex}\dfrac{1}{2}kx^2+\dfrac{1}{2}mv_{x}^2\]
However, at room temperature, many degrees of freedom do not contribute to a molecule's thermal energy. The explanation lies in quantum mechanics, as we will see in chapter 3.
\end{ex}
\vspace{2ex}
{\bf 1.4}\hspace{2ex}Heat and Work
\\
\begin{thm}
(First law of thermodynamics) The energy change of a system is given by the sum of heat and work. Heat is any spontaneous flow of energy caused by temperature difference while work is any other transfer of energy except heat.
\[\Delta U=Q+W\]
The unit of energy is given as Joules or Calories, where 1 Calorie is 4.184 Joules. 1 Calorie is equivalent to the amount of heat needed to raise the temperature of a gram of water by $1^{\circ}C$
\end{thm}
\vspace{2ex}
{\bf 1.3}\hspace{2ex}Compression of Work
\\\\
Imagine the work done by gas contained in a cylinder.
\begin{align*}
	dW=&{\vec F}\cdot d{\vec r}\\
	=&F\cdot dx\\
	=&-P\cdot dV\\
W=&-\int _{V_{i}}^{V_{f}}P\cdot dV
\end{align*}
\\
\begin{defi}
There are two large classes of compression of ideal gases, which are isothermal ($T=$ constant) and adiabatic compression ($Q=0$\; due to insulation or fast change). During isothermal compression (expansion), 
\begin{align*}
	W=&-\dfrac{V_{f}}{V_{i}}P\cdot dV=-\int ^{V_{f}}_{V_{i}}\dfrac{NkT}{V}\;dV\\
	=&NkT\ln\Big(\dfrac{V_{i}}{V_{f}}\Big)\hspace{2ex}(>0)\\
\end{align*}
from this, we know that $f\cdot Nk\Delta T/2=0=\Delta U=Q+W$. On the other hand, during Adiabatic compression (expansion), 
\[\Delta U=Q+W=W\]
Since $W>0$, we know that $\Delta U>0$ and that temperature increases. Continuing,
\begin{align*}
	\Delta U&=W\\
	\dfrac{f}{2}Nk\;dT=&-P\;dV\\
	\dfrac{3}{2}Nk\;dT=&-P\;dV=-\dfrac{NkT}{V}\;dV\\
	\dfrac{3}{2}\dfrac{dT}{T}=&-\dfrac{dV}{V}\\
	\dfrac{3}{2}\ln\Big(\dfrac{T_{f}}{T_{i}}\Big)=&-\ln\Big(\dfrac{V_{f}}{V_{i}}\Big)\\
\ln\Big(\dfrac{T_{f}}{T_{i}}\Big)^{3/2}=&\ln\Big(\dfrac{V_{i}}{V_{f}}\Big)\\
V_{i}T_{i}^{3/2}=&V_{f}T_{f}^{3/2}\\
\implies& VT^{3/2}=\mathrm{const.}\\
\implies& V\cdot \Big(\dfrac{PV}{Nk}\Big)^{3/2}=\mathrm{const.}\\
	\implies &(P^{3/2}V^{5/2})^{3/2}=\mathrm{const.}
\end{align*}
Where we have assumed that $f=3$ and the gas is monatomic.
\end{defi}
\vspace{2ex}
{\bf 1.6}\hspace{2ex}Heat Capacities
\\
\begin{defi}
The heat capacity and the specific heat capacity is defined as
\[C=\dfrac{Q}{\Delta T}\hspace{5ex}c=\dfrac{C}{m}\]
There are two ways we can obtain heat capacity, (1) at constant volume and (2) at constant pressure. At constant volume, we know that $W=-P\Delta V=0$ and that
\[C_{V}=\Big(\dfrac{\Delta U}{\Delta T}\Big)_{V}=\Big(\dfrac{\partial U}{\partial T} \Big)_{T}\]
At constant temperature,
\[C_{P}=\Big(\dfrac{\Delta U+P\Delta V}{\Delta T}\Big)_{P}=\Big(\dfrac{\partial U}{\partial T} \Big)_{P}+P\Big(\dfrac{\partial V}{\partial T} \Big)_{P}\]
\end{defi}
\vspace{2ex}
\begin{defi}
Enthalpy is defined as 
\[H=U+PV\hspace{5ex}\Delta H=\Delta U+P\Delta V\]
Constant-pressure processes occur quite often both in the natural world and in the lab. Keeping track of the compression-expansion work done during these processes gets to be a pain after a while, but there is a convenient trick.
\end{defi}
\vspace{2ex}
\begin{ex}
Assume taking a single mole of water and making it vapor. Some of the energy is used in changing the chemical bonding from water to vapor ad some of energy is used to create the volume for the vapor. Calculating this,
\[W=P\Delta V=P(V_{vapor}-V_{water})\approx PV_{vapor}\approx 1 \cdot R\cdot 373 = 3100\;J\]
thus, we see how $8\%$ of $\Delta H$ is used for $W=P\Delta V$. We now know that out of the 40660 Joules used in the process above, 3100 Joules is used in creating volume and the remainder is used in breaking chemical bonds. 
\end{ex}
\vspace{2ex}

