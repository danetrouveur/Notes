\section{Lecture 8 (March 31st)}
{\bf Chapter 2.6}\hspace{2ex}Entropy
\begin{defi}
Any large system in equilibrium will be found in the macrostate with the greatest multiplicity. We define entropy as
\[S=k\;\ln \mathrm{\Omega} \]
\end{defi}
\vspace{2ex}
\begin{ex}
The entropy of an ideal gas is what is known as the Sakur-Tetrode equation.
\[\mathrm{\Omega} _{N}\approx \dfrac{1}{N!}\dfrac{V^{N}}{h^{3N}}\dfrac{\pi^{3N/2}}{(3N/2)!}(\sqrt{2mU})^{3N}\]
using the definition above,
\[S=k\ln\mathrm{\Omega} _{N}=Nk\Big[\ln\Big(\dfrac{V}{N}\Big(\dfrac{4\pi mU}{3Nh^2}\Big)^{3/2}\Big)+\dfrac{5}{2}\Big]\]
From here, we know that the change of entropy is only given by the following equation when there is no change in $U$ and $N$.
\[\Delta S=Nk\ln\Big(\dfrac{V_{f}}{V_{i}}\Big)\]
Two examples of such systems are isothermal expansion and free expansion.
\end{ex}
\vspace{2ex}

