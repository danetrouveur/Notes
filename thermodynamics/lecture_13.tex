\section{Lecture 13 (April 16th)}
\begin{recall}
We have previously defined chemical potential,
\[\mu =-T\Big(\dfrac{\partial S}{\partial N} \Big)_{U,V} \]
\end{recall}
\vspace{2ex}
\begin{thm}
Now, with the definition of chemical potential, we can newly define the thermodynamic identity as (note that $S=S(U,V,N)$)
\begin{align*}
dS=&\Big(\dfrac{\partial S}{\partial U}\Big)_{V,N}\;dU+\Big(\dfrac{\partial S}{\partial V} \Big)_{U,N}\;dV+\Big(\dfrac{\partial S}{\partial N}\Big)_{U,V}\;dN\;
=&\dfrac{1}{T}\;dU+\dfrac{P}{T}\;dV-\dfrac{\mu }{T}\;dN
\end{align*}
which implies that
\[dU=T\;dS-P\;dV+\mu \;dN\]
This generalised thermodynamic identity is a great way to remember the various partial-derivative formulas for $T$, $P$, and $\mu $, and to generate other similar formulas. 
\begin{itemize}
\item[(i)] With $U,V$ fixed, $0=T\;dS+\mu \;dN$
\item[(ii)] With $S,V$ fixed, $dU=\mu \;dN$
\item[(iii)] With $S,N$ fixed, $dU=-P\;dV$
\end{itemize}
\end{thm}
\vspace{2ex}
{\bf Chapter 4}\hspace{2ex}Engines and Refrigerators
\\\\
{\bf Chapter 4.1}\hspace{2ex}Heat Engines
\\
\begin{defi}
A heat engine is any device that absorbs heat and converts parts of that energy into work. It takes heat from a hot reservoir ($Q_{h}$) and converts it into work ($W$) and heat directed at the cold reservoir ($Q_{c}$). The efficiency of the system is defined as
\[\varepsilon =\dfrac{\mathrm{benifit}}{\mathrm{cost}}=\dfrac{W}{Q_{h}}\]
The total entropy is the sum of the entropy increase in each reservoir and the engine. The entropy change in the engine is taken to be zero and we have
\[\Delta S_{tot}=\dfrac{Q_{c}}{T_{c}}-\dfrac{Q_{h}}{T_{h}}>0\]
by the first law of thermodynamics. We then come to the conclusion that the efficiency has a upper limit.
\[\varepsilon <1-\dfrac{Q_{c}}{Q_{h}}\leq 1-\dfrac{T_{c}}{T_{h}}\]

\end{defi}
\vspace{2ex}

