\section{Lecture 7 (March 26th)}
\begin{recall}
We have previously learned about multiplicities of large Einstein solids (number of microstates for a certain macrostate) denoted by $\mathrm{\Omega} (N,q)$ where $N$ was the number of oscillators and $q$ were the number of energy units. Assuming that there were many more energy units than oscillators, we could show that 
\[\mathrm{\Omega} (N,q)=\begin{pmatrix}
q+N-1\\q
\end{pmatrix}=\dfrac{(q+N-1)!}{q!(N-1)!}\approx \dfrac{(q+N)!}{q!N!}\]
Additionally, we can use Stirling's approximation ($\ln N!\approx N\ln N- N$) to express its logarithm as
\begin{align*}
\ln \mathrm{\Omega} =& \ln \Big(\dfrac{(q+N)!}{q!N!}\Big)=\ln (q+N)!-\ln q!-\ln N!\\
\approx & (q+N)\ln (q+N)-(q+N)-q\ln q+q-N\ln N +N\\
=& (q+N)\ln \Big[q\Big(1+\dfrac{N}{q}\Big)\Big]-q\ln q-N\ln N\\
\approx &(q+N)\Big[\ln q+\dfrac{N}{q}\Big]-q\ln q-N\ln N\\
=& N\ln q+N+\dfrac{N^2}{q}-N\ln N\\
=& N\ln \Big(\dfrac{q}{N}\Big)+N+\dfrac{N^2}{q}\\ 
\approx &N\ln \Big(\dfrac{q}{N}\Big)+N
\end{align*}
In conclusion, we have the approximation
\[\mathrm{\Omega} (N,q)\approx \Big(\dfrac{eq}{N}\Big)^{N}\]
for $q>>N$.
\end{recall}
\vspace{2ex}
\begin{thm}
The sharpness of the multiplicity of a system of two large Einstein solids with many energy units per oscillator can be obtained by considering 
\[\mathrm{\Omega} =\Big(\dfrac{eq_{A}}{N}\Big)^{N}\Big(\dfrac{eq_{B}}{N}\Big)^{N}=\Big(\dfrac{e}{N}\Big)^{2N}(q_{A}q_{B})^{N}\]
where the width of the distribution can be found to be $x=q/2\sqrt{N}$. This result tells us that when two large Einstein solids are in thermal equilibrium with each other, any random fluctuations away from the most likely macrostate will be utterly unmeasurable.
\end{thm}
\vspace{2ex}
{\bf Chapter 2.5}\hspace{5ex}The Ideal Gas
