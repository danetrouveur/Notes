\section{Lecture 23 (May 28th)}
{\bf Chapter 6.1}\hspace{2ex}The Boltzmann factor
\newline
\begin{defi}
(Boltzmann factor) The Boltzmann factor is defined as the increase of the number of states as a system gains a small amount of energy $\varepsilon $ from the ground state.
\[\dfrac{\mathrm{\Omega} (E_{T}-\varepsilon )}{\mathrm{\Omega} (E_{T})}=\dfrac{\exp ({S(E_{T}-\varepsilon )/k})}{\exp ({S(E_{T})/k})}\]
\end{defi}
\vspace{2ex}
\begin{defi}
(Partition function) The partition function is defined as the sum of all Boltzmann factors
\[Z=\sum _{S}\exp \Big(-\dfrac{E(s)}{kt}\Big)\]
Since the probability to be at the state $S$ is proportional to the Boltzmann factor,
\[P(S)\propto \exp \Big(-\dfrac{E(s)}{kt}\Big)\]
and
\[P(S)=\dfrac{1}{Z}\exp\Big(-\dfrac{E(S)}{kt}\Big)=\dfrac{1}{Z}\exp (-\beta E(S))\]
where $\beta =1/kT$.
\end{defi}
\vspace{2ex}
{\bf Chapter 6.2}\hspace{2ex}Average values
\newline
\begin{defi}
(Energy average value) The enerrgy average value is defined as the weighted average of the energy of a system.
\[\bar{E}=\sum _{S}E(S)\cdot P(S)=\dfrac{1}{Z}\sum _{S}E(S)\exp \Big(-\dfrac{E(S)}{kT}\Big)\]
Extending this argument, for any quantity $X(S)$,
\[\bar{X}=\dfrac{1}{Z}\sum _{S}X(S)\exp \Big(-\dfrac{E(S)}{kt}\Big)\]
\end{defi}
\vspace{2ex}
\begin{defi}
Previously, we looked at {\it micro-canonical ensembles}. However, now on, we look at {\it canonical ensembles}.
\end{defi}
\vspace{2ex}
\begin{thm}
The average energy is related with the partion function by
\[\bar{E}=-\dfrac{1}{Z}\dfrac{\partial Z}{\partial \beta }=-\dfrac{\partial }{\partial \beta }\ln Z  \]
or, equivalently,
\[\bar{E}=-\dfrac{\partial T}{\partial \beta }\dfrac{\partial }{\partial T}\ln Z=kT^2\cdot \dfrac{\partial }{\partial T}\ln Z   \]
\end{thm}
\vspace{2ex}
\begin{proof}
\[Z=\sum _{S}e^{-\beta E(S)}\]
and 
\[\dfrac{\partial Z}{\partial \beta }=\sum _{S}(-E(S))e^{-\beta E(S)} \]
leading to
\[\bar{E}=\dfrac{1}{Z}\sum _{S}E(S)e^{-\beta E(S)}=-\dfrac{1}{Z}\dfrac{\partial Z}{\partial \beta } \]
\end{proof}
\vspace{2ex}
\begin{ex}
(Paramagnetism) Let's apply the Boltzmann factor to the ideal two-state paramagnet. The partition function is given by 
\[Z=\sum _{S}e^{-\beta E(S)}=e^{\beta \mu B}+e^{-\beta \mu B}=2\cosh (\beta \mu B)\]
The average energy, on the other hand, is
\[\bar{E}=\dfrac{1}{Z}\sum _{S}E(S)e^{-\beta E(S)}=\dfrac{-\mu B[e^{\beta \mu B}-e^{-\beta \mu B}]}{2\cosh (\beta \mu B)}=-\mu B\tanh (\beta \mu B)\]
For a collection of $N$ dipoles, $U=N\bar{E}=-N\mu B\tanh (\beta \mu B)$.
\end{ex}
\vspace{2ex}
\begin{ex}
(Equipartition theorem) Classically, rotation of diatomic molecules are given as
\[E=\dfrac{L_{x}^2}{2I_{x}}+\dfrac{L_{y}^2}{2I_{y}}\]
while in a quantum mechanical setting,
\[E(j)=j(j+1)\varepsilon \]
with degeneracy $2j+1$. The partition function is given as
\[Z=\sum ^{\infty }_{j=0}(2j+1)\exp \Big(-\dfrac{j(j+1)\varepsilon }{kT}\Big)\]
For $kT\gg \varepsilon $, the number of terms that contribute significantly to the partition function will be quite large. So
\[Z\approx \int ^{\infty }_{0}(2j+1)e^{-j(j+1)\varepsilon /kT}\,dj=\int^{\infty }_{0}\dfrac{kT}{\varepsilon }e^{-x}\,dx=\dfrac{kT}{\varepsilon }\Big[-e^{-x}\Big]^{\infty }_{0}=kT/\varepsilon \]
Now using $\bar{E}=-\partial \ln Z/\partial \beta $,
\[\bar{E}=-\dfrac{\partial }{\partial \beta }\ln \Big(\dfrac{1}{\beta \varepsilon }\Big)=\dfrac{1}{\beta }=kT \]
This is just the prediction of the equipartition theorem! 
\[\bar{E}=\dfrac{f}{2}\cdot kT\]
\end{ex}
\vspace{2ex}
\begin{rmk}
For diatomic molecules made of distinguishable atoms (for example, carbon dioxide), previous results are correct. However, for molecules made of identical atoms (for example, nitrogen or oxygen dioxide), indistinguishability should be considered.
\end{rmk}
\vspace{2ex}
{\bf Chapter 6.3}\hspace{2ex}The equipartition theorem
\newline
\begin{recall}
The equipartition theorem stated that for systems whose energy is in the form of quadratic degrees of freedom,
\[E=\sum_{i}^{f}c_{i}q_{i}^2\quad\implies \quad \bar{E}=\dfrac{f}{2}\cdot kT\]

\end{recall}
\vspace{2ex}

