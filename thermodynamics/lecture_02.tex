\section{Lecture 2 (March 10th)}
\begin{rmk}
The postulate of equal a priori probability is a fundamental assumption that states that all accessible quantum states are assumed to be equally probable (there is no reason to prefer some acessible states over other acessible states). Examples are infinite potential wells and particles in a box. A corollary is that macrostates that affords largest possible number of microstates are most probable states. When we look at a macroscopic state, we have microstates that form it, and its probability is proportional to the number of such possible microstates.
\end{rmk}
\vspace{2ex}
\begin{ex}
Imagine a system of $N$ electrons with positive and negative spin. Its energy would be govern by the following set of equations
\begin{align*}
N_{\uparrow}+N_{\downarrow}=N\hspace{5ex}N_{\uparrow}-N_{\downarrow}=2s\hspace{5ex}U=-2sB
\end{align*}
The full width half maximum is given approximately as $FWHM\sim 1/\sqrt{2N}\sim 10^{-12}$ for $N=10^{24}$. In this way, when there are many particles, we can become certain of the macrostate prevalent.
\end{ex}
\vspace{2ex}
\begin{defi}
Engraved on Boltzmann's tombstone is the Entropy equation, given by
\[S=k_{B}\ln \mathrm{\Omega }\]
where $S$ is the definition of entropy, $\mathrm{\Omega } $ is the number of possible microstates, and $k_{B}$ is the possible microstates. Obviously, when there are higher accessible microstates, there is a higher entropy for the system. Well why was entropy defined this way? Consider two systems $A_1$ and $A_2$ described by triplets $(N_{1},V_{1},E_{1})$ and $(N_{2},V_{2},E_{2})$ denoting number of particles, volume, and energy. Let energy be transferable between the two systems. Being a closed system altogether, energy conservation dictates and $E_{\mathrm{tot}}=E_1+E_2=\mathrm{constant}$. The accessible states would be given by
\begin{align*}
	\mathrm{\Omega }_{\mathrm{tot}}(E_1,E_2)=\mathrm{\Omega}_{1}(E_{1})\mathrm{\Omega}_{2}(E_2)&=\mathrm{\Omega }_{1}(E_{1})\mathrm{\Omega }_{2}(E_{\mathrm{tot}}-E_{1})\\
	0=\dfrac{\partial \mathrm{\Omega}  _{\mathrm{tot}}}{\partial E_{1}}=\dfrac{\partial \mathrm{\mathrm{\Omega} }_1(E_1)}{\partial E_1}\mathrm{\Omega} _2(E_2)&+\mathrm{\Omega} _1(E_1)\dfrac{\partial \mathrm{\Omega} _2(E_2)}{\partial E_2}\dfrac{\partial E_2}{\partial E_1}
\end{align*}
Continuing,
\begin{align*}
	\dfrac{\partial E_2}{\partial E_1}=&-1\\
	\dfrac{\partial \mathrm{\Omega} _1(E_1)}{\partial E_1}\mathrm{\Omega} _2(E_2)=&\mathrm{\Omega} _1(E_1)\dfrac{\partial \mathrm{\Omega} _2(E_2)}{\partial E_2}\\  
	\dfrac{\partial \ln\mathrm{\Omega} _1(E_1)}{\partial E_1}=&\dfrac{\partial \ln\mathrm{\Omega} _2(E_2)}{\partial E_2}  
\end{align*}
Knowing that $dS=dQ/T$, we have the final form we introduced above.
\end{defi}
\vspace{2ex}
\begin{thm}
	(Entropy and the 2nd law of thermodynamics) A physical system, left to itself, proceeds naturally in a direction that enables it to assume an ever-increasing number of microstates until it finally settles down in a macrostate that affords the largest number of microstates.
\end{thm}
\vspace{2ex}
\begin{defi}
Take a single particle system that interacts with a larger reservoir system with temperature $T$. The total energy would be constant, and the number of states for the former system's (system A) energy being $\varepsilon  $ and 0 would be given as
\[1\times \mathrm{\Omega} (E_{\mathrm{tot}}-\varepsilon )=1\times e^{S(E_{\mathrm{tot}}-\varepsilon )/k}\hspace{5ex}1\times \mathrm{\Omega} (E_{\mathrm{tot}})=1\times e^{S(E_{\mathrm{tot}})/k}\]
Taylor expanding $S(E_{\mathrm{tot}}-\varepsilon )$ we have
\[S(E_{\mathrm{tot}}-\varepsilon )=S(E_{\mathrm{tot}})-\varepsilon \Big(\dfrac{\partial S}{\partial E} \Big)_{E=E_{\mathrm{tot}}}+\ldots=S(E_{\mathrm{tot}})-\varepsilon \dfrac{1}{T} \]
where $\partial S/\partial E=1/T$. Thus the ratio between the number of states for system energy with energy $\varepsilon $ and $0$ we have
\[\dfrac{E_{\varepsilon }}{E_{0}}=\dfrac{1\times \mathrm{\Omega} (E_{tot}-\varepsilon )}{1\times \mathrm{\Omega} (E_{\mathrm{tot}})}\]
which is approximately the Boltzmann factor.
\[e^{-\varepsilon /kT}\]
The probability for system $A$ to have energy 0 or $\varepsilon $ would then by given by 
\[\dfrac{1}{1+e^{-\varepsilon /kT}}\hspace{5ex}\dfrac{e^{-\varepsilon /kT}}{1+e^{-\varepsilon /kT}}\]
respectively.
\end{defi}
\vspace{2ex}
\begin{defi}
Helmholtz free energy is defined for a system with energy $E$ and entropy $S$ in a reservoir with temperature $T$ as
\[F=E-TS\]

\end{defi}
\vspace{2ex}

