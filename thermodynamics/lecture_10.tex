\section{Lecture 10 (April 7th)}
{\bf Chapter 3.1}\hspace{2ex}Temperature
\begin{defi}
The second law of thermodynamics states that when two objects are in thermal equilibrium, their total entropy has reached its maximum value. However, we have also defined temperature as something that is the same when two objects are in thermal equilibrium. Through this, we can mathematical define temperature. We notice that in thermal equilibrium, 
\[\dfrac{\partial S_{tot}}{\partial U_{A}}0\hspace{4ex}\mathrm{and}\hspace{4ex}\dfrac{\partial S_{A}}{\partial U_{A}}=\dfrac{\partial S_{B}}{\partial U_{B}}\]
Likewise, we define 
\[T=\Big(\dfrac{\partial U}{\partial S} \Big)\]
\end{defi}
\vspace{2ex}
{\bf Chapter 3.1}\hspace{2ex}Entropy and Heat
\\
\begin{defi}
Heat capacity is defined as 
\[C_{V}=\Big(\dfrac{\partial U}{\partial T} \Big)_{N,V}\]
\end{defi}
\vspace{2ex}
\begin{thm}
(Third law of thermodynamics) At zero temperature, a system should settle into its unique lowest energy state.
\end{thm}
\vspace{2ex}
{\bf Chapter 3.3}\hspace{2ex}Paramagnetism
\\
\begin{defi}
A two-state paramagnet consisting of $N$ microscoptic magnetic dipoles respond only to the influence of the external magnetic field $B$. In a "down" state, they have an energy of $+\mu B$ while in an "up" state they have an energy of $-\mu B$. The total energy is then given as
\[U=\mu B(N_{\downarrow}-N_{\uparrow})\]
The magnetization is given as
\[M=\mu (N_{\uparrow}-N_{\downarrow})=-\dfrac{U}{B}\]
\end{defi}
\vspace{2ex}

