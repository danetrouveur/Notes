\section{March 24th}
{\bf Chapter 2}\hspace{2ex}The Second Law
\\
{\bf Chapter 2.1}\hspace{2ex}Two State System
\\
\begin{defi}
The multiplicity is defined as the number of microstates for a given macrostate. For example, the number of microstates (the number of outcomes) for getting $n$ coins from 100 throws is
\[\mathrm{\Omega} (n)=_{100}C_{n}=\dfrac{100!}{(100-n)!n!}\]
An analogous situation is an array of two state paramagnets. The number of microstates for $N$ upside dipoles is
\[\mathrm{\Omega} (N_{\uparrow})=_{N}C_{N_{\uparrow}}=\dfrac{N!}{N_{\uparrow}!(N-N_{\uparrow}!)}=\dfrac{N!}{N_{\uparrow}!N_{\downarrow}!}\]
\end{defi}
\vspace{2ex}
{\bf Chapter 2.2}\hspace{2ex}The Einstein Model of a Solid
\\
\begin{defi}
The Einstein model considers a solid as a collection of identical oscillators with quantized energy units.\[E=\underbrace{\dfrac{p^2}{2m}+\dfrac{1}{2}kx^2}_{\mathrm{harmonic osillator}}=\dfrac{p^2}{2m}+\dfrac{1}{2}m\omega ^2x^2\implies E_{n}=\underbrace{\hbar \omega \Big(n+\dfrac{1}{2}\Big)}_{\mathrm{quantization}}\]
\end{defi}
\vspace{2ex}
\begin{ex}
By modeling particles this way, we can seek the multiplicity of energy states in a Einstein solid. 
\[\mathrm{\Omega} (N,q)=_{N+q-1}C_{q}=\dfrac{(N+q-1)!}{(N-1)!q!}\]
\end{ex}
\vspace{2ex}
{\bf Chapter 2.3}\hspace{2ex}Interacting Systems
\\
\begin{defi}
We say that two systems are weakly coupled if the exchange of energy between them is much slower than the exchange of energy among atoms within each solid.
\end{defi}
\vspace{2ex}
\begin{thm}
 (Fundamental assumption of statistical mechanics) In thermodynamics, we assume that every microstate is equally probable.
\end{thm}
\vspace{2ex}
{\bf Chapter 2.4}\hspace{5ex}Large Systems
\\
\begin{defi}
Stirling's approximation is given as
\[N!\approx N^{N}e^{-N}\sqrt{2\pi N}\]
We then have
\[\ln N!\approx N\ln N-N+\dfrac{1}{2}\ln(2\pi N)\approx N\ln N-N\]
Using this approximation, we can calculate the multiplicity of a large Einstein solid (assuming $q>>N$).
\end{defi}
\vspace{2ex}


