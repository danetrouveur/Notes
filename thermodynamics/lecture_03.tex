\section{Lecture 3 (March 12th)}
\begin{thm}
$F=E-TS$ is minimum at thermal equilibrium.
\end{thm}
\vspace{2ex}
\begin{proof}
\begin{align*}
	S_{tot}=&S_{R}+S_{S}\\
	=&S_{R}(E_{tot}-E_{S})+S_{S}(E_{S})\\
	=&S_{R}(T_{tot})-E_{S}\Big(\dfrac{\partial S_{R}}{\partial E} \Big)_{E=E_{tot}}+S_{S}(E_{S})\\
	=&S_{R}(E_{tot})-E_{S}\dfrac{1}{T_{R}}+S_{S}(E_{S})\\
	=&S_{R}(E_{tot})-E_{S}\dfrac{1}{T_{S}}+S_{S}(E_{S})
\end{align*}
Where we have used that $E_{S}\ll E_{tot}$ and that at equillibrium, $T_{R}=T_{S}$. As the above value has the maximum value when at equilibrium, the Helmholtz free energy defined as $E-TS$ has its minimum value at equillibrium. Note that we have only used two concepts. (1) $1/T=\partial S/\partial E$ and that $\mathrm{\Omega} $ or $S$ is maximum at equilibrium.
\end{proof}
\vspace{2ex}
\begin{defi}
Temperature governs the flow of energy while chemical potential, on the other hand, governs the flow of particles. Take a thermal reservoir of temperature $T$ and a system in it. Alike how we defined the Boltzmann factor, we obtain the entropy of the reservoir after a single particle leaves it and enters our system.
\begin{align*}
	S_{R}(E_0-\varepsilon ,N_0-1)\approx & S_{R}(E_{0},N_0)-\varepsilon \Big(\dfrac{\partial S_{R}}{\partial E_{R}} \Big)_{V,N}-1\cdot \Big(\dfrac{\partial S_{R}}{\partial N_{R}} \Big)_{V,E}\\
	=&S_{R}(E_{0},N_{0})-\varepsilon \dfrac{1}{T}+\mu \dfrac{1}{T}\\
	=&S_{R}(E_{0},N_{0})-\dfrac{\varepsilon }{T}+\dfrac{\mu }{T}
\end{align*}
where we have used $\partial S_{R}/\partial E_{R}=1/T$, $\partial S_{R}/\partial N_{R}=-\mu /T$ and for $E_{0}\gg \varepsilon $ with $N_{0}\gg 1$
\[\dfrac{\mu }{T}=-\dfrac{\partial S}{\partial N} \]
The Gibbs factor is then given as the probability ratio
\[\dfrac{P(\varepsilon ,t)}{P(0,0)}=\exp\{(\mu -\varepsilon )/kT\}\]
\end{defi}
\vspace{2ex}
\begin{thm}
Particle flows from higher chemical potential to lower chemical potential.
\end{thm}
\vspace{2ex}
\begin{proof}
Let two systems be undergoing energy transfer. Let $1/T_{A}=(\partial S_{A}/\partial E_{A})_{V,N}$ and $1/T_{B}=(\partial S_{B}/\partial E_{B})_{N,V}$. We examine two cases, first when energy $\Delta \varepsilon (>0)$ flows from A to B. Then,
\begin{align*}
	\Delta S_{A}=&-\dfrac{1}{T}\Delta \varepsilon \\
	\Delta S_{B}=&\dfrac{1}{T_{B}}\Delta \varepsilon \\
	\Delta S_{tot}=&\Delta S_{A}+\Delta S_{B}=\Delta \varepsilon \Big(-\dfrac{1}{T_{A}}+\dfrac{1}{T_{B}}\Big)>0
\end{align*}
Lets contrarily assume energy $\Delta \varepsilon (>0)$ goes from B to A.
\[\Delta S_{tot}=\Delta \varepsilon \Big(\dfrac{1}{T_{A}}-\dfrac{1}{T_{B}}\Big)<0\]
Therefore, the first case occurs. Likewise, for chemical potential $\mu /T=-(\partial S/\partial N)_{V,E}$, particles flow from a system with a high chemical potential value to one with a lower chemical potential value (as $\Delta S \propto 1/T$ and $\Delta S\propto -\mu $). 
\end{proof}
\vspace{2ex}
\begin{defi}
The Fermi-Dirac distribution takes a quantum state interacting with a reservoir. The occupation probability can be given by the ratio between the number of occupied states and the number of total states.
\[\dfrac{\exp\{(\mu -\varepsilon )/kT\}}{1+\exp\{(\mu -\varepsilon )/kT\}}=\dfrac{1}{1+\exp\{(\varepsilon -\mu )/kT\}}\approx \dfrac{1}{\exp\{(\varepsilon -\mu )/kT\}}\]
\end{defi}
\vspace{2ex}
\begin{rmk}
In a semiconductor, the chemical poetntials are the Fermi levels. 
\end{rmk}
\vspace{2ex}
\begin{rmk}
To summarize, we have the following definitions for entropy, free energy, Boltzmann factor, chemical potential, and the Gibbs factor.
\[
S=k\ln \mathrm{\Omega} \hspace{4ex}
e^{-\varepsilon /kT}\hspace{4ex}
F=E-TS\hspace{4ex}
\dfrac{\mu }{T}=-\Big(\dfrac{\partial S}{\partial N} \Big)_{V,E}\hspace{4ex}
\exp\{(\mu -\varepsilon )/kT\}
\]
\end{rmk}
\vspace{2ex}
{\bf Chapter 1}\hspace{2ex}Energy in Thermal Physics
\\\\
{\bf 1.1}\hspace{2ex}Thermal Equilibrium
\\
\begin{defi}
Temperature is what you measure with a thermometer operationally. Theoretically, temperature is a measure of the tendency of an object to spontaneously give up energy to its surroundings. When two objects are in thermal contact, the one that tends to spontaneously lose energy is at the higher temperature. The absolute temperature is given as $T_{abs}=T_{cel}+273^{\circ}C$.
\end{defi}
\vspace{2ex}
{\bf 1.2}\hspace{2ex}The Ideal Gas
\\
\begin{defi}
Ideal gas is defined as gas whose particles do not interact with each other. Gas with low density theoretically approach this limit. In this case, the gas follows the following ideal gas law.
\[PV=nRT=NkT\]
$n$ is defined as the number of moles, where 1 mole is defined as $N=N_{A}=6.02\times 10^{23}$ (Avagadro's number). Accordingly, $R=N_{A}k$ which implies $k=R/N_{A}=8.31/6.02\times 10^{23}=1.38\times 10^{-23}\;J/K$. 
\end{defi}
\vspace{2ex}
\begin{thm}
The kinetic energy of gas can be given as
\[E_{k}=\dfrac{3}{2}kT\cdot N\]
\end{thm}
\vspace{2ex}
\begin{proof}
Consider a cylinder filled with gas molecules. We assume that the height and diameter of the cylinder is given as $L$. The pressure would be given as
\[P=\dfrac{F_{x,\mathrm{on\;the\;piston}}}{A}=\dfrac{-F_{x,\mathrm{on\;the\;molecule}}}{A}=-\dfrac{m(\Delta v_{x}/\Delta t)}{A}\]
Here, $\Delta t=2L/v_{x}$ and $\Delta v_{x}-2v_{x}$ where $v_{x}$ is the velocity of the particles. Therefore,
\[\Big(\dfrac{\Delta v_{x}}{\Delta t}\Big)=\Big(\dfrac{-2v_{x}}{2L/v_{x}}\Big)=-\dfrac{v_{x}^2}{L}\]
which allows us to conclude that
\[\bar{P}=\dfrac{m}{A}\cdot \dfrac{v_{x}^2}{L}=\dfrac{mv_{x}^2}{V}\]
knowing that $\bar{P}V=mv_{1x}^2+mv_{2x}+\ldots $, we can use the ideal gass law to obtain
\[PV=Nm\dfrac{1}{3}\bar{v^2}=N\cdot m\cdot \dfrac{1}{3}v_{rms}^2=\dfrac{2}{3}N\cdot \dfrac{1}{2}mv^2=\dfrac{2}{3}\bar{K}_{trans}\]
From the ideal gas law, we arrive at
\[\bar{K}_{trans}=\dfrac{3}{2}kT\cdot N\]

From this we know that the energy of gas in room temperature is given as $25\;meV$. 
\end{proof}
\vspace{2ex}
