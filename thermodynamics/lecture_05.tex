\section{Lecture 5 (March 19th)}
{\bf Chapter 1.7}\hspace{2ex}Rates of Processes
\\
\begin{rmk}
Thermodynamics, by many people's definitions, include only the study of equilibrium states themselves. Questions about time and rates of processes are then considered a separate (though related) subject (transport theory, kinetics). 
\end{rmk}
\vspace{2ex}
\begin{defi}
 (Heat conduction) Consider systems (inside and outside) separated by a wall of thickness $\Delta x$ and area $A$. The rate of transfer of head can be expected to be proportional to both the area of the wall and the temperature gradient (the ratio between temperature and displacement between the systems)
 \[\dfrac{\Delta Q}{\Delta x}=-kA\dfrac{dT}{dx}\] 
\end{defi}
\vspace{2ex}
\begin{defi}
 (Conductivity of an idea gas) We first learn the concept of mean free path. It is defined as the average distance traveled between collisions. Knowing that particles of radius $r$ collide once their distance converges to $2r$, we represent the path of a single particle as it travels a distance of $l$ by a cylinder of radius $r$ and length $l$. If the particle were to collide once, there must be a single particle within this volume and
 \[\dfrac{1}{V_{cylinder}}=\dfrac{N}{V_{gas}}\hspace{2ex}\implies \hspace{2ex}l=\dfrac{1}{2\pi r^2}\dfrac{N}{V_{gas}}\]
we assume that other particles are stationary, but nevertheless this is an appropriate approximation. To obtain conductivity, we now cut the wall made out of the ideal gas into thinner walls of length $l$. Assuming they all move in the same velocity, half of particles in one box would reach its adjacent one. Therefore, the heat crossing boundaries would be 
\[Q=\dfrac{1}{2}(U_1-U_2)=-\dfrac{1}{2}(U_2-U_1)=-\dfrac{1}{2}C_{V}(T_2-T_1)=-\dfrac{1}{2}C_V\cdot l \dfrac{d T}{d x} \]
We hereby obtain the conductivity of the gas $k_{t}=dk/dt$ as 
\[k_{t}=\dfrac{1}{2}\dfrac{C_{V}}{\Alpha }\dfrac{l^2}{\Delta t}=\dfrac{1}{2}\dfrac{C_{V}}{Al}\dfrac{l^2}{\Delta t}=\dfrac{1}{2}\dfrac{C_{V}}{V}l\cdot \bar{v}=\dfrac{1}{2}\dfrac{C_{V}}{V}\dfrac{1}{4\pi r^2}\dfrac{V}{N}\bar{v}\]
and we conclude $k_{t}\propto\bar{v}\propto\sqrt{T}$
\end{defi}
\vspace{2ex}
\begin{defi}
While heat conductivity is defined through energy transfer, viscosity is defined as momentum transfer.
\[F_{x}\propto \dfrac{A(u_{x,top}-u_{x,bottom})}{\Delta z}\]
and
\[\dfrac{|F_{x}|}{A}=\eta \dfrac{d u_{x}}{d z} \]
where the coefficient $\eta $ is defined as viscosity.
\end{defi}
\vspace{2ex}
\begin{defi}
 Diffusion is the transfer of particles, given by Fick's  law
 \[J_{x}=-D\dfrac{d n}{d x} \]
 The flux $J_{x}$ is defined as the net number of particles that cross the unit area per unit time.
\end{defi}
\vspace{2ex}
\begin{ex}
Consider a drop of food coloring added to a glass of water. Imagine that the dye has already spread uniformly through half of the glass. How long would it take for the dyed water to diffuse into the other half? Using Fick's law, 
\[J_{x}=-D\dfrac{d n}{d x}\hspace{2ex}\implies \hspace{2ex}\dfrac{N}{A\Delta t}=D\dfrac{N/V}{\Delta x}=D\dfrac{N/A\Delta x}{\Delta x} \]
Plugging $\Delta x=0.1\;m$ and $D=10^{-9}\;m^2/s$ in the equation, we obtain
\[\Delta t\approx 10^{7}\;s\]
\end{ex}
\vspace{2ex}

