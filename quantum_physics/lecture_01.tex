\section{Lecture 1 (Mar. 5th)}
\textbf{Chapter 1}\hspace{2ex}The Emergence of Quantum Physics
\\
\begin{thm} 
(A free particle as a plane wave) It was De Broglie that first suggested that all particles have wave-like properties and that equations for light can also be used to describe arbitrary particles
\[(E/c, p)=\hbar (\omega/c ,{ k} ) \]
As such, we will model a free particle with a precise momentum can as a complex plane wave.
\[\psi (x,t)=A \exp i\Big(\dfrac{{ p}{ x}}{\hbar}-\dfrac{Et}{\hbar}\Big)\]
Notice how the group velocity ($d\omega /dk$) is precisely the velocity of the particle as expected.\end{thm}
\vspace{2ex}
\begin{defi}
(Schrodinger's equation)
Differentiating, 
\[i\hbar\dfrac{\partial \psi }{\partial t}=E\psi\hspace{5ex}-i\hbar \dfrac{\partial \psi }{\partial x}  =p\psi   \]
Consider spliting the wave functions into two parts each dependent on $x$ and $t$
\[\psi (x,t)=\exp{\Big(\dfrac{-iEt}{\hbar}\Big)}\psi (x)\]
with $\psi (x)=\exp\{i{p}{ x}/\hbar\}$. In non-relativistic cases of free particles, energy is expressed as
\[E=\dfrac{1}{2}mv^2=\dfrac{p^2}{2m}\]
Attempting to obtain the form of this energy,
\[-\dfrac{\hbar^2}{2m}\dfrac{\partial ^2}{\partial x^2} \psi =\dfrac{p ^2}{2m}\psi =E\psi \]
The following equation is called the Schrodinger equation. Furthermore, we know that with scalar potential energy $V$, energy is expressed as $E=p^2/2m +V$. Then by extension we have
\[\Big(-\dfrac{\hbar^2}{2m}\dfrac{\partial ^2}{\partial x^2} +V(x)\Big)\psi (x)=E\psi (x)\]
which is called the time-independent Schrodinger equation. When we consider time-dependence, we have $E\psi(x)$ expressed as a derivation, or
\[\Big(-\dfrac{\hbar^2}{2m}\dfrac{d ^2}{d x^2} +V(x)\Big)\psi (x,t)=i\hbar\dfrac{\partial }{\partial t}\psi (x,t) \]
\end{defi}
\vspace{2ex}


