\section{Lecture 2 (March 10th)}
\begin{prop}
	 (Born Interpretation) In 1926, Born noticed that the Schrodinger equation must mathematically contain a radiation element, and proposed that it had to show the probability of the particle. The wave function being a complex function, it is natural to interpret its squared norm as probability, with
\[\int_{-\infty }^{\infty } |\psi  ({ x},t)|^2\;d{ x}=1\]
for all time $t$ and the integral being evaluated over all space. The important part is for the integral to be finite, or in other words, square integrable. 
\end{prop}
\vspace{2ex}
\begin{thm}
(Expectation of an observable quantity with zero standard deviation ) Recall that expectation $\langle A\rangle =a$ (observable with real expectation) for $0\leq P_{n}\leq 1$ and $\sum_{n}P_{n}=1 $ is given by
\[\langle A\rangle =\sum _{n}A_{n}P_{n}\rightarrow \int_{-\infty }^{\infty } d{ x}\;|\psi  ({ x})|^2A({x})\rightarrow \int_{-\infty }^{\infty } d{ x}\;\psi  ^{*}({ x})\Big(A({ x})\psi  ({x})\Big)\]
The right handside is a parallel statement written in operator notation. An important point is that when the expectation value converges to a singular value $\langle A\rangle =a$, we expect to have a negligible variance (standard deviation$^2$) which we can mathematically express as
\[\langle ( A-a)^2\rangle =\int_{-\infty }^{\infty } \psi  ^{*}(A-a)^2\psi =\int_{-\infty }^{\infty } \psi  ^{*}(A-a)\Big[(A-a)\psi  \Big]=\int_{-\infty }^{\infty } \Big[(A-a)\psi  \Big]^{*}\Big[(A-a)\psi  \Big]\]
If we were to take $A$ as an operator on $\psi $, the last equality would be only true when we impose that the operator $A$ is Hermitian which we will study later on. The above can be further simplified as
\[\int_{-\infty }^{\infty } |(A-a)\psi |^2\geq 0\]
For the above to be zero, we must have $A\psi  =a\psi $. In a general setting, we will show that, given that $A$ is a Hermitian operator, $A$'s expectation values will be seen as the eigenvalues of the operator. 
\end{thm}
\vspace{2ex}
\begin{thm}
(Shared eigenstates imply commutative operators) Suppose that two operators share eigenstates, that is, they share states in which their standard deviations are zero.
\[A(B\psi )=A(b\psi )=ba \psi =ab \psi =B(a\psi )=B(A\psi )\]
The above shows how $(AB-BA)\psi  =0$ and $[A,B]=0$ (the two operators commute). This implies that for non-commuting operators ($[\hat{x},\hat{p}]=i\hbar $), their eigenstates necessarily cannot be identical and that expectation values have non-zero variance ($\Delta x\Delta p\geq i\hbar $).
\end{thm}
\vspace{2ex}
