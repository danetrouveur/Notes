\section{Lecture 6 (March 24th)}
\begin{recall}
We have previously investigated a condition for which the probabilistic interpretation for quantum mechanics was valid.
\[0=\dfrac{d }{d x}\int dx\,\mathrm{\Psi} ^{*}\mathrm{\Psi} =\int dx\,\dfrac{\partial }{\partial t}(\mathrm{\Psi} ^{*}\mathrm{\Psi} )=-\int dx\,\dfrac{\partial }{\partial x}j(x,t)   \]
Where
\[j(x,t)=\dfrac{\hbar }{2im}\Big(\dfrac{\partial \mathrm{\Psi} }{\partial x}\mathrm{\Psi} ^{*}-\mathrm{\Psi} \dfrac{\partial \mathrm{\Psi}^{*} }{\partial x}  \Big)\]
is called the probability current density. We can see how the above formulation states that
\[-\oint dA\,{\bf n}\cdot {\bf j}(x,t)=0 \]

For the above to work, we require that the current density satisfies
\[\dfrac{\partial }{\partial t}(\mathrm{\Psi} ^{*}\mathrm{\Psi} )+\nabla \cdot {\bf j}(x,t)=0 \]
which is analogous to the continuity equation with $\rho =\mathrm{\Psi} ^{*}\mathrm{\Psi} $ and ${\bf j}(x,t)={\bf J}(x,t)$.
\end{recall}
\vspace{2ex}
\begin{ex}
We see that for the following wave function, our idea that current density is
equal to density times velocity aligns with our definition.
\[\psi (x)=C\exp \Big(i\dfrac{px}{\hbar }\Big)\]
we find the probability density to be $c^2p/m$ which exactly $(\psi \psi ^{*})v=\rho v$. This makes sense as this is exactly density times what we expect to be velocity.
\end{ex}
\vspace{2ex}
\begin{rmk}
(Expectation value) In quantum mechanics, we seek the expectation value of a measurable quantity $f(x)$. Given a probability wave $\mathrm{\Psi} (x,t)$, we have
\[\langle f(x)\rangle =\int dx\;\mathrm{\Psi} ^{*}(x,t)f(x)\mathrm{\Psi} (x,t)\]
we notice that as wave functions are dependent of time, the expectation values also have a time dependence. Let's verify that $\langle p\rangle =m d/dt\langle x\rangle $ with $p\rightarrow -i\hbar \partial /\partial x$
\begin{align*}
	\langle p\rangle =&m\dfrac{d }{d t}\langle x\rangle \\
	=&\int dx\;\mathrm{\Psi} ^{*}(x,t)x\mathrm{\Psi}(x,t)\\
	=&m\dfrac{\hbar }{2i}\int dx\;\Big(\underbrace{\Big(\dfrac{\partial ^2}{\partial x^2}\mathrm{\Psi} ^{*} \Big)x\mathrm{\Psi}}_{\dagger} -\mathrm{\Psi} ^{*}x\dfrac{\partial ^2}{\partial x^2}\mathrm{\Psi}  \Big)	
\end{align*}
Using Schrodinger's equation. Then,
\begin{align*}
	\dagger=&\dfrac{\partial }{\partial x}\Big(\dfrac{\partial \mathrm{\Psi} ^{*}}{\partial x}x\mathrm{\Psi}  \Big)-\dfrac{\partial \mathrm{\Psi} ^{*}}{\partial x}\dfrac{\partial }{\partial x}(x\mathrm{\Psi} )\\
=&\dfrac{\partial }{\partial x}\Big(\dfrac{\partial \mathrm{\Psi} ^{*}}{\partial x}x\mathrm{\Psi}  \Big)-\dfrac{\partial }{\partial x}\Big(\mathrm{\Psi} ^{*}\dfrac{\partial }{\partial x}(x\mathrm{\Psi} ) \Big)+\mathrm{\Psi} ^{*}\dfrac{\partial ^2}{\partial x^2}(x\mathrm{\Psi} )  
\end{align*}
However, the last term becomes (as $(fg)''=f''+2f'g'+g''$)
\[\mathrm{\Psi} ^{*}\Big(2\dfrac{\partial \mathrm{\Psi} }{\partial x}+x\dfrac{\partial ^2\mathrm{\Psi} }{\partial x^2}\Big)\]
Altogether,
\[\langle p\rangle =\mathrm{(boundary)}+(-i\hbar )\int dx\;\mathrm{\Psi} ^{*}\dfrac{d }{d x}\mathrm{\Psi}  \]
We now ask: is there no problem when the observable is a polynomial expression of $x$ and $p$? See that
\begin{align*}
	\hat{x}\hat{p}\mathrm{\Psi} (x)=&\hat{x}(-i\hbar \dfrac{\partial \mathrm{\Psi} }{\partial x} )\\
=&x(-i\hbar \dfrac{\partial \mathrm{\Psi} }{\partial x} )\\
\hat{p}\hat{x}\mathrm{\Psi}=&(-i\hbar )\dfrac{\partial }{\partial x}(\hat{x}\mathrm{\Psi} )\\
=&(-i\hbar )(\mathrm{\Psi} +x\dfrac{\partial \mathrm{\Psi} }{\partial x} )
\end{align*}
We thus see that
\[(\hat{x}\hat{p}-\hat{p}\hat{x})=[\hat{x}\hat{p}-\hat{p}\hat{x}]=i\hbar \]
This is called the fundamental canonical commutator relation. We find that we should be careful with the order in which $\hat{x}$ and $\hat{p}$ is applied ($\hat{x}^2\hat{p}^2\ne \hat{x}\hat{p}^2\hat{x}$).
\end{rmk}
\vspace{2ex}
\begin{rmk}
We emphasize that $\hat{A}$ must have a real expectation value given by
\[\langle \hat{A}\rangle =\int \mathrm{\Psi} ^{*}\hat{A}\mathrm{\Psi} \]
Or equivalently, $\hat{A}$ is Hermitian. We see that, indeed, for the momentum operator, 
\[\langle p\rangle ^{*}=\int \mathrm{\Psi} (i\hbar )\dfrac{\partial \mathrm{\Psi} ^{*}}{\partial x} \]
and that
\[\langle p\rangle ^{*}-\langle p\rangle =i\hbar \int dx\;\dfrac{\partial }{\partial x}(\mathrm{\Psi} ^{*}\mathrm{\Psi} )=i\hbar \mathrm{\Psi} ^{*}\mathrm{\Psi} \Big|^{\infty }_{-\infty } =0\]
\end{rmk}
In a similar manner, we can see that all polynomial expressions of $\hat{x}$ and $\hat{p}$ are Hermitian, and that the Hamiltonian operator also is. With the definition of the Hamiltonian operator, we can write the Schrodinger equation as
\[i\hbar \dfrac{\partial }{\partial t}\mathrm{\Psi} (x,t)=\hat{H}\mathrm{\Psi} (x,t) \]
\vspace{2ex}
\begin{rmk}
Let's see that the wave-packet in the momentum space can be normalized.
\begin{align*}
1=	\int _{-\infty }^{\infty }dx\;|\mathrm{\Psi} (x)|^2=&\int ^{\infty }_{-\infty }dx\;\mathrm{\Psi} ^{*}(x)\mathrm{\Psi} (x)\\
	=&\int dx\;\int \dfrac{dp'}{\sqrt{2\pi \hbar }}\int \dfrac{dp}{\sqrt{2\pi \hbar }}\phi ^{*}(p')\phi (p)\exp \Big\{\dfrac{-ip'x+ipx}{\hbar }\Big\}\\
	=&\int dp\;|\phi (p)|^2
\end{align*}
From the fact that the integral of the exponential term is
$2\pi \hbar \delta (p-p')$.
What have we learned? We have obtained the Parseval relation that tells us that if the wave function in the position space is squared-integrable, the function in the momentum space also is.
\\
We can also ask, then, what the momentum operator is in the momentum space.
\[\langle \hat{p}\rangle =\int \mathrm{\Psi} ^{*}(x)(-i\hbar )\dfrac{\partial \mathrm{\Psi} }{\partial x}=\int \dfrac{dp'}{\sqrt{2\pi \hbar }}\exp \Big\{\dfrac{-ip'x}{\hbar }\Big\}\phi ^{*}(p')\Big(-i\hbar \dfrac{\partial }{\partial x} \Big)\int \dfrac{dp}{\sqrt{2\pi \hbar }}\exp \Big\{\dfrac{ipx}{\hbar }\Big\}\phi (p)\]
which becomes
\[\int dp\;\phi ^{*}(p)p\phi (p)\]

\end{rmk}
\vspace{2ex}

