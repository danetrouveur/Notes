\section{Lecture 15 (April 30th)}
{\bf Chapter 5}\hspace{2ex}Essential Elements
\\
\begin{defi}
(Quantum State) A pure quantum state is defined as a ray in a Hilbert space. Here, a Hilbert space is a complex vector space with the inner product that is a complete. Some examples of Hilbert spaces are $L^{2}$, ${\bm C}^{n}$, and $M_{n\times n}$ 
\end{defi}
\vspace{2ex}
\begin{defi}
(Ray) By a ray, we mean that for $|\psi \rangle \in \mathcal{H}$, $\{c|\psi \rangle \}$ where $c\in \mathcal{C}$. As we are considering normalised wave functions, $\langle \psi |\psi \rangle =1$ and $|c|^2\langle \psi |\psi \rangle =1$ and $|c|^2=1$, leading to $c=e^{i\theta }$ and $\{e^{i\theta }\psi \}$. This is saying that a state is equivalent up to a transformation that conserves the modulus.
\end{defi}
\vspace{2ex}
\begin{defi}
(Composite system) Consider two Hilbert spaces $\mathcal{H}_{1}$ and $\mathcal{H}_{2}$. In the discussion of a system comprised of both Hilbert spaces, we necessarily look at bilinear maps.  A composite system of quantum states is therefore expressed as a tensor product
\[|\psi \rangle _{1}\otimes |\psi \rangle_{2}\in \mathcal{H}_{1}\otimes \mathcal{H}_{2}\]
When an element in such a tensor product can be expressed as a tensor product like the above, we say that such an element is separable. If not, we say that the element is entangled.
\end{defi}
\vspace{2ex}
\begin{ex}
(Kronecker product) We can consider the Kronecker product as a specific example of a tensor product. Let
\[A=\begin{pmatrix}
a&b\\c&d
\end{pmatrix}\quad B=\begin{pmatrix}
\alpha &\beta \\\gamma &\delta 
\end{pmatrix}
\]
We can use the representation
\[A\otimes B=\begin{pmatrix}
aB&bB\\cB&dB
\end{pmatrix}=\begin{pmatrix}
a\begin{pmatrix}
\alpha &\beta \\\gamma &\delta 
\end{pmatrix}&b\begin{pmatrix}
\alpha &\beta \\\gamma &\delta 
\end{pmatrix}\\\\c\begin{pmatrix}
\alpha &\beta \\\gamma &\delta 
\end{pmatrix}&d\begin{pmatrix}
\alpha &\beta \\\gamma &\delta 
\end{pmatrix}
\end{pmatrix}
\]
\end{ex}
\vspace{2ex}
\begin{ex}
Now, let $A=(x,y)$ and $B=(a,b)$. Then,
\[A\otimes B=(xB,yB)=\begin{pmatrix}
xa&ya\\xb&yb
\end{pmatrix}
\]
Also,
\[B\otimes A=\begin{pmatrix}
a(x,y)\\b(x,y)
\end{pmatrix}
=\begin{pmatrix}
ax&ay\\bx&by
\end{pmatrix}\]
\end{ex}
\vspace{2ex}
\begin{defi}
(Observables, their adjoint, and their expectation) An operator is a linear function from a Hilbert space to the complex numbers (linear map).
\[\hat{O}:\mathcal{H}\rightarrow {\bm C}\]
An adjoint of an operator is an operator that satisfies (we often denote $\hat{O}|\psi \rangle =|\hat{O}\psi \rangle $)
\[\langle \hat{O}\phi |\psi \rangle =\langle \phi |\hat{O}^{\dagger}|\psi \rangle \]
where $\hat{O}^{\dagger}$ is the adjoint. Notice how 
\[\langle \hat{O}\phi |\psi \rangle ^{*}=\langle \psi |\hat{O}\phi \rangle =\langle \hat{O}^{\dagger}\psi |\phi \rangle =\langle \phi |\hat{O}^{\dagger}\psi \rangle ^{*}\]
and that the definition for an adjoint operator can also be written as 
\[\langle \hat{O}^{\dagger}\phi |\psi \rangle =\langle \phi |\hat{O}|\psi \rangle \]
Now, for the expectation value for a measurement to be real, we have 
\[\langle \psi |\hat{O}|\psi \rangle =\langle \psi |\hat{O}|\psi \rangle^{*}=\langle \hat{O}\psi |\psi \rangle =\langle \psi |\hat{O}^{\dagger}|\psi \rangle  \]
That is, for an expectation value for a measurement to be real, the observable must be Hermitian. 
\end{defi}
\vspace{2ex}

