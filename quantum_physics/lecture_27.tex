\section{Lecture 27 (June 11th)}
\begin{thm}
We try to seek the eigenvectors for $S_{x}$ and $S_{y}$ in the basis generated by $S_{z}$. The characteristic equation for $S_{x}$ is
\[\lambda ^2-\Big(\dfrac{\hbar}{2}\Big)^2=0\]
and $\lambda=\pm \hbar /2$. Let the first eigenvector be denoted as $v=(v_1,v_2)$ we find that $v_1=v_2$ and from the normalisation condition, $v_1^2+v_2^2=1$ and we find that $v_1=v_2=1/\sqrt{2}$ where choose the sign to be positive by convention. We can also multiply $-1$ to either $v$ and $u$, but this would only change the spinor by a phase factor, and as same rays, they would have the same physical meaning.
\[v=\dfrac{1}{\sqrt{2}}\begin{pmatrix}
1\\1
\end{pmatrix}=|S_{x},\uparrow\rangle \quad \mathrm{and}\quad u=\dfrac{1}{\sqrt{2}}\begin{pmatrix}
1\\-1
\end{pmatrix}=|S_{x},\downarrow\rangle 

\]
for eigenvalues $\lambda_u=\hbar /2$ and $\lambda_v=-\hbar /2$ respectively. Similarly, the characteristic equation for $S_{y}$ is identical with $S_{x}$ and along with the normalisation condition, we find 
\[v=\dfrac{1}{\sqrt{2}}\begin{pmatrix}
1\\i
\end{pmatrix}=|S_{y},\uparrow\rangle \quad \mathrm{and}\quad u=\dfrac{1}{\sqrt{2}}\begin{pmatrix}
1\\-i
\end{pmatrix}=|S_{y},\downarrow\rangle 
\]
for eigenvalues $\lambda_v=\hbar /2$ and $\lambda_u=-\hbar /2$ respectively. We can additionally notice that they are perpendicular to each other.
\end{thm}
\vspace{2ex}
\begin{prop}
We now seek to the spin in an arbitrary direction $\hat{n}=(\sin \theta \cos \phi ,\sin \theta \sin \phi ,\cos \theta )$, the characteristic equation is still $\lambda^2=n_{x}^2+n_{y}^2+n_{z}^2=1$. The problem set related to this will definitely be in the test!
\end{prop}
\vspace{2ex}
\begin{rmk}
Consider an arbitrary spin state that is normalised. What is the probability that the result is $-\hbar /2$ when it is measured in the $\hat{n}$ direction? We simply expand the state with respect the basis and calculate the square of the coefficients.
\[|\psi \rangle =c_1|{\bf S}\cdot \hat{n},\uparrow\rangle +c_2|{\bf S}\cdot \hat{n},\downarrow\rangle \]
\end{rmk}
\vspace{2ex}
\begin{thm}
We show that there exists an operator that changes the spin state by an angle $\phi $.
\[\exp \Big(-\dfrac{i{\bf S}\cdot \hat{n}\,\phi }{\hbar }\Big)\]
For simplicity, take $\hat{n}=\hat{{\bf x}}$. 
\[\exp \Big(-\dfrac{i\sigma _{x}\phi }{2}\Big)=\sum _{n\in 2{\bm Z}}\dfrac{1}{n!}\Big(-\dfrac{i\phi }{2}\Big)^{n}(\sigma _{x})^{n}+\sum _{n\in 2{\bm Z}+1}\Big(-\dfrac{i\phi }{2}\Big)^{n}(\sigma _{x})^{n}\]
Which equal to 
\[\cos \Big(\dfrac{\phi }{2}\Big)I-i\sin \Big(\dfrac{\phi }{2}\Big)\sigma _{x}=\begin{pmatrix}
\cos \dfrac{\phi }{2}&-i\sin \dfrac{\phi }{2}\\
-i\sin \dfrac{\phi }{2}&\cos \dfrac{\phi }{2}
\end{pmatrix}
\]
Lets take the $(1,0)$ ket and rotate it by using the above matrix with $\phi =\pi /2$. What we expect is it to lie in $S_{y}$. This is true as we find
\[U_{\pi /2}(\hat{{\bf x}})=\dfrac{1}{\sqrt{2}}\begin{pmatrix}
1&-i\\-i&1
\end{pmatrix}\begin{pmatrix}
1\\0
\end{pmatrix}=\dfrac{1}{\sqrt{2}}\begin{pmatrix}
1\\-1
\end{pmatrix}
\]
We can caluclate to also find that after 360 degrees, the matrix is
\[\begin{pmatrix}
-1&0\\0&-1
\end{pmatrix}
\]
and after 720 degrees, the matrix is 
\[\begin{pmatrix}
1&0\\0&1
\end{pmatrix}
\]
In this sense, spinors are double valued.
\end{thm}
\vspace{2ex}
\begin{rmk}
There exists (god given) magnetic moment for electrons. To properly approach this, we require the Dirac equation, where we would apply an operator to the equation and do unrelativistic approximations. We then find
\[\hat{{\bf H}}=-{\bm \mu } \cdot {\bf B}(t)\]
with
\[{\bm \mu } =\dfrac{q(-e)}{2m_{e}}{\bf S}\]
and $g=2\times 1.0011596\ldots $. For orbital angular momentum, $g=1$. Let try and use this Hamiltonian to solve the equations of motion. What we expect is precessional motion.
\end{rmk}
\vspace{2ex}
\begin{proof}
Take 
\[|\psi (t)\rangle =\begin{pmatrix}
\alpha _{\uparrow}(t)\\
\alpha _{\downarrow}(t)
\end{pmatrix}
\]
and we have
\[\dfrac{eg\hbar }{4m_{e}}{\bm \sigma }\cdot {\bf B}(t)\begin{pmatrix}
\alpha _{\uparrow}(t)\\
\alpha_{\downarrow}(t)
\end{pmatrix}=i\hbar \begin{pmatrix}
\dot{\alpha} _{\uparrow}(t)\\
\dot{\alpha}_{\downarrow}(t)
\end{pmatrix}
\]
Taking ${\bf B}=B_{0}\hat{{\bf z}}$, we simply have the differential equations
\[i\dot{\alpha }_{\uparrow}(t)=\dfrac{egB_0}{4m_{e}}\alpha _{\uparrow}(t)\]
and
\[i\dot{\alpha }_{\downarrow}(t)=-\dfrac{egB_{0}}{4m_{e}}\alpha _{\downarrow}(t)\]
These have simple solutions,
\[\begin{cases}
\alpha _{\uparrow}(t)=\alpha _{\uparrow}(0)e^{-iw_{0}t}\\
\alpha _{\downarrow}(t)=\alpha _{\downarrow}(0)e^{iw_0t}
\end{cases}\]
Finding the expectation values with assuming hat initially the spin was up with $\alpha _{\uparrow}(0)=\alpha _{\downarrow}(0)=1/\sqrt{2}$, we have
\[\langle \psi (t)|\hat{S}_{x}|\psi (t)\rangle =\dfrac{\hbar }{2}\dfrac{1}{\sqrt{2}}(e^{i\omega _0t},e^{-i\omega _0t})\begin{pmatrix}
1&0\\0&1
\end{pmatrix}\dfrac{1}{\sqrt{2}}\begin{pmatrix}
e^{-i\omega_0t}\\e^{i\omega t}
\end{pmatrix}
=\dfrac{\hbar }{4}(e^{2i\omega_0t}+e^{-2i\omega_0t})=\dfrac{\hbar }{2}\cos (2\omega_0t)
\]
for $S_{y}$, we find the same result but with the sine function.
\end{proof}
\vspace{2ex}

