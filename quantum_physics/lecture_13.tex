\section{Lecture 13 (April 16th)}
\begin{thm}
Last class we've learned that in the double well problem, the ground state is given by an even solution. The odd solution would then be an excited state. The superposition of the two ($\psi _{e}-\psi _{o}$ or $\psi _{e}+\psi _{o}$) gives a asymmetrical distribution. Let's say that the wave function is initially in the right:
\begin{align*}
\psi (t>0)=&\exp \Big(-\dfrac{iE_{e}t}{\hbar }\Big)\psi _{e}(x)+\exp \Big(-\dfrac{iE_0t}{\hbar }\Big)\psi _{o}(x)\\
=&\exp \Big(-\dfrac{iE_{e}t}{\hbar }\Big)\Big(\psi _{e}+\exp \Big[-\dfrac{i(E_{o}-E_{e})t}{\hbar }\Big]\psi _{o}\Big)
\end{align*}
Given that $E_{o}>E_{e}$, we notice that when
\[\dfrac{E_{o}-E_{e}}{\hbar }t'=\pi \]
the sign of the function flips and the wave function moves to the left.
\end{thm}
\vspace{2ex}
\begin{thm}
(Potential problem for dirac distribution) Consider the potential
\[V(x)=-\dfrac{\hbar ^2}{2m}\dfrac{\lambda }{a}\delta (x)\]
where $\lambda $ is a dimensionless constant. The Schrodinger equation is
\[\psi ''-\kappa ^2\psi =-\dfrac{\lambda }{a}\delta (x)\psi \]
where $-\kappa^2=2mE/\hbar  $.
\[\begin{cases}
x>0\quad c_1e^{-\kappa x}\\
x<0\quad c_2e^{\kappa x}
\end{cases}\]
We require that the wave function is continuous at $x=0$, and discover that $c_1=c_2=c$. What about the derivative at $x=0$? The technique is to integrate the differential equation in a small ball around the origin.
\[\int ^{\varepsilon }_{-\varepsilon }dx\;\psi ''-\int ^{\varepsilon }_{-\varepsilon }dx\;\kappa ^2\psi =-\dfrac{\lambda }{a}\int ^{\varepsilon }_{-\varepsilon }dx\;\delta (x)\psi (x)\]
We then have
\begin{align*}
\psi '(x=0^{+})-\psi '(x=0^{-})=&\dfrac{\lambda }{a}\psi (0)\\
c(-\kappa )-c(\kappa )=&-c\dfrac{\lambda }{a}
\end{align*}
which allows us to have a single solution with $2\kappa =\lambda /a$. Two good exercises is when there is a double delta function well and there is a positive delta function barrier (this is in the test)!
\end{thm}
\vspace{2ex}
\begin{defi}
(Harmonic oscillator) (important) The Hamiltonian of a harmonic oscillator is given as 
\[\hat{H}\psi (x)=\Big(-\dfrac{\hbar ^2}{2m}\dfrac{d ^2}{d x^2}+\dfrac{1}{2}m\omega ^2x^2\Big) \psi (x)=E\psi (x)\]
Note that $\varepsilon =2E/\hbar \omega $ and $y=\sqrt{mx/\hbar }x$ are dimensionless constants (check this). Then,
\[\dfrac{d ^2\psi  }{d y^2}+(\varepsilon -y^2)\psi =0 \]
is an adjoint operator as $c_0'=c_1=0$. When $y$ is large, the $\varepsilon $ term can be ignored. Multiplying $\psi_0 '$ to both sides,
\begin{align*}
\psi_0 '\psi_0 ''-y^2\psi_0 \psi_0 '=&0 \\
\dfrac{d }{d y}[(\psi_0 ')^2-y^2\psi_0 ^2]=&-2y\psi_0 ^2 
\end{align*}
at this point, we can impose that $\psi \rightarrow 0$ and fastly and $|y|\rightarrow \infty $ and say:
\[(\psi_0 ')^2-y^2\psi_0 ^2=c\]
implying that
\[\dfrac{d \psi_0 }{d y}=\pm\sqrt{c+y^2\psi_0 ^2} =-y\psi_0 \]
As imposing boundary conditions, we find $c=0$ and as we want an exponential decay, we choose the negative sign. We find
\[\psi_0=e^{-y^2/2}\]
Setting $\psi (y)=e^{-y^2/2}h(y)$ as our ansatz we find
\[h''-2yh'+(\varepsilon -1)h=0\]
Notice that this is a Hermite differential equation. We can choose the following idicial equation near $y=0$
\[h(y)=\sum ^{\infty }_{j=0}y^{k}y^{j}a_{j}\]
We then obtain
\[0=\sum ^{\infty }_{j=0}(k+j)(k+j-1)a_{j}y^{k+j-2}+\sum ^{\infty }_{j=0}\Big[-2(k+j)+(\varepsilon -1)\Big]a_{j}y^{k+1}\]
where $j=0,1$ have no matching indexes. We then consider these separately, arriving at an indicial equation ($a_0\ne 0$ is obvious). For $k=0,1$ we have
\[\begin{cases}
j=0\quad 0=k(k-1)a_0\\
j=1\quad 0=(k+1)ka_1
\end{cases}\]
\[0=\sum ^{\infty }_{j=0}(k+j)(k+j-1)a_{j+2}y^{k+j}+\sum ^{\infty }_{j=0}\Big[-2(k+j)+(\varepsilon -1)\Big]a_{j}y^{k+j}\]
We arrive at a recurrence equation,
\[a_{j+2}=\dfrac{2(k+j)-(\varepsilon -1)}{(k+j+2)(k+j+1)}a_{j}\]
When $k=0$,
\begin{align*}
a_2=&\dfrac{2(0-(\varepsilon -1)/2)}{2\cdot 1}a_0\\
a_{4}=&\dfrac{2^2}{4!}\Big(2-\dfrac{(\varepsilon -1)}{2}\Big)\Big(0-\dfrac{(\varepsilon -1)}{2}\Big)a_0
\end{align*}
When $k=1$,
\begin{align*}
a_2=&2-\dfrac{1-(\varepsilon -1)/2}{2\cdot 3}a_0\\
a_4=&\dfrac{2^2}{5!}\Big(3-\dfrac{(\varepsilon -1)}{2}\Big)\Big(1-\dfrac{(\varepsilon -1)}{2}\Big)
\end{align*}
For an arbitrary $\varepsilon $, such a sequence explodes, and we must choose a special $\varepsilon $ and truncate the infinite series. In the case that $k=0$, we truncate when
\[\dfrac{\varepsilon -1}{2}=0,2,4,\ldots \]
and in the case that $k=1$, we truncate when 
\[\dfrac{\varepsilon -1}{2}=1,3,5,\ldots \]
\end{defi}
\vspace{2ex}

