\section{Lecture 4 (March 17th)}
\begin{rmk}
The metric $d:V\times V\rightarrow {\bm R}$ in the $L^2$ space is defined as
\[d(f,g)=\sqrt{\int _{I}|f-g|^2}\geq 0\]
We have previously made the remark that
\[d(f,g)=0\iff [f]=[g]\]
We note that the metric satisfies the triangle inequality,
\[d(f,g)\leq d(f,h)+d(h,g)\]
Which completes our proof that the function is indeed a metric. 
\end{rmk}
\vspace{2ex}
\begin{defi}
A Cauchy sequence is a sequence $\{x_{n}\}$ such that for all $\varepsilon >0$ there exists a $N$ such that for all $m,n>N$,
\[d(x_{n},x_{m})<\varepsilon \]
\end{defi}
\vspace{2ex}
\begin{defi}
A set is called complete if all Cauchy sequences in the set converge to a element in the set.
\[\lim _{n\rightarrow \infty }x_{n}=x\in V\]
It is important to note that $L^2(I)$ is complete (Riesz-Fisher theorem). 
\end{defi}
\vspace{2ex}
\begin{thm}
(Uncertainty principle seen as the product of distribution deviations) We have previously shown that a wave function can be expressed as
\[\mathrm{\Psi} (x,t)=\int \dfrac{dk}{2\pi }\;A(k)\exp(ikx-i\omega (k)t)\]
Assume that the distribution of $A(k)$ is given as
\[A(k)= \exp\Big[-\dfrac{(k-k_0)^2}{\alpha }\Big]\]
Then the approximate width (the distance between the maximum of the distribution and the point where the graph falls to $1/e$) would be given as $(\Delta k/2)^2=\alpha $ and $\Delta k=2\sqrt{\alpha }$. Integrating the function at $t=0$, 
\begin{align*}
	\psi  (x,0)=&\int_{-\infty }^{\infty } \dfrac{dk}{2\pi }\exp\Big[-\dfrac{(k-k_0)^2}{\alpha }+ikx\Big]\\
	=&\int _{-\infty }^{\infty }\dfrac{d k'}{2\pi }\exp\Big[-\dfrac{k'^2}{\alpha }+i(k_0+k')x\Big]\\
	=&\exp(ik_0x)\int _{-\infty }^{\infty }\dfrac{dk'}{2\pi }\exp\Big[-\dfrac{ k'^2}{\alpha }+ik'x\Big]\\
=&\exp\Big(ik_{0}x-\dfrac{\alpha x^2}{4}\Big)\int _{-\infty }^{\infty }\dfrac{dk'}{2\pi }\exp\Big[-\dfrac{1}{\alpha }\Big( k'-\dfrac{i\alpha x}{2}\Big)^2\Big]
\end{align*}
with $ k'=k-k_0$ leading to $k= k'+k_0$. Recall that for $a\in {\bm R}>0$,
\[\int _{-\infty }^{\infty }e^{-ax^2}=\sqrt{\dfrac{\pi }{a}}\]
However, the integral above can be done on a closed contour $\mathcal{C}$. Cauchy's theorem states that on this contour,
\[\oint_{\mathcal{C}}dz\;f(z)=0\]
if $f(z)$ is analytic. An example of such a function $e^{-z^2}$. With this knowledge, we preform the integral without the imaginary part shift along the real axis from $+\infty $ to $-\infty $ and down to $y=q-i\alpha x/2$ and back to positive infinity. 
\[\oint_{C}dk'\; f(k')=0=(\mathrm{wanted})-\exp\Big[ik_0x-\dfrac{\alpha x^2}{4}\Big]\int _{-\infty }^{\infty }\dfrac{d k'}{2\pi }\exp\Big[-\dfrac{k'^2}{\alpha }\Big] \]
which results in
\[(\mathrm{wanted})=\exp\Big[ik_0x-\dfrac{\alpha x^2}{4}\Big]\int _{-\infty }^{\infty }\dfrac{dk'}{2\pi }\exp\Big[-\dfrac{k'^2}{\alpha }\Big]=\exp\Big[ik_0x-\dfrac{\alpha x^2}{4}\Big]\dfrac{1}{2}\sqrt{\dfrac{\alpha }{\pi }}\]
In sum,
\[\mathrm{\Psi} (x,0)=\dfrac{1}{2}\sqrt{\dfrac{\alpha }{\pi }}\exp\Big[ik_0x-\dfrac{\alpha x^2}{4}\Big]\]
Preforming the operation to find the width again, we find $\Delta x=4/\sqrt{\alpha }$ and thus $\Delta k\Delta x=8$, a single number. 
\end{thm}
\vspace{2ex}
\begin{thm}
(The group velocity of the wave packet as the velocity of the particle)
What about the case where the wave function evolves throughout time?
\[\psi (x,t)=\int ^{\infty }_{-\infty }\dfrac{dk}{2\pi }\;A(k)e^{ikx-i\omega (k)t}\]
Preforming a Taylor expansion about $k_0$, 
\[\omega (k)=\omega (k_0)+\Big(\dfrac{d \omega }{d k} \Big)_{k=k_0}(k-k_0)+\dfrac{1}{2}\Big(\dfrac{d ^2\omega }{d k^2} \Big)_{k=k_0}(k-k_0)^2+\ldots\]
Here, we can substitute $(d\omega /dk)_{k=k_0}=v_{g}$ and $(k-k_0)=q$. Ignoring the higher order terms, we have for the wave function
\[\psi (x,t)=e^{ i(k_0x-\omega (k_0)t)}\int_{-\infty }^{\infty }\dfrac{dq}{2\pi }\;A(q+k_0)e^{iq(x-v_{g}t)}\exp \Big[-\dfrac{iq^2t}{2}\Big(\dfrac{\partial^2 \omega }{\partial k^2}\Big)_{k=k_0} \Big]\]
We realise that we can preform the integral alike the one we did beforehand and results in (with the substitution of $\beta $ for the second derivative),
\[\psi (x,t)=\sqrt{\dfrac{2\pi }{\alpha +i\beta t}}\exp\Big[ik_0x-i\omega (k_0)t-\dfrac{1}{2}\dfrac{(x-v_{g}t)^2}{\alpha +i\beta t}\Big]\]
The result's main punchline is that the wave has a maximum when $x-v_{g}t=0$. That is, the wave-particle has the group velocity of the wave! ($v_{g}|_{\mathrm{wave}}\iff v|_{\mathrm{particle}}$)! The probability $|\psi  (x,t)|^2$ becomes
\[\dfrac{2\pi }{\sqrt{\alpha ^2+\beta^2 t^2}}\exp\Big\{-\dfrac{\alpha (x-v_{g}t)^2}{\alpha ^2+\beta ^2t^2}\Big\}\]
The width for this is $2\sqrt{(\alpha ^2+\beta ^2t^2)/\alpha }$, which poses a problem. Despite this, $\int^{\infty }_{-\infty } dx\;|\psi  (x,t)|^2$ is a constant independent of time, which alleviates the problem theoretically.
\[\int ^{\infty }_{-\infty }dx\;|\psi (x,t)|^2=2\pi \sqrt{\dfrac{\pi }{\alpha }}\]
\end{thm}
\vspace{2ex}

