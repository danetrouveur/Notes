\section{Lecture 12 (April 14th)}
\begin{rmk}
We previously studied the case where the total energy was less than the potential barrier ($E<V_0$). In a general case, we can consider a barrier as a composition of step barriers. The transmission coefficient then follows the following proportionality:
\[|T|^2\propto \exp \Big(-2\int ^{x_2}_{x_1}\sqrt{\dfrac{2m}{\hbar ^2}(V(x)-E)}\Big)\]
where $x_1$ and $x_2$ are the endpoints of the barrier. This should definitely be memorised.
\end{rmk}
\vspace{2ex}
\begin{ex}
Consider a metal apparatus where an electric field of magnitude $\varepsilon $ is applied. The potential energy is given by $V(x)=-x(e\varepsilon )$. A electron at zero energy, the potential energy is then given by $W-x(e\varepsilon )$ where $W$ is thework-function of the metal. Then, from the above equation,
\[|T|^2\propto \exp \Big(-2\int ^{a}_{0}\sqrt{\dfrac{2m}{\hbar ^2}(W-x(e\varepsilon ))}\Big)=\exp \Big(-\dfrac{4\sqrt{2}}{3}\sqrt{\dfrac{mWa^2}{\hbar ^2}}\Big)\]
with $W/e\varepsilon =a$. This equation is called the Fowler-Nordheim equation. For the potential well, the scattering state can be considered as unbounded orbital motion while the bounded state can be considered as bounded orbital motion. When it comes to bounded states, parity symmetry is important, as the potential function is a symmetric function $V(x)=V(-x)$. This implies that the eigenfunctions $\psi (x)$ are also even or odd. It now suffices to consider the boundary condition only at $x=a$. 
\[\psi ''+\dfrac{2m}{\hbar }(E-V(x))\psi =0\]
Then,
\[\begin{cases}
a<x\quad \alpha ^2=\dfrac{2m}{\hbar }(-E)
0\leq x<a\quad \dfrac{2m}{\hbar }(E+V_0)
\end{cases}\]
we then have $\psi \propto e^{-\alpha x}$ and $\psi \propto (\sin qx,\cos qx)$ as general ansatz. Requiring the continuity of $\psi '/\psi $ at $a$ for the even (odd) part,
\[-\dfrac{\alpha e^{-\alpha x}}{e^{-\alpha x}}\Big|_{a^{+}}=\dfrac{-q\sin qx}{\cos qx}\Big|_{a^{-}}\]
and we have $\alpha a =q\tan qa$ for even and $\alpha a=-q\cot qa$ for the odd part of the function. Notice how $y=qa$ and $\lambda =2mV_0a^2/\hbar ^2$ are dimensional quantities that we shall now define. 
We can then obtain for the even case
\[\dfrac{\sqrt{\lambda -y^2}}{y}=\tan y\]
and for the odd case
\[\dfrac{\sqrt{\lambda -y^2}}{y}=-\cot y\]
From this we get that for large $\lambda $, we get more solutions. Note that in the ground state, the solution must have an even parity. 
\end{ex}  
\vspace{2ex}
\begin{thm}
Another interesting apparatus is the double well. As asymmetric potential functions are a result of a composition of even and odd functions with different energy values, it cannot be 
\end{thm}
\vspace{2ex}

