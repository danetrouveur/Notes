\section{Lecture 21 (May 21st)}
\begin{thm}
(Exsistence of simultaneous eigenkets for commuting operators) Let $[\hat{A},\hat{B}]=0$, that is, $\hat{A}$ and $\hat{B}$ be commuting operators. Consider the set of eigenstates $\{|u_{a}\rangle \}$ of $\hat{A}$, satisfying
\[\hat{A}|u_{a}\rangle =a|u_{a}\rangle \]
Observe that
\[\hat{A}(\hat{B}|u_{a}\rangle )=\hat{B}(\hat{A}|u_{a}\rangle )=a(\hat{B}|u_{a})\]
How many eigenkets (eigenstates) does the eigenvalue $a$ have? Assume that it has $1$ and we have:
\[\hat{B}|u_{a}\rangle =b|u_{a}\rangle \]
where $|u_{a}\rangle $ is a simultaneous eigenket for both $\hat{A}$ and $\hat{B}$. In this case, we say that the eigenvalue is nondegenerate. Assume that it has $2$, $|u_{a}^{(1)}\rangle $ and $|u_{a}^{(2)}\rangle $. Applying the operator $\hat{B}$, we expect
\[\begin{cases}
\hat{B}|u_{a}^{(1)}\rangle =c_{11}|u_{a}^{(1)}\rangle +c_{21}|u_{a}^{(2)}\rangle \\
\hat{B}|u_{a}^{(2)}\rangle =c_{21}|u_{a}^{(1)}\rangle +c_{22}|u_{a}^{(2)}\rangle 
\end{cases}\]
As $\hat{B}$ is Hermitian, the following matrix would have a diagonal form,
\[\begin{pmatrix}
c_{11}&c_{12}\\c_{21}&c_{22}
\end{pmatrix}\longrightarrow \begin{pmatrix}
\lambda _{b}^{(1)}&0\\0&\lambda ^{(2)}_{b}
\end{pmatrix}
\]
By which we conclude
\[\begin{cases}
\hat{B}|v^{(1)}_{a}\rangle =\lambda _{b}^{(1)}|v_{a}^{(1)}\rangle \\
\hat{B}|v^{(2)}_{a}\rangle =\lambda _{b}^{(2)}|v_{a}^{(1)}\rangle \\
\end{cases}\]
for some vectors $|v_{a}^{(i)}\rangle $. We therefore find $|u_{a}\rangle $ to be simultaneous eigenkets of $\hat{A}$ and $\hat{B}$. In this case, we say that the eigenvalue is degenerate.
\end{thm}
\vspace{2ex}
\begin{defi}
(Expectation values) Consider the expectation value of $\hat{A}$, defined as
\[\langle \psi (t)|\hat{A}|\psi (t)\rangle \]
For kets, the Schrodinger equation is given as
\begin{align*}
i\hbar \dfrac{\partial }{\partial t}|\psi (t)\rangle =\hat{H}|\psi (t)\rangle \\
-i\hbar \dfrac{\partial }{\partial t}\langle \psi (t)=\langle \psi (t)|\hat{H} 
\end{align*}
For now, we take $\hat{H}$ to be time-independent. We find
\begin{align*}
\dfrac{d }{d t}\Big(\langle \psi (t)|\hat{A}|\psi (t)\rangle \Big)=&\Big(\dfrac{d }{d t}\langle \psi (t)| \Big)\hat{A}|\psi (t)\rangle +\langle \psi (t)|A\Big(\dfrac{d }{d t}|\psi (t)\rangle\Big) +\langle \psi (t)|\dfrac{\partial A}{\partial t}|\psi (t)\rangle   \\
=&\dfrac{i}{\hbar }\langle \psi(t)|[\hat{H},\hat{A}]|\psi (t)\rangle +\langle \psi (t)|\dfrac{\partial A}{\partial t}|\psi (t)\rangle 
\end{align*}
From this we learn that expectation values (given that the observable is time independent) are only conserved through time when the commutator with the Hamiltonian operator is 0.
\[[\hat{A},\hat{H}]=0\]
\end{defi}
\vspace{2ex}
\begin{ex}
(Ehrenfest theorem) Take $\hat{A}=\hat{x}$ and 
\[\hat{H}=\dfrac{\hat{p}^2}{2m}+V(x)\]
The derivative of the expectation value is given as
\[\dfrac{d }{d t}\langle \hat{x}\rangle =\dfrac{i}{\hbar }\langle \psi |[H,\hat{x}]|\psi \rangle =\Big\langle \dfrac{\hat{p}}{m}\Big\rangle  \]
On the other hand,
\[\dfrac{d }{d t}\langle \hat{p}\rangle =\dfrac{i}{\hbar }\langle \psi |[\hat{H},\hat{p}]|\psi \rangle  =-\Big\langle \dfrac{d V}{d x}\Big\rangle  \]
Derivating the top equation one more time,
\[m\dfrac{d ^2}{d t^2}\langle \hat{x}\rangle =\dfrac{d }{d t}\langle \hat{p}\rangle =-\Big\langle \dfrac{d V}{d x}\Big\rangle    \]
we find the quantum mechanical version of Newton's theorem.
\end{ex}
\vspace{2ex}
\begin{defi}
(Schrodinger and Heisenberg picture of quantum theory) Like the above, when we take wavefunctions to be time independent, we call the construction Schrodinger's picture of quantum mechanics. On the other hand, suppose that time evolution is introduced by 
\[|\psi (t)\rangle=\exp \Big(-\dfrac{\hat{H}t}{\hbar }\Big)|\psi (0)\rangle  \]
where we call the exponential term (an unitary operator) the time evolution operator. The time dependent ket satisfies the Schrodginer equation, as
\[i\hbar \dfrac{\partial }{\partial t}|\psi (t)\rangle =i\hbar \Big(-\dfrac{i\hat{H}}{\hbar }\Big)\exp \Big(-\dfrac{\hat{iH}t}{\hbar }\Big)|\psi (0)\rangle =\hat{H}|\psi (t)\rangle  \]
The expectation value, with this formulation, becomes
\[\langle \psi (0)|\exp \Big(\dfrac{i\hat{H}t}{\hbar }\Big)\hat{A}\exp \Big(-\dfrac{i\hat{H}t}{\hbar }\Big)|\psi (0)\rangle \]
In this case, we see how instead of the states, the operators evolve throughout time ($\hat{A}=\hat{A}(t)$), and quantum mechanics seen in this manner is called Heisenberg's picture. We emphasize that the above is a formal solution, meaning that we aren't caring about details. In this process, we care about the time derivative of $\hat{A_{H}}(t)$ which we find to be
\begin{align*}
\dfrac{d }{d t}\hat{A_{H}}(t)=&\dfrac{i\hat{H}}{\hbar }\exp \Big(\dfrac{iHt}{\hbar }\Big)\hat{A}_{S}\exp \Big(-\dfrac{i\hat{H}t}{\hbar }\Big)+\exp \Big(\dfrac{iHt}{\hbar }\Big)\hat{A}_{S}\exp \Big(-\dfrac{iHt}{\hbar }\Big)\Big(-\dfrac{i\hat{H}}{\hbar }\Big)\\
&+\exp \Big(\dfrac{i\hat{H}t}{\hbar }\Big)\dfrac{\partial A_{S}}{\partial t}\exp \Big(-\dfrac{i\hat{H}t}{\hbar }\Big) \\
=&\dfrac{i\hat{H}}{\hbar }\hat{A}_{H}(t)-\dfrac{i}{\hbar }A_{H}(t)\cdot \hat{H}\\=&\dfrac{i}{\hbar }[\hat{H},\hat{A}_{H}(t)]+\exp \Big(\dfrac{i\hat{H}t}{\hbar }\Big)\dfrac{\partial A_{S}}{\partial t}\exp \Big(-\dfrac{i\hat{H}t}{\hbar }\Big) 
\end{align*}
We the parallel between Schrodinger and Hamilton's equations, and Heisenberg's and Poisson's equation.
\end{defi}
\vspace{2ex}
\begin{prop}
Does the commutator relation $[\hat{x}(t),\hat{p}(t)]=i\hbar $ hold in Heisenberg's picture? We see that
\begin{align*}
\hat{x}(t)\hat{p}(t)-\hat{p}(t)\hat{x}(t)=&\exp \Big(\dfrac{i\hat{H}t}{\hbar }\Big)\hat{x}\Big(-\dfrac{i\hat{H}t}{\hbar }\Big)\exp \Big(\dfrac{i\hat{H}t}{\hbar }\Big)\hat{p}\Big(-\dfrac{i\hat{H}t}{\hbar }\Big)\\&-\exp \Big(\dfrac{i\hat{H}t}{\hbar }\Big)\hat{x}\Big(-\dfrac{i\hat{H}t}{\hbar }\Big)\exp \Big(\dfrac{i\hat{H}t}{\hbar }\Big)\hat{x}\Big(-\dfrac{i\hat{H}t}{\hbar }\Big)\\
=&\exp \Big(\dfrac{i\hat{H}t}{\hbar }\Big)(\hat{x}\hat{p}-\hat{p}\hat{x})\Big(-\dfrac{i\hat{H}t}{\hbar }\Big)\\
=&i\hbar 
\end{align*}
It is then important to check that $[\hat{a},\hat{a}^{\dagger}]=1$ too!
\end{prop}
\vspace{2ex}

