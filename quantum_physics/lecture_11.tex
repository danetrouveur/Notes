\section{Lecture 11 (April 9th)}
Today, we will deal with the potential well.
\\
\begin{thm}
Consider the potential well problem. 
\[V(x)=
\begin{cases}
0\hspace{5ex}x< -a\\
-V_0\hspace{5ex}-a<x<a\\
0\hspace{5ex}a<x
\end{cases}\]
For this,
\[\begin{cases}
E<-V_{0}:\quad\mathrm{no\ solutions}\\
-V_0<E<0:\quad\mathrm{bound\ state}\\
0<E:\quad\mathrm{scattering\ state}
\end{cases}\]
We will deal the last case first, which we call the scattering state. Then,
\[\psi ''+\dfrac{2m}{\hbar }(E-V)\psi =0\]
and 
\[\begin{cases}
|x|>a\hspace{3ex}\psi ''+k^2\psi =0\\
|x|<a\hspace{3ex}\psi ''+q^2\psi =0
\end{cases}\]
for $k^2=2mE/\hbar ^2$ and $q^2=2m(E+V_0)/\hbar ^2$. Then, 
\[\begin{cases}
x<-a\hspace{9ex}e^{ikx}+Re^{-ikx}\\
-a<x<a\hspace{4ex}Ae^{iqx}+Be^{-iqx}\\
a<x\hspace{11ex}Te^{ikx}
\end{cases}\]
To obtain the coefficients, we can impose either that the wave function and its derivative is continuous or that
\begin{itemize}
\item[(i)] (Continuity of logarithm)\[\dfrac{1}{\psi }\dfrac{\partial \psi }{\partial x}=\dfrac{\partial }{\partial x}(\ln \psi )  \]
\item[(ii)] ${\bf J}(x)$ is continuous
\end{itemize}
From (ii) we have, at $x=a$,
\[\dfrac{\hbar ka}{m}(1-|R|^2)=\dfrac{\hbar q}{m}(|A|^2-|B|^2)=\dfrac{\hbar k}{m}|T|^2\]
The point is that we have four equations, two for $x=a$ and two for $x=-a$. In conclusion, we have
\[\begin{cases}
\vspace{2ex}
R=\dfrac{ie^{-2ika}(q^2-k^2)\sin 2qa}{2kq\cos 2qa-i(q^2+k^2)\sin 2qa}\\
T=\dfrac{e^{-2ika}2kq}{2kq\cos 2qa-i(q^2+k^2)\sin 2qa}
\end{cases}\]
The fact that the $\sin 2qa$ exists on the numerator of $R$ is rather shocking as when $4qa=n(s\pi )$ we have $R=0$. This can be understood through the wave property of the particle. This was vigourously studied in Ramsauer-Townsend resonance.
\end{thm}
\vspace{2ex}
\begin{thm}
Consider the barrier problem.
\[V(x)=\begin{cases}
0\hspace{4ex}x<-a\\
V_0\hspace{4ex}-a<x<a\\
0\hspace{4ex}a<x
\end{cases}\]
Then,
\[\psi ''+\dfrac{2m}{\hbar ^2}(E-V)\psi =0\hspace{5ex}\psi ''-\kappa ^2\psi =0\]
with $\kappa ^2=2m(V_0-E)/\hbar $.
We have the ansatz
\[\begin{cases}
x< -a\hspace{4ex}e^{ikx}+Re^{-ikx}\\
-a<x<a\hspace{4ex}Ae^{-\kappa x}+Be^{\kappa x}\\
x>a\hspace{4ex}Te^{ik x}
\end{cases}\]
The story is the same, with two values each at $-a$ and $a$. We have the following simultaneous equations.
\begin{itemize}
\item[(i)] ($\psi $ continuity at $x=-a$) $e^{-ika}+Re^{ika}=Ae^{\kappa a}+Be^{-\kappa a}$
\item[(ii)] ($\psi $ continuity at $x=a$) $Ae^{-\kappa a}+Be^{\kappa a}=Te^{ik a}$
\item[(iii)] ($\psi '$ continuity at $x=-a$) $ike^{-ika}+R(-ik)e^{ika}=-\kappa Ae^{\kappa a}+B\kappa e^{-\kappa a}$
\item[(iv)] ($\psi '$ continuity at $x=a$) $-\kappa Ae^{-\kappa a}+\kappa Be^{\kappa a}=ikTe^{ika}$
\end{itemize}
We arrive at
\[\begin{cases}
\vspace{2ex}
|T|^2=\dfrac{(2\kappa k)^2}{(2k\kappa )^2\cosh ^2(2\kappa a)+(k^2-\kappa ^2)^2\sinh ^2(2\kappa a)}\\
|R|^2=\dfrac{(k^2+\kappa ^2)\sinh ^2(2\kappa a)}{(2k\kappa )^2+(k^2+\kappa ^2)^2\sinh ^2(2\kappa  a)}
\end{cases}\]
where $\cosh^2x -\sinh ^2x=1 $. Notice flux conservation, $|R|^2+|T|^2=m/\hbar k$. We naturally question whether flux conservation satisfies in the barrier itself. Applying the formula for probability flux,
\begin{align*}
j=&\dfrac{\hbar }{2im}\Big(\psi ^{*}\dfrac{\partial \psi }{\partial x}-\psi \dfrac{\partial \psi ^{*}}{\partial x}\Big)\\
=&\dfrac{\hbar }{2im}2\kappa (-AB^{*}+A^{*}B)=\dfrac{\hbar k}{m}|T|^2
\end{align*}
When $x$ is larger than 1, we can aproximate $\sinh x\sim e^{x}/2$ and we can approximate the transmission constant squared as
\[|T|^2\sim Ce^{-2(\kappa 2a)}\]
for some constant $C$. $2a$ symbolises the barrier's width and we know that $\kappa =2m\sqrt{(V_0-E)}/\hbar ^2$. As such, we understand that as the barrier's width increases, the probability that the particle penetrates it exponentially decreases. More generally, for an arbitrary potential function,
\[|T|^2\sim C\exp \Big(-2\int _{x_1}^{x_2}\dfrac{2m}{\hbar ^2}\sqrt{V(x)-E}\Big)\]

\end{thm}
\vspace{2ex}

