\section{Lecture 8 (March 31st)}
\begin{recall}
The time independent Schrodinger equation was given by
\[E\psi  (x)=\hat{H}\psi  (x)\]
In this manner, we can take $E$ as an eigenvalue for the operator $\hat{H}$. 
\end{recall}
\vspace{2ex}
\begin{ex}
We can consider the infinite potential well, where
\[V(x)=\begin{cases}
\infty \hspace{5ex}x<0\\
0\hspace{5ex}0<x<a\\
\infty \hspace{5ex}a<x
\end{cases}\]
The Schrodinger equation becomes
\[\Big(-\dfrac{\hbar^2}{2m}\dfrac{d ^2}{d x^2}+V(x)\Big)\psi (x)=E\psi  \]
Due to the infinities, we require that the wave function degenerates at the endpoints and beyond along with continuity. Imposing the boundary conditions, we obtain
\[\psi (x)=\sqrt{\dfrac{2}{a}}\sin \Big(\dfrac{n\pi x}{a}\Big)\]
for $n\in {\bm N}$. We then have 
\[E_{n}=\dfrac{\hbar ^2}{2m}\Big(\dfrac{n\pi }{a}\Big)^2\]
as discrete eigenvalues. Note that the functions are mutually orthogonal. We can then express an arbitrary wave function with multiple energy states as a linear combination of these values.
\[\psi  (x)=\sum _{n}c_{n}\psi  _{n} \]
Imposing the Born interpretation of quantum mechanics,
\[1=\int \psi ^{*}\psi =\sum _{n,m}c_{m}^{*}c_{n}\int \psi _{m}^{*}\psi _{n}=\sum _{n}|c_{n}|^2\]
If we ask whether any function on the well can written as the basis above, we confront the problem that $L^2$ spaces are of infinite dimension. However, in this case, Fourier already discovered that cosine and sine functions space any function space, and we know that the set of basis functions are complete.
\end{ex}
\vspace{2ex}
\begin{rmk}
In the above, example, we saw how solving the time independent Schrodinger equation is equivalent to an eigenvalue problem, with certain wave functions forming a basis of the solution space. What about general potential? As the Hamiltonian operator is a self-adjoint differential operator, according to the Strum-Liouville theorem, eigenfunction do indeed form a basis for the solution space. Note that for Hermitian operators,
\begin{itemize}
\item[(i)] Eigenvalues are real
\item[(ii)] Eigenfunctions are orthogonal
\item[(iii)] The vector space spanned by the eigenfunctions are complete
\end{itemize}
We could've also asked why the ground state of the solution space ($n=1$) is non-degenerate. The answer lies in Heisenberg's uncertainty principle. 
\end{rmk}
\vspace{2ex}

