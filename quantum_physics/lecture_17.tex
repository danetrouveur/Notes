\section{Lecture 17 (May 7th)}
\begin{defi}
(Adjointness as a suffient condition for real eigenvalues) An adjoint operator is defined as 
\[\langle \psi|A^{\dagger}|\phi \rangle =\langle A\psi | \phi \rangle\]
Where a self-adjoint operator is an operator that satisfies
\[A=A^{\dagger}\]
We have previously seen how the following inner product is real if self-adjointness holds.
\[\langle \psi |A|\psi \rangle  \]
\end{defi}
\vspace{2ex}
\begin{thm}
(Completeness relation) A basis is complete if and only if the completeness relation holds. 
\[{\bf 1}=\sum _{n}|n\rangle \langle n|\]
\end{thm}
\vspace{2ex}
\begin{proof}
Suppose that a Hilbert space has a complete orthonormal basis satisfying $\langle n|m\rangle =\delta _{nm}$. For $|\psi \rangle \in \mathcal{H}$, we have the following representation.
\[|\psi \rangle =\sum _{n}c_{n}|n\rangle \]
The coefficients are given by
\[\langle m|\psi \rangle =\sum _{n}c_{n}\langle m|n\rangle =c_{m}\quad \mathrm{and}\quad |\psi \rangle =\sum _{n}\langle n|\psi \rangle |n\rangle=\sum _{n}|n\rangle \langle n|\psi \rangle  \]
From the above, we have
\[{\bf 1}=\sum _{n}|n\rangle \langle n|\]
which is called the completeness relation. Conversely, we can assume that the above is true for any $|\psi \rangle $, and any vector could be expanded in terms of the basis. We have thus proved the theorem.
\end{proof}
\vspace{2ex}
\begin{defi}
(Dual spaces and the Riesz Lemma) The dual space of a vector space $V$ is the vector space that contains all the linear maps $T:V\rightarrow {\bm C}$ that act on $V$. There is a one-to-one correspondence between the dual space and the inner products $\langle\, \cdot \,|w\rangle $ and moreover the dual space and $V$. 
\[ \underbrace{\langle \psi |}_{\mathrm{bra}}\longleftrightarrow \underbrace{|\psi \rangle}_{\mathrm{ket}}\quad \mathrm{and}\quad \langle \psi |A^{\dagger}\longleftrightarrow A|\psi \rangle \]
Notice that from $\langle \psi |\phi \rangle ^{*}=\langle \phi |\psi \rangle $, we can deduce the conjugate linearity of the first argument. Consider how 
\begin{align*}
\langle \beta |c_1v_1+c_2v_2\rangle =c_1\langle \beta |v_1\rangle +c_2\langle \beta |v_2\rangle \\
\langle c_1v_1+c_2v_2|\beta \rangle =c_1^{*}\langle v_1|\beta \rangle +c_2^{*}\langle v_2|\beta \rangle 
\end{align*}
\end{defi}
\vspace{2ex}
\begin{defi}
(Operator algebra) Notice that operators form a ring with an underlying structure of a vector space which is exactly the definition of an algebra. 
\end{defi}
\vspace{2ex}
\begin{rmk}
(Identities)
\begin{itemize}
\item[(i)] $(A^{\dagger})^{\dagger}=A$
\item[(ii)] $(AB)^{\dagger}=B^{\dagger}A^{\dagger}$
\item[(iii)] $(\alpha A)^{\dagger}=\alpha ^{*}A^{\dagger}$
\end{itemize}
\end{rmk}
\vspace{2ex}
\begin{proof}
For (i), let $B=A^{\dagger}$.
\begin{align*}
\langle \phi ,B^{\dagger}\psi \rangle 	=&\langle B\phi ,\psi \rangle =\langle \psi ,B\phi \rangle ^{*}=\langle \psi ,A^{\dagger}\phi \rangle ^{*}=\langle A\psi ,\phi \rangle ^{*}=\langle \phi ,A\psi \rangle 
\end{align*}
This will be on the test! For (ii),
\begin{align*}
\langle \phi ,(AB)^{\dagger}\psi \rangle =\langle AB\phi ,\psi \rangle =\langle B\phi ,A^{\dagger}\psi \rangle =\langle \phi ,B^{\dagger}A^{\dagger}\psi \rangle 
\end{align*}
For (iii),
\[\langle \phi ,(\alpha A)^{\dagger}\psi \rangle =\langle \alpha A\phi ,\psi \rangle =\alpha ^{*}\langle A\phi ,\psi \rangle =\alpha ^{*}\langle \phi ,A^{\dagger}\psi \rangle \]
\end{proof}
\vspace{2ex}
\begin{thm}
(Eigenvalue problem for Hermitian operators) Let $A|\psi _{n}\rangle =a_{n}|\psi _{n}\rangle $ and $A|\psi _{m}\rangle =a_{m}|\psi _{m}\rangle $. Applying the dual of $|\psi _{m}\rangle $ and $|\psi _{n}\rangle $ each, 
\[\langle \psi _{m}|A|\psi _{n}\rangle =a_{n}\langle \psi _{m}|\psi _{n}\rangle \quad \langle \psi _{n}|A|\psi _{m}\rangle =a_{m}\langle \psi _{n}|\psi _{m}\rangle \]
Taking the complex conjugate of the right, we have
\[\langle \psi _{m}|A^{\dagger}|\psi _{n}\rangle =\langle A\psi _{m}|\psi _{n}\rangle =\langle \psi _{n}|A\psi _{m}\rangle ^{*}=a_{m}^{*}\langle \psi _{m}|\psi _{n}\rangle \]
We conclude that 
\[a_{m}^{*}\langle \psi _{m}|\psi _{n}\rangle =a_{n}\langle \psi _{m}|\psi _{n}\rangle \quad \mathrm{and}\quad (a^{*}_{m}-a_{n})\langle \psi _{m}|\psi _{n}\rangle =0\]
when $m=n$, due to the positive definiteness of the inner product, we conclude that the eigenvalues are real. When $m\ne n$, we require that the inner product is zero, which proves orthogonality. In conclusion, we have proved the for self-adjoint operators, eigenvalues are real and they form an orthonormal set. Whether they form a complete basis is another difficult problem which we take for granted.
\end{thm}
\vspace{2ex}
\begin{prop}
(Probability interpretation) For an observable quantity $A$ measured with respect to $|\psi \rangle $, once a observation is made, one of $\{a_{n}\}$ is observed and the probability that this happens is $|a_{n}|^2$.
\end{prop}
\vspace{2ex}

