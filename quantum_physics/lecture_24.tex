\section{Lecture 24 (June 2nd)}
\begin{recall}
We have the following operators generating the time, translation, and rotation groups. 
\[U_{\mathrm{time }}=\exp \Big(-i\dfrac{Ht}{\hbar }\Big)\qquad U_{\mathrm{tran}}=\exp \Big(-i\dfrac{{\bf p}\cdot {\bf a}}{\hbar }\Big)\qquad U_{\mathrm{rot}}=\exp \Big(-i\dfrac{{\bf J}\cdot {\bf n}\,\theta }{\hbar }\Big)\]
\end{recall}
\vspace{2ex}
\begin{prop}
For the orbital angular momentum, we consider the transformation from cartesian to cylindrical coordinates
\[{\bf L}={\bf r}\times {\bf p}\]
with $(x,y,z)\rightarrow (r,\theta ,\phi )$ through
\[\begin{cases}
x=r\sin \theta \cos \phi \\
y=r\sin \theta \sin \phi \\
z=r\cos \theta 
\end{cases}\]
In the infinitesimal sense,
\[L_{z}=xP_{y}-yP_{x}=x\Big(i\hbar \dfrac{\partial }{\partial y} \Big)-y\Big(-i\hbar \dfrac{\partial }{\partial x} \Big)=-i\hbar \dfrac{\partial }{\partial \phi } \]
which should be calculated atleast once. By extension, we can also express $L_{\pm}$ in cartesian coordinates. Lastly, we can find ${\bf L}^2=L^{2}_{x}+L^{2}_{y}+L^{2}_{z}$ to be
\[{\bf L}^2=-\hbar ^2\Bigg[\dfrac{1}{\sin \theta }\dfrac{\partial }{\partial \theta }\Big(\sin \theta \dfrac{\partial }{\partial \theta } \Big)+\dfrac{1}{\sin ^2\theta }\dfrac{\partial ^2}{\partial \phi ^2}  \Bigg]\]
which is the Laplacian in terms of spherical coordinates. 
\end{prop}
\vspace{2ex}
\begin{thm}
Memorize, again,
\[L_{z}|l,m\rangle =m\hbar |l,m\rangle \quad \mathrm{and}\quad {\bf L}^2|l,m\rangle =l(l+1)\hbar ^2|l,m\rangle \]
We define spherical harmonics as
\[\langle \theta ,\phi |l,m\rangle =Y_{l,m}(\theta ,\phi )\]
We can write
\[\int d{\bf x}\,|{\bf x}\rangle \langle {\bf x}|={\bf 1}\quad\mathrm{and}\quad
\int d\mathrm{\Omega} \,|\theta ,\phi \rangle \langle \theta ,\phi |={\bf 1}\]
Then,
\[\int \sin \theta d\theta d\phi \,|\theta ,\phi \rangle \langle \theta ,\phi |\theta ',\phi '\rangle \]
We then require that
\[\langle \theta ,\phi |\theta ',\phi '\rangle =\dfrac{\delta (\theta -\theta ')\delta (\phi -\phi ')}{\sin \theta }=\delta (\cos \theta -\cos \theta ')\delta (\phi -\phi ')\]
We now seek
\[\langle \theta ,\phi |\hat{L}_{z}|l,m\rangle =m\hbar \langle \theta ,\phi |l,m\rangle \]
implying
\[-i\hbar \dfrac{\partial }{\partial \phi }\langle \theta ,\phi |l,m\rangle =m\hbar \langle \theta ,\phi |l,m\rangle  \]
and
\[\dfrac{\partial }{\partial \phi }Y_{l,m}(\theta ,\phi )=imY_{l,m}(\theta ,\phi ) \]
We now know that $Y_{l,m}(\theta ,\phi )=e^{im\phi }F(\theta )/\sqrt{2\pi }$ with $m$ an integer, as the function should be periodic with respect to $\phi $. We now use the  operator ${\bf L}^2$ to determine $F(\theta )$.
\[\langle \theta ,\phi |{\bf L}^2|l,m\rangle =l(l+1)\hbar ^2\langle \theta ,\phi |l,m\rangle \]
We have
\[-\hbar ^2\Bigg[\dfrac{1}{\sin \theta }\dfrac{\partial }{\partial \theta }\Big(\sin \theta \dfrac{\partial }{\partial \theta } \Big)+\dfrac{1}{\sin^2\theta }\dfrac{\partial ^2}{\partial \phi ^2} \Bigg]\dfrac{e^{im\phi }}{\sqrt{2\pi }}F_{l,m}(\theta  )=l(l+1)\hbar ^2\dfrac{e^{im\phi }}{\sqrt{2\pi }}F_{l,m}(\theta )\]
and 
\[-\Bigg[\dfrac{1}{\sin \theta }\dfrac{\partial }{\partial \theta }\Big(\sin \theta \dfrac{\partial }{\partial \theta }\Big)-\dfrac{m^2}{\sin ^2\theta } \Bigg]F_{l,m}(\theta )=l(l+1)F_{l,m}(\theta )\]
This is the associated Legendre polynomial with $m\ne 0$ (Legendre if $m=0$). Finite solutions with $0\leq \theta \leq \pi $ require that $l\geq 0$ and $l\in{\bm Z}$. 
\[F_{l,m}(\theta )=P_{l,m}(\cos \theta ) \]
Substituting $z=\cos \theta $, we have the more famililar 
\[\dfrac{d }{d z}\Big[(1-z^2)\dfrac{d P_{l,m}}{dz} \Big]+\Big[l(l+1)-\dfrac{m^2}{1-z^2}\Big]P_{l,m}=0 \]
with solutions
\[P_{l,m}(z)=(1-z^2)^{m/2}\dfrac{d ^{m}}{d z^{m}}P_{l} \]
Normalisation can be done through setting
\[\langle l',m'|l,m\rangle =\delta _{ll'}\delta _{mm'}\]
We have, then,
\[\int d\mathrm{\Omega} \langle l',m'|\theta ,\phi \rangle \langle \theta ,\phi |l,m\rangle =\int d\mathrm{\Omega}\,Y^{*}_{l',m'}(\theta ,\phi )Y_{l,m}(\theta ,\phi )=\delta _{ll'}\delta _{mm'}\]
and set coefficients accordingly. Resultantly,
\[Y_{l,m}(\theta ,\phi )=(-1)^{m}\sqrt{\dfrac{2l+1}{4\pi }\dfrac{(l-m)!}{(l+m)!}}P_{l,m}(\cos \theta )e^{im\phi }\]
for $m\geq 0$ and
\[Y_{l,-m}=(-1)^{m}(Y_{l,m})^{*}\]
for $m<0$. 
\end{thm}
\vspace{2ex}
\begin{cor}
Consider 
\[\sum _{l,m}\langle \theta',\phi '|l,m\rangle \langle l,m|\theta ,\phi \rangle = \sum _{l,m}Y_{l,m}(\theta ',\phi ')Y_{l,m}^{*}(\theta ,\phi )=\langle \theta ',\phi '|\theta ,\phi \rangle=\dfrac{\delta (\theta -\theta ')\delta(\phi -\phi ')}{\sin \theta } \]
this is called the closure relation.
\end{cor}
\vspace{2ex}
\begin{defi}
The parity operator was defined as the operator that sends ${\bf x}\rightarrow -{\bf x}$. How does this work in spherical polar coordinates? We require that
\[\theta \rightarrow \pi -\theta \quad\mathrm{and}\quad \phi \rightarrow \phi +\pi \]
This turns $\cos \theta \rightarrow -\cos \theta $, and to an extension,
\[Y_{l,m}(-{\bf n})=(-1)^{l}Y_{l,m}({\bf n})\]
under parity.
\end{defi}
\vspace{2ex}

