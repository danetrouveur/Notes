\section{Lecture 5 (March 19th)}
\begin{thm}
(Explaination of the collapse of the wave function) Consider the double slit experiment with electrons. The electron going through the first slit would be expressed as $\psi _{1}$ and the electron going through the second slit would be expressed as $\psi _{2}$. The resulting distribution on the screen would be expressed as
\[|\psi _{1}+\psi _{2}|^2\]
giving an interference pattern. However, if we were to disturb the first slit's wave with photons used for observation, we could achieve
\begin{align*}
	|e^{i\phi_1}\psi _{1}+\psi _{2}|^2=&(e^{i\phi }\psi _{1}+\psi _{2})(e^{-i\phi}\psi _{1}^{*}+\psi_{2}^{*})\Big|_{\mathrm{avg}}\\
=&|\psi _{1}|^2+|\psi _{2}|^2+e^{-i\phi}\psi _{1}^{*}\psi _{2}+e^{i\phi }\psi _{1}\psi _{2}^{*}\Big|_{\mathrm{avg}}\\
=&|\psi _{1}|^2+|\psi _{2}|^2
\end{align*}
as $\langle \cos \phi\pm i\sin \phi\rangle _{\mathrm{avg}}=0$. This is the result we actually see experimentally, with two Gaussian distributions.
\end{thm}
\vspace{2ex}
\begin{defi}
(Wave packets) We have previously learned the wave packet description
\[\mathrm{\Psi}(x,t)=\int \dfrac{dp}{\sqrt{2\pi \hbar } }\phi (p)\exp\Big(\dfrac{ipx}{\hbar }-\dfrac{iE(p)t}{\hbar }\Big)\]
The idea by De Broglie to apply the logic of waves to particles was revolutionary. At $t=0$, the energy term disappears, resulting in
\[\mathrm{\Psi} (x,t=0)=\int \dfrac{dp}{\sqrt{2\pi \hbar }}\phi (p)\exp\Big(\dfrac{ipx}{\hbar }\Big)\longleftrightarrow \phi (p)=\int \dfrac{dx}{\sqrt{2\pi \hbar }}\mathrm{\Psi} (x,t=0)\exp\Big(\dfrac{-ipx}{\hbar }\Big) \]
In this way, we see two different wavefunctions (each in the position space and momentum space space) which give equivalent physical information.
\end{defi}
\vspace{2ex}
Previously, we have found that the plane wave equation, that models a free particle with definite momentum and indefinite position, follows the Schrodinger equation. We now look for the differential equation in the quantum landscape.
\\
\begin{defi}
(Schrodinger Equation) We have verified that complex plane waves satisfy the Schrodinger equation. We now verify that wave packets do also.
\[i\hbar \dfrac{\partial }{\partial t}\mathrm{\Psi} (x,t)=\int \dfrac{dp}{\sqrt{2\pi \hbar }}\phi (p)E(p)\exp\Big(\dfrac{ipx}{\hbar }-\dfrac{iE(p)t}{\hbar }\Big)\]
and
\[\dfrac{1}{2m}\Big(-i\hbar \dfrac{\partial }{\partial x}\Big)^2\mathrm{\Psi} (x,t)=\int \dfrac{dp}{\sqrt{2\pi \hbar }}\phi (p)\dfrac{p^2}{2m}\exp\Big(\dfrac{ipx}{\hbar }-\dfrac{iE(p)t}{\hbar }\Big) \]
(We can of course extend our definition to multiple dimensions and we would have $\partial ^2/\partial x^2\rightarrow \nabla ^2$) If there exists potential, we can simply add $V(x)$ to the differential operator, giving
\[i\hbar \dfrac{\partial }{\partial t}\mathrm{\Psi} (x,t)=\Big(-\dfrac{\hbar ^2}{2m}\dfrac{\partial ^2}{\partial x^2}+V(x)\Big)\mathrm{\Psi} (x,t)  \]
for a free particle. What we have obtained is the time dependent Schrodinger equation.

\end{defi}
\vspace{2ex}
\begin{rmk}
(Boundary condition for normalisation and the probability current) Let's do a very important calculation! As we mentioned that the wave function denotes probability, we require that the wave functions are normalized, that is,
\[\int dx\;|\mathrm{\Psi} (x,t)|^2=1\]
for all time t. Note that $\int dx\,|\mathrm{\Psi} (x,t=0)|^2=1$ is rather simple to normalise, but the tricky part is when $t>0$, where the result might be a function of $t$. In a more formally manner, we ask: does time evolution preserves total probability? The answer lies in the Schrodinger equation, whose left-hand-side generates time evolution. We impose the following boundary condition:
\[\dfrac{\hbar }{2im}\Big(\mathrm{\Psi} \dfrac{\partial \mathrm{\Psi} ^{*}}{\partial x}-\dfrac{\partial \mathrm{\Psi} }{\partial x}\mathrm{\Psi} ^{*} \Big)\Big|^{b}_{a}\]
\end{rmk}
\vspace{2ex}
\begin{proof}
Note that
\[
\left\{
\begin{aligned}
\left( -\frac{\hbar^2}{2m} \frac{\partial^2}{\partial x^2} + V(x) \right) \mathrm{\Psi}  &= i\hbar \frac{\partial \mathrm{\Psi} }{\partial t} \\
\left( -\frac{\hbar^2}{2m} \frac{\partial^2}{\partial x^2} + V(x) \right) \mathrm{\Psi} ^* &= -i\hbar \frac{\partial \mathrm{\Psi} ^*}{\partial t}
\end{aligned}
\right.
\]
We then have
\begin{align*}
0 &= \frac{d}{dt} \int_a^b dx\, \mathrm{\Psi} \cdot \mathrm{\Psi} ^* 
= \int_a^b dx \left( \frac{\partial \mathrm{\Psi} 	}{\partial t} \mathrm{\Psi} ^* + \mathrm{\Psi}  \frac{\partial \mathrm{\Psi} ^*}{\partial t} \right) \\
&= \int_a^b \frac{1}{i\hbar} \left( -\frac{\hbar^2}{2m} \frac{\partial^2 \mathrm{\Psi} }{\partial x^2} \right) \mathrm{\Psi} ^* \, dx 
+ \frac{1}{i\hbar} \left( \frac{\hbar^2}{2m} \frac{\partial^2 \mathrm{\Psi} ^*}{\partial x^2} \right) \mathrm{\Psi}  \, dx\\=& \int_a^b dx \, \frac{\hbar}{2im} \frac{\partial}{\partial x} \left( \mathrm{\Psi}  \frac{\partial \mathrm{\Psi} ^*}{\partial x} - \frac{\partial \mathrm{\Psi} }{\partial x} \mathrm{\Psi} ^* \right) = \frac{\hbar}{2im} \left( \mathrm{\Psi} \frac{\partial \mathrm{\Psi} ^*}{\partial x} - \frac{\partial \mathrm{\Psi} }{\partial x} \mathrm{\Psi} ^* \right) \Big|_a^b \\
&= - j(x, t) \Big|_a^b
\end{align*}
From here, we impose that the last term is equal to zero, which allows the probability interpretation to be valid. This is a boundary condition for the wavefunction to follow. In multiple dimensions, 
\[\int _{V}d{\bf x}\,\dfrac{\hbar }{2im}\nabla \cdot [(\nabla \mathrm{\Psi} ^{*})\mathrm{\Psi} -\mathrm{\Psi} ^{*}(\nabla \mathrm{\Psi} )]\]
and using Stoke's law, we change the requirement to
\[-\dfrac{\hbar }{2im}\int {\bf j}(x,t)\cdot d{\bf a}=0\]
\end{proof}
\vspace{2ex}

