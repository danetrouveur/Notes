\section{Lecture 9 (April 2nd)}
\begin{defi}
A parity transformation is defined as the exchange of sign of the coordinate (in three dimensions it is called inversion). In one dimension, we have
\[(\hat{P}\psi )(x)=\psi (-x)\hspace{5ex}\hat{P}^2\psi =\psi \]
Considering the above as a eigenvalue problem, we have the eigenvalues $\lambda ^2=1$. Therefore, we have even and odd functions as eigenfunctions. We now have two operators and two corresponding eigenvalues,
\[\begin{cases}
\hat{P}\psi _{n}=(-1)^{n+1}\psi _{n}\\
\hat{H}\psi _{n}=E_{n}\psi _{n}
\end{cases}\]
As the wavefunctions are simutaneously satisfy the eigenfunctions of $\hat{H}$ and $\hat{P}$, we now derive that $[H,P]=0$.
\begin{align*}
(\hat{H}\hat{P})\psi(x)=&\hat{H}(\hat{P}\psi (x))=\hat{H}(\psi (-x))\\
\hat{P}(\hat{H}\psi (x))=&\hat{P}\Big(-\dfrac{d ^2}{d x^2}\psi (x)+V(x)\psi (x) \Big)=\hat{H}\psi (-x)
\end{align*}
What about $\hat{P}$ and $\hat{p}=-i\hbar\;d/dx$?
\begin{align*}
\hat{P}(\hat{p}\psi (x))=&\hat{P}\Big(-i\hbar \dfrac{d \psi (x)}{d x} \Big)=-\hbar \dfrac{d \psi (-x)}{d (-x)}\\
\hat{p}(\hat{P}\psi (x))=&\hat{p}(\psi (-x))=-i\hbar \dfrac{d }{d x}\psi (-x) 
\end{align*}
We further note that $[x_{i},x_{j}]=0$ and that $[x_{a},p_{b}]=i\hbar \delta _{ab}$.
\end{defi}
\vspace{2ex}
\begin{thm}
For an Hermitian matrix $A$, we claim that $A$'s eigenfunctions form an orthonormal basis (with real eigenvalues). An arbitrary wave function would then be expressed as
\[\psi (x)=\sum _{n}c_{n}\psi _{n}(x)\]
We do require that
\[\int |\psi |^2=1=\sum _{n}|c_{n}|^2\]
The QM postulate is that $|c_{n}|^2$ is the probability that the eigenvalue $A_{n}$ would occur. If the operators $A$ and $B$ commute, the eigenvalues would be shared and the same eigenvector would come out.
\end{thm}
\vspace{2ex}
\begin{ex}
For the momentum operator, we have the solution
\[\psi (x)=\dfrac{1}{\sqrt{2\pi \hbar }}e^{ipx/\hbar }\]
Calculating the inner product between two wave functions we have
\[\langle \psi_{p'}\;|\;\psi _{p}\rangle =\int ^{\infty }_{-\infty }dx\;\psi _{p'}^{\ast}(x)\psi _{p}(x)=\int ^{\infty }_{-\infty }dx\;\exp \Big(\dfrac{i(p-p')x}{\hbar }\Big)=\delta (p-p')\]
Let's now impose a periodic boundary condition, or $\psi _{p}(x)=\psi _{p}(x+L)$, we have
\[\exp \Big(\dfrac{ipL}{\hbar }\Big)=1\hspace{2ex}\rightarrow\hspace{2ex}p_{n}=\Big(\dfrac{2\pi }{L}\Big) n\hbar \]
for $n\in {\bm Z}$. We then have, for $p=p'$, 
\[\langle \psi _{p'}\;|\; \psi _{p}\rangle =L|c|^2=1\]
and obtain the condition $|c|=1/\sqrt{L}$. This process involving the boundary condition is called box quantisation.
\end{ex}
\vspace{2ex}
\begin{ex}
For a wave function given a cyclic boundary condition $\phi (x)=\phi (x+L)$ on $[-L/2,L/2]$, the Fourier expansion is given by
\[\psi (x)=\sum _{n\in {\bm Z}}A_{n}\exp \Big(\dfrac{i2\pi nx}{L}\Big)\]
with coefficients
\[A_{m}=\dfrac{1}{L}\int ^{L/2}_{-L/2}dx\;\psi (x)\exp \Big(-\dfrac{i2\pi mx}{L}\Big)=\dfrac{1}{L}\sum _{n\in {\bm N}}\Big[\int ^{L/2}_{-L/2}\exp\Big(\dfrac{i2\pi (n-m)x}{L}\Big) \Big]a_{n}=\delta _{nm}a_{n}\]

\end{ex}
\vspace{2ex}

