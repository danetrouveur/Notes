\section{Lecture 18 (May 12th)}
\begin{prop}
The dynamics, or evolution of a quantum state is given by
\[i\hbar \dfrac{\partial }{\partial t}|\psi (t)\rangle =\hat{H}|\psi (t)\rangle  \]
whereas, because $\hat{H}=\hat{H}^{\dagger}$,
\[-i\hbar \dfrac{\partial }{\partial t}\langle \psi (t)|=\langle \psi (t)|\hat{H} \]
\end{prop}
\vspace{2ex}
\begin{defi}
For any linear map $\hat{A}:V\rightarrow V$ between vector spaces, we can create a matrix representation according to
\[\hat{A}({\bf e}_{i})=\sum _{j}{\bf e}_{j}A_{ji}\quad\mathrm{and}\quad \langle {\bf e}_{j},\hat{A}({\bf e}_{i})\rangle =\sum _{k}A_{ki}\langle {\bf e}_{j},{\bf e}_{k}\rangle =\sum _{k}A_{ki}\delta _{jk}=A_{ji}\]
with respect to the basis $\{{\bf e}_{i}\}$. Now, let $\mathcal{H}$ be a Hilbert space with respect to $\{|n\rangle \}$. For an operator $\hat{A}:\mathcal{H}\rightarrow \mathcal{H}$, we have
\[\hat{A}|n\rangle =\sum _{n}A_{mn}|m\rangle \quad\mathrm{or}\quad \langle m|\hat{A}|n\rangle =A_{mn}\]
Now, by extension, we can create a matrix representation for any operator by employing the completeness relation
\[\hat{A}={\bf 1}\hat{A}{\bf 1}=\sum _{n,m}|n\rangle \langle n|\hat{A}|m\rangle \langle m|=\sum _{m,n}A_{nm}|n\rangle \langle m|\]
In the language of matrices, for any inner product, we can express
\[\langle \mathrm{\Phi} |\mathrm{\Psi} \rangle =\langle \mathrm{\Phi} |\sum _{n}|n\rangle \langle n|\mathrm{\Psi} \rangle =\sum _{n}\langle \mathrm{\Phi} |n\rangle \langle n|\mathrm{\Psi} \rangle =\sum _{n}(\langle 1|\mathrm{\Phi} \rangle^{*} ,\langle 2|\mathrm{\Phi} \rangle^{*} ,\ldots)
\begin{pmatrix}
\langle 1|\mathrm{\Psi} \rangle \\
\langle 2|\mathrm{\Psi} \rangle \\
\vdots
\end{pmatrix}\]
Therefore, kets can be seen as column vectors while bras can be seen as complex conjugated and transposed rows. Likewise, the dual relationship between the Hermitian and row vectors are parallel to the dual relation between bras and kets. 
\end{defi}
\vspace{2ex}
\begin{ex}
(Quantum gates) We now learn the NOT quantum gate $\hat{O}:{\bf C}^2\rightarrow {\bf C}^2$. If suffices to define the linear function on the basis like the following:
\[\begin{cases}
\hat{O}|0\rangle =|1\rangle \\
\hat{O}|1\rangle =|0\rangle 
\end{cases}\]
We can now find the matrix representation using the formalism above,
\begin{align*}
A_{11}=\langle 0|\hat{O}|0\rangle =0\quad A_{12}=\langle 0|\hat{O}|1\rangle =1\\
A_{21}=\langle 1|\hat{O}|0\rangle =1\quad A_{22}=\langle 1|\hat{O}|1\rangle =0
\end{align*}
Where we get
\[\hat{A}=\begin{pmatrix}
0&1\\1&0
\end{pmatrix}\quad \mathrm{or}\quad \hat{A}=|1\rangle \langle 0|+|0\rangle \langle 1|
\]
\end{ex}
\vspace{2ex}
\begin{defi}
The Pauli matrices are defined like the following
\[\sigma _{x}=\begin{pmatrix}
0&1\\1&0
\end{pmatrix}\quad \sigma _{y}=\begin{pmatrix}
0&-i\\i&0
\end{pmatrix}\quad
\sigma _{z}=\begin{pmatrix}
1&0\\
0&-1
\end{pmatrix}
\]
Note that they are Hermitian, unitary, and has a trace of $0$.
\end{defi}
\vspace{2ex}
\begin{defi}
(Mixed state) A general (or mixed) quantum state $\rho :\mathcal{H}\rightarrow \mathcal{H}$ is defined as an operator that is 
\begin{itemize}
\item[(i)] Hermitian 
\item[(ii)] $\mathop{\mathrm{Tr}}(\rho )=1$
\item[(iii)] Positive operator (also called a positive semidefinite operator), meaning $\langle \psi |\rho |\psi \rangle \geq 0$ for all $|\psi \rangle $
\end{itemize}
A general quantum state is also called a density operator.
\end{defi}
\vspace{2ex}
\begin{defi}
In the Dirac sense, the wavefunctions $\psi (x)$ and $\phi (x)$ in the position space and the momentum space respectively are the coefficents of the state vector when projected into the position space and momentum space respectively. That is, by defining the eigenstates $|x\rangle $ and $|p\rangle $ to form a generalised orthonormal basis satisfying the following condition,
\[\langle x|x'\rangle =\delta (x-x')\]
states can be expanded as 
\[|\psi \rangle =\int ^{\infty }_{-\infty }dx'\,\psi (x')|x'\rangle \quad |\psi \rangle =\int ^{\infty }_{-\infty }dp'\,\phi (p')|p'\rangle \]
with
\[\langle x|\psi \rangle =\int ^{\infty }_{-\infty }dx'\,\psi (x')\langle x|x'\rangle =\int ^{\infty }_{-\infty }dx'\,\psi (x')\delta (x-x')=\psi (x)\]
and likewise for the momentum function. 
\end{defi}
\vspace{2ex}

