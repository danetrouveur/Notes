\section{Lecture 10 (April 7th)}
\begin{recall}
In the most general case, we model states of quantum mechanical system as a complex Hilbert space (a vector space that is complete with respect to its inner product). In this regard, a wave function 
\end{recall}
\vspace{2ex}
{\bf Chapter 4}\hspace{2ex}1-dimensional Potential Problem
\\
\begin{rmk}
We solve the following boundary value problem with various potential functions:
\[
H\psi (x)=E\psi (x)
\hspace{3ex}\mathrm{with}\hspace{3ex}H=-\dfrac{\hbar ^2}{2m}\dfrac{d ^2}{dx^2}+V(x)\\ \]
Some examples of potential functions are steps, wells, barriers, delta functions, and simple harmonic oscillators.
\end{rmk}
\vspace{2ex}
\begin{thm}
(Step function) 
\[V(x)=\begin{cases}
V_0\hspace{5ex}x>0\\
0\hspace{5ex}x<0
\end{cases}\]
\end{thm}
\vspace{2ex}
\begin{proof}
We impose that 
\begin{itemize}
\item[(i)] (continuous at zero) $\psi (x\rightarrow 0^{-})=\psi (x\rightarrow 0^{+})$
\item[(ii)] (derivative is continuous at zero) $\psi '(x\rightarrow 0^{-})=\psi '(x\rightarrow 0^{+})$
\end{itemize}
Imposing these two conditions we have a general solution
\[\psi (x)=\begin{cases}
e^{ikx}+Re^{-ikx}\hspace{5ex}x<0\\
Te^{iqx}\hspace{5ex}x>0
\end{cases}\]
with $k^2=2mE/\hbar ^2$ and $q^2=2m(E-V_0)/\hbar >0$. With the conditions
\[\begin{cases}
1+R=T\\
ik(1-R)=iqT
\end{cases}\]
we find
\[\begin{cases}
\vspace{2ex}
R=\dfrac{k-q}{k+q}\\
T=\dfrac{2k}{k+q}
\end{cases}\]
Therefore with all $E>V_0$, solutions exist. We have previously defined the current density:
\[{\bf J}(x)=\dfrac{\hbar }{2im}(\psi ^{*}\nabla \psi -\nabla \psi ^{*}\psi )\]
Using the definition of current density as density times velocity,
\[\dfrac{\hbar k}{m}(1-|R|^2)=|T|^2\dfrac{\hbar q}{m}\]
we notice that the first condition along with this condition also gives the same solutions for $R$ and $T$. For another situation, we have 
\[\begin{cases}
1+R=T\\
ik(1-R)=-q'T
\end{cases}\]
with solutions
\[\begin{cases}
\vspace{2ex}
R=\dfrac{ik+q'}{ik-q;}\\
T=\dfrac{2k}{k+iq'}
\end{cases}\]
with $|R|^2=1$. Notice how $T$ does not vanish. 
\end{proof}
\vspace{2ex}

