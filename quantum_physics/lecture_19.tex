\section{Lecture 19 (May 14th)}
\begin{lem}
We first try to obtain $\langle x|p\rangle $ and $\langle p|x\rangle $. Notice that
\begin{align*}
\langle x|\hat{p}|p\rangle =&p\langle x|p\rangle =-i\hbar \dfrac{d }{d x}\langle x|p\rangle 
\end{align*}
Solving the differential equation we find
\begin{align*}
\langle x|p\rangle =&A(p)e^{ipx/\hbar }
\end{align*}
Noting the identity
\[\int ^{\infty }_{-\infty }dx\,e^{ikx}=2\pi \delta (k)\]
we find
\begin{align*}
\langle p|p'\rangle =&\delta (p-p')\\
\int ^{\infty }_{-\infty }dx\,\langle p|x\rangle \langle x|p'\rangle =&\delta (p-p')\\
\int ^{\infty }_{-\infty }dx\,|A(p)|^2e^{ix(p-p')/\hbar }=&\delta (p-p')\\
|A(p)|^22\pi \delta \Big(\dfrac{p-p'}{\hbar }\Big)=&\delta (p-p')\\
A(p)=\dfrac{1}{\sqrt{2\pi \hbar }}
\end{align*}
\end{lem}
\vspace{2ex}
\begin{thm}
(Fourier transform) We find that the inverse Fourier and Fourier transforms are given as 
\begin{align*}
\psi (x)=\langle x|\psi \rangle =&\int ^{\infty }_{-\infty }dp\,\langle x|p\rangle \langle p|x\rangle\\ 
=&\int ^{\infty }_{-\infty }dp\,e^{ipx/\hbar }\dfrac{1}{\sqrt{2\pi \hbar }}\phi (p)
\end{align*}
and
\[\phi (p)=\langle p|\psi \rangle =\int ^{\infty }_{-\infty }dx\,e^{ipx/\hbar }\dfrac{1}{\sqrt{2\pi \hbar }}\psi (x)\]
\end{thm}
\vspace{2ex}

\begin{rmk}
\begin{prop}
(Matrix representation) We can express the below basis independent expression of an arbitrary linear transformation in terms of matrices,
\[|\psi \rangle =A|\phi \rangle \]
by projecting into the basis $\langle m|$ and inserting the completeness relation,
\[\langle m|\psi \rangle =\sum _{n}\langle m|A|n\rangle \langle n|\phi \rangle \quad \mathrm{or}\quad \psi _{m}=\sum_{n}A_{mn}\phi _{n}\]
where $\phi _{m}$ and $\phi _{n}$ are ordinary matrices. On the other hand,
\[A=BC\]
can be expressed as
\[\langle m|A|n\rangle =\sum _{k}\langle m|B|k\rangle \langle k|C|n\rangle \quad \mathrm{or}\quad A_{mn}=\sum _{k}B_{mk}C_{kn}\]
\end{prop}
\vspace{2ex}
\begin{rmk}
(Adjoint matrices) We now seek the matrix expression for an adjoint operator.
\[(A_{nm})^{\dagger}=\langle n|A^{\dagger}|m\rangle =\langle An|m\rangle =\langle m|An\rangle ^{*}=(\langle m|A|n\rangle )^{*}=A^{*}_{mn}\]
We find that an adjoint of an operator in terms of its matrix is simply the complex transpose of the operator itself.
\end{rmk}
\vspace{2ex}
\begin{defi}
(Unitary operators) A key property of unitary operators (operators satisfying $O^{\dagger}O=OO^{\dagger}={\bf 1}$) is that they preserve inner products. This can be seen from the fact that as $|\phi \rangle \rightarrow U|\phi \rangle $ and $|\psi \rangle \rightarrow U|\psi \rangle $, 
\[\langle U\psi |U\phi \rangle =\langle \psi |U^{\dagger}U|\phi \rangle \]
\end{defi}
\vspace{2ex}
\begin{defi}
(Trace) The trace of a matrix is defined like the following
\[\mathop{\mathrm{Tr}}(A)=\sum _{n}\langle n|A|n\rangle =\sum _{n}A_{nn}\]
However, see how the definition relies of a certain basis $\{|n\rangle \}$. We now prove that the definition is independent on this set. Take another basis $\{|m\rangle \}$. We see that
\begin{align*}
\sum _{n}\langle n|A|n\rangle =&\sum _{n,m_1,m_2}\langle n|m_1\rangle \langle m_1|A|m_2\rangle \langle m_2|n\rangle \\=&\sum _{n,m_1,m_2}\langle m_1|A|m_2\rangle \langle m_2|n\rangle \langle n|m_1\rangle\\ =&\sum _{m_1,m_2}\langle m_1|A|m_2\rangle \delta _{m_1m_2}
\end{align*}
\end{defi}
\vspace{2ex}
\begin{defi}
(Spectral decomposition of a Hermitian operator) Consider a Hermitian operator 
\[\hat{A}=\sum _{n,m}|n\rangle \langle n|A|m\rangle \langle m|\]
and select the basis to be eigenkets of $\hat{A}$. We have
\[\langle n|\hat{A}|m\rangle =\langle n|A_{m}|m\rangle =A_{m}\langle n|m\rangle =A_{m}\delta _{nm}\]
When inserting this into the above, we diagonalise the matrix by obtaining
\[\sum _{m}|m\rangle A_{m}\langle m|\]
We see that the eigenvalues form the diagonals of the Hermitian matrix. 
\end{defi}
\vspace{2ex}
\begin{thm}
(Operator solution of the simple harmonic oscillator) We have previously witnessed the simple harmonic oscillator whose Hamiltonian was given by
\[\hat{H}=\dfrac{\hat{p}^2}{2m}+\dfrac{1}{2}m\omega ^2\hat{x}^2\]
We are certain that $\langle \psi |\hat{H}|\psi \rangle\geq 0 $ from the fact that $\langle \psi |\hat{x}\hat{x}|\psi \rangle =\langle \hat{x}^{\dagger}\psi |\hat{x}\psi \rangle \geq 0$. Define the following operator
\begin{align*}
\hat{a}=\sqrt{\dfrac{m\omega }{2\hbar }}\hat{x}+i\dfrac{\hat{p}}{\sqrt{2m\omega \hbar }}\\
\hat{a}^{\dagger}=\sqrt{\dfrac{m\omega }{2\hbar }}\hat{x}-i\dfrac{\hat{p}}{\sqrt{2m\omega \hbar }}
\end{align*}
If possible, memorise these. Knowing that $[\hat{x},\hat{p}]=i\hbar $, we can compute that $[\hat{a},\hat{a}^{\dagger}]={\bf 1}$. We then find
\[\hat{x}=\sqrt{\dfrac{\hbar }{2m\omega }}(\hat{a}+\hat{a}^{\dagger})\quad \hat{p}=\ldots \]
In terms of $\hat{a}$, we therefore find the Hamiltonian above to be (it is extremely important to be careful of the order of $\hat{a}$ and $\hat{a}^{\dagger}$):
\[\hbar \omega \Big(\hat{a}^{\dagger}\hat{a}+\dfrac{1}{2}\Big)\]
What is the commutator relation between $\hat{H}$ and $\hat{a}$? We see that
\[[H,a]=\hbar \omega \Big[a^{\dagger}a+\dfrac{1}{2}a\Big]=\hbar \omega (a^{\dagger}[a,a]+[a^{\dagger},a]a)=\hbar \omega (-a)\quad\mathrm{with}\quad [H,a^{\dagger}]=\hbar \omega a^{\dagger}\]
Physically, each operator is defined as operators that raise and lowers energy levels. Now we actually solve for energy levels.
\begin{align*}
\hat{H}|E\rangle =E|E\rangle \\
\hat{H}(a|E\rangle )=(aH+[\hat{H},a])|E\rangle =(\hat{a}E-\hbar \omega \hat{a})|E\rangle =(E-\hbar \omega )(\hat{a}|E\rangle )
\end{align*}

\end{thm}
\vspace{2ex}

