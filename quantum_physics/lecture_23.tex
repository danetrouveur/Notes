\section{Lecture 23 (May 28th)}
\begin{recall}
Last time we have learned that
\[[L_{i},L_{j}]=i\hbar \varepsilon _{ijk}L_{k}\]
and that for ${\bf L}^2=L_{x}^2+L_{y}^2+L_{z}^2$ and $L_{\pm}=L_{x}\pm iL_{y}$, the following identities hold.
\[\begin{cases}
[{\bf L}^2,L_{a}]=0&[L_{z},L_{+}]=\hbar L_{+}\\
[L_{+},L_{-}]=2\hbar L_{z}&[L_{z},L_{-}]=-\hbar L_{-}
\end{cases}\]
Also, we defined the simultaneous eigenvectors $|l,m\rangle $ to satifisfy
\[{\bf L}^2|l,m\rangle =\hbar ^2l(l+1)|l,m\rangle \quad\mathrm{and}\quad L_{z}|l,m\rangle =m\hbar |l,m\rangle \]
We notice that from $[{\bf L}^2,L_{a}]=0$, we have $[{\bf L}^2,L_{\pm}]=0$, or 
\[{\bf L}^2(L_{+}|l,m\rangle )=L_{+}{\bf L}^2|l,m\rangle =l(l+1)\hbar ^2(L_{+}|l,m\rangle )\]
Therefore, $L_{+}|l,m\rangle $ belongs to the $l$-multiplet. Continuing,
\begin{align*}
L_{z}(L_{+}|l,m\rangle )=&([L_{z},L_{+}]+L_{+}L_{z})|l,m\rangle\\=&(\hbar L_{+}+m\hbar L_{+})|l,m\rangle \\
=&(m+1)\hbar (L_{+}|l,m\rangle )
\end{align*}
From this we obtain the fact that
\[L_{+}|l,m\rangle =C_{lm}^{+}|l,m+1\rangle\quad\mathrm{or}\quad L_{-}|l,m\rangle =C^{-}_{lm}|l,m-1\rangle  \]
Taking the dual of the first, 
\[\langle l,m|L_{-}=(C^{+}_{lm})^{*}\langle l,m+1|\]
Applying this to the ket $|l,m\rangle $,
\[\langle l,m|L_{-}L_{+}|l,m\rangle =|C^{+}_{lm}|^2\langle l,m+1|l,m+1\rangle \]
The operator on the left is equal to ${\bf L}^2-L_{z}^2-\hbar L_{z}$. We then have
\[|C_{lm}^{+}|^2=\hbar ^2(l-m)(l+m+1)\]
and further that
\[C_{lm}^{+}=\pm\hbar \sqrt{(l-m)(l+m+1)}\]
where we take the positive sign. On the other hand,
\[C_{lm}^{-}=\hbar \sqrt{(l+m)(l-m+1)}\]
you MUST do this. Now we have accumulated the following facts
\[\begin{cases}
l\geq 0\\
l(l+1)-m(m+1)\geq 0\\
l(l+1)-m(m-1)\geq 0
\end{cases}\]
From this we obtain that
\[-l\leq m\leq l\]
which must be memorised. We lastly remark that operators like ${\bf L}^2$ are Casimir operators.
\end{recall}
\vspace{2ex}
\begin{cor}
We now have that
\begin{itemize}
\item[(i)] $l\geq 0$ 
\item[(ii)] $-l\leq m\leq l$
\item[(iii)] $m$ has a minimum value $m_{\mathrm{min}}$, abeit that it doesn't have to be $-l$
\end{itemize}
\[L_{-}|l,m_{\mathrm{min}}\rangle =0=C^{-}_{l,m_{\mathrm{min}}}|l,m_{\mathrm{min}}-1\rangle \]
which tells us that $C^{-}_{l,m_{\mathrm{min}}} =0$. In otherwords,
\[\sqrt{(l+m_{\mathrm{min}})(l-m_{\mathrm{min}}+1)}=0\]
which teaches us that $m_{\mathrm{min}}$ is exactly $-l$. You should separately prove that identically, $m$ has a maximum of $m_{\mathrm{max}}=+l$. Do this calculation.
\begin{itemize}
\item[(iv)] $m$ has a maximum value $m_{\mathrm{max}}=l$
\item[(v)] The possible values of $l$ values are given as $l\in {\bm Z}\cup \dfrac{1}{2}{\bm Z}\cup \{0\}$. We often write $l$ for integers, $s$ for fractional values, or $j$ altogether.
\end{itemize}
\end{cor}
\vspace{2ex}
\begin{thm}
Now we seek the matrix representation of these operators. Consider the following for $\langle l',m'|\hat{O}|l,m\rangle $:
\begin{itemize}
\item[(i)] $\hat{O}={\bf L}^2$, $l(l+1)\hbar \langle l',m'|l,m\rangle =l(l+1)\hbar ^2\delta _{ll'}\delta _{mm'}$
\item[(ii)] $\hat{O}=L_{z}$, $m\hbar \delta _{ll'}\delta _{m m'}$
\item[(iii)] $\hat{O}=L_{+}$, $C^{+}_{lm}\delta _{ll'}\delta _{m'm+1}$ and therefore is not diagonal
\end{itemize}
\end{thm}
\vspace{2ex}
\begin{defi}
(Conserved quantities) Consider, for $\partial \hat{O}_{S}/\partial t$, the Heisenberg equation
\[\dfrac{d \hat{O}_{H}}{d t}=\dfrac{i}{\hbar }[H,\hat{O}_{H}] \]
For $\hat{O}_{H}$ to be a conserved quantity,
\[\dfrac{d \hat{O}_{S}}{d t}=0\quad \iff\quad [H,\hat{O}_{H}]=0 \]
\end{defi}
\vspace{2ex}
\begin{thm}
We found that the momentum operator was the generator of the unitary translational operator,
\[T({\bf a})=\exp \Big(-\dfrac{i{\bf p}\cdot {\bf a}}{\hbar }\Big)\]
with $T({\bf a})|{\bf x}\rangle =|{\bf x}+{\bf a}\rangle $. Along this logic, what is the momentum operator ${\bf L}$ the generator of? We expect it to be rotation!
\end{thm}
\vspace{2ex}
\begin{proof}
For a small rotation counter clock-wise,
\[\begin{cases}
x'=x\cos\theta  -y\sin \theta \approx x-y\theta  \\
y'=x\sin \theta +y\cos \theta \approx x\theta +y
\end{cases}\]
Then, the wave function can be expressed as
\begin{align*}
\psi (x',y')=&\psi (x-y\theta ,y+x\theta )\\
=&\psi (x,y)-y\theta \dfrac{\partial \psi }{\partial x}+x\theta \dfrac{\partial \psi }{\partial y}\\
=&\psi (x,y)-y\theta \Big(\dfrac{i}{\hbar }\Big)(-i\hbar )\dfrac{\partial \psi }{\partial x}+x\theta \Big(\dfrac{i}{\hbar }\Big)(-i\hbar )\dfrac{\partial \psi }{\partial y}\\
=&\dfrac{i\theta }{\hbar }\langle x,y|(-yP_{x}+xP_{y})|\psi \rangle 
\end{align*}
We thus have found that the angular momentum operator is the generator of the unitary rotation operator.
\[U(\hat{n},\theta )=\exp \Big(-\dfrac{i{\bf J}\cdot {\bf n}\theta }{\hbar }\Big)\]

\end{proof}
\vspace{2ex}

