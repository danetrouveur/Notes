\section{Lecture 22 (May 26th)}
\begin{rmk}
Last class we have learned the time derivative of an operator in Heisenberg's picture, given by
\[\dfrac{d }{d t}\hat{A}_{H}(t)=\dfrac{i}{\hbar }[\hat{H},\hat{A}_{H}(t)]+\exp\Big(\dfrac{i\hat{H}t}{\hbar }\Big)\dfrac{\partial\hat{A}_{S} }{\partial t}(t_{\mathrm{s}}) \exp \Big(-\dfrac{i\hat{H}t}{\hbar }\Big) \]
The hamilotian in the Heisenberg picture is
\[\hat{H}_{H}=\exp \Big(\dfrac{i\hat{H}_{S}t}{\hbar }\Big)\hat{H_{S}}\exp \Big(-\dfrac{i\hat{H}_{S}t}{\hbar }\Big)=\hat{H}_{S}\]
from the fact that $[H^{n},H]=H^{n}H-HH^{n}=0$. 
\end{rmk}
\vspace{2ex}
\begin{thm}
(Harmonic oscillator) From $\hat{H}_{S}=\hbar \omega (\hat{a}^{\dagger}\hat{a}+1/2)=\hbar \omega (\hat{a}^{\dagger}(t)\hat{a}(t)+1/2)=\hat{H}_{H}$ we find
\begin{align*}
\dfrac{d \hat{a}(t)}{d t}=&\dfrac{i}{\hbar }[\hat{H},\hat{a}(t)]=i\omega [\hat{a}^{\dagger}(t)\hat{a}(t),\hat{a}(t)]\\=&i\omega [\hat{a}^{\dagger}(t),\hat{a}(t)]\hat{a}(t)=-i\omega (t)\hat{a}(t)
\end{align*}
Solving the following the differential equation we find
\[\hat{a}(t)=e^{-i\omega t}\hat{a}(0)\]
Preforming the adjoint,
\[\hat{a}^{\dagger}(t)=e^{i\omega t}\hat{a}^{\dagger}(0)\]
We can then seek (do this!) $\hat{x}(t)$ and $\hat{p}(t)$ and you will find dependence on $\hat{x}(0)$ and $\hat{p}(0)$ as if you solved the Newton's equations.
\end{thm}
\vspace{2ex}
\begin{ex}
Seek, for example,
\[\langle 0|\hat{a}(t)\hat{a}^{\dagger}(0)|0\rangle=e^{-i\omega t}\langle 0|\hat{a}(0)\hat{a}^{\dagger}(0)|0\rangle=e^{-i\omega t}  \]
these functions are called correlation functions. Harmonic oscillators and angular momentum questions will be dealt in the final exam.
\end{ex}
\vspace{2ex}
\begin{prop}
Suppose you want to translate a wave function. Such a such that preforms this operation can be expressed as
\[\langle x|T(a)|\psi \rangle \]
where we want to move the wave function $a$ to the right.
\begin{align*}
\psi (x-a)=&\sum ^{\infty }_{n=0}\dfrac{1}{n!}(-a)^{n}\dfrac{d ^{n}}{d x^{n}}\psi (x)=\sum ^{\infty }_{n=0}\dfrac{1}{n!}\Big(-\dfrac{ia}{\hbar }\Big)^{n}(-i\hbar )^{n}\dfrac{d ^{n}}{d x^{n}}\psi (x)\\=&\langle x|\sum ^{\infty }_{n=0}\dfrac{1}{n!}\Big(-\dfrac{ia\hat{p}}{\hbar }\Big)^{n}|\psi \rangle =\langle x|\exp \Big(-\dfrac{ia\hat{p}}{\hbar }\Big)|\psi \rangle   
\end{align*}
we thus find the translation operator to be a unitary operator
\[T(a)=\exp \Big(-\dfrac{ia\hat{p}}{\hbar }\Big)\]
\end{prop}
\vspace{2ex}
\begin{defi}
(Angular momentum) Angular momentum is classically given as
\[{\bf L}={\bf r}\times {\bf p}\]
We find that
\[\begin{cases}
L_{x}=yP_{z}-zP_{y}\\
L_{y}=zP_{x}-xP_{z}\\
L_{z}=xP_{y}-yP_{x}
\end{cases}\]
Finding the comutators,
\begin{align*}
[L_{x},L_{y}]=&[yP_{z}-zP_{y},zP_{x}-xP_{z}]\\
=&(yP_{z}-zP_{y})(zP_{x}-xP_{z})-(zP_{x}-xP_{z})(yP_{z}-zP_{y})\\
=&[yP_{z},zP_{x}]-[yP_{z},xP_{z}]-[zP_{y},zP_{x}]+[zP_{y},xP_{z}]\\
=&y[P_{z},z]P_{x}-y[P_z,P_z]x-P_y[z,z]P_x+x[zP_y,P_z]+[zP_y,x]P_z\\
=&y(-i\hbar )P_x+x(i\hbar)P_y=i\hbar L_z
\end{align*}
as
\[x[zP_{y},P_{z}]+[zP_{y},x]P_{z}=x(z[P_{y},P_{z}]+(i\hbar )P_{y})\]
Along this line, 
\[[L_{x},L_{y}]=i\hbar L_{z}\quad [L_{y},L_{z}]=i\hbar L_{x}\quad [L_{z},L_{x}]=i\hbar L_{y}\]
and we have
\[[L_{a},L_{b}]=i\hbar \epsilon _{abc}L_{c}\]
with $(a,b,c)=(x,y,z)$. This forms a $SU(2)$ Lie algebra, and applies to spin also. Now take
\[{\bf L}\cdot {\bf L}=L_{x}^2+L_{y}^2+L_{z}^2\]
Finding the commutator with $L_{z}$,
\begin{align*}
[{\bf L}\cdot {\bf L},L_{z}]=[L_{x}^2+L_{y}^2,L_{z}]=[L_{x}^2,L_{z}]+[L_{y}^2,L_{z}]=0
\end{align*}
where
\[[L_{x}^2,L_{z}]=L_{x}[L_{x},L_{z}]+[L_{x},L_{z}]L_{x}=L_{x}(-i\hbar L_{y})+(-i\hbar L_{y})L_{x}\]
and 
\[[L_{y}^2,L_{z}]=L_{y}[L_{y},L_{z}]+[L_{y},L_{z}]L_{y}=i\hbar L_{y}L_{x}+i\hbar L_{x}L_{y}\]
In quantum physics, we choose ${\bf L}^2$ and $L_{z}$ to be the commuting set with simulataneous eigenkets $|l,m\rangle $, which we define to satisfy the equations
\[{\bf L}^2|l,m\rangle =\hbar^2 l(l+1)|l,m\rangle \quad \mathrm{and}\quad L_{z}|l,m\rangle =\hbar m|l,m\rangle \]
From here, we see that
\[\langle l,m|{\bf L}\cdot {\bf L}|l,m\rangle =\hbar l(l+1)=\langle L_{x}lm|L_{x}lm\rangle +\ldots \geq 0\]
and we impose that $l\geq 0$. As $|l,m\rangle $ are eigenkets, we have
\[\langle l',m'|l,m\rangle =\delta _{ll'}\delta _{mm'}\]
These calculations are 1000\% in the exams and must be memorised.
\end{defi}
\vspace{2ex}
\begin{rmk}
We now investigate
\[\begin{cases}
L_{+}=L_{x}+iL_{y}\\
L_{-}=L_{x}-iL_{y}
\end{cases}\]
$[{\bf L}^2,L_{\pm}]=0$ is trivial, meanwhile
\[[L_{z},L_{+}]=\hbar L_{+}\quad [L_{z},L_{-}]=-\hbar L_{-}\]
\end{rmk}
\vspace{2ex}

