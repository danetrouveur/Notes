\section{Lecture 26 (June 9th)}
\begin{rmk}
Last class we have arrived at the differential equation
\[\rho H''+(2l+2-\rho )H'+(\lambda -l-1)H=0\]
which followed the form of an associated Laguerre equation.
\[\rho (L^{k}_{n_{r}})''+(k+1-\rho )(L^{k}_{n_{r}})'+n_{r}L_{n_{r}}^{k}=0\]
Such a differential equation can be solved through the Frobenius method. We note that it is required that $n_{r}\in {\bm Z}^{+}$. We can equate the two equations with $k=2l+1$ and $\lambda =n=l+1+n_{r}\geq 1$ ($n\in {\bm N}$). This $n$ is what we've learned in chemistry as the principle quantum number. With this information, we find the energy levels to be
\[E_{n}=-\dfrac{Z^2\alpha ^2mc^2}{2n^2}=-\dfrac{Z^2}{n^2}(13.6\mathrm{\ eV})\]
In this equation, the mass $m$ is technically the reduced mass,
\[m=\dfrac{Zm_{e}m_{p}}{m_{e}+Zm_{p}}\]
The total wavefunction then becomes
\[\psi _{n,l,m}=R_{n,l}(r)Y_{l,m}(\theta ,\phi )\]
For a particular $n$ and a corresponding energy level, there are $2n^2$ degeneracies, meaning that there are $2n^{2}$ states for a single energy level.
\end{rmk}
\vspace{2ex}
\begin{thm}
(Separation of variables for the harmonic oscillator) Consider the Hamiltonian
\[H=\dfrac{p_{x}^2}{2m}+\dfrac{p_{y}^2}{2m}+\dfrac{p_{z}^2}{2m}+V(x,y,z)\]
We sometimes take the potential to be a summation of component-dependent potentials $V(x,y,z)=V(x)+V(y)+V(z)$. Then, we can take $\psi (x,y,z)=X(x)Y(y)Z(z)$ as a separable solution. Also, take
\[V(x,y,z)=\dfrac{1}{2}k_{x}x^2+\dfrac{1}{2}k_{y}y^2+\dfrac{1}{2}k_{z}z^2\]
and we can find the energy levels to be
\[\hbar \omega \Big(n_{x}+n_{y}+n_{z}+\dfrac{3}{2}\Big)\]
Notice, how, we can also write the potential energy as $kr^2/2$. Then, we can also do variable separables and obtain a function in the form of
\[\psi (r,\theta ,\phi )=R(r)Y(\theta ,\phi )\]
We notice how solutions can become very complex when using different coordinate systems.
\end{thm}
\vspace{2ex}
\begin{defi}
We previously noted that the angular momentum can take half-integer values where we denote the values by $s$. Take $S=1/2$, and we use the following conventions for the possible kets
\[\Big|S=\dfrac{1}{2},S_{z}=\dfrac{1}{2}\Big\rangle =\left|\uparrow \right\rangle\qquad \Big|S=\dfrac{1}{2},S_{z}=-\dfrac{1}{2}\Big\rangle =\left|\downarrow\right\rangle \]
We then see that $\langle \uparrow |\uparrow \rangle =\langle  \downarrow |\downarrow \rangle =1$ and $\langle \uparrow |\downarrow\rangle=0$. With this information, we can find the matrix representation of $\hat{S}_{z}$ with respect to the above eigenkets that construct a basis, and we find
\[\hat{S}_{z}=\dfrac{\hbar }{2}\begin{pmatrix}
1&0\\0&{-1}
\end{pmatrix}=\dfrac{\hbar }{2}\sigma _{z}
\]
The eigenkets are named qubits. We now define
\[S_{\pm}=S_{x}\pm iS_{y} \]
and seek their matrix representations. We find that
\[S_{+}=\hbar \begin{pmatrix}
0&1\\0&0
\end{pmatrix}\quad \mathrm{and} \quad S_{-}=\hbar \begin{pmatrix}
0&0\\
1&0
\end{pmatrix}
\]
from which follows that
\[S_{x}=\dfrac{\hbar }{2}\begin{pmatrix}
0&1\\1&0
\end{pmatrix}=\dfrac{\hbar }{2}\sigma _{x}\quad \mathrm{and}\quad S_{y}=
\begin{pmatrix}
0&-i\\
i&0
\end{pmatrix}=\dfrac{\hbar }{2}\sigma _{y}
\]
The Pauli matrices together satisfy the following properties
\begin{itemize}
\item[(i)] Hermitian
\item[(ii)] $\sigma _{x}^2=\sigma _{y}^2+\sigma _{z}^2=I$ and hence unitary
\item[(iii)] $\sigma _{i}\sigma _{j}=i\varepsilon _{ijk}\sigma k=-\sigma _{j}\sigma _{i}$ where $\{i,j,k\}=\{x,y,z\}$
\item[(iv)] $\{\sigma _{i},\sigma _{j}\}= \sigma_{i}\sigma _{j}-\sigma_{j}\sigma _{i}=\delta _{ij}2I $ which is to say that the Pauli matrices satisfy the Clifford algebra
\end{itemize}
\end{defi}
\vspace{2ex}

