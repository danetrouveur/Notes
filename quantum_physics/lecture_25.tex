\section{Lecture 25 (June 4th)}
\begin{thm}
In spherical polar coordinates, the Hamiltonian is given as
\[\hat{H}=\dfrac{{\bf p}^2}{2m}+V({\bf r})=-\dfrac{\hbar ^2}{2m}\nabla ^2_{{\bf r}}+V({\bf r})\]
The Laplacian in spherical coordinates is given as
\[\nabla ^2=\dfrac{\partial ^2}{\partial r^2}+\dfrac{2}{r}\dfrac{\partial }{\partial r}+\dfrac{1}{r^2}\underbrace{\Big(\dfrac{\partial ^2}{\partial \theta ^2}+\cot\theta \dfrac{\partial }{\partial \theta } +\dfrac{1}{\sin^2\theta  }\dfrac{\partial ^2}{\partial \phi ^2}  \Big)}_{-{\bf L}^2/\hbar ^2} \]
We want to solve
\[\hat{H}\psi (r,\theta ,\phi )=E\psi (r,\theta ,\phi )\]
When solving a problem in QM, we need to first find all operators that commute with the Hamiltonian. For a potential that only depends on the modulus of the distance $V(|{\bf r}|)$, we can compute the commutator
\[[{\bf p}^2,L_{z}]=[p_{x}^2+p_{y}^2+p_{z}^2,xp_{y}-yp_{x}]=0\]
check this later on. On the otherhand, we can compute the following,
\begin{align*}
[L_{z},V(|{\bf r}|)]=&[xP_{y}-yP_{x},V(|{\bf r}|)=x[P_{y},V(|{\bf r}|)]-y[P_{x},V(|{\bf r}|)]\\=&x(-i\hbar )V'(|{\bf r}|)\dfrac{y}{r}-y(-i\hbar )V'(|{\bf r}|)\dfrac{x}{r}=0
\end{align*}
In sum,
\[[H,L_{z}]=0\quad \mathrm{and}\quad [H,{\bf L}^2]=0\]
and there exists simultaneous eigenfunctions of $H$ and $L_{z}$ and ${\bf L}^2$. We previously saw that $Y_{l,m}(\theta ,\phi )$ is a set of eigenfunctions of $L_{z}$ and ${\bf L}^2$. We take the simultaneous eigenfunctions for the three operators to be $Y_{lm}(\theta ,\phi )R_{l}(r)$. Under the assumption that we are dealing with eigenfunctions of the ${\bf L}^2$ operator, we have
\[\nabla\rightarrow \dfrac{\partial ^2}{\partial r^2}+\dfrac{2}{r}\dfrac{\partial }{\partial r}-\dfrac{l(l+1)}{r^2}  \]
and the full Schrodinger equation becomes
\[\Bigg[-\dfrac{\hbar ^2}{2m}\Big(\dfrac{d ^2}{d r^2}+\dfrac{2}{r}\dfrac{d }{d r} \Big)+V(r)+\dfrac{l(l+1)\hbar ^2}{2mr^2}\Bigg]R_{nl}(r)=ER_{nl}(r)\]
\end{thm}
\vspace{2ex}
\begin{rmk}
(Solutions of the above formulation) Imagine solving the equation in the free-particle case, which we know the solution of to be $\exp (ip\cdot x)/(\sqrt{2\pi \hbar })^2$. With the substitution $\rho =kr$ and $k=2mE/\hbar ^2>0$, we have
\[\rho ^2\dfrac{d ^2R}{d \rho ^2}+2\rho \dfrac{d R}{d \rho }+(\rho ^2-l(l+1))R=0  \]
which is the spherical Bessel equation, an self-adjoint equation. Substitute $R(\rho )=Z(\rho )/\sqrt{\rho }$ to find
\[\rho ^2Z''+\rho Z'+\Big(\rho ^2-\Big(l+\dfrac{1}{2}\Big)^2\Big)Z=0\]
which is the original Bessel equations. The solutions are given as the Bessel functions $J_{l+1/2}(\rho )$ and the Neumann functions $N_{l+1/2}(\rho )$. Reserving the substitutions, we have the spherical Bessel and spherical Neumann functions $j_{l}(\rho )$ and $n_{l}(\rho )$. We can create linear combinations of these functions to obtain spherical Hankel functions, which represent travelling waves. In the limit, we have
\[j_{l}(\rho )\sim \dfrac{1}{\rho }\sin \Big(\rho -\dfrac{l\pi }{2}\Big)\quad n_{l}(\rho )\sim \dfrac{1}{\rho }\cos \Big(\rho -\dfrac{l\pi }{2}\Big) \]
\end{rmk}
\vspace{2ex}
\begin{ex}
(Spherical well) Now consider the spherical well problem. When $r\geq a$, we have $R_{l}(r=a)=0$. Inside the well, we have $j_{l}(kr)$ and $n_{l}(kr)$, but we reject the latter as $n_{l}(kr)$ diverges at the origin. As the wavefunction needs to be continuous at $r=a$, we have
\[j_{l}(ka)=0\]
and $ka=x_{nl}$ where $x_{nl}$ denotes the $n$-th root of $j_{l}(x)$. We thus find $k$ to be $x_{nl}/a$ and the energy values to be
\[E_{nl}=\dfrac{\hbar ^2}{2m}\Big(\dfrac{x_{nl}}{a}\Big)^2\]
The last step is normalisation. We impose that
\[\int ^{\infty }_{0}r^2dr\,(j_{l})^2(Y_{lm})^2=1\]
to obtain the coefficients.
\end{ex}
\vspace{2ex}
\begin{ex}
(Hydrogen atom) For the Hydrogen atom, we need to substitute the Coulomb potential, 
\[V(|{\bf r}|)=\dfrac{1}{4\pi \varepsilon_0 }\dfrac{(-e)Ze}{r}\]
We note that we solve this for negative energy levels, we want bounded motion. Define $\rho =\sqrt{8m(-E)/\hbar ^2}r$ and $\lambda =Z\alpha \sqrt{mc^2/2(-E)}$. We then have the equation
\[R''+\dfrac{2}{\rho }R'-\dfrac{l(l+1)}{\rho ^2}R+\Big(\dfrac{\lambda }{\rho }-\dfrac{1}{4}\Big)R=0\]
We know that the asymptotic factor must be $\exp (-\rho /2)$, so we take the ansatz $R=\exp (-\rho /2)G(\rho )$ to find
\[G''-\Big(1-\dfrac{2}{\rho }\Big)G'+\Big[\dfrac{\lambda -1}{\rho }-\dfrac{l(l+1)}{\rho ^2}\Big]G=0\]
As $\rho \rightarrow 0$, we find $G=\rho ^{l}$ or $G=\rho ^{-(l+1)}$. However, as $l\geq 0$, we take the prior, and again set yet another ansatz $R=\exp (-\rho /2)\rho ^{l}H(\rho )$ to find
\[H''+\Big(\dfrac{2l+1}{\rho }-1\Big)H'+\dfrac{\lambda -l-1}{\rho }H=0\]
which follows the form of an Associated Laguerre equation. We can verify that for the Laguerre equations $L^{k}_{nr}$ satisfy
\[\rho (L^{k}_{n_{r}})''+(k+1-\rho )(L^{k}_{n_{r}})'+n_{r}L^{k}_{n_{r}}=0\]

\end{ex}
\vspace{2ex}

