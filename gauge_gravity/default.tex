% type of document
\documentclass[a4paper, 11pt]{article}
\usepackage{jheppub}

% formatting and spacing
% \usepackage{fancyhdr}
% \pagestyle{fancy}
\usepackage{setspace}
\usepackage{parskip}
\usepackage{lipsum}
\usepackage{enumitem}
% \usepackage{hyperref}

% math notation and fonts
\usepackage{amsmath}
\usepackage{amssymb}
\usepackage{amsfonts}
\usepackage{amsthm}
\usepackage{fixmath}
\usepackage{bm}
\usepackage{esint}
\usepackage{xcolor}
\usepackage{tikz}

\theoremstyle{definition}
\newtheorem*{defi}{Definition}
\newtheorem*{rmk}{Remark}
\newtheorem*{prop}{Proposition}
\newtheorem*{thm}{Theorem}
\newtheorem*{lem}{Lemma}
\newtheorem*{cor}{Corollary}
\newtheorem*{recall}{Recall}
\newtheorem*{ex}{Example}

% korean
% \usepackage{kotex}
% \usepackage{CJKutf8}

% rendering images and etcetera
\usepackage{graphicx}
% \usepackage{caption}

% bibliography
% \usepackage{natbib}
% \usepackage{csquotes}

% user-defined commands
% \newcommand{\unit}[1]{\ensuremath{\;\mathrm{#1}}}
% \newcommand{\ve}[1]{\bm{#1}}
% \newcommand{\vesy}[1]{\hat\bm{#1}}
% \newcommand{\kor}[1]{\begin{CJK}{UTF8}{mj}{#1}\end{CJK}}
% \newcommand{\te}[1]{\text{#1}}
% \newcommand{\quick}[2]{${#1}_\te{#2}$}
% \newcommand{\spac}[1]{\begingroup\addtolength{\jot}{1em}{#1}\endgroup}
% \newcommand{\pepsi}[0]{\mathrm\Psi}

\usepackage{cancel}

\title{}
\author{Dane Jeon}
\affiliation{Sogang University,\\
35 Baekbeom-ro, Mapo-gu, Seoul 04107, Republic of Korea}
\emailAdd{danetrouveur@gmail.com}
\abstract{}

\begin{document}

\begin{defi}
Consider the following Lagrangian density:
\[\mathcal{L}= \Big(R+4g^2\sum _{i}(e^{\phi_{i}}+e^{-\phi _{i}}+\chi_{i}^2e^{\phi_{i}})\Big)\star1-\dfrac{1}{2}\sum _{i}\Big(d\phi _{i}\wedge \star d\phi_{i}+e^{2\phi_{i}}d\chi_{i}\wedge \star d\chi_{i}\Big)+\mathcal{L}_{KinA}+\mathcal{L}_{CS}\]
\end{defi}
\vspace{2ex}
\begin{defi}
We define the truncation limit as the following set of limits
\[\begin{cases}
\phi_1\rightarrow\xi\\ \chi_1\rightarrow \chi\\ A_1\ \mathrm{and}\ A_2\rightarrow A_3\\   A_3 \ \mathrm{and}\ A_4\rightarrow A_1 
\end{cases}\]
While the remaining fields are put to $0$. We can additionally impose the following limits such that the equations of motion would reduce to the one in Cassani's paper. 
\[\begin{cases}
A_{i}\rightarrow \dfrac{1}{\sqrt{2}}A_{i}\\
g\rightarrow \dfrac{1}{2}g 
\end{cases}\]
\end{defi}
\vspace{2ex}
\begin{prop}
(Variation with respect to scalar fields $\phi_i$) The variation of the Lagrangian density with respect to the fields $\phi_{i}$ are 
\[\delta \mathcal{L}=\delta \phi_i\Big[4g^2(e^{\phi_i}-e^{-\phi_i}+\chi_{i}^2e^{\phi _{i}})\star 1+d(\star d\phi_i)-e^{2\phi_i}d\chi_i\wedge\star d\chi_i+\dfrac{\delta \mathcal{L}_{KinA}}{\delta \phi_i}+\dfrac{\delta \mathcal{L}_{CS}}{\delta \phi_i}\Big]\]
where we omitted the expressions for $\delta \mathcal{L}_{KinA}/\delta \phi _{i}$ and $\delta \mathcal{L}_{CS}/\delta \phi _{i}$ for time being.  We can easily see how, under the truncation limit,
\[\delta_{\phi_2}\mathcal{L}\rightarrow 0\quad\mathrm{and}\quad \delta _{\phi_3}\mathcal{L}\rightarrow 0\]
For the variation with respect to the field $\phi_1$, however, we have, along with the Cassani scaling conventions,
\[\dfrac{\delta \mathcal{L}_{KinA}}{\delta \phi_1}\rightarrow \dfrac{e^{-2\xi }(-e^{4\xi }\star F_1\wedge F_1+(1+e^{2\xi }\chi ^2)^2\star F_3\wedge F_3)}{(1+e^{2\xi }\chi ^2)^2} \tag{\dagger}\]
and
\[\dfrac{\delta \mathcal{L}_{CS}}{\delta \phi_1}\rightarrow \dfrac{e^{3\xi }\chi (3+e^{2\xi }\chi ^2)\star F_1\wedge F_1}{2(1+e^{2\xi }\chi ^2)^2}-\dfrac{1}{2}e^{-\xi }\chi (-1+e^{2\xi }\chi ^2)\star F_3\wedge F_3\tag{\dagger\dagger}\]
We therefore have, in the Cassani paper's conventions,
\[\delta \mathcal{L}=\delta \xi \Big[g^2(e^{\xi }-e^{\xi  }+\chi ^2e^{\xi })\star 1+d(\star d\xi )-e^{2\xi }d\chi \wedge \star d\chi +(\dagger)+(\dagger\dagger)\Big]\]

\end{prop}
\vspace{2ex}

\end{document}:
