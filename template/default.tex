% type of document
\documentclass[a4paper, 11pt]{article}
\usepackage{jheppub}

% formatting and spacing
% \usepackage{fancyhdr}
% \pagestyle{fancy}
\usepackage{setspace}
\usepackage{parskip}
\usepackage{lipsum}
\usepackage{enumitem}
% \usepackage{hyperref}

% math notation and fonts
\usepackage{amsmath}
\usepackage{amssymb}
\usepackage{amsfonts}
\usepackage{amsthm}
\usepackage{fixmath}
\usepackage{bm}
\usepackage{esint}
\usepackage{xcolor}
\usepackage{tikz}

\theoremstyle{definition}
\newtheorem*{defi}{Definition}
\newtheorem*{rmk}{Remark}
\newtheorem*{prop}{Proposition}
\newtheorem*{thm}{Theorem}
\newtheorem*{lem}{Lemma}
\newtheorem*{cor}{Corollary}
\newtheorem*{recall}{Recall}
\newtheorem*{ex}{Example}

% korean
% \usepackage{kotex}
% \usepackage{CJKutf8}

% rendering images and etcetera
\usepackage{graphicx}
% \usepackage{caption}

% bibliography
% \usepackage{natbib}
% \usepackage{csquotes}

% user-defined commands
% \newcommand{\unit}[1]{\ensuremath{\;\mathrm{#1}}}
% \newcommand{\ve}[1]{\bm{#1}}
% \newcommand{\vesy}[1]{\hat\bm{#1}}
% \newcommand{\kor}[1]{\begin{CJK}{UTF8}{mj}{#1}\end{CJK}}
% \newcommand{\te}[1]{\text{#1}}
% \newcommand{\quick}[2]{${#1}_\te{#2}$}
% \newcommand{\spac}[1]{\begingroup\addtolength{\jot}{1em}{#1}\endgroup}
% \newcommand{\pepsi}[0]{\mathrm\Psi}

\begin{document}
\title{Experimental Physics Final Notes}
\author{Dane Jeon}
\noaffiliation
\maketitle

\section{Torsional Oscillator}
\begin{itemize}
\item[(i)] In this experiment, we study an analogue of the simple harmonic oscillator, the torsional oscillator. In the simple harmonic case, we know that there are three variations of oscillations, undamped, damped, and driven. We study these exact three cases for the torsional harmonic oscillator and carry out around 3 experiments for all of them. 
\item[(ii)] The first step is {\bf calibrating the angular position transducer}. Before preforming the actual experiment, we try to find the relationship between the copper disk's angular displacement and the voltage, which is related through a device called the angular position transducer. We carefully rotate the oscillator into certain positions and see the voltage output of the transducer through oscilloscope.
\item[(iii)] The second step is the {\bf static measurement of the spring constant}. Begining the main course of the experiment, we place equal weights on both sides of a string that is connected to the torsional oscillator, causing it to find its equalibrium in different angular displacements. By creating a linear fit, we are able to find the spring constant.
\[\tau ={\bf r}\times {\bf F}=r\cdot  (2mg)=-\kappa \theta \]
where $m$ is the length of a single weight placed.
\item[(iv)] The third step is the {\bf dynamic measurement of the spring constant and moment of inertia}. Here, we try to obtain the period of the torsional oscillator and the same spring constant $\kappa $ from the above. The key idea is the torsional oscillator's period follows the formula $T=2\pi \sqrt{I/\kappa }$ and that as a single mass diisk is added the additional moment of inertia is given by
\[\Delta I=\dfrac{M}{2}(R_1^2+R_2^2)\]
We therefore expect the relationship
\[\Big(\dfrac{T}{2\pi }\Big)^2=\dfrac{1}{\kappa }(I+n\Delta I)\]
\item[(v)] The fourth step is {\bf applying static torque magnetically}. By using a Helmholtz coil, we find the reationship between the current applied and the angular displacement to be approximately linear.
\subsection{Damped Oscillations}
We observe {\bf Eddy current damping}, {\bf death spiral} which the relationship between angular velocity and angular position, and the time versus angular position for {\bf three different types of damping}.
\end{itemize}
\section{Magnetic Torque}

\section{Magnetic Hysteresis}

\section{Dispersion/Resolving Power of Prism}

\section{Specific Charge of the Electron $e$ by $m$}
\begin{itemize}
\item[(i)] To find the specific charge (charge per mass) of an electron, we simply rearrange the formula
\[eV=\dfrac{1}{2}mv ^2\]
where we substitute $v$ from
\[\dfrac{mv ^2}{r}=evB\]
\end{itemize}
\section{Polarisation \& Liquid Crystals}

\section{Interferometer}
\begin{itemize}
\item[(i)] 
\end{itemize}
\section{Quantum Analog I}

\end{document}
