\section{Lecture 11 (April 11)}
\begin{recall}
(Cauchy integral formula)
\[f(z)=\dfrac{1}{2\pi i}\int _{\partial D}\dfrac{f(\xi )}{\xi -z}\;d\xi =\sum ^{\infty }_{n=0}a_{n}(z-z_0)^{n}\]
for $z\in D(z_0,r)\subset D$.
\end{recall}
\vspace{2ex}
\begin{thm}
(Liouville's theorem) Bounded entire functions are constant.
\end{thm}
\vspace{2ex}
\begin{proof}
Let $f$ be an entire function such that $|f(z)|<M$ for all $z\in {\bm C}$. Take any $\varepsilon >0$ and any $z_0\in {\bm C}$. It is enough to show $|f'(z_0)|<\varepsilon $. Take $R>0$ so that $M/R<\varepsilon $. By the Cauchy integral formula, 
\[f'(z_0)=\dfrac{1}{2\pi i}\int _{|z-z_0|=R}\dfrac{f(z)}{(z-z_0)^2}\;dz\]
This implies that
\[|f'(z_0)|\leq \dfrac{1}{2\pi }\int _{|z-z_0|=R}\dfrac{M}{|z-z_0|^2}\;|dz|=\dfrac{1}{2\pi }\dfrac{M}{R^2}2\pi R=\dfrac{M}{R}<\varepsilon \]
\end{proof}
\vspace{2ex}
\begin{thm}
(Fundamental theorem of algebra) A complex polynomial
\[p_{n}(z)=a_{n}z^{n}+a_{n-1}z^{n-1}+\ldots +a_0\]
with $a_{n}\ne 0$ has a zero in ${\bm C}$.
\end{thm}
\vspace{2ex}
\begin{proof}
Suppose there is no $z\in {\bm C}$ with $p_{n}(z)=0$. Then $g(z)=1/p_{n}(z)$ is a bounded entire function which is a contradiction.
\end{proof}
\vspace{2ex}
\begin{cor}
If $f$ is a non-constant entire function, then $f({\bm C})$ is dense in ${\bm C}$.
\end{cor}
\vspace{2ex}
\begin{proof}
Suppose there exists $\delta >0$, $w_0\in {\bm C}$ such that $f({\bm C})\cap D(w_0,\delta )=\emptyset$. Then 
\[g(z)=\dfrac{1}{f(z)-w_0}\]
satisfies $|g(z)|\leq 1/\delta $ for all $z\in {\bm C}$. By Liouville's theorem, $g(z)={\bm C}$ which imlpies that $f(z)=w_0+1/C$ is constant.
\end{proof}
\vspace{2ex}
\begin{thm}
(Picard's theorem 1) If $f$ is non-constant entire function there is $w_0\in {\bm C}$ such that 
\[{\bm C}\;\backslash\;\{w_0\}\]
For example, for $f(z)=e^{z}$, $w_0=0$. 
\end{thm}
\vspace{2ex}
\begin{rmk}
We have a brief summary of power series. For a power series $f(z)=\sum ^{\infty }_{n=0}a_{n}(z-z_0)^{n}$,
\begin{itemize}
\item[(i)] there is $R\in [0,\infty ]$ such that the series converges absolutely on $D(z_0,R)$. If $R=0$ the series converges only at $z_0$ and if $R=\infty $ the series converges on all of ${\bm C}$
\item[(ii)] if $0<r<R$ the series converges uniformly on $\bar{D}(z_0,r)$
\item[(iii)] on $D(z_0,R)$,
\[f'(z)=\sum ^{\infty }_{n=1}na_{n}(z-z_0)^{n-1}\] 
as for a differentiable $f_{n}$, if $\sum f_{n}$ converges pointwise and $\sum f_{n}'$ converges uniformly, $(\sum f_{n})'=\sum f_{n}'$
\item[(iv)] for $n=0,1,\ldots $,
\[a_{n}=\dfrac{f^{n}(z_0)}{n!}\]
as $f^{(k)}(z)=\sum ^{\infty }_{n=k}n(n-1)\ldots (n-k+1)a_{n}(z-z_0)^{n-k}$
\end{itemize}
Take
\[R=\dfrac{1}{\limsup |a_{n}|^{1/n}}\in [0,\infty] \]
By applying the root test to $\sum ^{\infty }_{n=0}|a_{n}(z-z_0)^{n}|$, we have that
\[\limsup|a_{n}(z-z_0)^{n}|<1\]
implies that the absolute of the sequence converges. We know that
\[\limsup |a_{n}|^{1/n}|z-z_0|=|z-z_0|/R<1\]
and that it converges.
\end{rmk}
\vspace{2ex}
\begin{recall}
Let $D$ be a bounded simply connected domain with $\partial D$ being a POSCC and let $\bar{D}(z_0,r)\subset D$.
\[f(z)=\dfrac{1}{2\pi i}\int _{\partial D}\dfrac{f(\xi )}{\xi -z}\;d\xi =\sum ^{\infty }_{n=0}\Big[\dfrac{1}{2\pi i}\int _{\partial D}\dfrac{f(\xi )}{(\xi -z_0)^{n+1}}\;d\xi \Big](z-z_0)^{n}\]
for $z\in D(z_0,r)$. The coefficients can be obtained like the following
\[\dfrac{1}{2\pi i}\int _{\partial D}\dfrac{f(\xi )}{(\xi -z_0)^{n+1}}\;d\xi =\dfrac{f^{(n)}(z_0)}{n!}\]
and
\[f^{(n)}(z_0)=\dfrac{n!}{2\pi i}\int _{\partial D}\dfrac{f(\xi )}{(\xi -z_0)^{n+1}}\;d\xi \]
\end{recall}
\vspace{2ex}
\begin{ex}
On ${\bm R}^{2}$, consider $f(x,y)=xy$. Notice how it is zero in both the $x$ and $y$-axes. However, $f$ is not constant zero. 
\end{ex}
\vspace{2ex}
\begin{thm}
Let $f(z)=\sum ^{\infty }_{n=0}c_{n}(z-z_0)^{n}$ converges on $D(z_0,R)$. If $\{a_{n}\}$ is a sequence of distinct points in $D(z_0,R)$ such that 
\begin{itemize}
\item[(i)] $a_{n}\rightarrow z_0$
\item[(ii)] $f(a_{n})=0$ for every $n\in {\bm N}$
\end{itemize}
then $f(z)=0$ for all $z\in D(z_0,R)$. 
\end{thm}
\vspace{2ex}
\begin{proof}
For $0=f(a_{k})=\sum ^{\infty }_{n=0}c_{n}(a_{k}-z_0)^{n}$, taking $k\rightarrow \infty $ we get have $c_0=0$. 
\end{proof}
\vspace{2ex}

