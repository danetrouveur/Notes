\section{Lecture 14 (May 8)}
\begin{defi}
We say that $f$ has a singularity at $z_0$ if $f$ is not differentiable at $z_0$. We say that $f$ has an isolated singularity at $z_0$ provided that in addition to not being differentiable, there is $r>0$ such that $f$ is differentiable on $D^{*}(a,r)$
\end{defi}
\vspace{2ex}
\begin{ex}
All of these functions have a singularity at $z_0$
\begin{itemize}
\item[(i)] $f(z)=\dfrac{\sin z}{z}$ at $z=0$. Notice that when we define $f(0)=1$, then $f$ is entire.
\item[(ii)] $f(z)=\dfrac{e^{z}}{(z-1)^2}$ at $z_0=1$. Notice that 
\[\lim _{z\rightarrow 1}(z-1)^2f(z)=e\quad \mathrm{and}\quad \lim _{z\rightarrow 1}|f(z)|=\infty \]
That is, the function approaches infinity as you approach the singularity.
\item[(iii)] $f(z)=e^{1/z}$ at $z_0=0$. Notice that 
\[\lim _{x\rightarrow 0^{+}}e^{1/x}=+\infty \quad\mathrm{but}\quad\lim _{x\rightarrow 0^{-}}e^{1/x}=0\] 
\item[(iv)] $f(z)=\dfrac{1}{\sin (\pi /z)}$ at $z_0=0$. For $z_{n}=1/n$, $f$ has a singularity at each $z_{n}=1/n$. 
\end{itemize}
In each case, we see a removable, pole, essential, and non-isolated singularity!
\end{ex}
\vspace{2ex}
\begin{defi}
Let $f$ has an isolated singularity at $z_0$.
\begin{itemize}
\item[(i)] (Removable) If we can define $f(z_0)$ such that $f$ is analytic at $z_0$, then we say that $f$ has a removable singularity at $z_0$.
\item[(ii)] (Pole) If there is a $k\in {\bm N}$ such that 
\[\lim _{z\rightarrow z_0}(z-z_0)^{k}f(z)=\alpha \ne 0\]
we say that $f$ has a pole of order $k$ at $z_0$. If $k=1$, $f$ is said to have a simple pole.
\item[(iii)] (Essential) If $f$ satisfies neither of the two above, we then say that $f$ has an essential singularity.
\end{itemize}
\end{defi}
\vspace{2ex}
\begin{cor}
If $f$ has an isolated singularity at $z_0$, then the Laurent series of $f$,
\[f(z)=\sum ^{\infty }_{n=-\infty }a_{n}(z-z_0)^{n}\]
is available in some $D^{*}(z_0,r)$.
\end{cor}
\vspace{2ex}
\begin{thm}
(Riemann) If $f$ has an isolated singularity at $z_0$ and $|f(z)|$ is bounded and analytic on some $D^{*}(z_0,r)$, then the singularity at $z_0$ is removable.
\end{thm}
\vspace{2ex}
\begin{proof}
Define $h$ on $D(z_0,r)$ such that
\[h(z_0)=0\quad \mathrm{and}\quad h(z)=(z-z_0)^2f(z)\]
on $D(z_0,r)$. Then,
\[h'(z_0)=\lim _{z\rightarrow z_0}\dfrac{h(z)-h(z_0)}{z-z_0}=\lim _{z\rightarrow z_0}\dfrac{(z-z_0)^2f(z)}{z-z_0}=\lim _{z\rightarrow z_0}(z-z_0)f(z)=0\]
since $f(z)$ is bounded on $D^{*}(z_0,r)$. Thus $h$ is analytic on $D(z_0,r)$ so that 
\[h(z)=\sum ^{\infty }_{n=0}a_{n}(z-z_0)^{n}\]
As such, we see that on $D(z_0,r)$, 
\[f(z)=\dfrac{a_0}{(z-z_0)^2}+\dfrac{a_1}{(z-z_0)}+a_{2}+a_3(z-z_0)+a_{4}(z-z_0)^2+\ldots \]
However, we know that there should be no divergent terms as $f$ is bounded, and $a_0=a_1=0$. Accordingly, we can then define
\[f(z_0)=a_2\]
to create a power series expansion that is convergent on $z_0$ such that $f$ on the disk $D(z_0,r)$ is analytical.
\end{proof}
\vspace{2ex}
\begin{thm}
(Casorati-Weierstrass) If $f$ has an essential singularity at $z_0$, then for every $r>0$, $f(D^{*}(z_0,r))$ is dense in ${\bm C}$. 
\end{thm}
\vspace{2ex}
\begin{proof}
Suppose not, then there is $w_0\in {\bm C}$ and $\delta >0$ such that $f(D^{*}(z_0,r))\cap D(w_0,\delta )=\varnothing$. Then $|f(z)-w_0|\geq \delta $ for all $z\in D^{*}(z_0,r)$. Define 
\[g(z)=\dfrac{1}{f(z)-w_0}\]
on $D^{*}(z_0,r)$. Then $|g(z)|\leq 1/\delta $ for all $z\in D^{*}(z_0,r)$. By the previous theorem, we can define $g(z_0)$ so that $g(z)$ is analytic on $D^{*}(z_0,r)$. Let 
\[g(z)=\sum ^{\infty }_{n=0}a_{n}(z-z_0)^{n}=a_0+a_1(z-z_0)+a_2(z-z_0)^2+\ldots \]
on $D^{*}(z_0,r)$. That is,
\[\dfrac{1}{f(z)-w_0}=a_0+a_1(z-z_0)+a_2(z-z_0)^2+\ldots \]
on $D^{*}(z_0,r)$. In such a case, 
\[\lim _{z\rightarrow z_0}\dfrac{1}{f(z)-w_0}=a_0\]
Suppose, firstly, that $a_0\ne 0$. Then we have
\[f(z)=\dfrac{1}{g(z)}+w_0\]
and $f$ has a removable singularity at $z_0$. 
\\\\
In the second case, let $a_0=0$. Take $k\in {\bm N}$ to be the smallest $k$ integer such that $a_{k}\ne 0$. Then,
\[\dfrac{1}{f(z)-w_0}=a_{k}(z-z_0)^{k}+a_{k+1}(z-z_0)^{k+1}+\ldots \]
at $D^{*}(z_0,r)$. Thus, 
\[\dfrac{1}{(f(z)-w_0)(z-z_0)^{k}}=a_{k}+a_{k+1}(z-z_0)+\ldots \]
on $D^{*}(z_0,r)$. We see that
\[\lim _{z\rightarrow z_0}\dfrac{1}{(f(z)-w_0)(z-z_0)^{k}}=a_{k}\ne 0\]
with
\[\lim _{z\rightarrow z_0}(f(z)-w_0)(z-z_0)^{k}=\lim _{z\rightarrow z_0}(z-z_0)^{k}f(z)+\lim _{z\rightarrow z_0}w_0(z-z_0)^{k}=\dfrac{1}{a_{k}}\ne 0\]
Therefore, in the second case, $f$ has a pole of order $k$.
\end{proof}
\vspace{2ex}
\begin{thm}
(Picard's Great Theorem) If $f$ has an essential singularity at $z_0$ then for every $r>0$, $f(D^{*}(z_0,r))$ takes every complex number (except possibly one) infinitely many times.
\end{thm}
\vspace{2ex}
\begin{thm}
If $f$ has a pole of order $k\in {\bm N}$ at $z_0$, then $f(z)=\sum ^{\infty }_{n=-k}a_{n}(z-z_0)^{n}$ on some $D^{*}(z_0,r)$.
\end{thm}
\vspace{2ex}
\begin{proof}
Since $\lim _{z\rightarrow z_0}(z-z_0)^{k}f(z)=\alpha \ne 0$, $(z-z_0)^{k}f(z)$ has a removable singularity at $z_0$ so that 
\[(z-z_0)^{k}f(z)=\sum _{n=0}^{\infty }c_{n}(z-z_0)^{n}\]
on some $D^{*}(z_0,r)$. Thus 
\[f(z)=\sum ^{\infty }_{n=-k}c_{n+k}(z-z_0)^{n}\]
on some $D^{*}(z_0,r)$.
\end{proof}
\vspace{2ex}
\begin{cor}
If $f$ has a simple pole at $z_0$, then 
\[\mathrm{Res}_{z=z_0}f(z)=\lim _{z\rightarrow z_0}(z-z_0)f(z)\]
as 
\begin{align*}
f(z)=& a_{-1}(z-z_0)^{-1}+a_0+a_1(z-z_0)+\ldots \\
(z-z_0)f(z)=&a_{-1}+a_0(z-z_0)+a_1(z-z_0)^2+\ldots 
\end{align*}
\end{cor}
\vspace{2ex}

