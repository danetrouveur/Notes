\section{Lecture 10 (April 8th)}
\begin{thm}
(Cauchy Formula) Let $f$ be analytic in $\bar{D}=D\cup \partial D$ for a simply connected domain $D$ whose boundary is POSCC $\partial D$. Then for every $z\in D$ we have 
\[f(z)=\dfrac{1}{2\pi i}\int _{\partial D}d\xi \;\dfrac{f(\xi )}{\xi -z} \]
\end{thm}
\vspace{2ex}
\begin{ex}
Consider the following integral
\[\int _{|z-i|=1/2}dz\;\dfrac{3z^2}{(z-2)(z^2+1)}=2\pi if(i)=2\pi i\dfrac{3(i)^2}{(i-2)(2i)}\]
where
\[f(z)=\dfrac{3z^2}{(z-2)(z+i)}\]
is analytic in $|z-i|=1/2$.
\end{ex}
\vspace{2ex}
\begin{thm}
(Cauchy's differentiation formula) Consider how
\begin{align*}
\dfrac{d }{d z}(a-z)^{-1}=&(a-z)^{-2}\\
\dfrac{d }{d z}(a-z)^{-2}=&2(a-z)^{-3}\\
\dfrac{d }{d z}2(a-z)^{-3}=&6(a-z)^{-4}\\
\vdots\\
\dfrac{d ^{n}}{d z^{n}}(a-z)^{-n}=&n!(a-z)^{-(n+1)}
\end{align*}
Employing this to Cauchy's formula,
\[f^{(n)}(z)=\dfrac{n!}{2\pi i}\int _{\partial D}d\xi \;\dfrac{f(\xi )}{(\xi -z)^{n+1}} \]
where $f$ is analytic in $D$ and $z\in D$. 
\end{thm}
\vspace{2ex}
\begin{ex}
\[\int _{|z-1|=3}dz\;\dfrac{4-2\cos z}{\Big(z-\dfrac{\pi }{2}\Big)^2}=2\pi if'\Big(\dfrac{\pi }{2}\Big)=2\pi i\cdot  2\sin \dfrac{\pi }{2}=4\pi i\]
Here, $f(z)=4-2\cos z$ implies that $f'(z)=2\sin z$ and $n=1$, $\partial D\;:\;|z-1|=3$ and $z_0=\pi /2$.
\end{ex}
\vspace{2ex}
\begin{ex}
\[\int _{|z|=3}dz\;\dfrac{4z^{5}-7z^2+2}{(z-1)^3}=2\pi i\dfrac{1}{2}f''(1)\]
where $f(z)=4z^{5}-7z^2+2$. 
\end{ex}
\vspace{2ex}
\begin{rmk}
There are functions that are infinitely differentiable but not analytic (Taylor series representable).
\[f(t)=\begin{cases}
e^{-1/t}\hspace{4ex}t>0\\
0\hspace{8ex}t\leq 0
\end{cases}\]
Notice how $f^{(n)}(0)=0$ and $f\in C^{\infty }({\bm R})$ but not power series representable at 0.
\end{rmk}
\vspace{2ex}
\begin{rmk}
Notationwise, we write 
\[D(z_0,r)=\{z\in D\;|\;|z-z_0|<r\}\]
\end{rmk}
\vspace{2ex}
\begin{proof}
We prove Cauchy's differentiation formula by first showing a power series representation of the original Cauchy's formula. In this process, we will need to show uniform convergence of the series. Then, we will differentiate the power series to obtain the conclude the proof. We first state that $z\in D(z_0,r) $ and $\bar{D}(z_0,r)\subset D$. For $|z|<1$ we have 
\[\dfrac{1}{1-z}=\sum ^{\infty }_{n=0}z^{n}\]
and 
\[\dfrac{1}{\xi -z}=\dfrac{1}{\xi -z_0-(z-z_0)}=\dfrac{1}{\xi -z_0}\dfrac{1}{1-\dfrac{z-z_0}{\xi -z_0}}\]
where
\[\Big|\dfrac{z-z_0}{\xi -z_0}\Big|<1\]
becoming
\[\dfrac{1}{\xi -z}=\dfrac{1}{\xi -z_0}\sum ^{\infty }_{n=0}\Big(\dfrac{z-z_0}{\xi -z_0}\Big)^{n}=\sum ^{\infty }_{n=0}\dfrac{(z-z_0)^{n}}{(\xi -z_0)^{n+1}}\]
Then, for $f$ analytic in $D$ and $\bar{D}(z_0,r)\subset D$ and $z\in D(z_0,r)$ we have
\begin{align*}
f(z)=&\dfrac{1}{2\pi i}\int _{\partial D}\dfrac{f(\xi )}{\xi -z}\;d\xi \\
=&\dfrac{1}{2\pi i}\int _{\partial D}f(\xi )\sum ^{\infty }_{n=0}\dfrac{(z-z_0)^{n}}{(\xi -z_0)^{n+1}}\;d\xi \\
=&\dfrac{1}{2\pi i}\sum ^{\infty }_{n=0}\Big[\int _{\partial D}\dfrac{f(\xi )}{(\xi -z_0)^{n+1}}\;d\xi \Big](z-z_0)^{n}
\end{align*}
We can take out the summation due to uniform convergence. We also note that
\[f(z)=\sum ^{\infty }_{n=0}(a-a_0)^{n}\hspace{3ex}\mathrm{implies}\hspace{3ex}a_{n}=\dfrac{f^{(n)}(z_0)}{n!}\]
Two points will be later investigated: (1) that the series converges uniformly and that the summation can be taken out and (2) the implication above. 
\end{proof}
\vspace{2ex}
\begin{defi}
For $E\subset {\bm C}$ and $f_{n}:E\rightarrow  {\bm C}$ with a limit of $f:E\rightarrow {\bm C}$ we say that $f_{n}\rightarrow f$ uniformly on $E$ provided that
\[T_{n}=\sup_{z\in E}|f_{n}(z)-f(z)|\rightarrow 0\]
\end{defi}
\vspace{2ex}
\begin{thm}
If $f_{n}$ is continuous on $E$ for each $n$ and $f_{n}\rightarrow f$ uniformly on $E$, then $f$ is continuous on $E$.
\end{thm}
\vspace{2ex}
\begin{thm}
(Weierstrass M-test) For $f_{n}:E\rightarrow {\bm C}$, for each $n$, there is $M_{n}\geq 0$ such that $|f_{n}(z)|\leq M_{n}$ for all $z\in E$ and $\sum ^{\infty }_{n=1}M_{n}<\infty $ then $\sum ^{\infty }_{n=1}f_{n}$ converges uniformly on $E$.
\end{thm}
\vspace{2ex}
\begin{thm}
If $\{f_{n}\}$ is continuous on a contour $C$ such that $f_{n}\rightarrow f$ uniformly on $C$ then
\[\int _{C}f(z)\;dz=\lim _{n\rightarrow \infty }\int _{C}f_{n}(z)\;dz\]
\end{thm}
\vspace{2ex}
\begin{proof}
Let $L$ be the length of $C$. Take any $\varepsilon >0$ there is $N\in {\bm N}$ such that if $n>N$ then $|f_{n}(z)-f(z)|<\varepsilon /L$ for all $z\in {\bm C}$. Then if $n>N$, we have
\[\Big|\int _{C}f_{n}\;dz-\int _{C}f\;dz\Big|=\Big|\int _{C}(f_{n}-f)\;dz\Big|\leq\int _{C}|f_{n}-f|\;|dz|<\dfrac{\varepsilon }{L}\cdot L=\varepsilon \]
\end{proof}
\vspace{2ex}
\begin{rmk}
In our setting, $z\in D(z_0,r)$ and $\bar{D}(z_0,r)\subset D$ with
\[f(z)=\dfrac{1}{2\pi i}\int _{C}\sum ^{\infty }_{n=0}\dfrac{(z-z_0)^{n}}{(\xi -z_0)^{n+1}}f(\xi )\;d\xi \]
For fixed $z\in D(z_0,r)$,
\begin{align*}
\Big|\dfrac{z-z_0}{\xi -z_0}\Big|<&\dfrac{|z-z_0|}{r}\\
\sum ^{\infty }_{n=1}\Big|\dfrac{z-z_0}{\xi -z_0}\Big|^{n}\leq &\sum ^{\infty }_{n=1}\Big(\dfrac{|z-z_0|}{r}\Big)^{n}<\infty 
\end{align*}
for all $\xi \in \partial D$ and the series converges uniformly on $C$ with respect to $\xi $.
\end{rmk}
\vspace{2ex}
\begin{ex}
If $g$ is continuous on $D$ and
\[f(z)=\int _{\partial D}\dfrac{g(\xi )}{\xi -z}\;d\xi \]
then $f$ is power series representable for the same reason.
\end{ex}
\vspace{2ex}

