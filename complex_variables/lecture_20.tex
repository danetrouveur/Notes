\section{Lecture 20 (May 29th)}
\begin{thm}
(Argument principle) The number of times $f(C)$ winds up the origin in the positive sense is
\[\dfrac{1}{2\pi }\Delta _{C}f(z)\]
which in turn equal to the number of zeros inside $C$. 
\end{thm}
\vspace{2ex}
\begin{proof}
Let $f$ be analytic inside and on a POSCC $C$ with no zeros on $C$. If $f$ has zeros at $z_1,z_2,\ldots ,z_{n}$ inside $C$ with multiplicity $\alpha_1,\alpha_2,\ldots ,\alpha _{n}$ respectively, then 
\[f(z)=(z-z_1)^{\alpha _1}\ldots (z-z_{n})^{\alpha _{n}}F(z)\]
with $F$ being analytic with no zeros inside and on $C$. Thus on $C$,
\[\dfrac{f'(z)}{f(z)}=\sum ^{n}_{n=1}\dfrac{\alpha _{k}}{z-z_{k}}+\dfrac{F'(z)}{F(z)}\]
Where $F'(z)/F(z)$ is analytic inside and on $C$. Thus
\[\int _{C}\dfrac{f'(z)}{f(z)}\,dz=\sum ^{n}_{k=1}\int _{C}\dfrac{\alpha _{k}}{z-z_{k}}\,dz+\int _{C}\dfrac{F'(z)}{F(z)}\,dz=2\pi i\sum ^{n}_{k=1}\alpha _{k}\]
Thus 
\[\dfrac{1}{2\pi i}\int _{C}\dfrac{f'(z)}{f(z)}\,dz\]
is the number of zeros of $f$ inside $C$. On the other hand, just on $C$,
\[\int _{C}\dfrac{f'(z)}{f(z)}\,dz=\Big[\log f(z)\Big]_{C}=\Big[\ln |f(z)|+i\mathop{\mathrm{arg}}f(z)\Big]_{C}=i\Delta _{C}\mathop{\mathrm{arg}}f(z)\]
where $\Delta _{C}\mathop{\mathrm{arg}}f(z)$ is the change of $\mathop{\mathrm{arg}}f(z)$ as $z$ tranverses $C$. This implies that
\[\dfrac{1}{2\pi i}\int _{C}\dfrac{f'(z)}{f(z)}\,dz=\dfrac{1}{2\pi }\Delta _{C}\mathop{\mathrm{arg}}f(z)\]
The RHS is the number of times $f(C)$ winds up the origin in the positive sense. 
\end{proof}
\vspace{2ex}
\begin{thm}
(Rouche's theorem) Let $f$ and $g$ be analytic inside and on a POSCC $C$. If $|f(z)|>|g(z)|$ for every $z\in C$, then $f+g$ and $f$ have the same number of zeros (counting multiplicities) inside $C$. 
\end{thm}
\vspace{2ex}
\begin{proof}
On $C$, 
\[f(z)+g(z)=f(z)\Big[1+\dfrac{g(z)}{f(z)}\Big]\]
so that
\[\Delta _{C}\mathop{\mathrm{arg}}(f(z)+g(z))=\Delta _{C}\mathop{\mathrm{arg}}f(z)+\Delta _{C}\mathop{\mathrm{arg}}\Big(1+\dfrac{g(z)}{f(z)}\Big)\]
As the modulus of $g(z)/f(z)<1$, the second term vanishes, and the arguments are equal and the number of zeros are equal also. 
\end{proof}
\vspace{2ex}
\begin{recall}
(Fundamental theorem of algebra) Let $P_{n}(z)=a_{n}z^{n}+a_{n-1}z^{n-1}+\ldots +a_0=a_{n}z^{n}+P_{n-1}(z)$ where $a_{n}\ne 0$. Assume $r\geq 1$. If $|z|=r$, then
\[\Big|\dfrac{P_{n-1}(z)}{a_{n}z^{n}}\Big|=\dfrac{|P_{n-1}(z)|}{|a_{n}|r^{n}}\leq \dfrac{(|a_0|+|a_1|+\ldots +|a_{n-1}|)r^{n-1}}{|a_{n}|r^{n}}=\dfrac{|a_0|+|a_1|+\ldots +|a_{n-1}|}{|a_{n}|r}<1\]
if 
\[r>\dfrac{|a_0|+\ldots +|a_{n-1}|}{|a_{n}|}\]
We can now apply Rouche's theorem for $f(z)=a_{n}z^{n}$, $g(z)=P_{n-1}(z)$ on $C:|z|=r$, and $r>(|a_0|+\ldots )/|a_{n}|$. We see that $P_{n}(z)$ has $n$ zeros inside $|z|=r$.
\end{recall}
\vspace{2ex}
\begin{rmk}
Consider the function $f(x)=x^2$. Notice how $f((-1,1))=[0,1)$, and how an open set maps to a not open set. 
\end{rmk}
\vspace{2ex}
\begin{thm}
(Open mapping theorem) Suppose $f$ is a non-constant analytic function in a domain $D$. Then, $f(D)$ is open. 
\end{thm}
\vspace{2ex}
\begin{proof}
Take any $w_0\in f(D)$. We have to find $m>0$ such that $D(w_0,m)\subset f(D)$. By defintion of $w_0$, there is $z_0\in D$ such that $f(z_0)=w_0$. Since $D$ is open, there is $\delta >0$ such that 
\begin{itemize}
\item[(i)] $\bar{D}(z_0,\delta )\subset D$ \item[(ii)] $f(z)-w_0$ has no zeros in $0<|z-z_0|\leq \delta $ by the identity theorem
\end{itemize}
Let $m=\min _{|z-z_0|=\delta }|f(z)-w_0|>0$. Now we'll show $D(w_0,m)\subset f(D)$. Take any $w\in D(w_0,m)$. Then, $|w_0-w|<m\leq |f(z)-w_0|$ on the contour $|z-z_0|=\delta $. By Rouche's theorem, $f(z)-w_0$ and 
\[(f(z)-w_0)+(w_0-w)=f(z)-w\]
has the same number of zeros inside $|z-z_0|=\delta $. Since $(f(z)-w_0)$ has a zero in $|z-z_0|\leq \delta $ (at $z_0$), $(f(z)-w)$ has a zero inside $|z-z_0|=\delta \subset D$. In otherwords, $(f(z)-w)$ has a zero in $D$ and $w\in f(D)$. 
\end{proof}
\vspace{2ex}
\begin{ex}
Show that all roots of $z^{5}+6z^{3}+2z+10=0$ lies in $1<|z|<5$. 
\end{ex}
\vspace{2ex}
\begin{proof}
\begin{itemize}
\item[(i)] On $|z|=1$, let $f(z)=10$, $g(z)=z^{5}+6z^3+2z$. $10>|g(z)|\leq 1+6+2\leq 9$. This implies that $f+g$ has the same numbers of zeros as $10$ inside $|z|=1$. That is, it has no zeros on $|z|=1$. 
\item[(ii)] On $|z|=5$, let $f(z)=z^{5}$, $g(z)=6z^3+2z+10$. $|g(z)|\leq 6\cdot 5^3+2\cdot 5+10<5^{5}=|f(z)|$. $f$ and $f+g$ have the same number of zeros inside $|z|=5$. 
\end{itemize}
\end{proof}
\vspace{2ex}
