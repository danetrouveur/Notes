\section{Lecture 16 (May 13th)}
\begin{ex}
Consider the Laurent series of $f(z)=\pi\cot \pi z$ at $z=0$. From
\[\sin z=z-\dfrac{z^3}{3!}+\dfrac{z^{5}}{5!}+\dfrac{z^{7}}{7!}+\ldots \quad \cos z=1-\dfrac{z^2}{2!}+\dfrac{z^{4}}{4!}-\dfrac{z^{6}}{6!}+\ldots \]
we have
\begin{align*}
\pi\cot \pi z=&\pi \dfrac{\cos \pi z}{\sin \pi z}\\
=&\pi \dfrac{1-\dfrac{(\pi z)^2}{2!}+\dfrac{(\pi z)^{4}}{4!}-\ldots }{\pi z-\dfrac{(\pi z)^3}{3!}+\dfrac{(\pi z)^{5}}{5!}-\ldots }\\
=&\pi \dfrac{1}{\pi z}\dfrac{1-\dfrac{\pi ^2z^2}{2}+\dfrac{\pi^{4}z^{4}}{24}}{1-\dfrac{\pi ^2z^2}{6}+\dfrac{\pi ^{4}z^{4}}{120}}
\end{align*}
After preforming long division, we get 
\[\pi \cot \pi z=\dfrac{1}{z}\Big[1-\dfrac{\pi ^2z^2}{3}-\dfrac{\pi ^{4}z^{4}}{45}-\dfrac{2\pi ^{6}z^{6}}{945}+\ldots \Big]\]
From this, we see how
\[g(z)=\dfrac{\pi \cot \pi z}{z^2}\implies \mathop{\mathrm{Res}}_{z=0}g(z)=-\dfrac{\pi ^2}{3}\]
and
\[h(z)=\dfrac{\pi\cot \pi z}{z^{4}}\implies \mathop{\mathrm{Res}}_{z=0}h(z)=-\dfrac{\pi ^{4}}{45}\]
We can continue this indefinitely,
\[\mathop{\mathrm{Res}}_{z=0}\dfrac{\pi \cot \pi z}{z^{6}}=-\dfrac{2\pi ^{6}}{945}\]
The reason why this is important is as follows. 
\end{ex}
\vspace{2ex}
\begin{defi}
(Squares lemma)
In advanced mathematics, we often see the square contour $C_{N}$, with edges at $\pm(N+1/2)i$ and $\pm (N+1/2)$ (taking $N\in {\bm N}$). There is an upper bound $M$ ($=2$) such that 
\[| \cot \pi z|\leq M\]
for all $z\in C_{N}$ and for all $N\in {\bm N}$. 
\end{defi}
\vspace{2ex}
\begin{proof}
Take $z=x+iy$. Consider cutting the square in three parts with $y=1/2$ and $y=-1/2$. We show that ff $y>1/2$, then $|\cot \pi z|\leq 2$ and if $y<-1/2$ then $|\cot \pi z|\leq 2$. We also show that on the left and right edges of the middle cut, $|\cot \pi z|\leq 1$. Note that
\[\cot \pi z=\dfrac{\cos \pi z}{\sin \pi z}=\dfrac{\dfrac{e^{i\pi z}+e^{-i\pi z}}{2}}{\dfrac{e^{i\pi z}-e^{-i\pi z}}{2i}}\]
and that 
\[|\cot \pi z|=\Big|\dfrac{e^{i\pi z}+e^{-i\pi z}}{e^{i\pi z}-e^{-i\pi z}}\Big|=\Big|\dfrac{e^{i\pi x}e^{-\pi y}+e^{-i\pi x}e^{\pi y}}{e^{i\pi x}e^{-\pi y}-e^{i\pi x}e^{\pi y}}\Big|\]
We see that
\[|e^{i\pi z}|=|e^{i\pi (x+iy)}|=e^{-\pi y}\quad \mathrm{and}\quad |e^{-i\pi z}|=e^{\pi y}\]
assuming $y>1/2$, we automatically have $e^{\pi y}>e^{-\pi y}$ and that
\[|\cot \pi z|\leq \dfrac{e^{-\pi y}+e^{\pi y}}{e^{\pi y}-e^{-\pi y}}=\dfrac{1+e^{-2\pi y}}{1-e^{-2\pi y}}\leq \dfrac{1+e^{-\pi }}{1-e^{-\pi }}<2\]
Meanwhile, if $y<-1/2$, then
\[|\cot \pi z|\leq \dfrac{e^{-\pi y}-e^{\pi y}}{e^{-\pi y}-e^{\pi y}}=\dfrac{1+e^{2\pi y}}{1-e^{2\pi y}}\leq \dfrac{1+e^{-\pi }}{1-e^{-\pi }}<2\]
Now if $-1/2\leq y\leq 1/2$ and $z=(N+1/2)+iy$,
\[|\cot \pi z|=\Big|\cot \pi \Big(N+\dfrac{1}{2}+iy\Big)\Big|=\Big|\cot\Big(\dfrac{\pi }{2}+i\pi y\Big)\Big|=|\tan (i\pi y)|=\Big|\dfrac{e^{-\pi y}-e^{\pi y}}{e^{-\pi y}+e^{\pi y}}\Big|\leq 1\]
Also, if $-1/2\leq y \leq 1/2$ and $z=-(N+1/2)+iy$, the same thing happens and 
\[|\cot\pi z|=|\tan (i\pi y)|\leq 1\]
\end{proof}
\vspace{2ex}
\begin{ex}
Consider $g(z)=\pi \cot \pi z/z^2$ on the square contour $C_{N}$. We can use both (1) the residue theorem and (2) parameterisation. By the residue theorem,
\begin{align*}
\int _{C_{N}}\dfrac{\pi \cot \pi z}{z^2 }\,dz=&2\pi i\sum _{k=-n}^{n}\mathop{\mathrm{Res}}_{z=k}g(z)\\
=&2\pi i\Big[\mathop{\mathrm{Res}}_{z=0}g(z)+2\sum ^{N}_{k=1}\mathop{\mathrm{Res}}_{z=k}g(z)\Big]\\
=&2\pi i\Big[-\dfrac{\pi ^2}{3}+2\sum ^{N}_{k=1}\dfrac{1}{k^2}\Big]
\end{align*}
where 
\[\mathop{\mathrm{Res}}_{z=k, k\ne 0}g(z)=\lim _{z\rightarrow k}(z-k)\dfrac{\pi \cot \pi z}{z^2}=\dfrac{1}{k^2}\]
and
\[\mathop{\mathrm{Res}}_{z=0}g(z)=-\dfrac{\pi ^2}{3}\]
On the other hand,
\[\Big|\int _{C_{N}}g(z)\,dz\Big|=\Big|\int _{C_{N}}\dfrac{\pi \cot \pi z}{z^2}\Big|\leq \dfrac{M}{N^2}(8N+2)\rightarrow 0\]
with $M$ being less than $2\pi $. We thus find 
\[2\pi i\Big[2\sum ^{\infty }_{k=1}\dfrac{1}{k^2}-\dfrac{\pi ^2}{3}\Big]=0\]
and
\[\sum ^{\infty }_{k=1}\dfrac{1}{k^2}=\dfrac{\pi ^2}{6}\]
Also, we can find
\[\int _{C_{N}}\dfrac{\pi \cot \pi z}{z^{4}}\,dz=2\pi i\Big[2\sum ^{N}_{k=1}\dfrac{1}{k^{4}}-\dfrac{\pi ^{4}}{45}\Big]\]
which leads to 
\[\sum ^{\infty }_{k=1}\dfrac{1}{k^{4}}=\dfrac{\pi ^{4}}{90}\]
In this manner,
\[\xi (2n)=\sum ^{\infty }_{k=1}\dfrac{1}{k^{2n}}\]
can be found. 
\end{ex}
\vspace{2ex}
\begin{ex}
Find
\[1-\dfrac{1}{2^2}+\dfrac{1}{3^2}-\dfrac{1}{4^2}+\dfrac{1}{5^2}+\ldots =\sum ^{\infty }_{k=1}\dfrac{1}{k^2}-2\cdot \dfrac{1}{4}\sum ^{\infty }_{k=1}\dfrac{1}{k^2}=\dfrac{\pi }{6}-\dfrac{\pi }{12}=\dfrac{\pi }{12}\]
and 
\[\dfrac{1}{2^2}+\dfrac{1}{4^2}+\dfrac{1}{6^2}+\ldots =\sum ^{\infty }_{k=1}\dfrac{1}{(2k)^2}=\dfrac{1}{4}\sum ^{\infty }_{k=1}\dfrac{1}{k^2}=\dfrac{\pi }{24}\]
\end{ex}
\vspace{2ex}
\begin{ex}
Evaluate
\[\sum ^{\infty }_{k=-\infty }\dfrac{1}{(n+a)^2}\]
for $a\notin {\bm Z}$
\end{ex}
\vspace{2ex}
\begin{proof}
Define, on $C_{N}$
\[f(z)=\dfrac{\pi \cot\pi z}{(z+a)^2}\]
on $C_{N}$. Then
\[0=2\pi i\Big[\sum ^{\infty }_{n=-\infty }\mathop{\mathrm{Res}}_{z=n}f(z)+\mathop{\mathrm{Res}}_{z=-a}f(z)\Big]\]
where
\[\mathop{\mathrm{Res}}_{z=-a}f(z)=\lim _{z\rightarrow -a}[\pi \cot \pi z]'=-\pi ^2\csc^2\pi a\]
Noting that $f(z)$ has a pole of order 2, we then have
\[\sum ^{\infty }_{n=-\infty }\dfrac{1}{(n+a)^2}=\Big(\dfrac{\pi }{\sin \pi a}\Big)^2\]
\end{proof}
\vspace{2ex}


