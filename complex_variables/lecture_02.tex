\section{Lecture 2 (March 6th)}
We have learnt last class that when $z_1,z_2\ne 0$, we express these complex numbers as $z_1=r_1e^{\theta_2},z_2=r_2e^{\theta_2}$ and that $z_1z_2=r_1e^{i\theta_1}r_2e^{i\theta_2}=r_1r_2e^{i(\theta_1+\theta_2)}$. We know that if $z=ere^{i\theta }$ then $|z|=r$ and $\mathrm{arg}\;z=\theta +2n\pi $ for $n\in {\bm Z}$. Therefore we can note that $\mathrm{arg}\;z_1z_2=\theta_1+\theta_2+2n\pi =\mathrm{arg}\;z_1+\mathrm{arg}\;z_2$. In other words, we know that the argument turns multiplication into addition. 
\\
\begin{ex}
Lets say that $n\in {\bm N}$, $z_{0}\ne 0$ and that $e$. How do you find all $z\in {\bm C}$ satisfying $z^{n}=z_0$ (where such $z$ would be denoted as $z_0^{1/n}$)? The conclusion is that for every $z_0\ne 0$, $z^{n}_{0}$ has exactly one value for every integer. What if the exponent is not necessarily a integer? 
\[\begin{cases}
z_{0}^{n}\;\;\mathrm{has\; exactly\; one\; value}\\
z^{r}_{0}\;\;\mathrm{has}\;m\;\mathrm{values}\\
z_{0}^{q}\;\;\mathrm{has\; infinitely\; many\; values}
\end{cases}\]

If the exponent is a quotient $r=n/m$ which is reducible, it would have exactly $m$ roots. On the other hand, if the exponent is in $q\in {\bm R}\,\backslash\,{\bm Q}$, there would be an infinite number of roots. In all cases, there would be a single root if we impose that $-\pi <\mathrm{arg}\;z<\pi $. Note that all cases do not work for the case that the exponent is $e$.
\end{ex}
\vspace{2ex}
\begin{proof}
First denote $z=re^{i\theta }$ and $z_0=r_0e^{i\theta_0}$. Then $z^{n}=z_0$ means that $r^{n}e^{in \theta }=r_0e^{i \theta_0}$ and therefore $r^{n}=r_0$ and $n\theta =\theta_0+2k\pi $ (for $k\in {\bm Z}$) such that $r=r_0^{1/n}$ and $\theta =\theta_0/n+2k\pi /n$. In sum, we would have
\[z=re^{i\theta }=\sqrt[n]{r_0}\exp\Big[i\Big(\dfrac{\theta_0}{n}+\dfrac{2k\pi }{n}\Big)\Big]=\sqrt[n]{r_0}\exp\Big(i\dfrac{\theta }{n}\Big)\exp\Big(i\dfrac{2k\pi }{n}\Big)\]
this number is indeed finite from the fact that the later term has $n$ possible values for $k\in {\bm Z}$ (for $k=0,\ldots ,n-1$ and the other values would be redundant).
\end{proof}
\vspace{2ex}
\begin{ex}
Find all $z$ satisfying $z^3=-8i$. Then,
\[-8i=8e^{-i\pi /2}\]
Using the formula above for $k=0,1,2$, 
\[z=\sqrt[3]{8}\exp\Big[i\Big(\dfrac{-\pi /2}{3}+\dfrac{2k\pi }{3}\Big)\Big]\]
when $k=0$, 
\[z=2e^{-\pi i/6}=2\Big(\cos\Big(-\dfrac{\pi }{6}\Big)+i\sin \Big(-\dfrac{\pi }{6}\Big) \Big)=\sqrt{3}-i\]
for $k=1$,
\[z=2e^{i\pi /2}=2\Big(\cos \dfrac{\pi }{2}+i\sin \dfrac{\pi }{2}\Big)=2i\]
and for $k=2$,
\[r=2e^{i7\pi /6}=2\Big(\cos \dfrac{7\pi }{6}+i\sin\dfrac{7\pi }{6} \Big)=-\sqrt{3}-i\]
\end{ex}
\vspace{2ex}
We now observe many theorems that follow from properties of ${\bm R}^2$ onto ${\bm C}$.
\\
\begin{defi}
In the complex plane, $G\subset {\bm C}$ is called open provided that for every $z_0\in G$, there is a $\delta >0$ such that $D(z_0,\delta )\subset G$. On the other hand, $F\subset {\bm C}$ is closed if $F^{C}={\bm C}\;\backslash\;F$ is open. Lastly, $K\subset {\bm C}$ is called compact provided that every open cover of $K$ has a finite subcover.
\end{defi}
\vspace{2ex}
\begin{thm}
	(Heine-Borel) Satisfied in ${\bm R}^2$, the Heine-Borel theorem works for the complex plane which states $K\subset {\bm C}$ is compact if and only if $K$ is closed and bounded.
\end{thm}
\vspace{2ex}
\begin{defi}
A sequence $\{z_{n}\}$ in ${\bm C}$ is said to converge to $\alpha $ provided that for every $\varepsilon >0$, there exists an $N\in {\bm N}$ such that if $n>N$, $|z_{n}-\alpha |<\varepsilon $. This is equivalent to saying that any given disk around $\alpha $ contains all but a finite number of $z_{n}$. If this is true, we denote this as
\[\lim_{n\rightarrow \infty }z_{n}=\alpha \]
\end{defi}
\vspace{2ex}
\begin{defi}
	Let $z_{n}=x_{n}+iy_{n}$ and $\alpha =a+ib$. $\{z_{n}\}$ converges to $\alpha $ if  and only if $\{x_{n}\}$ converges to $a$ and $\{y_{n}\}$ converges to $b$.
\end{defi}
\vspace{2ex}
\begin{defi}
$\{z_{n}\}$ is called a Cauchy sequence provided that for every $\varepsilon >0$ there is $N\in {\bm N}$ such that if $n,m>N$ then $|z_{n}-z_{m}|<\varepsilon $.
\end{defi}
\vspace{2ex}
\begin{thm}
We say that a sequence $\{z_{n}\}$ converges if and only if it is a Cauchy sequence.
\end{thm}
\vspace{2ex}
\begin{rmk}
We know that 
\[C(I)\supset D(I)\supset C^{1}(D)\supset \ldots \supset C^{\infty }\supset \mathrm{Analytic} \]
however, for complex functions, we have
\[C(G)\supset D(G)=\mathrm{Analytic}=\mathrm{Polynomial}\]
\end{rmk}
We at this point make three quick definitions.
\\
\begin{defi}
\begin{itemize}
\item[(i)] A set $G\subset {\bm C}$ is called {\it polygonally connected} provided that if $z_1,z_2\in G$ then there is a polygonal line in $G$ connecting $z_1$ and $z_2$. 
\item[(ii)] Also, $G\subset {\bm C}$ is called {\it simply connected} if every closed curve in $G$ contains points of $G$ only (that is, $G$ has no holes)
\item[(iii)] If $G$ is open and connected, then $G$ is called a {\it domain}.
\end{itemize}	
\end{defi}
\vspace{2ex}


\vspace{2ex}




