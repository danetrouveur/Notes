\section{Lecture 3 (March 11th)}
\begin{defi}
Let $D$ be a domain (open and connected set). A function is normally defined as $f:D\rightarrow {\bm C}$. Then for $u=\mathrm{Re}\;f$ and $v=\mathrm{Im}\;f$, 
\[f(x+iy)=u(x,y)+iv(x,y)\]
and $u,v:D\rightarrow {\bm R}$. We say that $\lim _{z\rightarrow z_0}f(z)=w_{0}$ if and only if for all $\varepsilon >0$ there exists a $\delta >0$ such that if $z\in D^{*}(z_{0},\delta )$ (a deleted neighborhood of $0<|z-z_0|<\delta $) then $f(z)\in D(w_0,\varepsilon )$ ($|f(z)-w_0|<\varepsilon $). 

If $f(z)=u(x,y)+iv(x,y)$ where $z=x+iy$ and $w_0=u_0+iv_0$ then
\[f(z)-w_0=[u(x,y)-u_0]+i[v(x,y)-v_0]\]
and
\[|f(z)-w_0|=\sqrt{(u(x,y)-u_0)^2+(v(x,y)-v_0)^2}\]
so that $\lim _{z\rightarrow z_0}f(z)=w_0$ if and only if $\lim _{(x,y)\rightarrow (x_0,y_0)}u(x,y)=u_0$ and $\lim _{(x,y)\rightarrow (x_0,y_0)}v(x,y)=v_0$. 
\end{defi}
\vspace{2ex}
\begin{thm}
$f(z)$ be continuous at $z_0$ if and only if $u(x,y)$ and $v(x,y)$ are continuous at $(x_0,y_0)$.
\end{thm}
\vspace{2ex}
\begin{defi}
If $u$ and $v$ belongs to $C^{1}(D)$ ($u,v$ and their partial derivatives are all continuous). Then we say that $f\in C^{1}(D)$. We note that this assumption is very weak. In advanced calculus, we defined differentiability as the following. If $u\in C^{1}(D)$ and $(x_0,y_0)\in D$, then 
\[
u(x,y)-u(x_0,y_0)=u_{x}(x_0,y_0)(x-x_0)+u_{y}(x_0,y_0)(y-y_0)+\varepsilon_1(x,y)(x-x_0)+\varepsilon _{2}(x,y)(y-y_0)\] 
for $(x,y)$ in a neighborhood of $(x_0,y_0)$ where $\lim _{(x,y)\rightarrow (x_0,y_0)}\varepsilon_1=\lim _{(x,y)\rightarrow (x_0,y_0)}\varepsilon_2=0$. This implies that the following expression is possible
\[\Delta u=u_{x}+u_{y}\Delta y+\varepsilon _{1}\Delta x+\varepsilon _{2}\Delta y\]
where $\varepsilon_1\rightarrow 0$ and $\varepsilon _{2}\rightarrow 0$ as $(\Delta x,\Delta y)\rightarrow (0,0)$. 
\end{defi}
\vspace{2ex}
\begin{ex}
\begin{itemize}

	\item[] If $f(z)=z^2$ then $u(x,y)=x^2-y^2$ and $v(x,y)=2xy$. 
	\item[] When $f(z)=\bar{z}=x-iy$ then $u(x,y)=x$ and $v(x,y)=-y$ 
	\item[] When $f(z)=|z|^2=z\bar{z}=x^2+y^2$ then $u(x,y)=x^2+y^2$ and $v(x,y)=0$. \end{itemize}
For these functions, $f\in C^{\infty }(D)$.
\end{ex}
\vspace{2ex}
\begin{defi}
We define
\[f'(z_0)=\lim _{z\rightarrow z_0}\dfrac{f(z)-f(z_0)}{z-z_0}\]
if it exists. In this case, we call $f'(z_0)$ the complex derivative of $f$ at $z_0$. If $D$ is open and $f'(z_0)$ exists for all $z_0\in D$, then we say $f$ is differentiable on $D$. From this definition, the product rule, quotient rule, and chain rule all exist and are defined in the same way as in calculus 1.
\end{defi}
\vspace{2ex}
\begin{ex}
Where is $f(z)=\bar{z}$ differentiable?
\[\lim_{h\rightarrow 0}\dfrac{f(z+h)-f(z)}{h}=\lim _{h\rightarrow 0}\dfrac{\overline{z+h}-\bar{z}}{h}=\lim _{h\rightarrow 0} \dfrac{\bar{h}}{h}\]
The above limit corresponds to
\[\begin{cases}
1\hspace{5ex}\mathrm{if}\;h\in {\bm R}\\
-1\hspace{5ex}\mathrm{if}\;h\in i{\bm R}
\end{cases}\]
where $i{\bm R}$ denotes pure imaginary. The function is differentiable nowhere!
\end{ex}
\vspace{2ex}
\begin{ex}
Where is $f(z)=|z|^2=z\bar{z}$ differentiable?
\begin{align*}
	\lim_{h\rightarrow 0}\dfrac{f(z+h)-f(z)}{h}=&\lim _{h\rightarrow 0}\dfrac{|z+h|^2-|z|^2}{h}\\=&\lim _{h\rightarrow 0}\dfrac{(z+h)(\bar{z}+\bar{h})-z\bar{z}}{h}\\=&\lim _{h\rightarrow 0}\Big(\bar{z}+z\dfrac{\bar{h}}{h}+\bar{h}\Big) 
\end{align*}
Notice how this limit exists only when $z=0$ and $f'(0)=0$!
\end{ex}
\vspace{2ex}
In this way, our prior notion of differentiability doesn't really work alone. We again emphasize that this is because the limit differs by direction. Let's try this differently.
\\
\begin{defi}
$f$ is analytic at $z_0$ if there is $\delta >0$ such that $f$ is differentiable on $D(z_0,\delta )$. If $f$ is analytic at every $z_0\in D$ then we say that $f$ is entire on $D$. We make this definition because if $D\subset {\bm C}$ is open then $f$ is differentiable on $D$ if and only if $f$ is analytic on $D$. Therefore, $f(z)=\bar{z}$ is nowhere differentiable and $f(z)=|z|^2$ is differentiable at $0$ but analytic nowhere. 
\end{defi}
\vspace{2ex}
\begin{thm}
\begin{itemize}
	\item[(i)] If $f$ is continuous on a connected set $D$ then $f(D)$ is connected. 
	\item[(ii)] If $f$ is continuous on a compact set $K$ then $f(D)$ is compact
	\item[(iii)] If $f$ is continuous on a compact set $K$ then $|f|$ takes a maximum and minimum on $K$. 
\end{itemize}
\end{thm}
\vspace{2ex}
\begin{thm}
(Cauchy-Riemann equation) The conclusion is that if $f$ is differentiable in an open set $D$ such that $f(x+iy)=u(x,y)+iv(x,y)$ then $u_{x}=v_{y}$ and $v_{x}=-u_{y}$ on $D$. The converse is also true if $f\in C^{1}(D)$. $f$ is analytic $\iff$ $u_{x}=v_{y}$ and $v_{x}=-u_{y}$. Notice how from the left to the right, we need nothing but coming back we need $C^{1}(D)$. 
\end{thm}
\vspace{2ex}
\begin{proof}
Suppose that $f$ is differentiable at $z_0=x_0+iy_0$. Then for $h,k\in {\bm R}$,
\[f'(z_0)=\lim _{h\rightarrow 0}\dfrac{f(z_0)-f(z_0)}{h}=\lim _{k\rightarrow 0}\dfrac{f(z_0+ik)-f(z_0)}{ik}\]
where 
\begin{align*}
	f(z_0+h)-f(z_0)=&u(x_0+h,y_0)+iv(x_0+h,y_0)-u(x_0,y_0)-iv(x_0,y_0)\\
	=&u(x_0+h,y_0)-u(x_0,y_0)+i[v(x_0+h,y_0)-v(x_0,y_0)]\\
	f'(z_0)=&\dfrac{u(x_0+h,y_0)-u(x_0,y_0)}{h}+i\dfrac{v(x_0+h,y_0)-v(x_0,y_0)}{h}\\
=& u_{x}(x_0,y_0)+iv_{x}(x_0,y_0)
\end{align*}
that is, $f'(z_0)=u_{x}(x_0,y_0)+iv_{x}(x_0,y_0)$.
\end{proof}
\vspace{2ex}

