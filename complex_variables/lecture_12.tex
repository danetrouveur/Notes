\section{Lecture 12 (April 15th)}
\begin{thm}
In a disk $D(a,R)$, consider the power series 
\[f(z)=\sum ^{\infty }_{n=0}c_{n}(z-a)^{n}\]
If there exists a sequence $\{a_{n}\}\rightarrow a$ with $f(a_{n})=0$ for all $n\in {\bm N}$, $f(z)=0$ for all $z\in D(a,R)$
\end{thm}
\vspace{2ex}
\begin{proof}
Let $f(z)=c_0+c_1(z-a)+c_2(z-a)^2+\ldots $. By putting $z=a_{n}$ we have
\[0=f(a_{n})=c_0+c_1(a_{n}-a)+\ldots \]
Taking $n\rightarrow \infty $, we get $c_0=0$. Then,
\[f(z)=(z-a)(c_1+c_2(z-a)+c_3(z-a)^2+\ldots )\]
and
\[\dfrac{f(z)}{z-a}=c_1+c_2(z-a)+c_3(z-a)^2\ldots \]
now, taking $z=a_{n}$, then
\[0=\dfrac{f(a_{n})}{a_{n}-a}=c_1+(a_{n}-a)c_2+\ldots \]
taking $n\rightarrow  \infty $, we then get $c_1=0$. As $c_0=c_1=\ldots =c_{k}=0$,
\[f(z)=(z-a)^{k+1}(c_{k+1}+c_{k+2}(z-a)+c_{k+3}(z-a)^2+\ldots )\]
so that
\[\dfrac{f(z)}{(z-a)^{k+1}}=c_{k+1}+(z-a)\Big[c_{k+2}+\ldots \Big]\]
put $z=a_{n}$ then take $n\rightarrow \infty $ we get $c_{k+1}=0$. Therefore, $c_{n}=0$ for all $n=0,1,2,\ldots $.
\end{proof}
\vspace{2ex}
\begin{thm}
If $f$ is analytic in a domain $D$ such that there is a sequence of distinct points $\{a_{n}\}$ in $D$ with $a_{n}\rightarrow a\in D$ and $f(a_{n})=0$ for every $n\in {\bm N}$, then $f(z)=0$ for all $z\in D$. Remark that this time, we converge to an arbitrary point in the disk and the sequence is in a domain, not a disk.
\end{thm}
\vspace{2ex}
\begin{proof}
We can use the overlapping disk technique. In the domain, we know that the function is zero on a disk with center $a$. Take any $b\in D$ with $a\ne b$. Then there is a polygonal line segment $L$ from $a$ to $b$. Notice that $L$ is compact, and $d(L,\partial D)=r>0$. Now create an overlapping disk with radius $\delta $ from $a$ to $b$ such that every disk contains the center of the previous disk. As there exists distinct points that converge to a point whose function values are all zero, all disks have $f(z)=0$. 
\end{proof}
\vspace{2ex}
\begin{defi}
(Identity theorem) Let $f$ and $g$ be analytic in a domain $D$. If there is a sequence $\{z_{n}\}$ which has a limit point in $D$ such that $f(z_{n})=g(z_{n})$ for every $n\in {\bm N}$, then $f(z)=g(z)$ for all $z\in D$.
\end{defi}
\vspace{2ex}
\begin{ex}
Notice how $f(z)=e^{i/1-z}$ is analytic in $|z|<1$ ($D(0,1)$).
\[z_{n}=1-\dfrac{1}{2n\pi }\]
satisfies $|z_{n}|<1$, and is a sequence in $D$ with $f(z_{n})=1$ for all $n\in {\bm N}$. The limit point of $\{z_n\}$ is $1\notin D(0,1)$, telling us that the identity theorem cannot be used. 
\end{ex}
\vspace{2ex}
\begin{ex}
Let $f$ and $g$ be analytic in a domain $D$ such that $f(z)g(z)= 0$ for all $z\in D$. Show that $f(z)=0$ for all $z\in D$ or $g(z)=0$ for all  $z\in D$. 
\end{ex}
\vspace{2ex}
\begin{proof}
If $f(z_0)\ne 0$ then there is $\delta >0$ such that $f(z)\ne 0$ for any $z\in D(z_0,\delta )$. Thus $g(z)=0$ for all $z\in D(z_0,\delta )$. We can now apply the identity theorem.
\end{proof}
\vspace{2ex}
\begin{thm}
If $f$ is analytic in $D$ such that $\bar{D}(z_0,r)\subset D$, then by the Cauchy integral formula
\[f(z_0)=\dfrac{1}{2\pi i}\int _{|z-z_0|=r}\dfrac{f(z)}{z-z_0}\;dz\]
Then, $|z-z_0|=r$ is a positively oriented simple curve, which we can parametrise like the following.
\[\dfrac{1}{2\pi i}\int ^{2\pi }_{0}\dfrac{f(z_0+re^{it})}{re^{it}}ire^{it}\;dt=\dfrac{1}{2\pi }\int ^{2\pi }_{0}f(z_0+re^{it })\;dt\]
The latter is the expression for the average value of $f$ on $|z-z_0|=r$.
\end{thm}
\vspace{2ex}
\begin{thm}
If $f(z)=\sum ^{\infty }_{n=0}c_{n}(z-a)^{n}$ in $D(a,R)$ converges, then for every $0<r<R$,
\[\dfrac{1}{2\pi }\int ^{2\pi }_{0}|f(a+re^{it})|^2\;dt=\sum ^{\infty }_{n=0}|c_{n}|^2r^{2n}\]
\end{thm}
\vspace{2ex}
\begin{proof}
\[f(a+re^{it})=\sum ^{\infty }_{n=0}c_{n}(re^{it})^{n}=\sum ^{\infty }_{n=0}c_{n}r^{n}e^{int}\]
This series converges uniformly on $[0,2\pi ]$ (since $r<R$). Therefore,
\begin{align*}
\dfrac{1}{2\pi }\int ^{2\pi }_{0}|f(a+re^{it})|^2\;dt=&\dfrac{1}{2\pi }\int ^{2\pi }_{0}\sum ^{\infty }_{n=0}c_{n}r^{n}e^{int }\sum ^{\infty }_{m=0}\bar{c}_{m}r^{m}e^{-imt}\;dt\\
=&\dfrac{1}{2\pi }\sum ^{\infty }_{n=0}\sum ^{\infty }_{m=0}\int^{2\pi }_{0}c_{n}\bar{c}_{m}r^{n+m}e^{i(n-m)t}\;dt\\
=&\sum ^{\infty }_{n=0}\sum ^{\infty }_{m=0}c_{n}\bar{c}_{m}r^{n+m}\dfrac{1}{2\pi }\int ^{2\pi }_{0}e^{i(n-m)t}\;dt
\end{align*}
where the far right integral is $1$ only when $n=m$ and otherwise 0.
\end{proof}
\vspace{2ex}
\begin{thm}
(Maximum modulus theorem I) Let $f$ be analytic in a domain $D$. Suppose that $|f(z_0)|$ is a local maximum for some $z_0\in D$, then $f$ is constant in $D$. 
\end{thm}
\vspace{2ex}
\begin{proof}
Suppose $\sup _{t\in [0,2\pi ]}|f(z_0+re^{it})|\leq |f(z_0)|$ for some $r>0$. If $f(z)=\sum ^{\infty }_{n=0}c_{n}(z-z_0)^{n}$ in $\bar{D}(z_0,r)$ then
\[\sum ^{\infty }_{n=0}|c_{n}|^2r^{2n}=\dfrac{1}{2\pi }\int ^{2\pi }_{0}|f(z_0+re^{it})|^2\;dt\leq \dfrac{1}{2\pi }\int^{2\pi }_{0}|f(z_0)|^2\;dt=|c_0|^2\]
which means that $c_{n}=0$ for all $n\geq 1$. By the identity theorem, $f$ is constant in $D$. 
\end{proof}
\vspace{2ex}
\begin{thm}
(Maximum modulus theorem II) If $f$ is analytic inside and on $a$ bounded domain $D$, then $|f|$ attains its maximum on $\partial D$. 
\end{thm}
\vspace{2ex}
\begin{proof}
$|f|$ is a real valued continuous on a compact set $\bar{D}$ so that $|f|$ has its maximum in $\bar{D}$. But $|f|$ cannot take a maximum in $D$. 
\end{proof}
\vspace{2ex}
\begin{thm}
(Fundamental theorem of algebra) A complex polynomial
\[p_{n}(z)=a_{n}z^{n}+a_{n-1}z^{n-1}+\ldots +a_0\]
with $a_{n}\ne 0$ has a zero in ${\bm C}$.
\end{thm}
\vspace{2ex}
\begin{proof}
Suppose that there is no $z_0\in {\bm C}$ such that $p(z_0)= 0$ then $f(z)=1/p(z)$ is a entire function with $\lim _{|z|\rightarrow \infty }|f(z)|=0$. Take $R>0$ such that $|f(z)|<|f(0)|$ for $|z|=R$. This contradicts the maximum modulus theorem. 
\end{proof}
\vspace{2ex}
\begin{ex}
Suppose $f$ is entire with $f(0)=1$. Let $|f(z)|\leq |e^{z}|$ for all $z\in {\bm C}$. What is $f$?
\end{ex}
\vspace{2ex}
\begin{proof}
Let $g(z)=f(z)/e^{z}$. As this function is entire, $|g|\leq 1$. 
\end{proof}
\vspace{2ex}

