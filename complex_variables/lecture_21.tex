\section{Lecture 21 (June 5th)}
\begin{recall}
A $C^2$ function $u:D\rightarrow {\bm R}$ is harmonic if $\Delta u=u_{xx}+u_{yy}=0$ on $D$. 
\end{recall}
\vspace{2ex}
\begin{recall}
(Cauchy) If $f$ is analytic in a simply connected domain $D$ then
\[\int _{C}f(z)\,dz=0\]
for every closed contour $C$ in $D$. This is due to the fact that for every fixed $z_0\in D$, 
\[F(z)=\int _{z_0}^{z}f(\xi )\,d\xi \]
is well-defined and is independent of path.
\end{recall}
\vspace{2ex}
\begin{thm}
If $f$ is analytic in a simply connected domain $D$, then $f=F'$ for some $F$ that is analytic in $D$. 
\end{thm}
\vspace{2ex}
\begin{proof}
Take any $z_0\in D$, and we'll show that $F(z)=\int ^{z}_{z_0}f(\xi )\,d\xi $ satisfies $F'=f$ on $D$. Take any $w\in D$. We'll show for each $\varepsilon >0$, there is $\delta >0$ such that if $0<|h|<\delta $ then 
\[\Big|\dfrac{F(w+h)-F(w)}{h}-f(w)\Big|<\varepsilon \]
Since $D$ is open, there is $\delta_1 >0$ such that $\bar{D}(w,\delta_1)\subset D$. Since $f$ is continuous at $w$, there is $\delta_2>0$ such that if $|\xi -w|<\delta_2$ then $|f(\xi )-f(w)|<\varepsilon$. Take $\delta =\mathop{\mathrm{min}}(\delta_1,\delta _2)$. Then we'll complete the proof. If $0<|h|<\delta $ then 
\[\dfrac{F(w+h)-F(w)}{h}=\dfrac{1}{h}\Big[\int ^{w+h}_{z_0}f(\xi )\,d\xi -\int ^{w}_{z_0}f(\xi )\,d\xi \Big]\]
so that if $0<|h|<\delta $, then
\[\Big|\dfrac{F(w+h)-F(w)}{h}-f(w)\Big|=\Big|\dfrac{1}{h}\int ^{w+h}_{w}(f(\xi )-f(w))\,d\xi \Big|<\varepsilon \]
As the path can be made a straight line, we can reduce the right handside smaller than the length of the path and the integeral can be reduced to be smaller than $\varepsilon $.
\end{proof}
\vspace{2ex}
\begin{thm}
If $u$ is harmonic in a simply connected domain $D$, then there is a function $f$ that is analytic in $D$ such that $u=\mathop{\mathrm{Re}}(f)$ on $D$. 
\end{thm}
\vspace{2ex}
\begin{proof}
Define $g=u_{x}-iu_{y}$ in $D$, and $g$ is analytic on $D$ since it satisfies the Cauchy-Riemann equation. Since $D$ is simply connected, there is $F$ such that $F'=g$ on $D$. Suppose that $F=U+iV$, and $F'=U_{x}+iV_{x}=U_{x}-iV_{y}=g=u_{x}+iu_{y}$ due to the Cauchy Riemann equation. Thus $U_{x}=u_{x}$ and $U_{y}=u_{y}$ in $D$. We see that $U=u+c$ for some $c\in {\bm R}$. Hence $u=\mathop{\mathrm{Re}}[F-c]$.
\end{proof}
\vspace{2ex}
\begin{ex}
Let $u(x,y)=\ln \sqrt{x^2+y^2}=\ln |z|$. Then $f(x)=\log z$ is analytic in ${\bm C}\,\backslash\,(-\infty ,0]$. Additionally, $\mathop{\mathrm{Re}}f(z)=\ln |z|=u$. However, there is no $g$ analytic in $0<|z|<1$ such that $u=\mathop{\mathrm{Re}}(g)$. 
\end{ex}
\vspace{2ex}
\begin{rmk}
If $u$ and $v$ are harmonic in $D$, then for every $a,b\in {\bm R}$, $au+bv$ is harmonic.
\end{rmk}
\vspace{2ex}
\begin{ex}
If $u$ is non-constant harmonic in a domain $D$ and $v=u^2$, $v_{x}=2uu_{x}$ such that $v_{xx}=2u_{x}^2+2uu_{xx}$. Therefore, 
\[\Delta v=v_{xx}+v_{yy}=2u(u_{xx}+u_{yy})+2(u_{x}^2+u_{y}^2)=2(u_{x}^2+u_{y}^2)>0\]
We therefore find $u^2$ to be never harmonic.
\end{ex}
\vspace{2ex}
\begin{rmk}
If $f$ is analytic and $u$ is harmonic, then $u\circ f$ is harmonic. The key point is that $u$ is locally the real part of an analytic function such that $u\circ f=\mathop{\mathrm{Re}}(g\circ f)$. 
\end{rmk}
\vspace{2ex}
\begin{thm}
If $u$ is bounded and harmonic in ${\bm C}$, then $u$ is constant. 
\end{thm}
\vspace{2ex}
\begin{proof}
Since ${\bm C}$ is simply connected, there exists an entire function $f$ such that $u=\mathop{\mathrm{Re}}(f)$. Then, $|\exp f(z)|=\exp u(z)$ and $\exp f(z)$ is a bounded entire function. We therefore find that $\exp f(z)$ is constant and that $f(z)$ is constant. 
\end{proof}
\vspace{2ex}
\begin{thm}
If $u$ is a positive harmonic function on ${\bm C}$, then $u$ is constant. 
\end{thm}
\vspace{2ex}
\begin{proof}
There is an entire function $f$ such that $u=\mathop{\mathrm{Re}}(f)$. Thus $|\exp f(z)|=\exp (u)\geq 1$. We find $g(z)=1/\exp f(z)$ to be a bounded entire function which implies that $g$ is constant. Then, $f$ is constant and $u$ is constant.
\end{proof}
\vspace{2ex}
\begin{thm}
If $u$ is harmonic in a domain $D$, then $u$ can not take a maximum in $D$. 
\end{thm}
\vspace{2ex}
\begin{proof}
Suppose that you take a maximum at $z_0=x_0+iy_0$ in $D$. Then, there is $r>0$ such that $\bar{D}(z_0,r)\subset D$. Since $D(z_0,r)$ is simply connected, there is $f$ that is analytic in $D(z_0,r)$ such that $u=\mathop{\mathrm{Re}}f$ on $D(z_0,r)$. Then $|\exp f(z_0)|=\exp u(z_0)=$ is a maximum in $D(z_0,r)$. Therefore, $\exp (f)$ is constant. 
\end{proof}
\vspace{2ex}
\begin{thm}
(Mean value property of harmonic functions) If $f$ is analytic in $D$ such that $\bar{D}(z_0,R)\subset D$, then 
\[f(z_0)=\dfrac{1}{2\pi }\int ^{2\pi }_{0}f(z_0+re^{it})\,dt\]
for every $0<r\leq R$. If $u$ is harmonic in a domain containing $\bar{D}(a,R)$, then $u=\mathop{\mathrm{Re}}(f)$ in $\bar{D}(z_0,R)$ resulting in
\[u(z_0)=\mathop{\mathrm{Re}}f(z_0)=\mathop{\mathrm{Re}}\Big[\dfrac{1}{2\pi }\int ^{2\pi }_{0}f(z_0+re^{it})\,dt\Big]=\dfrac{1}{2\pi }\int ^{2\pi }_{0}u(z_0+re^{it})\,dt \]
for $r\leq R$. We then have, by extension,
\begin{align*}
u(z_0)=&\dfrac{1}{2\pi }\int ^{2\pi }_{0}u(z_0+re^{it})\,dt\\
\int ^{R}_{0}u(z_0)r\,dr=&\dfrac{1}{2\pi }\int ^{2\pi }_{0}\int^{R}_{0}u(z_0+re^{it})r\,drdt\\
\dfrac{R^2}{2}u(z_0)=&\dfrac{1}{2\pi }\iint _{D(z_0,R)}u(x,y)\,dA\\
u(z_0)=&\dfrac{1}{\pi R^2}\iint _{D(z_0,R)}u\,dA
\end{align*}
and the mean value theorem works on a disk also.
\end{thm}
\vspace{2ex}
\begin{thm}
If $u$ is bounded harmonic in ${\bm C}$, then $u$ is constant.
\end{thm}
\vspace{2ex}
\begin{proof}
Suppose that $|u(z)|\leq M$. Then, for each $z_0\in {\bm C}$, 
\begin{align*}
u(z_0)-u(0)=&\dfrac{1}{\pi R^2}\iint_{D(z_0,R)}u\,dA-\dfrac{1}{\pi R^2}\iint_{D(0,R)}u\,dA\\
\leq &\dfrac{M}{\pi R^2}(\mathrm{symmetric\ difference\ between\ two\ discs}) 
\end{align*}
\end{proof}
\vspace{2ex}

