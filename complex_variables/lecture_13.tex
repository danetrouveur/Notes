\section{Lecture 13 (April 29th)}
\begin{prop}
We shall often use the following power series representations.
\begin{align*}
e^z=\sum ^{\infty }_{n=0}\dfrac{z^{n}}{n!}\qquad
\sin z=&\sum ^{\infty }_{n=0}(-1)^{n}\dfrac{z^{2n+1}}{(2n+1)!}\qquad 
\cos z=\sum ^{\infty }_{n=0}(-1)^{n}\dfrac{z^{2n}}{(2n)!}
\end{align*}
where $z\in {\bm C}$. An effective way to create a power series representation is to substitute 
\[\dfrac{1}{1-z}=\sum ^{\infty }_{n=0}z^{n}\]
for $z\in D(0,1)$. 
\end{prop}
\vspace{2ex}
\begin{ex}
Power series expansions allow us to calculate integrals on the complex plane. For example, consider
\[e^{1/z}=\sum ^{\infty }_{n=0}\dfrac{z^{-n}}{n!}\]
for $z\ne 0$. Then, 
\[\int _{|z|=1}e^{1/z}\,dz=\sum ^{\infty }_{n=0}\int _{|z|=1}\dfrac{1}{n!z^{n}}\,dz=\int _{|z|=1}\dfrac{1}{z}\,dz=2\pi i\]
as each term is equal to 0 if $n\ne 1$.
\end{ex}
\vspace{2ex}
\begin{thm}
(Laurent theorem) Let $f$ be analytic in a multiply connected domain of the form $D=\{z\in {\bm C} \;|\; r<|z-z_0|<R\}$. Also, let $C$ be a POSCC in $D$ such that $z_0$ is inside $C$. Then,
\[f(z)=\sum ^{\infty }_{n=-\infty }a_{n}(z-z_0)^{n}\]
where 
\[a_{n}=\dfrac{1}{2\pi i}\int _{C}\dfrac{f(z)}{(z-z_0)^{n+1}}\,dz\]
In particular, when $n=-1$,
\[a_{-1}=\dfrac{1}{2\pi i}\int _{C}f(z)\,dz\]
We will use $a_{-1}$ to find the integral of $f(z)$ on $C$ in the future. 
\end{thm}
\vspace{2ex}
\begin{proof}
There is $R_1<R_2$ such that $R_1\leq |z-z_0|\leq R_2$ contains $C$ and $R_1\leq |z-z_0|\leq R_2$ is in $D$. Then $f$ is analytic in $R_1\leq |z-z_0|\leq R_2$. By the Cauchy integral formula, for $R_1<|z-z_0|<R_2$,
\begin{align*}
f(z)=&\dfrac{1}{2\pi i}\int _{|\xi -z_0|=R_2}\dfrac{f(\xi )}{\xi -z}\,d\xi -\dfrac{1}{2\pi i}\int _{|\xi -z_0|=R_1}\dfrac{f(\xi )}{\xi -z}\,d\xi\\
=&\sum ^{\infty }_{n=0}a_{n}(z-z_0)^{n}-\dfrac{1}{2\pi i}\int _{|\xi -z_0|=R_1}\dfrac{f(\xi )}{\xi -z}\,d\xi 
\end{align*}
where both are positively oriented simply curves and 
\[a_{n}=\dfrac{1}{2\pi i}\int _{|\xi -z_0|=R_2}\dfrac{f(\xi )}{(\xi -z_0)^{n+1}}\,d\xi \]
For the second part, 
\begin{align*}
\dfrac{1}{z-\xi }=\dfrac{1}{z-z_0-(\xi -z_0)}=\dfrac{1}{z-z_0}\dfrac{1}{1-\Big(\dfrac{\xi -z_0}{z-z_0}\Big)}=\sum ^{\infty }_{n=0}\dfrac{(\xi -z_0)^{n}}{(z-z_0)^{n+1}}
\end{align*}
since $|(\xi -z_0)/(z-z_0)|<1$. As the series converges uniformly on the compact set, we have, for the second part,
\begin{align*}
\dfrac{1}{2\pi i}\int _{|\xi -z_0|=R_1}\dfrac{f(\xi )}{\xi -z}\,d\xi =&\dfrac{1}{2\pi i}\sum ^{\infty }_{n=0}\int _{|\xi -z_0|=R_1}\dfrac{(\xi -z_0)^{n}}{(z-z_0)^{n+1}}f(\xi )\,d\xi \\=&\dfrac{1}{2\pi i}\sum ^{\infty }_{n=0}\Big[\int_{|\xi -z_0|=R_1}f(\xi )(\xi -z_0)^{n}\,d\xi \Big](z-z_0)^{-(n+1)}\\
=&\sum ^{-1}_{m=-\infty }\Big[\dfrac{1}{2\pi i}\int _{|\xi -z_0|=R_1}\dfrac{f(\xi )}{(\xi -z_0)^{m}}\,d\xi \Big](z-z_0)^{m}
\end{align*}
Therefore,
\[f(z)=\sum ^{\infty }_{n=0}a_{n}(z-z_0)^{n}+\sum ^{-1}_{m-\infty }a_{m}(z-z_0)^{m}\]
where 
\begin{align*}
a_{n}=&\dfrac{1}{2\pi i}\int _{|\xi -z_0|=R_2}\dfrac{f(\xi )}{(\xi -z_0)^{n+1}}\,d\xi &n\geq 0\\
a_{m}=&\dfrac{1}{2\pi i}\int _{|\xi -z_0|=R_1}\dfrac{f(\xi )}{(\xi -z_0)^{m+1}}\,d\xi &m<0
\end{align*}
as the integrand is analytic between the concentric circles with radius $R_1$ and $R_2$, we can generalise to a curve in this domain and say
\[a_{n}=\dfrac{1}{2\pi i}\int _{C}\dfrac{f(\xi )}{(\xi -z_0)^{n+1}}\,d\xi \]
for $n\in {\bm Z}$.
\end{proof}
\vspace{2ex}
\begin{ex}
By cleverly manipulating functions to be expressed in terms of an infinite summation of a geometric sequence, we can obtain various Laurent series expansions at different regions on the complex plane. Let
\[f(z)=\dfrac{1}{(z-1)(z-2)}\]
be an analytic function in 
\begin{itemize}
\item[(i)] Consider $D_1=\{ 0<|z-1|<1\}$
\[f(z)=-\dfrac{1}{(z-1)(1-(z-1))}=\sum ^{\infty }_{n=-1}(z-1)^{n}\]
\item[(ii)] Consider $D_2=\{0<|z-2|<1\}$
\[f(z)=\dfrac{1}{((z-2)+1)(z-1)}=\dfrac{1}{z-2}\sum ^{\infty }_{n=1}(-1)^{n}(z-2)^{n}=\sum ^{\infty }_{n=-1}(-1)^{n+1}(z-2)^{n}\]
\item[(iii)] Consider $D_3=\{|z|<1\}$
\[f(z)=-\dfrac{1}{z-1}+\dfrac{1}{z-2}\]
\item[(iv)] Consider $D_4=\{1<|z|<2\}$
\[f(z)=-\dfrac{1}{z-1}+\dfrac{1}{z-2}\]
\item[(v)] Consider $D_5=\{|z|>2\}$
\[f(z)=-\dfrac{1}{z-1}+\dfrac{1}{z-2}\]
\end{itemize}
\end{ex}
\vspace{2ex}


