\section{Lecture 4 (March 18th)}
\begin{recall}
\[f(z)=\lim _{\Delta z\rightarrow 0}\dfrac{f(z+\Delta z)-f(z)}{\Delta z}\]
if $f'(z)$ exists and $\Delta z=h\in {\bm R}$ 
\[f'(z)=u_{x}+iv_{x}\]
where $f=u+iv$. If $\Delta =ih$ ($h\in {\bm R}$), then
\begin{align*}
	\lim _{\Delta z\rightarrow 0}\dfrac{f(z+\Delta z)-f(z)}{\Delta z}=& \lim _{h\rightarrow 0}\dfrac{u(x,y+h)+iv(x,y+h)-u(x,y)-iv(x,y)}{ih}\\
=&-i\lim _{h\rightarrow 0}\dfrac{u(x,y+h)-u(x,y)}{h}+\lim _{h\rightarrow 0}\dfrac{v(x,y+h)-v(x,y)}{h}\\
=&-iu_{y}(x,y)+v_{y}(x,y)
\end{align*}
We desire that these two expressions are equal, and we get $u_{x}=v_{y}$ and $v_{x}=-u_{y}$. In sum, if $f$ is differentiable at $z_0=x_0+iy_0$, then $u_{x}=v_{y}$ and $v_{x}=-u_{y}$ at $(x_0,y_0)$. This equality is called the Cauchy-Riemann equation.
\end{recall}
\vspace{2ex}
\begin{ex}
Note that the function $f(z)=\bar{z}=x-iy$ is a $C^{\infty }$ function everywhere, but nowhere differentiable. 
\end{ex}
\vspace{2ex}
\begin{thm}
If $f\in C^{1}(D)$ where $D$ is open, and $u_{x}=v_{y}$ and $v_{x}=-u_{y}$, then $f$ is differentiable (or analytic) in $D$. 
\end{thm}
\vspace{2ex}
\begin{recall}
If $f=u+iv\in C^{1}(D)$, then $u$, $v$ satisfies 
\[\begin{cases}
	\Delta u=u_{x}\Delta x+u_{y}\Delta y+\varepsilon_1\Delta x+\varepsilon_2\Delta y\\
\Delta v=v_{x}\Delta x+v_{y}\Delta y+\eta _{1}\Delta x+\eta _{2}\Delta y
\end{cases}\]
where $\varepsilon_1,\varepsilon_2,\eta_1,\eta_2\rightarrow 0$ as $(\Delta x,\Delta y)\rightarrow (0,0)$.
\end{recall}
\vspace{2ex}
\begin{proof}
If $\Delta z=h+ik$ where $h,k\in {\bm R}$, then 
\begin{align*}
	f(z_0+\Delta z)-f(z)=&u(x_0+h,y_0+k)-u(x_0,y_0)+i[v(x_0+h,y_0+k)-v(x_0,y_0)]\\
	=&u_{x}(x_0,y_0)h+\underbrace{u_{y}(x_0,y_0)}_{-v_{x}(x_0,y_0)}k+\varepsilon_1h+\varepsilon_2k\\&+i[v_{x}(x_0,y_0)h+\underbrace{v_{y}(x_0,y_0)}_{u_{x}(x_0,y_0)}k+\eta_1\Delta x+\eta_2\Delta y]\\
	=&u_{x}(x_0,y_0)(h+ik)+i[v_{x}(x_0,y_0)(h+ik)]+\varepsilon_1h+\varepsilon_2k+i[\eta_1h+\eta_2k]\\
=&u_{x}(x_0,y_0)\Delta z+iv_{x}(x_0,y_0)\Delta z
\end{align*}
where $z_0=x_0+iy_0\in D$. Then,
\begin{align*}
	\lim _{\Delta z\rightarrow 0}\dfrac{f(z_0+\Delta z)}{\Delta z}=&\lim _{\Delta z\rightarrow 0}[u_{x}(x_0,y_0)+iv_{x}(x_0,y_0)]+\lim _{\Delta z\rightarrow 0}\dfrac{\varepsilon_1h+\varepsilon_2k+i(\eta_1h+\eta_2k)}{h+ik}\\
	=&u_{x}(x_0,y_0)+iv_{x}(x_0,y_0)+\lim _{\Delta z\rightarrow 0}\dfrac{h}{h+ik}\varepsilon_1+\lim _{\Delta z\rightarrow 0}\dfrac{k}{h+ik}\varepsilon_2+\ldots\\
=&u_{x}+iv_{x}
\end{align*}
The limit thus exists at $z_0$! The latter parts all go to zero as $h\rightarrow 0$ and $k\rightarrow 0$ because
\[\Big|\dfrac{h}{h+ik}\Big|=\dfrac{|h|}{\sqrt{h^2+k^2}}\leq 1\]
\end{proof}
\vspace{2ex}
\begin{ex}
For $z=x+iy$, define $e^{z}=e^{x}e^{iy}=e^{x}(\cos y+i\sin y)$. Then, if $f(z)=e^{z}$ then $u(x,y)=e^{x}\cos y$ and $v(x,y)=e^{x}\sin y$. We will show that these satisfy the C-R conditions.
\[\begin{cases}
u_{x}=e^{x}\cos y=v_{y}\\
v_{x}=e^{x} \sin y=-u_{y}
\end{cases}\]
The function $f$ is also thus analytic on ${\bm C}$ and $f'(z)=u_{x}+iv_{x}=e^{x}\cos y+ie^{x}\sin y=e^{x}(\cos x+i\sin y)=e^{x}e^{iy}=e^{z}$. Simply put, if $f(z)=e^{z}$ then $f'(z)=e^{z}$. 
\end{ex}
\vspace{2ex}
\begin{defi}
We now see the C-R equation in polar coordinates. Let $f=u+iv$ where $z=x+iy=re^{i\theta }$. If $u,v\in C^{1}(D)$ satisfies the C-R equation, 
\begin{spacing}{2}
\[\begin{cases}
u_{r}=\dfrac{\partial u}{\partial x}\dfrac{\partial x}{\partial r}+\dfrac{\partial u}{\partial y}\dfrac{\partial y}{\partial r}=u_{x}\dfrac{\partial x}{\partial r}+u_{y}\dfrac{\partial y}{\partial r}=u_{x}\cos \theta +u_{y}\sin \theta  \\
u_{\theta }=u_{x}\dfrac{\partial x}{\partial \theta }+u_{y}\dfrac{\partial y}{\partial \theta }=-u_{x}r\sin \theta +u_{y}r\cos \theta \\
v_{r}=v_{x}\dfrac{\partial x}{\partial r}+v_{y}\dfrac{\partial y}{\partial r}=v_{x}\cos \theta +v_{y}\sin \theta=-u_{y}\cos \theta +u_{x}\sin \theta  \\
v_{\theta }=u_{x}\dfrac{\partial x}{\partial \theta }+u_{y}\dfrac{\partial y}{\partial \theta }=u_{y}r\sin \theta +u_{x}r\cos \theta 
\end{cases}\]
\end{spacing}
We conclude that
\[\begin{cases}
v_{\theta }=ru_{r}\\
u_{\theta }=-rv_{r}
\end{cases}\]
\end{defi}
\vspace{2ex} 
\begin{ex}
For a multivariable function $u(x_1,\ldots ,x_{n})\in C^2$, we can define the Laplacian as 
\[\Delta u=\sum ^{n}_{k=1}\dfrac{\partial ^2u}{\partial x_{k}^2} \]
If $\Delta u=0$ then we call $u$ harmonic.
\\\\
If $f\in C^2(D)$ with $f=u+iv$ is analytic on $D$, then from the Cauchy-Riemann equation we have $u_{x}=v_{y}$ and $v_{x}=-u_{y}$. Thus $u_{xx}=v_{yx}$ and $u_{yy}=-v_{xy}$. As we know that $v_{xy}=v_{yx}$, we conclude that on $D$,
\[u_{xx}+u_{yy}=0\]
Similarly, 
\[v_{xx}+v_{yy}=0\]
on $D$.
\end{ex}
\vspace{2ex}
\begin{thm}
We have proved the following theorem. If $f=u+iv\in C^2(D)$ is analytic, then $u,v$ are harmonic.
\end{thm}
\vspace{2ex}
\begin{thm}
If $u\in C^2({\bm R}^2)$ is harmonic, then $u=\mathrm{Re}\;f$ for some entire function $f$. 
\end{thm}
\vspace{2ex}
\begin{ex}
If $u(x,y)=y^3-3x^2y$,
\[\begin{cases}
u_{x}=-6xy\hspace{5ex}u_{xx}=-6x\\
u_{y}=3y^2-3x^2\hspace{5ex}u_{yy}=6y
\end{cases}\]
and we can conclude that on ${\bm R}^2$, $u_{xx}+u_{yy}=0$.
\\\\
For a given $u$, how do we find $v\in C^2({\bm R}^2)$, called harmonic conjugate, that allows $x+iv=f$ to be entire (analytic)? We require that $v_{x}=-u_{y}$ and $v_{y}=u_{x}$. 
\end{ex}
\vspace{2ex}
