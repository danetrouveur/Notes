\section{Lecture 23 (June 12th)}
\begin{recall}
We have learned multiple expressions for the Poisson integral. For a continuous function $h:T\rightarrow {\bm R}$, we define
\begin{align*}
p[h](z)=&\dfrac{1}{2\pi }\int ^{2\pi }_{0}h(e^{it})\dfrac{1-|z|^2}{|e^{it}-z|^2}\,dt=\int _{T}h(\xi )\dfrac{1-|z|^2}{\xi -z}\,d\sigma \\
=&\dfrac{1}{2\pi }\int ^{2\pi }_{0}h(e^{it})\dfrac{1-r^2}{1-2r\cos (\theta -t)+r^2}\,dt=\mathop{\mathrm{Re}}\Big(\dfrac{1}{2\pi }\int ^{2\pi }_{0}h(e^{it})\dfrac{e^{it}+z}{e^{it}-z}\,dt\Big)\\
=&\dfrac{1}{2\pi }\int ^{2\pi }_{0}h(e^{it})P_{r}(\theta -t)\,dt
\end{align*}
where the expression in the bracket is analytic and $p[h](z)$ is harmonic in $U$.
\end{recall}
\vspace{2ex}
\begin{thm}
If $f:T\rightarrow {\bm R}$ is continuous, define $g:\bar{U}\rightarrow {\bm R}$ by $g(z)=f(z)$ for $z\in T$ and $g(z)=p[f](z)$ for $z\in U$. Then $g$ is continuous on $\bar{U}$. 
\end{thm}
\vspace{2ex}
\begin{proof}
Define for $z\in U$ and $\xi \in T$ the Poisson kernel 
\[p(z,\xi )=\dfrac{1-|z|^2}{|\xi -z|^2}\]
then,
\begin{itemize}
\item[(i)] $p(z,\xi )>0$ for all $u\in U$ and $\xi \in T$
\item[(ii)] For all $z\in U$, \[\int _{T}p(z,\xi )\,d\sigma (x)=1\]
\item[(iii)] For every $\delta >0$ and $\eta \in T$,
\[\lim _{z\rightarrow \eta }\int _{|\xi -\eta |>\delta }p(z,\xi )\,d\sigma (\xi )=\lim _{z\rightarrow \eta }\int _{|\xi -\eta |>\delta }\dfrac{1-|z|^2}{|\xi -z|^2}\,d\sigma (\xi )=0\]
\end{itemize}
Now, with this information, we prove that $g$ is continuous on $\bar{U}$ as this suffices in showing that $g$ is continuous on $\bar{U}$. Take any $\eta \in T$, and it is enough to show that $g$ is continuous at $\eta $. Take any $\varepsilon>0$, we have to find $\delta >0$ so that if 
\[|z-\eta |<\delta $, for $z\in U\] then 
\[|g(z)-g(\eta )|<\varepsilon \]
For one, there is $M>0$ such that $|f(\xi )|\leq M$ for all $\xi \in T$. For two, there is $\delta_1>0$ such that if $|\xi -\eta |<\delta_1$ then $|f(\xi )-f(\eta )|<\varepsilon /2$ for $\xi ,\eta \in T$. 
\begin{align*}
g(z)-g(\eta )=&\int _{T}f(\xi )p(z,\xi )d\sigma (\xi )-g(\eta )\\
=&\int _{T}[f(\xi )-f(n)]p(z,\xi )\,d\sigma (\xi )
\end{align*}
Thus, 
\begin{align*}
|g(z)-g(\eta )|\leq &\int _{T}|f(\xi )-f(\eta )|p(z,\xi )\,d\sigma (\xi )\\
=&\int _{|\xi -\eta |\leq \delta_1}|f(\xi )-f(\eta )|p(z,\xi )d\sigma (\xi )+\int _{|\xi -\eta |<\delta_1}|f(\xi )-f(\eta )|p(z,\xi )d\sigma (\xi )\\
=&2M\int _{|\xi -\eta |\leq \delta_1}p(\xi ,z)\,d\sigma (\xi )+\dfrac{\varepsilon }{2}\int _{|\xi -\eta |<\delta_1}p(z,\xi )d\sigma (\xi )
\end{align*}
Now, take $\delta >0$ so that if $|z-\eta |<\delta $ then
\[\int _{|\xi -\eta |}p(z,\xi )\,d\sigma (\xi )<\dfrac{\varepsilon }{4M}\]
Then if $|z-\eta |<\delta $ then $|g(z)-g(\eta )|<\varepsilon $. Then,
\[|g(z)-g(\eta )|<\varepsilon \]
\end{proof}
\vspace{2ex}

