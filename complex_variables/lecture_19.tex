\section{Lecture 19 (May 27th)}
\begin{ex}
(Reverse parametrisation) Solve
\[\int ^{2\pi }_{0}\dfrac{1}{5+4\sin t}\,dt\]
using the following equality
\[\int _{0}^{2\pi }f(e^{it})ie^{it}\,dt=\int _{|z|=1}f(z)\,dz\]
\end{ex}
\vspace{2ex}
\begin{proof}
\begin{align*}
\int ^{2\pi }_{0}\dfrac{dt}{5+4\sin t}=&\int _{|z|=1}\dfrac{1}{5+4\dfrac{z-z^{-1}}{2i}}\dfrac{1}{iz}\,dz\\
=&\int _{|z|=1}\dfrac{1}{5iz+2z^2-1}\,dz\\
=&2\pi i\mathop{\mathrm{Res}}_{z=-i/2}\dfrac{1}{2z^2+5iz-2}=2\pi i\Big[\dfrac{1}{4z+5i}\Big]_{z=-i/2}\\
=&2\pi i\dfrac{1}{3i}=\dfrac{2\pi}{3}
\end{align*}
Where utilizing the quadratic equation, the roots of the denominator are $(-5i\pm 3i)/4$. 
\end{proof}
\vspace{2ex}
\begin{ex}
If $-1<a<1$, then
\[\int ^{2\pi }_{0}\dfrac{1}{1+a\sin t}\,dt=\dfrac{2\pi }{\sqrt{1-a^2}}=\int ^{2\pi }_{0}\dfrac{1}{1+a\cos t}\,dt\]
\end{ex}
\vspace{2ex}
\begin{ex}
For $n\in {\bm N}$, evaluate 
\[\int _{0}^{2\pi }\sin^{2n}t\,dt\]
Let $z=e^{it}$ for $0\leq t\leq 2\pi $. Then, $\sin t=z-z^{-1}/2i$ and $dt=dz/iz$ so that 
\[\int ^{2\pi }_{0}\sin ^{2n}t\,dt=\int_{|z|=1}\Big(\dfrac{z-1/z}{2i}\Big)^{2n}\dfrac{dz}{iz}=2\pi i\mathop{\mathrm{Res}}_{z=0}f(z)\]
where
\[f(z)=\dfrac{1}{(2i)^{2n}}\dfrac{1}{iz}(z-\dfrac{1}{z})^{2n}\]
Here, 
\[\mathop{\mathrm{Res}}_{z=0}f(z)=\dfrac{1}{(2i)^{2n}}\dfrac{1}{i}{}^{2n}C_{n}(-1)^{n}=\dfrac{(2n)!}{(n!)^22^{2n}}\dfrac{1}{i}\]
The result of the integral is therefore
\[2\pi \dfrac{(2n)!}{(n!)^22^{2n}}\]
\end{ex}
\vspace{2ex}
\begin{defi}
(Fourier transform) With $(\hat{f})^{\vee}=f$, the Fourier transform is defined as 
\[\widehat{f}(x)=\int ^{\infty }_{-\infty }f(t)e^{-2\pi i xt}\,dt\quad \mathrm{and}\quad f^{\vee}(t)=\int ^{\infty }_{-\infty }f(x)e^{2\pi i xt}\,dx\]
if they exist.
\end{defi}
\vspace{2ex}
\begin{ex}
For $f(x)=e^{-\pi x^2}$, find the Fourier transform.
\[\widehat{f}(x)=\int ^{\infty }_{-\infty }e^{-\pi t^2}e^{-2\pi ixt}\,dt\]
Notice that
\[e^{-\pi t^2}e^{-2\pi xt}=e^{-\pi (t^2+2ixt)}=e^{-\pi (t+ixt)^2-\pi x^2}\]
Therefore,
\[\widehat{f}(x)=e^{-\pi x^2}\int ^{\infty }_{-\infty }e^{-\pi (t+ix)^2}\,dt\]
For $x\ne0$, in finding the integral, we can take the square contour with height $x$ and width $2R$ for the following function:
\[0=\int _{C_{R}}e^{-\pi z^2}\,dz=\int ^{R}_{-R}e^{-\pi t^2}\,dt+\int ^{x}_{0}e^{-\pi (R+iy)^2}i\,dy-\int ^{R}_{-R}e^{-\pi (t+ix)^2}\,dx+\int ^{0}_{x}e^{-\pi (-R+iy)^2}i\,dy\]
We can see that the second and last term approaches $0$ as $R\rightarrow \infty $.
\[\int ^{\infty }_{-\infty }e^{-\pi (t+ix)^2}\,dt=\int ^{\infty }_{-\infty }e^{-\pi t^2}\,dt=\dfrac{1}{\sqrt{\pi }}\int ^{\infty }_{-\infty }e^{-s^2}\,ds=1\]
taking $s=\sqrt{\pi }t$. We see that the Fourier transform of the function is itself.
\end{ex}
\vspace{2ex}
\begin{rmk}
For the remaining time being, we study (1) the argument principle, which uses Rouches theorem, and (2) the Poisson integral.
\end{rmk}
\vspace{2ex}
\begin{thm}
Let $f$ be analytic inside and on a POSCC $C$ with $f\ne 0$ on $C$. Then $f$ has finitely many zeros inside $C$. Then there exists $z_1,\ldots ,z_{n}$ inside $C$ and $\alpha _{k}\in {\bm N}$ ($1\leq k\leq n$) such that
\[f(z)=(z-z_1)^{\alpha }\ldots (z-z_{n})^{\alpha _{n}}F(z)\]
where $F$ is analytic with no zeros inside and on $C$. Then on $C$,
\[\dfrac{f'(z)}{f(z)}=\sum ^{n}_{k=1}\dfrac{\alpha _{k}}{z-z_{k}}+\dfrac{F'(z)}{F(z)}\]
Thus
\[\int _{C}\dfrac{f'(z)}{f(z)}\,dz=\sum _{k=1}^{n}\int _{C}\dfrac{\alpha _{k}}{z-z_{k}}\,dz+\int _{C}\dfrac{F'(z)}{F(z)}\,dz\]
\end{thm}
\vspace{2ex}
\begin{proof}
If $f(z_0)=0$ for some $z_0$ inside $C$, then $f(z)=a_1(z-z_0)+a_2(z-z_0)^2+\ldots $ in a neighborhood of $z_0$. Therefore,
\[f(z)=(z-z_0)\Big(a_1+a_2(z-z_0)+\ldots \Big)\]
such that $g(z)=f(z)/(z-z_0)$ has a removable singularity at $z_0$. As $g$ is analytic inside and on $C$, $f(z)=(z-z_0)g(z)$ for some analytic function $g$ inside and on $C$.
\end{proof}
\vspace{2ex}
