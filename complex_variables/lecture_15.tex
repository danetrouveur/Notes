\section{Lecture 15 (May 13th)}
\begin{rmk}
If $f$ has a simple pole at $z_0$, then 
\[\mathop{\mathrm{Res}}_{z=z_0}f(z)=\lim _{z\rightarrow z_0}(z-z_0)f(z)\]
We have previously remarked that L'Hospital's theorem works in the complex plane.
\end{rmk}
\vspace{2ex}
\begin{ex}
Let $f(z)=\pi \cot \pi z=\pi \dfrac{\cos \pi z}{\sin \pi z}$. This function has a simple pole at each $n\in {\bm Z}$.
\[\mathop{\mathrm{Res}}_{z=n}f(z)=\lim _{z\rightarrow n}(z-n)\pi \cot\pi z=\pi \cos n\pi \lim _{z\rightarrow n}\dfrac{z-n}{\sin \pi z}=\pi \cos n\pi \lim _{z\rightarrow n}\dfrac{1}{\pi \cos \pi z}=1\]
where in the second last line, we have used L'Hospital's theorem.
\end{ex}
\vspace{2ex}
\begin{ex}
Observe that
\[\mathop{\mathrm{Res}}_{z=\pi i}\Big(\dfrac{1}{e^{z}+1}\Big)=\lim _{z\rightarrow \pi i}(z-\pi i)\dfrac{1}{e^{z}+1}=\lim _{z\rightarrow \pi i}\dfrac{1}{e^{z}}=-1\]
\end{ex}
\vspace{2ex}
\begin{thm}
(Residue theorem 1) Let $f$ be analytic inside and on a POSCC except for an isolated singularity at $z_0$ inside $C$. Then 
\[\dfrac{1}{2\pi i}\int _{C}f(z)\,dz=\mathop{\mathrm{Res}}_{z=z_0}f(z)\]
\end{thm}
\vspace{2ex}
\begin{proof}
Note that due to the Cauchy theorem,
\[\int _{C}f(z)\,dz=\int _{|z-z_0|=\delta }f(z)\,dz=2\pi i\mathop{\mathrm{Res}}_{z=z_0}f(z)\]
\end{proof}
\vspace{2ex}
\begin{ex}
Evaluate 
\[\int _{-\infty }^{\infty }\dfrac{1}{1+x^2}\,dx\]
by using the residue theorem.
\end{ex}
\vspace{2ex}
\begin{proof}
Let the function $f(z)=1/(1+z^2)$ be on $D_{R}$, a positively oriented upper half circle with radius $R>1$. Then by the residue theorem, 
\[\int _{C_{R}}f(z)\,dz=2\pi i\mathop{\mathrm{Res}}_{z=i}f(z)=2\pi i\lim _{z\rightarrow i}(z-i)\dfrac{1}{1+z^2}=2\pi i \times \dfrac{1}{2i}=\pi \]
for all $R>1$. On the other hand, by parametrization,
\[\int _{C_{R}}f(z)\,dz=\int ^{R}_{-R}\dfrac{1}{1+x^2}\,dx+\int ^{\pi }_{0}\dfrac{1}{1+R^2e^{2it}}iRe^{it}\,dt\]
where we substituted $z=Re^{it}$. We then have
\[\Big|\int ^{\pi }_{0}\dfrac{1}{1+R^2e^{2it}}iRe^{it}\,dt\Big|\leq \int ^{\pi }_{0}\dfrac{R}{R^2-1}\,dt=\dfrac{\pi R}{R^2-1}\]
which $\rightarrow 0$ as $R\rightarrow \infty $. We thus found that the integral is equal to $\pi $.
\end{proof}
\vspace{2ex}
\begin{ex}
Consider
\[f(z)=\dfrac{e^{\alpha z}}{1+e^{z}}\]
for $0<\alpha <1$ on the rectangular contour with base $2R$ and height $2\pi $ with its base along the $x$-axis. Our aim is to evaluate 
\[\int ^{\infty }_{-\infty }\dfrac{e^{\alpha x}}{1+e^{x}}\,dx\]
\end{ex}
\vspace{2ex}
\begin{proof}
\[\mathop{\mathrm{Res}}_{z=\pi i}f(z)=\lim _{z\rightarrow \pi i}(z-\pi i)\dfrac{e^{\alpha z}}{1+e^{z}}=e^{\alpha \pi i}(-1)\]
Thus by the residue theorem,
\[\int_{C_{R}}\dfrac{e^{\alpha z}}{1+e^{z}}\,dz=2\pi i(-e^{\alpha \pi i})\]
for all $R>1$. On the other hand, 
\[\int _{C_{R}}f(z)\,dz=\int _{-R}^{R}f(x)\,dx+\int ^{2\pi }_{0}f(R+iy)i\,dy+\int _{R}^{-R}f(x+2\pi i)\,dx+\int ^{0}_{2\pi }f(-R+iy)i\,dy\]
or, equivalently,
\[\int _{C_{R}}f(z)\,dz=\int _{-R}^{R}\dfrac{e^{\alpha x}}{1+e^{x}}\,dx+\underbrace{\int ^{2\pi }_{0}\dfrac{e^{\alpha (R+iy)}}{1+e^{R+iy}}i\,dy}_{\mathrm{II}}-\int _{R}^{-R}\dfrac{e^{\alpha (x+2\pi i)}}{1+e^{x+2\pi i}}\,dx-\underbrace{\int _{0}^{2\pi }\dfrac{e^{\alpha (-R+iy)}}{1+e^{-R+iy}}i\,dy}_{\mathrm{IV}}\]
Note that
\begin{align*}
|\,\mathrm{II}\,|\leq & \int ^{2\pi }_{0}\dfrac{e^{\alpha R}}{e^{R}-1}\,dy=\dfrac{2\pi e^{\alpha R} }{e^{R}-1}\rightarrow 0\\
|\,\mathrm{IV}\,|\leq & \int ^{2\pi }_{0}\dfrac{e^{-\alpha R}}{1-e^{-R}}\,dy=\dfrac{2\pi e^{-\alpha R}}{1-e^{-R}}\rightarrow 0
\end{align*}
as $R\rightarrow \infty $ because $0<\alpha <1$. Accordingly,
\[\lim _{R\rightarrow \infty }\int _{C_{R}}f(z)\,dz=(1-e^{2\pi \alpha i})\int ^{\infty }_{-\infty }\dfrac{e^{\alpha x}}{1+e^{x}}\,dx=-2\pi ie^{\alpha \pi i}\]
Thus,
\[\int _{-\infty }^{\infty }\dfrac{e^{\alpha x}}{1+e^{x}}=-\dfrac{2\pi ie^{\alpha \pi i}}{1-e^{2\pi \alpha i}}=\dfrac{2\pi i}{e^{\alpha \pi i}-e^{-\alpha \pi i}}=\dfrac{\pi }{\sin \alpha \pi }\]
\end{proof}
\vspace{2ex}
\begin{rmk}
Here, substitute $e^{x}=t$. Then, $dt=e^{x}\,dx=t\,dx$. 
\[\int ^{\infty }_{0}\dfrac{t^{\alpha -1}}{1+t}\,dt=\dfrac{\pi }{\sin \alpha \pi }\]
Now, again let $t=x^{\beta }$ for $0<\beta <\infty $ to obtain $dt=\beta x^{\beta -1}\,dx$.
\[\int ^{\infty }_{0}\dfrac{x^{\alpha \beta -1}}{1+x^{\beta }}\,dx=\dfrac{1}{\beta }\dfrac{\pi }{\sin \alpha \pi }\]
Now as $0<\beta <\infty $, let  $\alpha  =1/\beta  $, getting
\[\int ^{\infty }_{0}\dfrac{1}{1+x^{\beta }}\,dx=\dfrac{1}{\beta }\dfrac{\pi }{\sin \pi /\beta }\]
Telling us that 
\[\int ^{\infty }_{0}\dfrac{1}{1+x^{\beta }}\,dx\]
converges when $\beta >1$. This example readily shows the beauty of complex integration, with parametrized integrals leaving us with powerful results. 
\end{rmk}
\vspace{2ex}
\begin{thm}
(Residue theorem 2) Let $f$ be analytic inside and on a POSCC $C$ except for finite isolated singularities at $z_1,\ldots ,z_{n}$ inside $C$. Then
\[\int_{C}f(z)\,dz=2\pi i\sum _{k=1}^{n}\mathop{\mathrm{Res}}_{z=z_k}f(z)\]
\end{thm}
\vspace{2ex}
\begin{proof}
As the singularities are isolated, there is $r>0$ such that 
\[\int _{C}f(z)\,dz=\sum ^{n}_{k=1}\int _{|z-z_{k}|=r}f(z)\,dz=2\pi i\sum ^{n}_{k=1}\mathop{\mathrm{Res}}_{z=z_{k}}f(z)\]
by the Cauchy theorem where $|z-z_{r}|=k$ is POS (positively oriented and simple).
\end{proof}
\vspace{2ex}
\begin{defi}
(Zeta function) The zeta function is given as
\[\xi(k)=\sum ^{\infty }_{n=1}\dfrac{1}{n^{k}}\]
\end{defi}
\vspace{2ex}
