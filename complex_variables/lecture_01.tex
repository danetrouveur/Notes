\section{Lecture 1 (March 4th)}
\begin{defi}
A complex number is expressed as $z=x+iy$ with $x,y\in {\bm R}$ with $i^2=-1$. Here,
\[\begin{cases}
x=\mathrm{Re}\;z\;\mathrm{(real\;part)}\\
y=\mathrm{Im}\;z\;\mathrm{(imaginary\;part)}
\end{cases}\]
\end{defi}
\vspace{2ex}
The cornerstone of imaginary numbers is that with the introduction of $i$, all operations become closed. We use the letter ${\bm C}$ to denote the set of complex numbers which is a complete field. Topologically, ${\bm C}$ and ${\bm R}^2$ have identical metrics and are topologically equivalent. In this way, continuity in both sets are defined identically. However, ${\bm C}$ has well defined multiplication and division which are very useful. Accordingly, the definition of differentiability is radically different.
\[f'(z)=\lim _{h\rightarrow 0}\dfrac{f(z+h)-f(z)}{h}\]
The concept of differentiability for functions on complex numbers are defined identically as the functions on real numbers. However, there is a critical complication regarding ``path". 

All of differentiation of complex functions stems from the ``Cauchy-Riemann equations". Once this strong condition is satisfied by functions, all miracles of complex numbers come from the theorem called the ``Cauchy integral formula".
\\
\begin{rmk}
The most important function in all of mathematics is the complex function
\[f(z)=e^{z}=\sum _{n=0}^{\infty }\dfrac{z^{n}}{n!}\]
for $z\in {\bm C}$. Given $z=x+iy$, we define this function using the following property
\[e^{z}=e^{x}e^{iy}\]
The equation used for the definition of $e^{z}$ is a very special definition indeed which we will investigate further later on.
\end{rmk}
\vspace{2ex}
\begin{defi}
For any complex number $z$, the conjugate is defined as $\bar{z}=x-iy$. For any number of the complex plane, the conjugate is simply its reflection along the real axis. The real and imaginary parts of a number can be, through this, defined as
\[\begin{cases}
	\mathrm{Re}\;z=&x=\dfrac{z+\bar{z}}{2}\\\\
	\mathrm{Im\;z}=&y=\dfrac{z-\bar{z}}{2i}
\end{cases}\]
We define addition, subtraction, and division of complex numbers like the following.
\begin{align*}
	z_1+z_2=&(x_1+x_2)+i(y_1+y_2)\\
	z_1z_2=&(x_1x_2-y_1y_2)+i(x_1y_2+x_2y_1)\\
	\dfrac{z_1}{z_2}=&\dfrac{(x_1x_2+y_1y_2)+i(x_1y_2-x_2y_1)}{x_2^2+y_2^2}
\end{align*}
Consequently,
\begin{align*}
	\overline{z_1+z_2}=&\bar{z_1}+\bar{z_2}\\
	\overline{z_1z_2}=&\bar{z_1}+\bar{z_2}\\
	\overline{\Big(\dfrac{z_1}{z_2}\Big)}=&\dfrac{\bar{z_1}}{\bar{z_2}}
\end{align*}
\end{defi}
\vspace{2ex}
\begin{defi}
We define the distance between two complex numbers $(x_1,y_1)$ and $(x_2,y_2)$ in ${\bm R}^2$ as
\[|z_1-z_2|=\sqrt{(x_2-x_1)^2+(y_2-y_1)^2}\]
For $z_{0}\in {\bm C}$ and $r>0$, we can define open and closed disks centered at $z_{0}$ with radius $r$ as like the following
\[D_{c}(z_0,r)=\{z \;|\; |z-z_0|\leq r\}\hspace{5ex}D_{o}(z_0,r)=\{z \;|\; |z-z_0|<r\}\]
\end{defi}
\begin{thm}
Note the triangle inequality
\begin{align*}
|z_1+z_2|\leq |z_1|+|z_2|
\end{align*}
From this we have $|z_1|=|z_1-z_2+z_2|\leq |z_1-z_2|+|z_2|$
and $|z_1|-|z_2|\leq |z_1-z_2|$
therefore we get
\[\big||z_1|-|z_2|\big|\leq |z_1-z_2|\]
\end{thm}
\vspace{2ex}

\begin{defi}
The polar expression of a complex number $z=x+iy$ not equal to zero is given as
\[z=x+iy=re^{i\theta }\]
For $\theta \in {\bm R}$, we define
\[e^{i\theta }=\cos \theta +i\sin \theta \]
where $r^2=|z|^2=x^2+y^2$ and $\tan \theta =y/x$. $r=|z|$ is called the modulus of $z$ while $\theta =\mathrm{arg}\;z$ is called the argument of $z$. Moreoever, $z^{n}=(re^{i\theta })^{n}=r^{n}e^{in \theta }$ for $n\in {\bm Z}$ and for $n,m\in {\bm Z}$, we have $z^{n}z^{m}=z^{n+m}$. Then,
\begin{align*}
	e^{i(\theta_1+\theta_2)}=& \cos (\theta_1+\theta_2)+i\sin (\theta_1+\theta_2)\\
	=&\cos \theta_1\cos \theta_2-\sin \theta_1 \sin \theta_2+i(\sin \theta_1\cos \theta_2+\cos \theta_1\sin \theta_2)\\
	=&(\cos\theta_1+i\sin \theta_1)(\cos\theta_2+i\sin \theta_2)\\
	=&e^{i\theta_1}e^{i\theta_2}
\end{align*}
\end{defi}
\vspace{2ex}
Note that when $z\ne 0$, $\theta =\mathrm{arg}\;z$ is not uniquely determined. If $\theta$ is an argument of $z$, then $\theta +2\pi $ is also an argument.
\\
\begin{thm}
(De-Moivre Theorem) When $n\in {\bm N}$, we have
\[e^{in \theta }=\cos n\theta +i\sin n\theta =(e^{i\theta })^{n}=(\cos \theta +i\sin \theta )^{n}\]
which can be proved by mathematical induction. A smiple corollary is that
\begin{align*}
	(e^{i\theta })^{-1}=\dfrac{1}{e^{i\theta }}=&\dfrac{1}{\cos \theta +i\sin \theta }\dfrac{\cos \theta -i\sin \theta }{\cos \theta -i\sin \theta }\\
	=&\dfrac{\cos \theta -i\sin \theta }{\cos ^2\theta +\sin ^2\theta }=\cos \theta -i\sin \theta =e^{-i\theta}
\end{align*}

\end{thm}
\vspace{2ex}
Additionally, from Euler's formula, we find
\begin{align*}
	e^{i(\theta +2\pi )}=&\cos (\theta +2\pi )+i\sin (\theta +2\pi )\\
=&\cos \theta +i\sin \theta =e^{i\theta }\\
e^{i2\pi }=&1\\
e^{i\pi }+1=&0
\end{align*}
Every $z\ne 0$ can be uniquely expressed as
\[z=re^{i\theta }\]
where $r=|z|$ and $\theta =\mathrm{Arg}\;z$ $(-\pi <\mathrm{Arg}\;z\leq \pi )$ is called the principle argument. Note that $\mathrm{arg}\;z=\mathrm{Arg}\;z+2n\pi $ for $n\in {\bm Z}$.
\\
\begin{ex}
Find $\mathrm{Arg}(-{2}/{1+i\sqrt{3}})$ (answer: $2\pi /3$).
\end{ex}
\vspace{2ex}



\vspace{2ex}

