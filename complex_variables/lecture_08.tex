\section{Lecture 8 (April 1st)}
\begin{defi}
For a closed contour $C$, we define $\Delta _{C}\;\mathrm{arg}\;z$ as the net change of the argument as $z$ traverses $C$. Then,
\[\dfrac{1}{2\pi }\Delta _{C}\;\mathrm{arg}\;z\]
is the number of times $C$ winds up the origin in the positive sense. On another note, we know that for a closed contour $0\in C$,
\[\int _{C}z^{n}\;dz=0\hspace{4ex}\mathrm{for}\hspace{4ex}z\in {\bm Z}\;\backslash\;\{-1\}\]
since
\[\Big[\dfrac{1}{z+1}z^{n+1}\Big]'=z^{n}\]
and
\[\int _{C}f(z)\;dz=\int ^{b}_{a}f(z(t))z'(t)\;d=F(z(b))-F(z(a))\]
for $F'=f$ on $C$. Continuing this argument, we have
\begin{align*}
\int _{C}\dfrac{1}{z}\;dz=&\log z(b)-\log z(a)=\ln |z(b)|+i\mathrm{arg}\;z(b)-[\ln |z(a)|+i\mathrm{arg}\;z(a)]\\=&i[\mathrm{arg}\;z(b)-\mathrm{arg}\;z(a)]=i\Delta _{C}\;\mathrm{arg}\;z
\end{align*}
for a closed contour $C:z=z(t)$, $a\leq t\leq b$ and $z(a)=z(b)$. In conclusion, for a closed contour where $0\notin C$,
\[\dfrac{1}{2\pi i}\int _{C}\dfrac{1}{z}\;dz\]
is the number of times $C$ winds up $0$ in the positive sense.
\end{defi}
\vspace{2ex}
\begin{thm}
If $C$ is a POSCC (positively oriented simple closed curve) and $0\notin C$), for $n\in {\bm Z}$,
\[\int _{C}z^{n}\;dz=\begin{cases}
0\hspace{5ex}n\ne -1\\
0\hspace{5ex}n=-1,\mathrm{\ origin\ is\ outside\ }C\\
2\pi i\hspace{3ex}n=-1,\mathrm{\ origin\ is\ inside\ }C
\end{cases}\]
If $C$ is a POSCC and $a\notin C$, for $z\in {\bm Z}$,
\[\int _{C}(z-a)^{n}\;dz=
\begin{cases}
0\hspace{5ex}n\ne -1\\
0\hspace{5ex}n=-1,\ a\mathrm{\ is\ outside\ }C\\
2\pi i\hspace{3ex}n=-1,\ a\mathrm{\ is\ inside\ }C
\end{cases}
\]
We can obtain this result by letting $z-a=w$ and $dz=dw$. Many (or most of) contour integrals are independent of path (dependent only on endpoints).
\end{thm}
\vspace{2ex}
\begin{ex}
Let $C$ be a contour in $\mathrm{Re}\;z\geq 0$ ($-\pi <\mathrm{arg}\;z<\pi $) and $0\notin C$. For $C$ that extends from $-2i$ to $2i$, evaluate
\[\int _{C}\dfrac{1}{z}\;dz\]
\end{ex}
\vspace{2ex}
\begin{proof}
This is equal to $\log 2i-\log -2i=\pi i$. If we contrarily limit the argument to $0<\mathrm{arg}\;z<2\pi $, we get $-\pi i$.
\end{proof}
\vspace{2ex}
\begin{ex}
Let $C$ be a contour in $\mathrm{Im}\;z\geq 0$ with $0\notin C$. Let $C$ be from 3 to -3. Evaluate
\[\int _{C}z^{1/2}\;dz\]
\end{ex}
\vspace{2ex}
\begin{proof}
Limiting the argument by creating a branch at $-\pi /2$, that is, limiting the argument to $-\pi /2<\mathrm{arg}\;z<3\pi /2$, $z^{1/2}$ is analytic with $(2z^{3/2}/3)'=z^{1/2}$, and we have
\[I=F(-3)-F(3)=\dfrac{2}{3}\Big(3e^{\pi i}\Big)^{3/2}-\dfrac{2}{3}\Big(3^{3/2}\Big)=\dfrac{2}{3}3\sqrt{3}\Big[e^{3\pi i/2}-1\Big]=2\sqrt{3}(-i-1)\]
\end{proof}
\vspace{2ex}
\begin{thm}
(Cauchy's theorem) Let $D$ be a simply connected domain. If $f$ is a $C^{1}$ analytic function in $D$, then
\[\int _{C}f(z)\;dz=0\]
for all simply closed contours $C$ in $D$. 
\end{thm}
\vspace{2ex}
\begin{proof}
Assume $C$ is a positive-oriented. Then 
\[\int _{C}f(z)\;dz=\int _{C}(u+iv)(dx+i\;dy)=\int _{C}u\;dx-v\;dy+i\int _{C}v\;dx+u\;dy\]
By Green's theorem, the interior of the contour becomes a simply connected set and
\[\int u\;dx-v\;dy=\iint_{\mathrm{\Omega} }\Big(-\dfrac{\partial v}{\partial x}-\dfrac{\partial u}{\partial y}  \Big)dxdy =0\]
where $\mathrm{\Omega} $ is the interior of $C$. Note well that Cauchy's theorem is strictly the property of the domain of the function whereas the theorem we dealt with beforehand is a result of line integrals and furthermore the fundamental theorem of calculus. Again, Cauchy's theorem is about the boundary while line integrals are about the existence of the anti-derivative!
\end{proof}
\vspace{2ex}
\begin{rmk}
For $P$ and $Q\in C^{1}(\bar{D})$,
\[\iint_{D}\Big(\dfrac{\partial Q}{\partial x}-\dfrac{\partial P}{\partial y}\Big)dxdy=\int _{\partial D}P\;dx+Q\;dy\]
\end{rmk}
\vspace{2ex}

