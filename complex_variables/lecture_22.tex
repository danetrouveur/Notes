\section{Lecture 22 (June 10th)}
\begin{rmk}
We use the following notation. 
\begin{itemize}
\item[(i)] $U=\{|z|<1\}$ for a unit disc
\item[(ii)] $T=\{|z|=1\}$ for a unit circle $z=re^{it}$ with $0\leq t\leq 2\pi $
\item[(iii)] $\bar{U}=\{|z|\leq 1 \}$ for a closed unit disc
\end{itemize}
\end{rmk}
\vspace{2ex}
\begin{lem}
If $f$ is analytic on $\bar{U}$, then for each $z\in U$ we have 
\[f(z)=\dfrac{1}{2\pi i}\int _{T}\dfrac{f(\xi )}{\xi -z}\,d\xi \]
and 
\[\dfrac{1}{2\pi i}\int _{T}\dfrac{f(\xi )\bar{z}}{1-\bar{z}\xi }\,d\xi =0\]
Parametrising the function $f(z)$,
\[f(z)=\dfrac{1}{2\pi i}\int ^{2\pi }_{0}\dfrac{f(e^{it})}{e^{it}-z}ie^{it}\,dt=\dfrac{1}{2\pi }\int ^{2\pi }_{0}\dfrac{f(e^{it})}{1-ze^{-it}}\,dt\]
We also see that
\[0=\dfrac{1}{2\pi i}\int ^{2\pi }_{0}\dfrac{f(e^{it})\bar{z}}{1-\bar{z}e^{it}}ie^{it}\,dt=\dfrac{1}{2\pi }\int ^{2\pi }_{0}\dfrac{f(e^{it})e^{it}\bar{z}}{1-\bar{z}e^{it}}\,dt\]
That is,
\[f(z)=\dfrac{1}{2\pi }\int ^{2\pi }_{0}f(e^{it})\Big[\dfrac{1}{1-ze^{-it}}+\dfrac{e^{it}\bar{z}}{1-\bar{z}e^{it}}\Big]\,dt\]
Here,
\[\dfrac{1}{1-ze^{-it}}+\dfrac{e^{it}}{1-\bar{z}e^{it}}=\dfrac{1-\bar{z}e^{it}+e^{it}\bar{z}-|z|^2}{|1-ze^{-it}|^2}=\dfrac{1-|z|^2}{|1-ze^{-it}|^2}=\dfrac{1-|z|^2}{|e^{it}-z|^2}\]
and we have
\[f(z)=\dfrac{1}{2\pi }\int ^{2\pi }_{0}f(e^{it})\dfrac{1-|z|^2}{|e^{it}-z|^2}\,dt\]
\end{lem}
\vspace{2ex}
\begin{thm}
If $u$ is harmonic in $\bar{U}$, then for every $z\in U$ we have
\[u(z)=\dfrac{1}{2\pi }\int ^{2\pi }_{0}u(e^{it})\dfrac{1-|z|^2}{|e^{it}-z|^2}\,dt\]
This tells us that any harmonic function can be expressed in terms of itself on a local disk.
\end{thm}
\vspace{2ex}
\begin{proof}
If $u$ is harmonic in a neighborhood of $\bar{U}$, then there is an analytic $f$ in the neighborhood of $U$ such that $u=\mathop{\mathrm{Re}}(f)$. Thus, for every $z\in U$,
\begin{align*}
u(z)=&\mathop{\mathrm{Re}}(f)=\mathop{\mathrm{Re}}\Big[\dfrac{1}{2\pi }\int ^{2\pi }_{0}f(e^{it})\dfrac{1-|z|^2}{|e^{it}-z|^2}\,dt\Big]\\
=&\dfrac{1}{2\pi }\int ^{2\pi }_{0}\mathop{\mathrm{Re}}\Big[f(e^{it})\dfrac{1-|z|^2}{|e^{it}-z|^2}\Big]\,dt
\\=&\dfrac{1}{2\pi }\int ^{2\pi }_{0}u(e^{it})\dfrac{1-|z|^2}{|e^{it}-z|^2}\,dt
\end{align*}
\end{proof}
\vspace{2ex}
\begin{defi}
For $h\in C(T)$, we define the Poisson integral $p[h]$ of $h$ on $U$ by 
\[p[h](z)=\dfrac{1}{2\pi }\int ^{2\pi }_{0}h(e^{it})\dfrac{1-|z|^2}{|e^{it}-z|^2}\,dt=\int _{T}h(\xi )\dfrac{1-|z|^2}{|\xi -z|^2}\,d\sigma (\xi )\]
where $\sigma(T)=1$ is a measure. Equivalently,
\[p[h](re^{i\theta })=\dfrac{1}{2\pi }\int ^{2\pi }_{0}h(e^{it})P_{r}(\theta -t)\,dt\]
where $P_{r}(\theta -t)$ is the Poisson kernel. Phrased differently, the above theorem states that $u=p[u]$ if it is harmonic. 
\end{defi}
\vspace{2ex}
\begin{defi}
Let $w=re^{it}$. Define
\[P_{r}(t)=\sum ^{\infty }_{n=-\infty }r^{|n|}e^{int }=1+\sum ^{\infty }_{n=1}r^{n}(e^{int}+e^{-in t})\]
Note how
\[\dfrac{1+w}{1-w}=(1+w)(1+w+w^2+\ldots )=1+2\sum ^{\infty }_{n=1}w^{n}=1+2\sum ^{\infty }_{n=1}r^{n}e^{in t}\]
Thus
\[\mathop{\mathrm{Re}}\Big(\dfrac{1+w}{1-w}\Big)=1+2\sum ^{\infty }_{n=1}r^{n}\cos nt=P_{r}(t)\]
where $(1+w)/(1-w)$ is harmonic in $w$. We see that the ugly expression that we first used can actually be expressed as 
\[P_{r}(t)=\mathop{\mathrm{Re}}\Big(\dfrac{1+re^{it}}{1-re^{it}}\Big)=\dfrac{1-r^2}{|1-re^{it}|^2}=\dfrac{1-|w|^2}{|1-w|^2}\]
We see exactly that 
\[P_{r}(\theta -t)=\dfrac{1-|w|^2}{|1-re^{it(\theta -t)}|^2}=\dfrac{1-r^2}{|e^{it}-re^{i\theta }|^2}=\dfrac{1-|z|^2}{|e^{it}-z|^2}\]
where $z=re^{i\theta }$. In sum, 
\[P_{r}(\theta -t)=\dfrac{1-|z|^2}{|e^{it}-z|^2}\]
Note that we can also write the Poisson kernal as 
\[\dfrac{1-r^2}{|1-re^{it}|^2}=\dfrac{1-r^2}{1-2r\cos t+r^2}\]
\end{defi}
\vspace{2ex}
\begin{thm}
For $0\leq r<1$ and $t\in {\bm R}$,
\[P_{r}(t)=\sum ^{\infty }_{n=-\infty }r^{|n|}e^{in t}=\mathop{\mathrm{Re}}\Big(\dfrac{1+re^{it}}{1-re^{it}}\Big)=\dfrac{1-r^2}{1-2r\cos t+r^2}\]
which implies that
\[P_{r}(\theta -t)=\dfrac{1-r^2}{1-2r\cos (\theta -t)+r^2}\]
The Poisson integral of $h\in C(T)$ is defined as 
\[p[h](re^{i\theta })=\dfrac{1}{2\pi }\int ^{2\pi }_{0}h(e^{it})P_{r}(\theta -t)\,dt=\dfrac{1}{2\pi }\int ^{2\pi }_{0}h(e^{it})\dfrac{1-r^2}{1-2r\cos (\theta -t)+r^2}\,dt\]
\end{thm}
\vspace{2ex}
\begin{thm}
If $h:T\rightarrow {\bm R}$ is continuous, then $p[h]$ is harmonic in $U$. 
\end{thm}
\vspace{2ex}
\begin{proof}
Recall that $P_{r}(t)=\mathop{\mathrm{Re}}(1+re^{it})/(1-re^{it})$ so that
\[P_{r}(\theta -t)=\mathop{\mathrm{Re}}\Big(\dfrac{1+re^{i(\theta -t)}}{1-re^{i(\theta -t)}}\Big)=\mathop{\mathrm{Re}}\Big(\dfrac{e^{it}+re^{i\theta }}{e^{it}-re^{i\theta }}\Big)=\mathop{\mathrm{Re}}\Big(\dfrac{e^{it}+z}{e^{it}-z}\Big)\]
which is harmonic in $z$. Then
\[p[h](z)=\dfrac{1}{2\pi }\int ^{2\pi }_{0}h(e^{it})P_{r}(\theta -t)\,dt=\mathop{\mathrm{Re}}\Big[\dfrac{1}{2\pi }\int ^{2\pi }_{0}h(e^{it})\dfrac{e^{it}+z}{e^{it}-z}\,dt\Big]\]
\end{proof}
\vspace{2ex}
\begin{defi}
(Dirichlet problem) For a continuous function $f$, is there a $g$ continuous inside and on $C$ such that $f=g$ on $C$ and $g$ is harmonic inside $C$? Yes, and $g$ is the Poisson integral of $f$. 
\end{defi}
\vspace{2ex}
\begin{thm}
Let $f:T\rightarrow {\bm R}$ be continuous. If we define $g:\bar{U}\rightarrow {\bm R}$ by $g(z)=f(z)$ if $z\in T$ and $g(z)=p[f](z)$ if $z\in U$, then $g\in C(\bar{U})$.
\end{thm}
\vspace{2ex}

