\section{Lecture 7 (March 27th)}
\begin{recall}
We have previously seen
	\[\int _{C}f(z)\;dz\]
where $C$ is a contour given by $z=z(t)$ with $z\leq t\leq b$ and $f$ is continuous on $C$. Note that $dz=z'(t)dt=dx+idy$. We could express this integral in two forms,
\begin{itemize}
\item[(i)] \[\int _{C}f(z)\;dz=\int ^{b}_{a}f(z(t))z'(t)\;dt\]
\item[(ii)] \[\int _{C}(u+iv)(dx+idy)=\int_{C}udx-vdy+i\int _{C}vdx+udy \]
\end{itemize}
The first is of the form $\int ^{b}_{a}w(t)\;dt$ where $w (t)=w _{1}(t)+iw _{2}(t)$ where $w_1(t)$ and $w_2(t)$ have real values, and the second is of the line integral form. From the second form, we learn that
\[\int _{-C}f(z)\;dz=-\int _{C}f(z)\;dz\]
where $-C$ denotes the same contour as $C$ but opposite direction. The central theme of complex functions is evaluating the integral in two ways, one through the residue theorem from Cauchy's formula and one through reparametrization. There is great power in comparing these two results. 
\end{recall}
\vspace{2ex}
\begin{ex}
Let $C$ be given by $z=3e^{it}$ where $0\leq t\leq \pi$. Find $\int _{C}z^{1/2}\;dz$. 
	\[\int ^{\pi }_{0}\Big(3e^{it}\Big)^{1/2}i 3e^{it}\;dt=i3\sqrt{3}\int _{0}^{\pi }e^{3ti/2}\;dt=i3\sqrt{3}\Big[\dfrac{2}{3i}e^{3ti/2}\Big]^{\pi }_{0}=-2\sqrt{3}(1+i)\]
\end{ex}
\begin{thm}
\[\Big|\int _{a}^{b}w (t)\;dt\Big|\leq \int _{a}^{b}|w (t)|\;dt\]
\end{thm}
\vspace{2ex}
\begin{proof}
	Let $\int ^{b}_{a}w(t)\;dt=re^{i\theta }$. Then,
	\[\Big|\int ^{b}_{a}w(t)\;dt\Big|=r=e^{-i\theta }\int ^{b}_{a}w(t)\;dt=\mathrm{Re}\;\int _{a}^{b}e^{-i\theta }w(t)\;dt=\int ^{b}_{a}\mathrm{Re}\;\Big(e^{-i\theta }w(t)\Big)dt\]
As we know that $x=\mathrm{Re}\;z\leq |z|=\sqrt{x^2+y^2}$, we have proven the statement for the real part.
\end{proof}
\vspace{2ex}
\begin{thm}
\[\Big| \int _{C}f(z)\;dz\Big|\leq ML\]
	where $M=\mathrm{max}_{z\in C}|f(z)|$ and $L$ is the length of $C$.
\end{thm}
\vspace{2ex}
\begin{proof}
The proof follows from the previous theorem.
\begin{align*}
	\Big|\int _{C}f(z)\;dz\Big|=&\Big|\int _{a}^{b}f(z(t))z'(t)dt\Big|\\
\leq& \int ^{b}_{a}|f(z(t))||z'(t)|\;dt\leq M\int ^{b}_{a}|z'(t)|\;dt\\
	=&M\int ^{b}_{a}\sqrt{\Big(\dfrac{d x}{d t} \Big)^2+\Big(\dfrac{d y}{d t} \Big)^2}dt=ML
\end{align*}
\end{proof}
\vspace{2ex}
\begin{ex}
Let $C$ denote $z=e^{it}$ with $0\leq t\leq \pi $. Then,
	\[\Big|\int _{C}\dfrac{2z+1}{3+z^2}\;dz\Big|\leq \dfrac{3}{2}\times l(C)=\dfrac{3\pi }{2}\] 
on $C$ with
\[\Big|\dfrac{2z+1}{3+z^2}\Big|\leq \dfrac{2|z|+1}{3-|z|^2}=\dfrac{3}{2}\]
noting that $|z|=1$.
\end{ex}
\vspace{2ex}
\begin{ex}
Given $C_{R}=Re^{it}$ with $0\leq t\leq \pi /4$ and $f(z)=e^{iz^2}$, show that 
\[\lim _{R\rightarrow \infty }\int _{C_{R}}f(z)\;dz=0\]
\end{ex}
\vspace{2ex}
\begin{proof}
On $C_{R}$, note that $|f(z)|=|e^{iz^2}|=e^{-R^2\sin 2t}$. Also, $dz=iRe^{it}\;dt$. We then have
\[\Big|\int _{C_{R}}e^{iz^2}\;dz\Big|\leq \int ^{\pi /4}_{0}e^{-R^2\sin 2t}R\;dt=R\int ^{\pi /4}_{0}e^{-R^2\sin 2t}\;dt=\dfrac{R}{2}\int ^{\pi /2}_{0}e^{-R^2\sin \theta }\;d\theta \]
Recall that for $0\leq x\leq \pi /2$,
\[\dfrac{2}{\pi }\leq \dfrac{\sin x}{x}\leq 1\]
We thus conclude that
	\[I\leq \dfrac{R}{2}\int ^{\pi /2}_{0}e^{-R^22t/\pi }\;dt=\dfrac{R}{2}\Big[-\dfrac{\pi }{2R^2}e^{-2R^2t/\pi }\Big]_{0}^{\pi /2}=\dfrac{\pi }{4R}(1-e^{-R^2})\]
and that the integral converges to zero as $R\rightarrow \infty $.
\end{proof}
\vspace{2ex}
\begin{thm}
	If $C$ is a closed contour and there is an analytic function $f$ on $C$ such that $F'=f$ then $\int _{C}f(z)\;dz=0$. 
\end{thm}
\vspace{2ex}
\begin{ex}
Let $C$ be any closed contour (not passing through the origin) and $n\in {\bm Z}$. We have $\int _{C}z^{n}\;dz=0$ for all $n\ne -1$. We cannot allow $n=-1$ as 
\[\dfrac{d }{d z}\dfrac{1}{n+1}z^{n+1}=z^{n} \]
on $C$. 
\end{ex}
\vspace{2ex}
\begin{proof}
\begin{align*}
\int _{C}f(z)\;dz=\int ^{b}_{a}f(z(t))z'(t)\;dt=&\int ^{b}_{a}F'(z(t))z'(t)\;dt\\
=&F(z(b))-F(z(a))=0
\end{align*}
as $z(a)=z(b)$. 
\end{proof}
\vspace{2ex}
\begin{ex}
Consider these three contours where the function $\int _{C}1/z\;dz$ is to be integrated.
\[\begin{cases}
	C_1:\hspace{2ex}z=z(t)=e^{it}\hspace{2ex}0\leq t\leq 2\pi \\
	C_2:\hspace{2ex}z=z(t)=e^{it}\hspace{2ex}0\leq t\leq 4\pi \\
	C_3:\hspace{2ex}z=z(t)=e^{-it}\hspace{2ex}0\leq t\leq 2\pi 
\end{cases}\]
We obtain
\begin{spacing}{2}
\[\begin{cases}
\int _{C_1}\dfrac{1}{z}\;dz=\int _{C_1}\dfrac{1}{e^{it}}ie^{it}dt=2\pi i\\
\int _{C_2}\dfrac{1}{z}\;dz=4\pi i\\
\int_{C_{3}}\dfrac{1}{z}\;dz=-2\pi i
\end{cases}\]
\end{spacing}
\end{ex}
\vspace{2ex}
\begin{defi}
	A contour $C=z:[a,b]\rightarrow {\bm C}$ is called positive oriented if it is counterclockwise. A contour is simple if $z(x)\ne z(y)$ for $x,y\in [a,b)$ with $x\ne y$. A contour is denoted POSCC if it is a positvely oriented simple closed contour.
\end{defi}
\vspace{2ex}
\begin{defi}
Let $C$ be a POSCC (passing the origin) with $n\in {\bm Z}$. Then,
\[\int _{C}z^{n}\;dz=
\begin{cases}
	0\hspace{5ex}\mathrm{if}\hspace{2ex}n\ne -1\\
	0\hspace{5ex}\mathrm{if}\hspace{2ex}n=-1\hspace{2ex}\mathrm{and\ 0\ is\ not\ in\ }C\\
2\pi i\hspace{3ex}\mathrm{if}\hspace{2ex}n=-1\hspace{2ex}\mathrm{and\ 0\ is\ in\ }C
\end{cases}
\]

\end{defi}
\vspace{2ex}

