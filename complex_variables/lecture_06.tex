\section{Lecture 6 (March 25th)}
\begin{defi}
If $z\ne 0$ and $e^{w}=z$, we then define $w=\log z$ ($=\ln|z|+i\mathrm{arg}\;z$). We note that $\log z$ is a multi-valued function. For this reason, we define the principle value of the logarithm as
\[\mathrm{Log}\;z=\ln|z|+\;i\mathrm{Arg}\;z\]
which is uniquely defined for $z\ne 0$ and $-\pi \leq\mathrm{Arg}\;z\leq \pi$. We very often use the principle branch of the logarithm defined as
\[\log\;z=\ln|z|+i\;\mathrm{arg}\;z\]
for $-\pi<\mathrm{arg}\;z<\pi$. The motivation for this inequality is continuity. The function $\mathrm{Log}\; z$ is not continuous at $\mathrm{Arg}\;z=\pi$, which corresponds to the negative $x$-axis. We can define different branches of the logarithm where $\log z=\ln |z|+i\theta $ given the condition $\alpha <\theta <\alpha +2\pi$. In the case of the principle branch, $\alpha =-\pi $. Notice how in this domain, the logarithm function becomes continuous. 
\\\\
Let's now check the differentiability of a  branch of the logarithm. The Cauchy-Riemann equations in polar coordinates were $u_{\theta }=-rv_{r}$ and $v_{\theta }=ru_{r}$. With $u=\ln r$ and $v=\theta $, we find that the conditions are indeed satisfied.
\end{defi}
\vspace{2ex}
\begin{rmk}
If $z_1=r_1e^{i\theta_1}$ and $z_2=r_2e^{i\theta_2}$, we know that
\[z_1z_2=r_1r_2e^{i(\theta_1+\theta_2)}\]
and thus
\begin{align*}
	\log z_1z_2=&\ln r_1r_2+i(\theta_1+\theta_2+2n\pi)\\
	=&\ln r_1+i(\theta_1+2k\pi)+\ln r_2+i(\theta_2+2n\pi)=\log z_1+\log z_2
\end{align*}
by setting $z_1=z_1z_2/z_2$, we can also get
\[\log \Big(\dfrac{z_1}{z_2}\Big)=\log z_1-\log z_2\]
\end{rmk}
\vspace{2ex}
\begin{ex}
\[\log (-1)=\ln |-1|+i\;\mathrm{arg}\;(-1)=i(2n+1)\pi\]
\end{ex}
\vspace{2ex}
\begin{recall}
$z^{c}$ for $c\notin {\bm Q}$ is defined as\[z^{c}=e^{c\log z}\]
We have previously mentioned that $z^{m}$ $(m\in {\bm Z})$ is single valued and that $z^{n/m}$ ($n/m$ is irreducible) has $m$ distinct values. These rules are satisfied by the definition above. We additionally note that we cannot apply power rules when the base isn't $e$. 
\end{recall}
\vspace{2ex}
\begin{ex}
\[i^{2i}=e^{2i\log i}=e^{2i\frac{(1+4n )\pi i}{2}}=e^{-(1+4n )\pi }\]
as
\[\log i=\ln|i|+i\;\mathrm{arg}(i)=\dfrac{(1+4n)\pi i}{2}\]
\end{ex}
\vspace{2ex}
\begin{rmk}
For $c\notin {\bm Z}$, $z^{c}$ is differentiable on any branch of the logarithm and $z_1^{c_1}z^{c_2}=z^{c_1+c_2}$ (since $e^{\alpha_1+\alpha_2}=e^{\alpha_1}e^{\alpha_2}$). Therefore,
\[\dfrac{d }{d z}z^{c}=\dfrac{d }{d z}e^{c\log z}=e^{c\log z}\dfrac{c}{z}=z^{c}\dfrac{c}{z}=cz^{c-1}  \]
\end{rmk}
\vspace{2ex}
\begin{defi}
We define the trigonometric functions as
\begin{spacing}{2}
\[\begin{cases}
\sin z=\dfrac{e^{iz}-e^{-iz}}{2i}\\
\cos z=\dfrac{e^{iz}+e^{-iz}}{2}\\
\tan z=\dfrac{\sin z}{\cos z}
\end{cases}\]
\end{spacing}
These functions are entire and satisfy
\begin{spacing}{2}
\[\begin{cases}
\dfrac{d }{dz}\sin z=\cos z\\
\dfrac{d }{d z}\cos z=-\sin z 
\end{cases}\]
\end{spacing}
Later, we will learn Liouville's theorem that states that bounded entire functions are constant. We will also later learn that all trigonometric identities work in the complex plane too through the identity theorem.
\end{defi}
\vspace{2ex}
\begin{defi}
In a similar fashion, we define the hyperbolic sine and cosine functions as 
\begin{spacing}{2}
\[\begin{cases}
\sinh z =\dfrac{e^{z}-e^{-z}}{2}\\
\cosh z =\dfrac{e^{z}+e^{-z}}{2}
\end{cases}\]
Like the above, all identities for real hyperbolic function hold for complex hyperbolic functions (again, due to the identity theorem). 
\end{spacing}
\end{defi}
\vspace{2ex}
{\bf Chapter 4}\hspace{2ex}Integrals
\\\\
\begin{defi}
All complex integrals are of the form
\[\int _{C}f(z)\;dz\]
where $C$ is a contour (range or image of continuous, piecewise $C^{1}$ curves on $[a,b]$) or disjoint union of contours. We require that $f(z)$ is continuous on the contour $C$ (or bounded discontinuous at finitely many points in $C$). Lastly, we define as  $dz=z'(t)dt$. There are two ways to express $\int _{C}f(z)\;dz$.
\end{defi}
\vspace{2ex}


