\section{Lecture 5 (March 20th)}
\begin{ex}
Determine $c$ so that $u(x,y)=cx^2+x+e^{-x}\cos y$ is harmonic and for such case find an entire function $f$ satisfying $u=\mathrm{Re}\;f$. 
\end{ex}
\vspace{2ex}
\begin{proof}
$u_{xx}+u_{yy}=2c+e^{-x}\cos y-e^{-x}\cos y=2c=0$. To obtain the harmonic conjugate, let $f=u-iv$ and $u_{x}=v_{y}$ with $u_{y}=-v_{x}$.
\[u_{x}=1+e^{-x}\cos y=v_{y}\hspace{2ex}\implies\hspace{2ex}v(x,y)=y-e^{-x}\sin y+\phi (x)\]
this implies
\[v_{x}(x,y)=e^{-x}\sin y+\phi '(x)=-u_{y}=e^{-x}\sin y\]
so that $\phi '(x)$ and $\phi (x)=k$. The entire function $f$ is $u+iv=x+e^{x}\cos y+i(y-e^{-x}\sin y+k)$. To obtain the function in terms of $z$, a good tip is to put $y=0$ and obtain $f(x)$. Replacing $x$ with $z$ grants us the function we wanted. 
\[f(z)=z+e^{-z}\]
\end{proof}
\vspace{2ex}
\begin{recall}
If $f$ is differentiable on an interval $I$ such that $f'\ne 0$ at any $z\in I$, then $f$ is 1-1 on $I$. However, note that the function $f(z)=e^{z}$ has a non-zero derivative but still is not 1-1.
\end{recall}
\vspace{2ex}
\begin{thm}
If $f$ is analytic in a domain $D$ (open \& connected) with $f'(z)=0$ for all $z\in D$, then $f$ is constant in $D$. 
\end{thm}
\vspace{2ex}
\begin{proof}
	If $f=u+iv$ then by C-R equations, $u_{x}=v_{y}$ and $v_{x}=-u_{y}$. Then, $f'=u_{x}+iv_{x}=0$ and $u_{x},v_{x}=0$. By a similar reasoning, $u_{y},v_{y}=0$ on $D$. Therefore, $f$ is constant on $D$. 
\end{proof}
\vspace{2ex}
\begin{thm}
If $f,g$ are differentiable at $z_0$ with $f(z_0)=g(z_0)=0$ and $g'(z_0)\ne 0$, then 
\[\lim _{z\rightarrow z_0}\dfrac{f(z)}{g(z)}=\dfrac{f'(z_0)}{g'(z_0)}\]
\end{thm}
\vspace{2ex}
\begin{proof}
\[\dfrac{f'(z_0)}{g'(z_0)}=\dfrac{\lim _{z\rightarrow z_0}\dfrac{f(z)-f(z_0)}{z-z_0}}{\lim _{z\rightarrow z_0}\dfrac{g(z)-g(z_0)}{z-z_0}}=\lim _{z\rightarrow z_0}\dfrac{\dfrac{f(z)-f(z_0)}{z-z_0}}{\dfrac{g(z)-g(z_0)}{z-z_0}}=\lim _{z\rightarrow z_0}\dfrac{f(z)}{g(z)}\]
\end{proof}
\vspace{2ex}
{\bf Chapter 3}\hspace{2ex}Elementary Functions
\\
\[f(z)=e^{z}=e^{x}e^{iy}=e^{x}(\cos y+i\sin y)\]
Note that $f(z+2\pi i)=f(z)$.
\\
\begin{rmk}
\[\begin{cases}
w=re^{i\theta }\implies |e^{iw^2}|=|e^{ir^2e^{i2\theta }}|=|e^{ir^2(\cos 2\theta +i\sin 2\theta )}|=e^{-r^2\sin 2\theta }\\
z=x+iy\implies |e^{z^2}|=|e^{(x^2-y^2)+2ixy}|=e^{x^2-y^2}
\end{cases}\]
Also, if $|e^{z}|<0$ means that $x<0$.
\end{rmk}
\vspace{2ex}
\begin{ex}
Find all $z=x+iy$ satisfying 
\[e^{z}=1+i\]
The answer is $e^{z}=e^{x}e^{iy}=1+i=\sqrt{2}e^{i\pi/4}$. $e^{x}=\sqrt{2}$ and $e^{iy}=e^{i\pi /4}$ making $x=\ln\sqrt{2}=\ln 2/2$ and $y=\pi/4+2n\pi $.
\end{ex}
\vspace{2ex}


