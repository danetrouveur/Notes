\section{Lecture 17 (May 20th)}
\begin{rmk}
If $f$ has a simple pole at $z_0$, then
\[\mathop{\mathrm{Res}}_{z=z_0}f(z)=\lim _{z\rightarrow z_0}(z-z_0)f(z)=\lim _{z\rightarrow z_0}\dfrac{p(z)}{(q(z)-q(z_0))/(z-z_0)}=\dfrac{p(z_0)}{q'(z_0)}\]
if we have $f(z)=p(z)/q(z)$ with $q(z_0)=0,p(z_0)\ne 0$.
\end{rmk}
\vspace{2ex}
\begin{ex}
For example, take
\[f(z)=\dfrac{e^{iz}z^3}{z^{4}+16}\]
in $\mathop{\mathrm{Im}}z\geq 0$. Observe that
\begin{align*}
z^{4}=-16=&2^{4}e^{\pi i}\\
z=&2\exp \Big(\dfrac{\pi i}{4}+\dfrac{2k\pi i}{4}\Big)
\end{align*}
for $k=0,1,2,3,\ldots $. Then
\[z_1=2\exp \Big(\dfrac{\pi i}{4}\Big)=\sqrt{2}+i\sqrt{2}\quad z_2=2\exp \Big(\dfrac{3\pi i}{4}\Big)=-\sqrt{2}+i\sqrt{2}\]
in $\mathop{\mathrm{Im}}z\geq 0$. $f$ therefore has simple poles at $z_1, z_2$ in the domain, and
\[\mathop{\mathrm{Res}}_{z=z_1}f(z)=\dfrac{e^{iz_1}z_1^3}{4z_1^{3}}=\dfrac{1}{4}e^{iz_1}=\dfrac{1}{4}e^{-\sqrt{2}+i\sqrt{2}}\]
or
\[\mathop{\mathrm{Res}}_{z=z_2}f(z)=\dfrac{e^{iz_2}z_2^3}{4z_2^{3}}=\dfrac{1}{4}e^{iz_2}=\dfrac{1}{4}e^{-\sqrt{2}-i\sqrt{2}}\]
\end{ex}
\vspace{2ex}
\begin{thm}
In complex analysis, we often use upper hemispheres and shells. There are two important theorems regarding the contour integrals at the rim $C_{R}$. 
\begin{itemize}
\item[(i)] Take $f(z)=\dfrac{p(z)}{q(z)}$ ($q\ne 0$ on $C_{R}$) with $\mathop{\mathrm{deg}}q(z)\geq \mathop{\mathrm{p(z)}}+2$. Then 
\[\lim _{R\rightarrow \infty }\int _{C_{R}}\dfrac{p(z)}{q(z)}\,dz=0\]
\item[(ii)] The Jordan lemma, which stems from the fact that
\[\Big|\int _{C_{R}}e^{iz}\,dz\Big|<\pi \]
for all $R>0$. Or, more generally,
\[\Big|\int _{C_{R}}e^{iaz}\,dz\Big|<\dfrac{\pi }{a}\]
for $a>0$. 
\end{itemize}
\end{thm}
\vspace{2ex}
\begin{proof}
\begin{align*}
\int _{C_{R}}e^{iz}\,dz=^{}\int ^{\pi }_{0}e^{iRe^{it}}iRe^{it}\,dt\\
\Big|\int _{C_{R}}e^{iz}\,dz\Big|\leq& \int ^{\pi }_{0}\Big|e^{iRe^{it}}iRe^{it}\Big|\,dt\\
=&R\int ^{\pi }_{0}e^{-R\sin t}\,dt=2R\int ^{\pi /2}_{0}e^{-R\sin t}\,dt\leq \pi (1-e^{-R})
\end{align*}
Noting that
\[\sin t\geq \dfrac{2}{\pi }t\]
for $0\leq t\leq \pi /2$.
\end{proof}
\vspace{2ex}
\begin{thm}
(Jordan lemma) If $|f(z)|\leq M_{R}$ for $z\in C_{R}$ and $\lim _{R\rightarrow \infty }M_{R}=0$, then
\[\lim _{R\rightarrow \infty }\int _{C_{R}}f(z)e^{iaz}\,dz=0\]
for $a>0$ since
\[\Big|\int _{C_{R}}f(z)e^{iaz}\,dz\Big|\leq M_{R}R\int ^{\pi }_{0}e^{-aR\sin t}\,dt\leq \dfrac{M_{R}\pi }{a}\rightarrow 0\]
\end{thm}
\vspace{2ex}
\begin{ex}
Evaluate 
\[\int ^{\infty }_{-\infty }\dfrac{x^2}{(x^2+1)(x^2+4)}\,dx\]
Let
\[f(z)=\dfrac{z^2}{(z^2+1)(z^2+4)}\]
defined on the upper hemisphere with radius $R$ (whole contour $C_{R}$ and $R>2$). By the residue theorem,
\[\int _{C_{R}}f(z)\,dz=2\pi i\Big(\mathop{\mathrm{Res}}_{z=i}f(z)+\mathop{\mathrm{Res}}_{z=2i}f(z)\Big)\]
We find
\[\mathop{\mathrm{Res}}_{z=i}f(z)=\mathop{\mathrm{Res}}_{z=i}\dfrac{\dfrac{z^2}{z^2+4}}{z^2+1}=-\dfrac{1}{6i}\]
and
\[\mathop{\mathrm{Res}}_{z=2i}f(z)=\mathop{\mathrm{Res}}_{z=2i}\dfrac{\dfrac{z^2}{z^2+1}}{z^2+4}=\dfrac{1}{3i}\]
Meanwhile,
\[\lim_{R\rightarrow \infty }\int_{C_{R}}f(z)\,dz=\int ^{\infty }_{-\infty }\dfrac{x^2}{(x^2+1)(x^2+4)}\,dx=0\]
since $\mathop{\mathrm{deg}}(z^2+1)(z^2+4)=\mathop{\mathrm{deg}}z^2+2$, and the above becomes
\[\int _{C_{R}}f(z)\,dz=2\pi i\Big[-\dfrac{1}{6i}+\dfrac{1}{3i}\Big]=\dfrac{\pi }{3}\]
\end{ex}
\vspace{2ex}
\begin{ex}
Evaluate
\[\int ^{\infty }_{-\infty }\dfrac{x^3\sin x}{x^{4}+16}\,dx\]
Define
\[f(z)=\dfrac{e^{iz}z^3}{z^{4}+16}\quad \mathrm{while}\quad \mathop{\mathrm{Im}}f(z)=\dfrac{x^3\sin x}{x^{4}+16}\]
on the upper hemisphere $C_{R}$ with $R>2$. Then,
\[\mathop{\mathrm{Res}}_{z=z_1}f(z)=\dfrac{1}{4}e^{-\sqrt{2}+i\sqrt{2}}\quad \mathop{\mathrm{Res}}_{z=z_2}f(z)=\dfrac{1}{4}e^{-\sqrt{2}-i\sqrt{2}}\]
Finishing off,
\begin{align*}
\int _{C_{R}}f(z)\,dz=2\pi i\Big[\dfrac{1}{4}e^{-\sqrt{2}+i\sqrt{2}}+\dfrac{1}{4}e^{-\sqrt{2}+i\sqrt{2}}\Big]\\
=&\dfrac{\pi i}{2}e^{-\sqrt{2}}\Big(e^{i\sqrt{2}}+e^{-i\sqrt{2}}\Big)\\
=&\pi ie^{-\sqrt{2}}\cos \sqrt{2}
\end{align*}
Meanwhile, 
\begin{align*}
\lim _{R\rightarrow \infty }\int _{C_{R}}f(z)\,dz=&\int ^{\infty }_{-\infty }\dfrac{e^{ix}x^3}{x^{4}+16}\,dx+\lim _{R\rightarrow \infty }\int ^{\pi }_{0}\cdot \,dz\\
=&\int ^{\infty }_{-\infty }\dfrac{x^3\cos x}{x^{4}+16}\,dx+i\int ^{\infty }_{-\infty }\dfrac{x^{3}\sin x}{x^{4}+16}\,dx
\end{align*}
where the term on the first line vanishes due to Jordan's lemma.
\end{ex}
\vspace{2ex}
\begin{ex}
Evaluate, for $a>0$,
\[\int ^{\infty }_{-\infty }\dfrac{\cos x}{x^2+a^2}\,dx\]
Define
\[f(z)=\dfrac{e^{iz}}{z^2+a^2}\]
on $C_{R}$ with $R>a$. By the residue theorem,
\[\int _{C_{R}}f(z)\,dz=2\pi i\mathop{\mathrm{Res}}_{z=ai}f(z)=2\pi i\dfrac{e^{-a}}{2ai}=\pi \dfrac{e^{-a}}{a}\]
On the other hand, 
\[\lim _{R\rightarrow \infty }\int _{C_{R}}f(z)\,dz=\int _{-\infty }^{\infty }\dfrac{e^{ix}}{x^2+a^2}\,dx+\lim _{R\rightarrow \infty }\int _{0}^{\pi }\cdot \,dx=\int ^{\infty }_{-\infty }\dfrac{\cos x}{x^2+a^2}\,dx+i\int ^{\infty }_{-\infty }\dfrac{\sin x}{x^2+a^2}\,dx\]
where the second term goes to zero by Jordan's lemma. So, 
\[\int ^{\infty }_{-\infty }\dfrac{\cos x}{x^2+a^2}\,dx=\pi \dfrac{e^{-a}}{a}\]
Now, put $x=\beta t$ for $\beta >0$, to obtain
\[\int ^{\infty }_{-\infty }\dfrac{\cos \beta t}{\beta ^2t^2+a^2}\beta \,dt=\dfrac{\pi e^{-a}}{a}\]
We succeedingly put $\alpha =a/\beta $ and we get
\[\int ^{\infty }_{-\infty }\dfrac{\cos \beta t}{t^2+\alpha ^2}\,dt=\dfrac{\pi }{\alpha }e^{-\alpha \beta }\]
for $\alpha ,\beta >0$. Lastly, differentiate the function with respect to $t$. 
\[\int ^{\infty }_{-\infty }\dfrac{-\beta \sin \beta t}{t^2+\alpha ^2}\,dt=-\pi e^{-\alpha \beta }\]
For $\beta =1$,
\[\int ^{\infty }_{-\infty }\dfrac{t\sin t}{t^2+\alpha ^2}\,dt=\pi e^{-\alpha }\]
with $\alpha >0$. Taking $\alpha \rightarrow 0^{+}$,
\[\int ^{\infty }_{-\infty }\dfrac{\sin t}{t}\,dt=\pi \]
Similarly, we can differentiate the above with respect to $\alpha $ instead of $t$ to get
\[\int ^{\infty }_{-\infty }\dfrac{x\sin x}{(x^2+a^2)^2}\,dx\]
\end{ex}
\vspace{2ex}
\begin{ex}
Find 
\[\int ^{\infty   }_{0}\dfrac{\sin x}{x}\,dx\]
Define
\[f(z)=\dfrac{e^{iz}}{z}\]
on the shell $C_{\varepsilon ,R}$ ($0<\varepsilon <1$ and $R>1$) for which it is analytic. We find
\[\int _{C_{\varepsilon ,R}}f(z)\,dz=\int ^{R}_{-R}\dfrac{e^{ix}}{x}\,dx+\int ^{\pi }_{0}f(Re^{it})iRe^{it}\,dt+\int _{-R}^{-\varepsilon }\dfrac{e^{ix}}{x}\,dx-\int ^{\pi }_{0}\dfrac{e^{i\varepsilon e^{it}}}{\varepsilon e^{it}}i\varepsilon e^{it}\,dt\]
the second term vanishes due to the Jordan lemma while the last term becomes
\[-\int ^{\pi }_{0}ie^{i\varepsilon e^{it}}\,dt\rightarrow -\pi i\]
as $\varepsilon \rightarrow 0$. 
\end{ex}
\vspace{2ex}

