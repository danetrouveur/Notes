\section{Lecture 10 (May 16th)}
\begin{defi}
(Lie group) A Lie group is defined as a group that is also a manifold whose inverse and multiplication operators are continuous. In studying Lie groups, we consider Lie algebras, which are the logarithms of the Lie group, which satisfying the BCH formula
\[\exp (A)\exp (B)=\exp\Big(A+B+\dfrac{1}{2}[A,B]+\ldots \Big) \]
telling us that we can know the Lie group completely if we know the algebra.
\end{defi}
\vspace{2ex}
\begin{recall}
Meanwhile, we have learned that the gamma matrices satisfy
\[
\Big[\dfrac{1}{4}\gamma _{ab},\gamma _{c}\Big]=\dfrac{1}{2}(\eta _{bc}\gamma _{a}-\eta _{ac}\gamma _{b})=\eta_{a[b}\gamma _{a]}\quad \Big[\dfrac{1}{4}\gamma _{ab},\gamma _{c}\Big]=-\eta _{c[a}\gamma _{b]}
\]
while 
\[\Big[\dfrac{1}{4}\omega ^{ab}\gamma _{ab},\gamma _{c}\Big]=-\omega _{c}^{\ d}\gamma _{d}\quad \Big[\dfrac{1}{4}\omega _{bc}\gamma ^{bc},\gamma ^{a}\Big]=-\omega ^{a}_{\ b}\gamma ^{b}\]
as
\[\mathop{\mathrm{ad}}\Big(\dfrac{1}{4}\omega _{bc}\gamma ^{bc}\Big)\gamma ^{a}=-\omega ^{a}_{\ b}\gamma ^{b}\quad \mathop{\mathrm{ad}}\Big(\dfrac{1}{4}\omega _{bc}\gamma ^{bc}\Big)^{2}\gamma ^{a}=(-\omega ^{a}_{\ b})(-\omega ^{b}_{\ c})\gamma ^{c}\]
which leads to
\[e^{-\omega _{bc}\gamma ^{bc}/4}\gamma^{a}e^{\omega _{bc}\gamma ^{bc}/4}=(e^{\omega })^{a}_{\ b}\gamma ^{b}\]
\end{recall}
\vspace{2ex}
\begin{defi}
Supersymmety (SUSY for short) is defined through Noether symmetries, which branches into supersymmetric QFT (global symmetries) and supergravity (local or gauge symmetries). We know for a fact that Noether symmetry can be taken as a direct product of Poincare symmetry (spacetime symmetry) and internal symmetry (rotation between fields). For example, consider the action
\[\int d^{4}x\,-\eta ^{\mu \nu }\partial _{\mu }\overline{\mathrm{\Phi}}  ^{i}\partial _{\nu }\mathrm{\Phi} ^{j}\delta _{ij}\]
with $i=1,\ldots ,N$. Then, the symmetry has a structure of 
\[\mathrm{SO}(1,3)\otimes U(N)\]
This is otherwise known as the Colemann-Mandela theorem. Supersymmetry states that there is a symmetry between the above bosonic symmetries and fermionic symmetries. In supersymmetry, alike how the generators for the Lorentz group are its Lie algebra, are denoted by
\[\{Q^{\alpha },\overline{Q}_{\beta }\}=\gamma ^{\alpha \mu }_{\ \ \beta }P_{\mu }\]
called the superalgebra. Observe how
\[\delta_1\delta_2B=\delta_1(\varepsilon_2F)=\varepsilon_2\delta_1F=\varepsilon_2\varepsilon_1\partial B\quad \delta_2\delta_1B=\varepsilon_1\varepsilon_2\partial B\]
leading to
\[[\delta_1,\delta_2]B=2\varepsilon_2\varepsilon_1\partial B\]
and
\begin{align*}
[\overline{\varepsilon }_{1}Q,\overline{\varepsilon }_{2}Q]B=&\overline{\varepsilon }_{1}Q\overline{\varepsilon }_{2}QB-\overline{\varepsilon }_{2}Q\overline{\varepsilon }_{1}QB\\
=&\overline{\varepsilon }_{1\alpha }Q^{\alpha }\overline{\varepsilon }_{2\beta }Q^{\beta }B-\overline{\varepsilon }_{2\beta }Q^{\beta }\overline{\varepsilon }_{1\alpha }Q^{\alpha }B\\
=&\overline{\varepsilon }_{2\beta }\overline{\varepsilon }_{1\alpha }(Q^{\alpha }Q^{\beta }B+Q^{\beta }Q^{\alpha }B)\\
=&\overline{\varepsilon }_{2\beta }\overline{\varepsilon }_{1\alpha }\{Q^{\alpha },Q^{\beta }\}B
\end{align*}
\end{defi}
\vspace{2ex}
\begin{rmk}
The super yang-mills action is given by
\[S_{\mathrm{YM}}=\int \dfrac{1}{4}\mathop{\mathrm{Tr}}(F^{\mu \nu }T_{\mu \nu })+\mathop{\mathrm{Tr}}(\overline{\psi }\slashed{D}\psi )\]
where $\slashed{D}\psi =\gamma ^{\mu }(\partial _{\mu }\psi -i[A_{\mu },\psi ])$. Taking 
\begin{align*}
\delta A_{\mu }=\overline{\varepsilon }\gamma _{\mu }\psi \\
\delta \psi =\dfrac{1}{2}F_{\mu \nu }\gamma ^{\mu \nu }\varepsilon 
\end{align*}
we can find that these are indeed symmetries of the action with
\[\delta \mathop{\mathrm{Tr}}(F^2)=2\mathop{\mathrm{Tr}}(F\delta F)\]
\end{rmk}
\vspace{2ex}
\begin{defi}
In SUGRA, we consider local SUSY. With $\varepsilon^{\alpha } (x)$ being a arbitrary function and $\overline{\varepsilon }_{1}\overline{\gamma }^{\mu }\varepsilon _{2}$ as a diffeomorphism, we consider the action
\[\int \sqrt{-g}R+\overline{\psi }_{\lambda }\gamma ^{\lambda \mu \nu }\slash{D}_{\mu }\psi \]
with $\slash{D}=\gamma ^{a}e_{a}^{\ \mu }(\partial _{\mu }+\omega _{\mu })$. The variations are given as
\[\begin{cases}
\delta e_{\mu }^{\ a}=\overline{\varepsilon }(x)\gamma ^{a}\psi _{\mu }\\
\delta \psi _{\mu }=D_{\mu }\varepsilon =\Big(\partial _{\mu }-\dfrac{1}{4}\omega _{\mu ab}\gamma ^{c b}\Big)\varepsilon 
\end{cases}\]
The Killing spinor equation is given by $D_{\mu }\xi =0$ where the Killing equation was given as
\[\mathcal{L}_{\xi }g_{\mu \nu }=\nabla _{\mu }\xi _{\nu }+\nabla _{\nu }\xi _{\mu }=0\]
\end{defi}
\vspace{2ex}
\begin{defi}
(Kosmann derivative) 
\[\hat{\mathcal{L}}_{\xi }\psi ^{\alpha }=\xi ^{\mu }D_{\mu }\psi ^{\alpha }+\dfrac{1}{4}\mathcal{D}_{[a}\xi _{b]}\gamma ^{ab}\psi ^{\alpha }\]
\end{defi}
\vspace{2ex}

