\section{Lecture 3 (March 21st)}
Last class, we learned about different types of symmetries. 
\\
\begin{recall}
Recall that all equations of motions are derived from the least action principle. Electricity and magnetism and the Euler-Lagrange equations are the result of varying the action, $\delta S=0$. In this mannerism, we can understand transformations that contain Noether symmetry as transformations that do not change the action (the minimum will be identical and the same solutions will come out).
\end{recall}
\vspace{2ex}
\begin{defi}
The Einstein-Hilbert action is given as
\[\int d^{D}x\;\Big(\sqrt{-g}R+S_{\mathrm{matter}}\Big)\]
There are multiple types of the action we can choose for matter $S_{\mathrm{matter}}$.
\begin{itemize}
	\item[(i)] For point particles we have the Nambu-Goto action,
	\[S=\int d\lambda \;\Big(-m\sqrt{-g_{\mu \nu }(x)\dot{x}^{\mu }\dot{x}^{\nu }}\Big)\]
For this action, we know that the function $\lambda $ we choose is irrelevant ($\lambda \rightarrow \lambda '(x)$). For massless particles, we take
\[S=\int d\lambda \;\Big(\dfrac{1}{2e}g_{\mu \nu }\dot{x}^{\mu }\dot{x}^{\nu }-\dfrac{1}{2}m^2e\Big)\]
where $e$ are called einbeins. We call this action the Polyakov action. 
\item[(ii)] For a scalar field, we have
	\[\int dx^{D}\sqrt{-g}\;\Big(-\dfrac{1}{2}g^{\mu \nu }\partial _{\mu }\phi \partial _{\nu }\phi -V(\phi )\Big)\]
for a Maxwell field,
\[\int dx^{P}\sqrt{-g}\;\Big(-\dfrac{1}{4}F_{\mu \nu }F_{\rho \sigma }g^{\mu \rho }g^{\nu \rho }\Big)\]
Notice how both fields are coupled with the Einstein-Hilbert action. Generalising this further,
\[\int dx^{D}\sqrt{-g}\;\Big(-\dfrac{1}{2}\nabla _{\lambda }T_{\mu_1\ldots \mu_{p}}\nabla _{\rho }T_{\nu_1\ldots \nu_{p}}g^{\lambda \rho }g^{\mu_1\nu_1}\ldots g^{\mu _{p}\nu _{p}}\Big)\]
\end{itemize}
For why the square root of the negative determinant is added, note the following list of transformations
\begin{spacing}{2}
\[\begin{cases}
d^{D}x\rightarrow d^{D}x'=||\dfrac{\partial x'}{\partial x}||d^{D}x\\
g_{\mu \nu }(x)\rightarrow g'_{\mu \nu }(x')=\dfrac{\partial x^{\rho }}{\partial x'^{\mu }}\dfrac{\partial x^{\sigma }}{\partial x'^{\nu }}g_{\rho \sigma }(x)=g'_{\mu \nu }(x')\\
g=||g||\rightarrow ||g'||=||\dfrac{\partial x}{\partial x'}||^2||g||\\
\sqrt{-g}\rightarrow \sqrt{-g'}=||\dfrac{\partial x}{\partial x'}||\sqrt{-g} 
\end{cases}\]
\end{spacing}
with $dx^{D}\sqrt{-g}=dx'^{D}\sqrt{-g}$ being invariant.
\end{defi}
\vspace{2ex}
\begin{defi}
We define $P$-branes. For $P=0,1,2$ we have particles, strings, and membranes respectively. We then denote the world volume by $\sigma $ (we could call this lines or sheets depending on the $P$ value).The embedding of the world volume into the target spacetime would be expressed as
\[x^{\mu }(\sigma )\]
The more general, and genuine, Nambu-Goto action for $P$-branes is then given as
\[S=-T\int d\sigma ^{P+1}\;\Big(\sqrt{-\mathrm{det}_{(P+1)\times (P+1)}(\delta _{\mu \nu }(x)\partial _{a}x^{\mu }\partial _{b}x^{\nu })}\Big)\]
This action has a world volume diffeomorphism, with $\sigma ^{a}\rightarrow \sigma '^{a}(\sigma )$.
\end{defi}
\vspace{2ex}
\begin{rmk}
We again emphasize that the action sum of the Einstein-Hilbert action and the action for matter is the action we apply the variation principle with varying the metric to obtain the Einstein field equations (or the Euler-Lagrangian). 
\begin{align*}
	\delta |g|=&|g|g^{\mu \nu }\delta g_{\mu \nu }\\
	\delta \sqrt{-g}=&\delta (-g)^{1/2}=\dfrac{1}{2}(-g)^{\dfrac{1}{2}-1}\delta (-g)=\dfrac{1}{2}(-g)^{\dfrac{1}{2}-1}(-gg^{\mu \nu }\delta g_{\mu \nu })\\
=&\dfrac{1}{2}\sqrt{-g}g^{\mu \nu }\delta g_{\mu \nu } 
\end{align*}
\end{rmk}
\vspace{2ex}
\begin{recall}
Recall that the Riemann curvature tensor was given as 
\[R^{\kappa }_{\ \lambda \mu \nu }=\partial _{\mu }\mathrm{\Gamma} ^{\kappa }_{\ \nu \lambda }-\partial _{\nu }\mathrm{\Gamma} ^{\kappa }_{\ \mu \lambda }+\mathrm{\Gamma} ^{\kappa }_{\ \mu \rho }\mathrm{\Gamma} ^{\rho }_{\ \nu \lambda }\mathrm{\Gamma} ^{\rho }_{\ \mu \lambda }\]
we know the variation of the following
\begin{spacing}{1.5}
\[\begin{cases}
	\delta g_{\mu \nu }=g_{\mu \nu }g^{\mu \nu }\delta g_{\mu \nu }\\
	\delta g^{\mu \nu }=-g^{\mu \rho }\delta g_{\rho \sigma }g^{\sigma \nu }\\
\delta \mathrm{\Gamma} ^{\lambda }_{\ \mu \nu }=-\dfrac{1}{2}g^{\lambda \alpha }\delta g_{\alpha \beta }g^{\beta \rho }(\partial _{\mu }g_{\rho \nu }+\partial _{\nu }g_{\rho \mu }-\partial _{\rho }g_{\mu \nu })+\dfrac{1}{2}g^{\lambda \rho }(\partial _{\mu }\delta g_{\rho \nu }+\partial _{\nu }\delta g_{\mu \rho }-\partial _{\rho }\delta g_{\mu \nu })
\end{cases}\]
which reduces to
\[\dfrac{1}{2}g^{\lambda \rho }(\nabla _{\mu }\delta g_{\rho \nu }-\nabla _{\nu }\delta g_{\rho \mu }-\nabla _{\rho }g_{\mu \nu })\]
We finally attempt to find the variation of the Ricci scalar
\begin{align*}
	\delta R^{\kappa }_{\ \lambda \mu \nu }=&\nabla _{\mu }\delta \mathrm{\Gamma} ^{\kappa }_{\ \nu \lambda }-\nabla _{\nu }\delta \mathrm{\Gamma} ^{\kappa }_{\ \mu \lambda }\\
	\delta R_{\lambda \nu }=&\nabla _{\kappa }\delta \mathrm{\Gamma} ^{\kappa }_{\ \nu \lambda }-\nabla _{\nu }\delta \mathrm{\Gamma} ^{\kappa }_{\ \kappa \lambda }\\
	\delta R=&-g^{\mu \rho }\delta g_{\rho \sigma }g^{\sigma \nu }R_{\mu \nu }+g^{\mu \nu }\delta R_{\mu \nu }\\
	=&-\delta g_{\rho \sigma }R^{\rho \sigma }+g^{\mu \nu }(\nabla _{\lambda }\delta \mathrm{\Gamma} ^{\lambda }_{\ \mu \nu }-\nabla _{\mu }\delta \mathrm{\Gamma} ^{\lambda }_{\ \nu \lambda })\\
	\delta (\sqrt{-g}R)=&\sqrt{-g}\dfrac{1}{2}g^{\mu \nu }\delta g_{\mu \nu }R+\sqrt{-g}(-\delta g_{\mu \nu }R^{\mu \nu }+g^{\mu \nu }\nabla _{\lambda }\delta \mathrm{\Gamma}^{\lambda }_{\ \mu \nu }-g^{\mu \nu }\nabla _{\mu }\delta \mathrm{\Gamma} ^{\lambda }_{\ \nu \lambda })\\
=&-\sqrt{-g}\delta g_{\mu \nu }(R^{\mu \nu }-\dfrac{1}{2}g^{\mu \nu }R)+\partial _{\lambda }(\sqrt{-g}g^{\mu \nu }\delta \mathrm{\Gamma} ^{\lambda }_{\ \mu \nu }-g^{\lambda \mu }\sqrt{-g}\delta \mathrm{\Gamma} ^{\nu }_{\ \mu \nu })
\end{align*}
\end{spacing}
\end{recall}
\vspace{2ex}
We then conclude
\[\int \sqrt{-g}\;R=\int\sqrt{-g}\; -G^{\mu \nu }\delta g_{\mu \nu }\]
In the case of matter,
\[\delta \int \mathcal{L}_{\mathrm{matter}}(\phi ,g)=\int \dfrac{\delta \mathcal{L}_{\mathrm{matter}}}{\delta g_{\mu \nu }}\]
and we have
\begin{align*}
	0=&-\sqrt{-g}G^{\mu \nu }+\dfrac{\delta \mathcal{L}_{\mathrm{matter}}}{\delta g_{\mu \nu }} \\
	G^{\mu \nu }=& \dfrac{1}{\sqrt{-g}}\dfrac{\delta \mathcal{L}_{\mathrm{matter}}}{\delta g_{\mu \nu }}=T^{\mu \nu }
\end{align*}
\begin{ex}
What is the energy momentum tensor for a scalar field?
\begin{align*}
&\delta \int \sqrt{-g}\;\Big(\dfrac{1}{2}g^{\mu \nu }\partial _{\mu }\phi \partial _{\nu }\phi -V(\phi )\Big)\\
&=\int\sqrt{-g}\; \dfrac{1}{2}g^{\alpha \beta }\delta g_{\alpha \beta }\Big(\dfrac{1}{2}(\partial \phi )^2-V\Big)+\sqrt{-g}\Big(-\dfrac{1}{2}\delta g_{\alpha \beta }\partial ^{\alpha }\phi \partial ^{\beta }\phi \Big)\\
&=\int \sqrt{-g}\;\dfrac{1}{2}\delta g_{\alpha \beta }\Big(-\partial ^{\alpha }\phi \partial ^{\beta }\phi +g^{\alpha \beta }(\dfrac{1}{2}(\partial \phi )^2-V)\Big)\\
\end{align*}
which results in 
\[T^{\alpha \beta }=\partial ^{\alpha }\phi \partial ^{\beta }\phi -g^{\alpha \beta }\Big(\dfrac{1}{2}(\partial \phi )^2-V\Big)\]
\end{ex}
\vspace{2ex}
\begin{ex}
What about a variation in $\phi $?
\begin{align*}
&\delta \int \sqrt{-g}\;\Big(\dfrac{1}{2}g^{\mu \nu }\partial _{\mu }\phi \partial _{\nu }\phi-V(\phi ) \Big)\\
&=\int \sqrt{-g}\;(g^{\mu \nu }\partial _{\mu }\delta \phi \partial _{\nu }\phi -\delta \phi \dfrac{\partial V}{\partial \phi } )\\
&=\int \sqrt{-g}\;(-\delta \phi )(\nabla _{\mu }(\partial ^{\mu }\phi )+\dfrac{\partial V}{\partial \phi } )
\end{align*}
Thus the equations of motion are 
\[\nabla _{\mu }(\partial ^{\mu }\phi )+\dfrac{\partial V}{\partial \phi }=0 \]
Lets check that the energy momentum tensor is conserved on-shell.
\begin{align*}
	\nabla T^{\mu \nu }=&\nabla _{\mu }\Big[\partial ^{\mu }\phi \partial ^{\nu }\phi -g^{\mu \nu }(\dfrac{1}{2}\partial _{\rho }\phi \partial ^{\rho }\phi -V)\Big]\\
=&\nabla _{\mu }(\partial ^{\mu }\phi )\partial ^{\nu }\phi +\partial ^{\mu }\phi \nabla _{\mu }(\partial ^{\nu }\phi )-\nabla ^{\nu }(\dfrac{1}{2}\partial _{\rho }\phi \partial ^{\rho }\phi -V)\\
=&\nabla _{\mu }(\partial ^{\mu }\phi )\partial ^{\nu }\phi+\partial ^{\mu }\phi \nabla _{\mu }(\partial^{\nu }\phi )-\nabla ^{\nu }(\partial _{\rho }\phi )\partial ^{\rho }\phi +\partial ^{\nu }\phi \dfrac{\partial V}{\partial \phi }\\
=&\Big[\nabla \mu (\partial ^{\mu }\phi )+\dfrac{\partial V}{\partial \phi } \Big]\partial ^{\nu }\phi 
\end{align*}
\end{ex}
\vspace{2ex}
\begin{ex}
What is the energy momentum tensor for a Maxwell field?
 \begin{align*}
 &\delta \int \sqrt{-g}\;-\dfrac{1}{4}F_{\mu \nu }F_{\rho \sigma }g^{\mu \rho }g^{\nu \sigma }\\
 &=\int\sqrt{-g}\;\dfrac{1}{2}g^{\alpha \beta }\delta g_{\alpha \beta }(-\dfrac{1}{4}F^2)+\dfrac{1}{2}F_{\mu \nu }F_{\rho \sigma }g^{\mu \alpha }g^{\rho \beta }\delta g_{\alpha \beta }g^{\nu \sigma }\\
 &=\int \sqrt{-g}\;\dfrac{1}{2}\delta g_{\alpha \beta }\Big(F^{\alpha }_{\ \sigma }F^{\beta \sigma }-\dfrac{1}{4}g^{\alpha \beta }F^2\Big)
 \end{align*}
and we arrive at
\[T^{\alpha \beta }=F^{\alpha }_{\ \rho }F^{\beta \rho }-\dfrac{1}{4}g^{\alpha \beta }F^2\]
The equations of motion are then
\[\nabla _{\mu }F^{\mu \nu }=0\]
Let's see that the tensor is conserved.
\begin{align*}
	&\nabla _{\mu }(F^{\mu }_{\ \rho }+\dfrac{1}{4}g^{\mu \nu }F^2)\\
&=\nabla _{\mu }F^{\mu }_{\ \rho }F^{\nu \rho }+F^{\mu }_{\ \rho }\nabla _{\mu }F^{\nu \rho }-\dfrac{1}{4}\nabla ^{\nu }(F_{\rho \sigma }F^{\rho \sigma })\\
&=(\nabla _{\mu }F^{\mu }_{\ \rho })F^{\nu \rho }+F_{\mu \rho }\nabla ^{\mu }F^{\nu \rho }-\dfrac{1}{2}F_{\rho \sigma }\nabla ^{\nu }F^{\rho \sigma }\\
&=(\nabla _{\mu }F^{\mu }_{\ \rho })F^{\nu \rho }+F_{\mu \rho }(\nabla ^{\mu }F^{\nu \rho }-\dfrac{1}{2}\nabla ^{\nu }F^{\mu \rho })\\
&=(\nabla _{\mu }F^{\mu }_{\ \rho })F^{\nu \rho }+\dfrac{1}{2}F_{\mu \rho }(\nabla ^{\mu }F^{\nu \rho }-\nabla ^{\rho }F^{\nu \mu }-\nabla ^{\nu }F^{\mu \rho })\\
&=(\nabla _{\mu }F^{\mu}_{\rho })F^{\nu \rho }-\dfrac{1}{2}F_{\mu \rho }(\nabla ^{\mu }F^{\rho \nu }+\nabla ^{\rho }F^{\nu \mu }+\nabla ^{\nu }F^{\mu \rho })
\end{align*}
\end{ex}
\vspace{2ex}

