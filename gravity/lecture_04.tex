\section{Lecture 4 (March 28th)}
\begin{recall}
Previously, we learned the general action
\[\int d^{D}x\;\sqrt{-g}R+\mathcal{L}_{\mathrm{matter}}(\phi ,A_{\mu },\ldots )\]
We now study Fermions coupled with gravity (GR) and Clifford algebra (gamma matrices) that describes them.
\end{recall}
\vspace{2ex}
\begin{defi}
The Klein-Gordan equation is the Lagrangian for scalar fields and is the relativistic wave equation. We know the Schrodinger equation as a non-relativistic wave equation, 
\[E=\dfrac{p^2}{2m}+V\hspace{2ex}\longrightarrow  \hspace{2ex}E\psi =\Big[\dfrac{(-i\hbar \nabla )^2}{2m}+V\Big]\mathrm{\Psi} =0\]
in parallel,
\[p_{\mu }p^{\mu }\hspace{2ex}\longrightarrow  \hspace{2ex}[(-i\hbar \partial _{\mu })^2+m^2]\phi =0\]
This equation had certain problems, with possibly negative energies. Dirac then though of an equation that was linear in momentum, 
\[(\gamma ^{\mu }p_{\mu }+m)\psi =0\]
which is the Dirac equation with $p_{\mu }=-\partial _{\mu }$. By multiplying $(\gamma ^{\nu  }p_{\nu  }+m)$ on both sides, we have
\begin{align*}
	0=&(\gamma ^{\mu}p_{\mu }\gamma ^{\nu }p_{\nu }-m^2)\psi \\
	=&\Big(\dfrac{1}{2}\Big(\gamma ^{\mu }\gamma ^{\nu }+\gamma ^{\nu }\gamma ^{\mu }\Big)p_{\mu }p_{\nu }-m^2\Big)\psi 
\end{align*}
we require that the gamma matrices satisfy the following relation such that it aligns with the Klein-Gordan equation
\[\gamma ^{\mu }\gamma ^{\nu }+\gamma ^{\nu }\gamma ^{\mu }=2\eta ^{\mu \nu }\]
which defines the Clifford algebra. The elements of the Clifford algebra are called gamma matrices. We can denote the condition compactly as
\[\{\gamma ^{a},\gamma ^{b}\}=\gamma ^{a}\gamma ^{b}+\gamma ^{b}\gamma ^{a}=2\eta ^{ab}\]
We have also expressed the relation locally, using local coordinate bases. By substituting
\[
f = \frac{1}{2}(\gamma^1 + i\gamma^2), \qquad
\bar{f} = \frac{1}{2}(\gamma^1 - i\gamma^2)
\]
we alternatively have
\[\{f,\bar{f}\}=1\]
\end{defi}
\vspace{2ex}
\begin{thm}
	If a square matrix $M$ satisfies $M^2=\lambda ^2I$ ($\lambda \ne 0$), then $M$ is diagonalisable. Further, if there is another invertible matrix $N$ which anti-commutes with $M$, that is, $\{M,N\}=0$, then $M$ is diagonalisable as
\[M=S\begin{pmatrix}
	\lambda &0\\0&-\lambda 
\end{pmatrix}S^{-1}\hspace{5ex}\mathrm{Tr}\;M=0
\]
\end{thm}
\vspace{2ex}
\begin{defi}
Define $\mathrm{\Gamma} ^{A}$ as the set of all possible gamma matrices up to the $D$th order, denotable as
\[\mathrm{\Gamma} ^{A}=\{I,\gamma ^{a_1},\gamma ^{a_1a_2},\ldots ,\gamma ^{a_{1}\ldots a_{D}} \}\]
for $a_1<\ldots <a_{D}$. Then, the number of possible $\mathrm{\Gamma} $'s is $2^{D}$. Note that $\mathrm{\Gamma} ^{A}$ is closed under multiplication up to the sign. Because of the above theorem, 
\[\mathrm{Tr}\;\mathrm{\Gamma} ^{A}=\begin{cases}
0\hspace{5ex}\mathrm{\Gamma} ^{A}\ne I\\
1\hspace{5ex}\mathrm{\Gamma} ^{A}=I
\end{cases}\]

\end{defi}
\vspace{2ex}

