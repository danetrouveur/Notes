\section{Lecture 6 (April 11th)}
\begin{rmk}
We take 
\[(\gamma ^{\mu })^{\dagger}=\gamma _{\mu }\]
and
\[\eta =\mathrm{diag}\;(\underbrace{+,\ldots ,+}_{\mathrm{tim}},\underbrace{-,\ldots ,-}_{\mathrm{space}})\]
Last class, we have derived that for 
\[A=\sqrt{(-1)^{t(t-1)/2}}\gamma ^{1}\gamma^{2}\ldots \gamma ^{t}\]
we have
\[A=A^{-1}=A^{\dagger}\]
and
\begin{align*}
(\gamma ^{\mu })^{\dagger}=&\gamma _{\mu }=(-1)^{t+1}A\gamma ^{\mu }A^{-1}\\
(-1)^{t+1}(\gamma ^{\mu })^{\dagger}=&A\gamma ^{\mu }A^{-1}\\
(-1)^{t}(\gamma ^{\mu })^{\dagger}=&\gamma ^{\mu }
\end{align*}
We have
\begin{align*}
\gamma ^{2(D+1)}=&1\\
\gamma ^{1}\gamma ^{2}\ldots \gamma ^{D}=&(-1)^{D(D-1)/2}\gamma ^{D}\ldots \gamma ^{2}\gamma ^{1}\\
(\gamma ^{1}\gamma ^{2}\ldots \gamma ^{D})^2=&(-1)^{D(D-1)/2}\gamma ^{1}\gamma ^{2}\ldots \mathrm{\Gamma} ^{D}\gamma^{D}\ldots \gamma ^{2}\gamma ^{1}\\
=&(-1)^{D(D-1)/2}(-1)^{S}=(-1)^{D/2-S}=(-1)^{t-s/2}
\end{align*}
We conclude that
\[\gamma ^{(D+1)}=\sqrt{(-1)^{t-s/2}}\gamma ^{1}\gamma ^{2}\ldots \gamma ^{D}\]
We can try Hermitian conjugating the array obtaining
\[(\gamma ^{1}\gamma ^{2}\ldots \gamma ^{D})^{\dagger}=(\gamma _{D}\ldots \gamma _{2}\gamma _{1})=(-1)^{D/2}\gamma _{1}\gamma _{2}\ldots \gamma _{D}=(-1)^{D/2-s}\gamma ^{1}\gamma ^{2}\ldots \gamma ^{D}=(-1)^{t-s/2}\gamma ^{1}\gamma ^{2}\ldots \gamma ^{D}\]
and we have
\[\gamma ^{(D+1)}=(\gamma ^{(D+1)})^{\dagger}=(\gamma ^{(D+1)})^{\dagger}\]
An important feature is that
\begin{align*}
\{\gamma ^{(D+1)},\gamma ^{\mu }\}=&0\\
\gamma ^{(D+1)}\gamma ^{\mu }+\gamma ^{\mu }\gamma^{(D+1)}=&0\\
\gamma ^{(D+1)}\gamma ^{\mu }(\gamma ^{(D+1)})^{-1}=-\gamma ^{\mu }
\end{align*}
from which we can derive 
\[(-1)^{t}(\gamma ^{\mu })^{\dagger}=A\gamma ^{(D+1)}\gamma ^{\mu }(A\gamma ^{(D+1)})^{-1}\]
we can diagonalise an odd power gamma matrix
\[\gamma ^{(D+1)}=\begin{pmatrix}
1&0\\0&-1
\end{pmatrix}
\]
where $+1$ is for a chrial spinor (Weyl spinor) and $-1$ for a anti-chiral spinor. An arbitrary matrix that anti-commutes with this matrix is of the form
\[\gamma ^{\mu }=\begin{pmatrix}
0&\sigma ^{\mu }\\
\bar{\sigma }^{\mu }&0
\end{pmatrix}
\]
and for two different gamma matrices, we have the equivalent identities
\begin{align*}
\sigma ^{\mu }\bar{\sigma }^{\nu }+\sigma ^{\nu }\bar{\sigma }^{\mu }=&2\eta ^{\mu \nu }\\
\bar{\sigma }^{\mu }\sigma ^{\nu }+\bar{\sigma }^{\nu }\sigma ^{\mu }=&2\eta^{\mu \nu } 
\end{align*}
\end{rmk}
\vspace{2ex}
\begin{ex}
For the metric $(-,+,+,+)$, we have
\[\gamma ^{0}=\begin{pmatrix}
0&1\\-1&0
\end{pmatrix}
\]
and
\[\gamma ^{i}=\begin{pmatrix}
0&\sigma ^{i}\\\bar{\sigma }^{i}&0
\end{pmatrix}
\]
for $i=1,2,3$ and $\bar{\sigma }^{\mu }=(\sigma _{\mu })^{\dagger}$. For $\gamma ^{5}$ we have
\[\begin{pmatrix}
1&0\\0&-1
\end{pmatrix}
\]
\end{ex}
\vspace{2ex}
\begin{defi}
We define the similarity transformations $B_{\pm}$ as
\[\pm (\gamma ^{\mu })^{\ast}=B_{\pm}\gamma ^{\mu }(B_{\pm})^{-1}\]
Applying the complex conjugate,
\begin{align*}
\gamma ^{\mu }=&B_{\pm}^{*}(\gamma ^{\mu })^{*}(B_{\pm}^{*})^{-1}\\
\gamma ^{\mu }=&B_{\pm}^{*}B_{\pm}\gamma ^{\mu }(B_{\pm})^{-1}(B_{\pm}^{*})^{-1}\\
=&(B^{*}_{\pm}B_{\pm})\gamma ^{\mu }(B^{*}_{\pm}B_{\pm})^{-1}
\end{align*}
For this to work, we require that
\[B^{*}_{\pm}B_{\pm}=\varepsilon _{\pm}I\]
In addition,
\begin{align*}
\pm(\gamma ^{\mu })^{\dagger\;\ast}=&(B_{\pm}^{\dagger})^{-1}(\gamma ^{\mu })^{\dagger}B_{\pm}^{\dagger}\\
\pm( \gamma _{\mu })^{\ast}=&(B_{\pm}^{\dagger})^{-1}\gamma _{\mu }B_{\pm}^{\dagger}\\
\pm(\gamma ^{\mu })^{\ast}=&(B^{\dagger}_{\pm})^{-1}\gamma ^{\mu }B_{\pm}^{\dagger}\\
=&B_{\pm}\gamma ^{\mu }B_{\pm}^{-1}
\end{align*}
which implies that
\[B^{\dagger}_{\pm}B_{\pm}\gamma ^{\mu }(B^{\dagger}_{\pm}B_{\pm})^{-1}=\gamma ^{\mu }\]
We now find that the matrix $B_{\pm}$ is positive definite and that
\[B^{\dagger}_{\pm}B_{\pm}=|\lambda |^{2}I\]
Now we can rescale $B_{\pm}/|\lambda |$ to obtain
\[B^{\dagger}_{\pm}B_{\pm}=1\]
we now have two conditions that result in 
\begin{align*}
B_{\pm}^{*}=&\varepsilon _{\pm}B_{\pm}^{\dagger}\\
B_{\pm}=&\varepsilon _{\pm}^{*}B^{t}_{\pm}\\
B_{\pm}^{t}=&\varepsilon _{\pm}^{*}B_{\pm}=(\varepsilon _{\pm}^2)^{*}B_{\pm}^{t}
\end{align*}
resulting in
\[\varepsilon _{\pm}^2=1\]
We therefore have
\begin{align*}
\pm(\gamma ^{\mu })^{*}=&B_{\pm}\gamma ^{\mu }B_{\pm}^{-1}\\
B_{\pm}^{\dagger}B_{\pm}=1\\
B_{\pm}^{t}=&\varepsilon _{\pm}B_{\pm}\\
\varepsilon _{\pm}=&(-1)^{(s-t)(s-t\pm 2)/8}
\end{align*}
\end{defi}
\vspace{2ex}
\begin{rmk}
Observe that
\begin{align*}
(\gamma ^{\mu })^{t}=&(\gamma ^{\mu \,*})^{\dagger}=(\gamma ^{\mu \,\dagger})^{*}\\
=&(\pm B_{\pm}\gamma ^{\mu }B^{-1}_{\pm})^{\dagger}\\
=&\pm B_{\pm}(\gamma ^{\mu })^{\dagger}B_{\pm}^{-1}\\
=&\pm (-1)^{t+1}B_{\pm}A\gamma ^{\mu }(B_{\pm}A)^{-1}
\end{align*}
and define $C_{\pm}=B_{\pm}^{t}A$ to obtain
\[\pm(-1)^{t+1}(\gamma ^{\mu })^{t}=C_{\pm}\gamma ^{\mu }C_{\pm}^{-1}\]
with $C_{\pm}^{\dagger}C_{\pm}=1$. Defining $\zeta = \pm(-1)^{t+1}$, we can derive that
\begin{align*}
C_{\pm}^{t}=&(\pm 1)^{t}(-1)^{t(t-1)/2}C_{\pm}\\
=&(-1)^{D(D-2\zeta)/8}C_{\pm}
\end{align*}
Now,
\begin{align*}
\varepsilon _{\pm}(\pm 1)^{t}(-1)^{t(t-1)/2}
=&(\pm 1)^{t}(-1)^{(s-t)(s-t\pm 2)/8 +t(t-1)/2}\\
=&(\pm 1)^{t}(-1)^{(D-2t)(D-2t\pm 2)/8+t(t-1)/2}\\
=&(\pm 1)^{t}(-1)^{[D^2-4tD+4t^2\pm 2(D-2)t]/8+t(t-1)/2}\\
=& (\pm 1)^{t}(-1)^{(D^2\pm 2D)/8+t^2/2-Dt/2\mp t/2+t^2/2-t/2}\\
=& (\pm 1)^{t}(-1)^{(D^2\pm 2D)/8+t^2/2-Dt/2\mp t/2+t^2/2-t/2}\\
=&(-1)^{D(D\pm 2)/8}(\pm 1)^{t}(-1)^{t^2-Dt/2-(1\pm 1)t/2}\\
=&(-1)^{D(D-4t\pm 2)/8}
\end{align*}
\end{rmk}
\vspace{2ex}
\begin{rmk}
To conclude,
\begin{align*}
\gamma ^{D+1}=&\sqrt{(-1)^{(t-s)/2}}\gamma^{12\ldots D}\\
(\gamma ^{D+1})^{\dagger}=&(-1)^{t}A\gamma ^{(D+1)}A^{-1}\\
=&(-1)^{t}(A\gamma ^{(D+1)})\gamma ^{(D+1)}(A\gamma ^{(D+1)})^{-1}\\
(\gamma ^{(D+1)})^{t}=&(-1)^{D/2}C_{\pm}\gamma ^{(D+1)}C_{\pm}^{-1}\\
(\gamma ^{(D+1)})^{*}=&(-1)^{(t-s)/2}\gamma ^{(D+1)}B_{\pm}^{-1}
\end{align*}
The implication of this to odd dimensions is that there is no more parity than $\{A_{\pm},B_{\pm},C_{\pm}\}$.
\end{rmk}
\vspace{2ex}

