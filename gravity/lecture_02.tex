\section{Lecture 2 (March 14th)}
\begin{recall}
Last class, we have learned about active and passive diffeomorphisms. Passive can be taken as change of charts while active can be taken both infinitesimally and finitely by $\mathcal{L}_{\xi}$ and $\exp\{\mathcal{L}_{\xi }\}$ respectively.
\end{recall}
\vspace{2ex}
\begin{defi}
Active transformations are closely related to Noether's theorem, which we will introduce now on. An action is defined as the integral of a Lagrangian like the following.
\[S=\int dx^{D}\;\mathcal{L}(\phi ,\partial _{\mu }\phi )\]
In this context, Noether symmetry refers to invariance in the Lagrangian as the field changes $\phi (x)\rightarrow \phi '(x)$ up to a surface integral.
\[\mathcal{L}(\phi ,\partial _{\mu }\phi )\rightarrow \mathcal{L}(\phi',\partial _{\mu }\phi ')=\mathcal{L}(\phi ,\partial _{\mu }\phi )+\partial _{\mu }K^{\mu }\]
We will investigate examples later. The action would then be invariant up to a bounder term!
\[S\rightarrow S'=S+\oint dA_{\mu }\;K^{\mu }\]
When we deal with actions, we normally refer to the Euler-Lagrange equations that are derived from the least action principle. Looking at the Lagrangian above, $\phi \rightarrow \phi +\delta \phi $ results in 
\begin{align*}
\delta \mathcal{L}(\phi ,\partial _{\mu }\phi )=&\delta \phi \dfrac{\partial \mathcal{L}(\phi ,\partial _{\mu }\phi )}{\partial \phi }+\delta (\partial _{\mu }\phi ) \dfrac{\partial \mathcal{L}(\phi ,\partial _{\mu }\phi )}{\partial (\partial _{\mu }\phi) }\\
=&\delta \phi \dfrac{\partial \mathcal{L}}{\partial \phi }+\partial _{\mu }\delta \phi \dfrac{\partial \mathcal{L}}{\partial( \partial _{\mu }\phi )}\\
=&\delta \phi \dfrac{\partial \mathcal{L}}{\partial \phi }+\partial _{\mu }\Big[\delta \phi \dfrac{\partial \mathcal{L}}{\partial( \partial _{\mu }\phi )} \Big]-\delta \phi \partial _{\mu }\Big(\dfrac{\partial \mathcal{L}}{\partial (\partial _{\mu }\phi )} \Big)\\
=&\delta \phi \Big(\dfrac{\partial \mathcal{L}}{\partial \phi }-\partial _{\mu }\dfrac{\partial \mathcal{L}}{\partial (\partial _{\mu }\phi )}  \Big)+\partial _{\mu }\Big(\delta \phi \dfrac{\partial \mathcal{L}}{\partial (\partial _{\mu }\phi )} \Big)
\end{align*}
We then arrive at
\[0=\delta S=\int dx^{D}\;\delta \mathcal{L}=\int dx^{D}\;\delta \phi \Big(\dfrac{\partial L}{\partial \phi }-\partial _{\mu }\Big(\dfrac{\partial \mathcal{L}}{\partial (\partial _{\mu }\phi )} \Big) \Big)+\mathrm{total\;derivative}\]
Noether current density is then defined as
\[J^{\mu }=\sum _{\phi }\delta \phi \dfrac{\partial \mathcal{L}}{\partial (\partial _{\mu }\phi )}-K^{\mu } \]
Note that Noether current density is on-shell conserved,
\begin{align*}
	\partial _{\mu }J^{\mu }=&\partial _{\mu }\Big[\delta \phi \dfrac{\partial \mathcal{L}}{\partial (\partial_{\mu }\phi )}-K^{\mu } \Big]\\
=&\partial _{\mu }\delta \phi \dfrac{\partial \mathcal{L}}{\partial (\partial _{\mu }\phi )}+\delta \phi \partial _{\mu }\Big(\dfrac{\partial \mathcal{L}}{\partial (\partial _{\mu }\phi )}\Big)-\partial _{\mu }K^{\mu }\\
=&\delta L+\delta \phi \Big[\partial _{\mu }\Big(\dfrac{\partial \mathcal{L}}{\partial (\partial _{\mu }\phi )} \Big)-\dfrac{\partial \mathcal{L}}{\partial \phi } \Big]-\partial _{\mu }K^{\mu }
\end{align*}
In sum, Noether theorem states that if the following off-shell equality holds,
\[\delta \mathcal{L}=\partial _{\mu }K^{\mu }\]
the following on-shell identity holds also.
\[\partial _{\mu }J^{\mu }=0\]
Consequently, Noether charge is defined as
\[Q=\int  dx^{D-1}\;J^{0}\]
Derivating this, 
\[\dfrac{d Q}{d t}=\int dx^{D-1}\;\partial _{t}J^{t}=\int dx^{D-1}\;-\sum ^{D-1}_{i=1}\partial _{i}J^{i}=-\oint dA_{i}\;J^{i}\approx 0  \]
\end{defi}
\vspace{2ex}
\begin{ex}
Consider the Lagrangian of a free particle,
\[\int dt\;\dfrac{1}{2}m\dot{x}^2\]
with $\mathcal{L}=m\dot{x}^2/2$. The Noether charge then becomes, according to our definition above,
\[Q=\dfrac{\partial \mathcal{L}}{\partial \dot{x}}=p \]
Consider simply translation, $x\rightarrow x'=x+C$, $\delta x=C$ and the Lagrangian transforms like $\mathcal{L}\rightarrow \mathcal{L}'=\mathcal{L}$ with $K^{t}=0$. However, consider $x(t)\rightarrow x'(t)=x(t+a)$, then $\delta x=\dot{x}$. Then,
\[\delta \mathcal{L}=\dot{x}\delta \dot{x}=m\dot{x}\ddot{x}=\dfrac{d }{d t}\mathcal{L} \]
Noether charge then becomes
\[Q=J^{t}=\dot{x}\dfrac{\partial \mathcal{L}}{\partial \dot{x}}-\mathcal{L}=H \]
\end{ex}
\vspace{2ex}
\begin{thm}
An important remark about Noether symmetry is that it preserves the solution. A brief sketch of a proof would be the following. Consider the following result from the least action principle
\[0\underbrace{=}_{\mathrm{on-shell}}\delta S=\int dx^{D}\;\delta \phi \Big[\dfrac{\partial \mathcal{L}}{\partial \phi }-\partial _{\mu }\Big(\dfrac{\partial \mathcal{L}}{\partial (\partial \phi )} \Big) \Big]\]
On the other hand, Noether symmetry refers to how 
\[S(\phi )\rightarrow S(\phi ')\underbrace{=}_{\mathrm{off-shell}}S(\phi )\]
We see how the there are multiple values of $\phi $ that would result in the same action and thus satisfying the Euler-Lagrange equations. 
\end{thm}
\vspace{2ex}
\begin{rmk}
Other than continuous symmetry and infinitesimal symmetry, we can also divide symmetries into global and gauge symmetries (or local symmetries). What do we mean by this? When we refer to global symmetries, we mean that parameters are constant. When we refer to local symmetries, we mean that parameters are arbitrary functions of spacetime coordinates $x^{\mu }$. 
\end{rmk}
\vspace{2ex}
\begin{ex}
Our first experience with gauge symmetry came from electromagnetism, where we have shifted our potential function like the following
\[A_{\mu }\rightarrow A'_{\mu }=A_{\mu }+\partial _{\mu }\Lambda  \]
where $\Lambda(x) $ is a function of $x^{\mu }$. This Results in 
\[F_{\mu \nu }\rightarrow F'_{\mu \nu }=F_{\mu \nu }\]
with the Lagrangian
\[\mathcal{L}=\dfrac{1}{4}F_{\mu \nu }F^{\mu \nu }=\dfrac{1}{2}({\bm E}^2-{\bm B}^2)\]
\end{ex}
\vspace{2ex}
The Euler-Lagrange equations were again given as
\[\partial _{\mu }\Big(\dfrac{\partial \mathcal{L}}{\partial (\partial _{\mu }\phi )} \Big)-\dfrac{\partial \mathcal{L}}{\partial \phi } =0\]
Classical mechanics suggested that with initial conditions for $\phi $ and $\partial _{\mu }\phi $ on a Cauchy surface, determinism dictates. But the fact that we can change gauges throughout time, we find a need to group gauges together through equivalence relations, or gauge orbits the reason why we call these orbits is because we can go from on solution to another using different gauges. One way to see local symmetries are as non-physical symmetries, meaning that they result in no physical difference but simply mathematical. On the other hand, global symmetries would create physical differences.
\\
\begin{ex}
Another good example of local symmetry are identical particles,
\[\dfrac{1}{2}m\dot{x}_{1}^2+\dfrac{1}{2}m\dot{x}_{2}^2\]
Where, in quantum mechanics, there would be no difference in the wave function if we swap the particles' position. 
\end{ex}
\vspace{2ex}
\begin{ex}
Consider the Lagrangian for electromagnetism
\[\mathcal{L}=-\dfrac{1}{4}F_{\mu \nu }F^{\mu \nu }=-\dfrac{1}{4}(\partial _{\lambda }A_{\rho }-\partial _{\rho }A_{\lambda })F^{\lambda \rho }\]
With $\delta A_{\mu }=\partial _{\mu }\Lambda $ and $\partial _{\mu }j^{\mu }=0$. The Noether current becomes
\begin{align*}
	J^{\mu }=&\dfrac{\partial \mathcal{L}}{\partial (\partial _{\mu }A_{\nu })}\delta A_{\nu }-K^{\mu } \\
=&-F^{\mu \nu }\partial _{\nu }\Lambda -\Lambda j^{\mu }
\end{align*}
with the Noether charge
\begin{align*}
	Q=&\int dx^3\;J^{0}\\
	=&\int dx^3\;-F^{0i}\partial _{i}\Lambda -\Lambda j^{0}\\
	=&\int dx^3\;{\bm E}\cdot \nabla \Lambda -\Lambda \rho \\
	=&\int dx^3\;-\Lambda (\nabla \cdot {\bm E}+\rho )=0
\end{align*}
As seen in this example, Noether charges of non-physical local symmetries are trivial.
\end{ex}
\vspace{2ex}

