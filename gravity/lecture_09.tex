\section{Lecture 8 (May 8th)}
\begin{recall}
We have learned the Weyl algebra, constructed of matrices that satisfy
\begin{align*}
\gamma ^{a}\gamma ^{b}+\gamma ^{b}\gamma ^{a}=2\eta ^{ab}\quad
(\gamma ^{a})^{\alpha }_{\ \beta }\quad
\psi ^{\alpha }
\end{align*}
where $a,b,c,\ldots =0,1,2,\ldots ,D-1$ and $\alpha, \beta, \gamma ,\ldots =1,2,\ldots ,2^{D/2}$. Vielbeins (when $D=4$, vierbeins) are transformations that transform to the locally Euclidean frame.
\[g_{\mu \nu }(x)=e_{\mu }^{\ a}(x)e_{\nu }^{\ b}(x)\eta _{ab}\]
Notice that there is local Lorentz symmetry (a gauge symmetry) of the above expression, where we can transform the vierbeins like the following
\[e^{\ a }_{\mu }(x)\rightarrow (e')_{\mu }^{\ a }(x)=L^{a}_{\ b}(x)e_{\mu }^{\ b}(x)\]
Consider the infinitesimal and we have
\[\begin{cases}
\delta e_{\mu }^{\ a}=\omega ^{a}_{\ b}(x)e_{\mu }^{\ b}\\
\delta \psi ^{\alpha }=-\dfrac{1}{4}\omega _{ab}(x)(\gamma ^{ab})^{\alpha }_{\ \beta }\psi ^{\beta }
\end{cases}\]
where $\omega _{ab}=-\omega_{ba}$. In curved space time, we have
\[\gamma ^{\mu }\gamma ^{\nu }+\gamma ^{\nu }\gamma ^{\mu }=g^{\mu \nu }\quad \mathrm{where}\quad \gamma ^{\mu }=\gamma ^{a}e_{a}^{\ \mu }\]
To make the equation gauge covariant, we want the master derivative,
\[\mathcal{D}_{\mu }=\partial _{\mu }+\gamma _{\mu }+\omega _{\mu }\]
where the Christoffel connection is for the diffeomorphism whereas the spin connection is for the local lorentz symmetry. For a vector, we can write
\[\mathcal{D}_{\mu }V^{\nu }=\nabla _{\mu }V^{\nu }\]
At this point, we mention that the Christoffel symbols were determined by both metric compatibility and torsionlessness. For the spin connection, we require vierbein compatibility,
\[\mathcal{D}_{\mu }e_{\nu }^{\ a}=\nabla _{\mu }e_{\mu }^{\ a}+\omega _{\mu\ b }^{\ a}e_{\nu }^{\ b}=\partial _{\mu }e_{\nu }^{\ a}-\mathrm{\Gamma} ^{\rho }_{\mu \nu }e_{\rho }^{\ a}+\omega _{\mu\ b }^{\ a}=0\]
where the coefficients can be promptly calculated as
\[\omega _{\mu \ b}^{\ a}=-e_{b}^{\ \nu }\nabla _{\mu }e_{\nu }^{\ a}=-\nabla _{\mu }(e_{b}^{\ \nu }e_{\nu }^{\ a})+\nabla _{\mu }e_{b}^{\ \nu }e_{\nu }^{\ a}\]
and
\[\omega _{\mu ab}=e_{\nu a}\nabla _{\mu }e_{b}^{\ \nu }\]
We notice that the expression is antisymmetric with respect to the exchange of indices $ab$. We find that a true invariant Dirac equation is given as
\[(\gamma ^{\mu }\mathcal{D}_{\mu }-m)\psi =0\quad \gamma ^{\mu }\mathcal{D}_{\mu }\psi =\gamma ^{a}e_{a}^{\ \mu }\Big(\partial _{\mu }\psi +\dfrac{1}{4}\omega _{\mu ab}\gamma ^{ab}\psi \Big)\]
It is the remaining task to see whether the Dirac equation is really covariant,
\[\delta (\gamma ^{\mu }\mathcal{D}_{\mu }\psi )=\delta \Big(\gamma ^{a}e_{a}^{\ \mu }\Big(\partial _{\mu }\psi +\dfrac{1}{4}\omega _{\mu ab}\gamma ^{ab}\psi \Big)\Big)=\dfrac{1}{4}\omega _{ab}\gamma ^{ab}(\gamma ^{\mu }\mathcal{D}_{\mu }\psi )\]
We see that
\begin{align*}
\delta \omega _{\mu ab}=&\delta (e^{\nu }_{\ a}\nabla _{\mu }e_{b\nu })=\delta (e^{\nu }_{\ a}\partial _{\mu }e_{b\nu }-e^{\nu }_{\ a}\mathrm{\Gamma}^{\rho }_{\mu \nu }e_{b\rho })\\
=&\omega _{a}^{\ c}e^{\nu }_{\ c}\partial _{\mu }e_{b\nu }+e^{\nu }_{\ a}\partial _{\mu }(\omega _{b}^{\ c}e_{c\nu })-\omega_{a}^{\ c}e^{\nu }_{\ c}\mathrm{\Gamma}^{\rho }_{\mu \nu }e_{b\rho }-\omega _{b}^{\ c}e^{\nu }_{\ a }\mathrm{\Gamma} ^{\rho }_{\mu \nu }e_{c\rho }\\
=&\omega _{a}^{\ c}\omega _{\mu cb}+\omega _{b}^{\ c}\omega _{\mu ac}-\partial _{\mu }\omega _{ab}
\end{align*}
Now,
\begin{align*}
&\delta (\mathcal{D}_{\mu }\psi )\\=&\delta \Big(\partial _{\mu }\psi +\dfrac{1}{4}\omega _{ab}\gamma ^{ab}\psi \Big)+\dfrac{1}{4}\omega _{\mu cd}\gamma ^{c d}\Big(\dfrac{1}{4}\omega _{ab}\gamma ^{ab}\psi \Big)+\dfrac{1}{4}(\omega _{a}^{\ c}\omega _{\mu cb}+\omega _{b}^{\ c}\omega _{\mu ac}-\partial _{\mu }\omega _{ab})\gamma ^{ab}\psi \\
=&\dfrac{1}{4}\omega _{ab}\gamma ^{ab}\partial _{\mu }\psi +\dfrac{1}{16}\omega _{\mu c d}\omega _{ab}\gamma ^{c d}\gamma ^{ab}\psi +\dfrac{1}{4}\Big(\omega _{a}^{\ c}\omega _{\mu c b}+\omega _{b}^{\ c}\omega _{\mu ac}\Big)\gamma ^{ab}\psi\\
=&\dfrac{1}{4}\omega _{ab}\gamma ^{ab}\Big(\partial _{\mu }\psi +\dfrac{1}{4}\omega _{\mu \cdot }\gamma ^{cd}\psi \Big)\\
=&\dfrac{1}{4}\omega _{ab}\gamma ^{ab}\mathcal{D}_{\mu }\psi 
\end{align*}
where we have used
\begin{align*}
\gamma ^{ab}\gamma ^{cd}-\gamma ^{c d}\gamma ^{ab}=&\gamma ^{ab}\gamma ^{c d}-(ab\leftrightarrow c d)\\
=&\gamma ^{abc d}+\eta ^{bc}\gamma ^{ad}-\eta ^{bd}\gamma ^{ac}+\eta ^{c d}\eta ^{bc}-\eta ^{ac}\gamma ^{bd}\\
&+\eta ^{bc}\eta ^{ad}-\eta ^{ac}\eta ^{bd}-(ab\leftrightarrow c d)\\
=&2(\eta ^{bc}\gamma ^{ad}-\eta ^{bd}\gamma ^{ac}+\eta ^{ad}\gamma ^{bc}-\eta ^{ac}\gamma ^{bd})\\
\Big[\dfrac{1}{4}\omega _{ab}\gamma ^{ab},\gamma ^{cd}\Big]=&\omega ^{c}_{\ e}\gamma ^{ed}+\omega ^{d}_{\ e}\gamma ^{ce}
\end{align*}
and
\[0=\omega ^{a}_{\ c}\gamma ^{c}+\Big[\gamma ^{a},\dfrac{1}{4}\omega _{c d}\gamma ^{cd}\Big]\]
which completes the proof.
\end{recall}
\vspace{2ex}
\begin{defi}
(Kosmann Derivative) The Kosmann derivative is the generalisation of the Lie derivative. 
\end{defi}
\vspace{2ex}

 
