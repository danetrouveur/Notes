\section{Lecture 11 (May 23rd)}
\begin{rmk}
Supersymmetry relates bosons (like $g_{\mu \nu }$) and fermion (like $A_{\mu }$). However, as their components differ by dimension, the maximal dimension that such a symmetry exists is $D=11$. At $D=11$, we have M-theory, describing membranes of the action
\[\int d\sigma ^{3}\,\sqrt{\mathop{\mathrm{det}}\partial _{\alpha }X^{\mu }\partial _{\beta }X^{\nu }g_{\mu \nu }}\]
where $\sigma^{\alpha }=(\tau ,\sigma ^{1},\sigma ^{2})$. However, quantisation of this action has failed. At $D=10$, we have superstring theory, describing superstrings with the action
\[\int d\sigma ^3\,\sqrt{\mathop{\mathrm{det}}\partial _{\alpha }x^{\mu }\partial _{\beta }x^{\nu }g_{\mu \nu }}\]
where $\sigma ^{\alpha }=(\tau,\sigma ^{1})$. Quanisation is possible for this action, whichis why we call this a critial dimension. We can Taylor expand world sheets in this dimension,
\[X^{\mu }(\tau ,\sigma )=\sum _{n}X^{\mu }_{n}(i)e^{in\sigma }\] 
and there would exist massless modes and excited modes. When we normalise the action, we obtain a factor of $1/2\pi \alpha '$ where the unit $[\alpha ']=\mathrm{length}^2$. We can square root this to obtain string length $l_{s}$. The annihilation operators of this action would satisfy (from the expansion), 
\[[a_{n}^{\mu },\overline{a}_{m}^{\nu }]=\eta ^{\mu \nu }\delta ^{m}_{n}\]
Massless modes give birth to closed strings (the metric $g_{\mu \nu }$, a $B$-field $B_{\mu \nu }$, and the string dilaton $\phi $) and open strings give birth to fields like $A_{\mu \nu }$.
\end{rmk}
\vspace{2ex}
\begin{defi}
Supergravity (SUGRA) has the action
\[\int dx^{10}\,\sqrt{-g}e^{-2\phi }\Big(R+4(\partial \phi )^2-\dfrac{1}{12}H^2\Big)\]
where the $H$-flux is defined as
\[H_{\lambda \mu \nu }=\partial _{\lambda }B_{\mu \nu }+\partial _{\mu }B_{\nu \lambda }+\partial _{\nu }B_{\lambda \mu }\]
There has also been attempts to obtain a 11D action, with 3-form gauge fields. However, reducing dimensions, we have unmotivated symmetries, and we also have nonsense that the metric, central in Einstein's gravity is pretty much indifferent from other fields. This is how double field theory was born.
\end{defi}
\vspace{2ex}
\begin{prop}
As you apply the variation principle to the abaove action, you obtain three equations,
\[\begin{cases}
0=&R_{\mu \nu }+2\nabla _{\mu }(\partial _{\nu }\phi )-\dfrac{1}{4}H_{\mu\rho \sigma  }-H_{\nu }^{\ \rho \sigma }\\
0=&\dfrac{1}{2}e^{2\phi }\nabla ^{\rho }(e^{-2\phi }H_{\rho \mu \nu })\\
0=&R+4\square \phi -4(\partial\phi )^2-\dfrac{1}{12}H^2
\end{cases}\]
In double field theory, these equations get unified into $G_{AB}=0$. The symmetry that unifies this is $O(D,D)$ symmetry.
\end{prop}
\vspace{2ex}
\begin{defi}
Indices that transform via $O(D,D)$ symmetry is written as $A,B$ or $M,N$. These indices are transformed via the metric
\[J_{AB}=\begin{pmatrix}
0&{\bf 1}\\{\bf 1}&0
\end{pmatrix}
\]
where coordinates are given by $x^{A}=(\tilde{x}_{\mu },x^{\nu })$ and $x_{A}=(x^{\nu },\tilde{x}_{\mu })$  with $\partial _{A}=(\tilde{\partial }^{\mu },\partial _{\nu })$ and $\partial ^{A}=(\partial _{\mu },\tilde{\partial }^{\nu })$. $\mathop{\mathrm{Spin}}(1,D-1)$ are dealt with $p,q$ and $\alpha ,\beta $ and $\mathop{\mathrm{Spin}}(D-1,1)$ are dealt with $\overline{p},\overline{q}$ and $\overline{\alpha },\overline{\beta }$. We take
\[\eta _{pq}=(-,+,\ldots,+)\quad \overline{\eta }_{pq}=(+,-,\ldots ,-)\]

\end{defi}
\vspace{2ex}

