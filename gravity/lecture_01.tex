\section{Lecture 1 (March 7th)}
We start with a general review of general relativity, learning a bit of Noether symmetry. We later learn Fermions, spinors, and the Dirac equation, along with gamma matrices that describe them. The mathematical background of this would be Clifford algebra. Gravitational coupling of Fermions are interesting topics, where we use vielbeins, spin connections, and local Lorentz symmetry (described by spin groups). We will then introduce supergravity (probably in 11 dimensions). Two branches of this 11D gravity is IIA SUGRA and IIB SUGRA. String theory implies double field theory which will also touch upon (we could also say that this is the gravity of string theory, it can also be seen as an $O(D,D)$ extension of general relativity). We possibly might deal with supersymmetric double field theory.
\\
\begin{recall}
Among diffeomorphisms, one could consider active and passive coordinate transformations. Normally, we would call the active transformations diffeomorphisms, which are given by Lie derivatives. 
\[\underbrace{x^{\lambda }\rightarrow x'^{\lambda }(x)}_{\mathrm{active}}\hspace{5ex}\underbrace{\phi(x)\rightarrow \phi '(x')=\phi (x) }_{\mathrm{passive}}\]
We have previously saw how passive transformations worked, for vectors, one-forms, and tensor densities (tensors with weights)
\[\begin{cases}
\vspace{2ex}\phi (x)\rightarrow \phi '(x')\\\vspace{1ex}
\partial _{\mu }\rightarrow \partial _{\mu }'=\dfrac{\partial x^{\nu }}{\partial x'^{\mu }}\partial _{\nu }\\\vspace{1ex}
dx^{\mu }\rightarrow dx'^{\mu }=\dfrac{\partial x'^{\mu }}{\partial x^{\nu }}dx^{\nu } \\||g||\rightarrow ||g'(x')||=||\dfrac{\partial x}{\partial x'}||^2||g(x)|| 
\end{cases}\]
as we already know, a tensor density with weight $w$ of order $(p,q)$ would transform like the following.
\[T^{\mu_1 \ldots \mu _{p} }_{\nu_1\ldots v_{q}}(x)\rightarrow T'^{\mu _1\ldots \mu _{p}}_{\nu _1\ldots \nu _{q}}(x')=||\dfrac{\partial x}{\partial x'}||^{w}\dfrac{\partial x'^{\mu_1}}{\partial x^{\rho_1}}\ldots \dfrac{\partial x^{\sigma _{q}}}{\partial x'^{\nu _{q}}}T^{\rho_1\ldots \rho _{p}}_{\sigma_1\ldots \sigma_{q} }(x)   \]
How does the Levi-civita tensor transform?
\[\varepsilon ^{\lambda_1\ldots \lambda _{D}}\rightarrow \dfrac{\partial x'^{\lambda_1 }}{\partial x^{\mu_1}}\ldots \dfrac{\partial x'^{\lambda_{D}}}{\partial x^{\mu _{D}}}\varepsilon ^{\mu_1\ldots \mu_{D}}=||\dfrac{\partial x'}{\partial x}||\varepsilon ^{\lambda_1\ldots \lambda _{D}}   \]
to turn the transformed tensor back into the Levi-civita, we would need to multiply $||\partial x/\partial x'||$. In the above sense, we would say that the Levi-civita tensor is a tensor density of weight $1$. Therefore, in literature, we often use the following weight $0$ tensor
\[\dfrac{\varepsilon ^{\lambda_1\ldots \lambda _{\mu }}}{\sqrt{||g||}}\]
A good question would be, what would the weight of the Levi-civita with lower indices (answer: it still has a weight of 1)? For further help, we note that the definition of a determinant was, for matrix $M_{ab}$
\[||M||=M_{1b_1}M_{2b_2}\ldots M_{Db_{D}}\varepsilon ^{b_1\ldots b_{D}}=\dfrac{1}{D!}M_{a_1b_1}\ldots M_{a_{D}b_{D}}\varepsilon ^{a_1\ldots a_{D}}\varepsilon ^{b_1\ldots b_{D}}\]
Above was for passive transformations. We again emphasize that for an active transformation, fields transform like the following
\[\phi (x)\rightarrow \phi '(x)=\phi (x'(x))\]
thus, differentiation works smoothly like the following
\[\partial _{\mu }\phi (x)\rightarrow \partial _{\mu }\phi '(x)=\dfrac{\partial x'^{\nu }}{\partial x^{\mu }}\dfrac{\partial }{\partial x'^{\nu }}\phi (x')  \]
from this we can know that vectors transform like the following.
\[\begin{cases}
\vspace{2ex}V_{\mu }(x)\rightarrow V'_{\mu }(x)=\dfrac{\partial x'^{\nu }}{\partial x^{\mu }}V_{\nu }(x')\\\vspace{2ex}
g_{\mu \nu }(x)\rightarrow g'_{\mu \nu }(x)=\dfrac{\partial x'^{\rho }}{\partial x^{\mu }}\dfrac{\partial x'^{\sigma }}{\partial x^{\nu }}g_{\rho \sigma }(x)\\
||g(x)||\rightarrow ||g'(x)||=||\dfrac{\partial x'}{\partial x}||^2||g(x')|| 
\end{cases}\]
Thus, an active transformation would be given as
\[T^{\mu_1\ldots \mu _{p}}_{\nu_1\ldots \nu _{q}}(x)\rightarrow T'^{mu_1\ldots \mu _{p}}_{\nu_1\ldots \nu _{q}}(x)=||\dfrac{\partial x'}{\partial x}||^{w}\dfrac{\partial x^{\mu_1}}{\partial x'^{\rho_1}}\ldots \dfrac{\partial x'^{\sigma _{q}}}{\partial x^{\nu _{q}}}T^{\rho_1\ldots \rho _{p}}_{\sigma_1\ldots \sigma _{q}}(x')  \]
\end{recall}
To this point we have learnt finite transformations. However, we can deal with infinitesimal transformations that could be expressed like the following
\[x^{\lambda }\rightarrow x'^{\lambda }(x)\approx x^{\lambda }+\xi^{\lambda }(x)\]
here, $\partial x^{\lambda }=\xi^{\lambda }$. In this way, we can express coodinate transformations like the following
\[\dfrac{\partial x'^{\lambda }}{\partial x^{\mu }}\approx \dfrac{\partial (x^{\lambda }+\xi^{\lambda })}{\partial x^{\mu }}=\delta ^{\lambda }_{\mu }+\partial _{\mu }\xi^{\lambda } \hspace{5ex}\dfrac{\partial x^{\mu }}{\partial x'^{\lambda }}\approx \delta ^{\mu }_{\lambda }-\partial _{\lambda }\xi^{\mu }\hspace{5ex}||\dfrac{\partial x'}{\partial x}||\approx 1+\partial _{\mu }\xi^{\mu }   \]
Using these three, we can express the actively transformed tensor as 
\[T^{\mu_1\ldots \mu _{p}}_{\nu _1\ldots \nu _{q}}(x)+w\partial _{\mu }\xi^{\mu }T^{\mu_1\ldots \mu _{p}}_{\nu_1\ldots \nu_{q} }(x)-\sum^{p}_{i=1}\partial _{\rho }\xi  ^{\mu_{i}}T^{\mu_1\ldots \rho \ldots \mu _{p}}_{\nu_1\ldots \nu_{q}}+\xi^{\rho }\partial _{\rho }T^{\mu_1\ldots \mu _{p}}_{\nu_1\ldots \nu_{q}}(x)+\sum ^{q}_{j=1}\partial _{\nu _{j}}\xi^{\sigma }T^{\mu_1\ldots \mu _{p}}_{\nu_1\ldots \sigma \ldots \nu _{q}}(x) \]
the terms other than the first is the definition of the Lie derivative. As a subtle reminder, we note that the variation of the determinant of a matrix is given as
\begin{align*}
	\delta \,\mathrm{det}\, M=&\mathrm{det}\,M(M^{-1})^{ab}\delta M_{ba}\\
	=&\mathrm{det}\,M\,\mathrm{Tr}\,(M^{-1}\delta M)\\
	\delta \,\mathrm{ln}\,||M||=&\mathrm{Tr}\,(M^{-1}\delta M)
\end{align*}
the proof was done through a barebone definition (which would help to be reviewed). We now recall the definition of a exponential map using an arbitrary vector $\xi^{\mu }(x)$. Given a real parameter $s \in [0,1]$, we can define $x^{\lambda }_{s}(x)$ like the following.
\[\begin{cases}\vspace{2ex}
x^{\lambda }_{s=1}(x)=x^{\lambda }\\\vspace{2ex}
x^{\lambda }_{s=1}(x)=x'^{\lambda }(x)\\\vspace{2ex}
\dfrac{d }{d s}x_{s}^{\lambda }(x)=\xi^{\lambda }(x_{s}(x))\\\vspace{2ex}
\dfrac{d }{d s}x^{\lambda }_{s}(x)|_{s=0}=\xi^{\lambda }(x) \\
\dfrac{d ^2}{d s^2}x^{\lambda }_{s}(x)=\dfrac{d x_{s}^{\mu }}{d s}\dfrac{d }{d x_{s}^{\mu }}\xi^{\lambda }(x_{s})=\xi^{\mu }(x_{s})\dfrac{\partial }{\partial x^{\mu }_{s}}\xi^{\lambda }(x_{s})=\Big(\xi^{\mu }(x_{s})\dfrac{\partial }{\partial x^{\mu }_{s}} \Big)^2x_{s}^{\lambda }
\end{cases}\]
thus, for an arbitrary $n\in {\bm N}$, we have
\begin{align*}
\vspace{2ex}
\dfrac{d ^{n}}{d s^{n}}x^{\lambda }_{s}(x)=&\Big(\xi^{\mu }(x_{s})\dfrac{\partial }{\partial x_{s}^{\mu }} \Big)^{n}x_{s}^{\lambda }\\\dfrac{d ^{n}}{d s^{n}}x_{s}^{\lambda }(x)|_{s=0}=&\Big(\xi^{\mu }(x)\dfrac{\partial }{\partial x^{\mu }} \Big)^{n} x^{\lambda }\\x^{\lambda }_{s}(x)=&\sum ^{\infty }_{n=0}\dfrac{s^{n}}{n!}\Big(\xi^{n}(\lambda )\dfrac{\partial }{\partial x^{n}} \Big)^{n}x^{\lambda }=e^{s\xi^{\mu }(x)\partial _{\mu }}x^{\lambda }
\end{align*}
and finally
\[x'^{\lambda }(x)=e^{\xi^{\mu }\partial _{\mu }}x^{\lambda }\approx x^{\lambda }+\xi^{\lambda }\]
in this way, we can parametrize transformations using exponentials of derivatives or more generally exponentials of Lie derivatives
\[\phi _{s}(x)=\phi (x_{s})=e^{s\xi^{\mu }\partial _{\mu }}\phi (x)\hspace{5ex}e^{s\mathcal{L}_{\xi}}T^{\mu_1\ldots \mu _{p}}_{\nu_1\ldots \nu _{q}}(x)=T^{\mu_1\ldots \mu _{p}}_{s\ \nu_1\ldots \nu _{q}}(x)\]

\vspace{2ex}
