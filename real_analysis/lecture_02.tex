\section{Lecture 2 (March 6th)}
\begin{defi}
Let $X$ denote a set and $\mathcal{P}(X)$ denote the power set of $X$ (the set of all subsets of $X$). Also, let $n(X)$ denote the number of elements of $X$. If $n(X)=k$, then $n(\mathcal{P}(X))=2^{k}$. We now define a $\sigma $-algebra (or $\sigma $-field) on $X$.
\[\mathcal{A}\subset \mathcal{P}(X)\]
is called a $\sigma $-algebra on $X$ provided that
\begin{itemize}
	\item [(i)] $\mathcal{A}$ is not empty (or $\emptyset \in \mathcal{A}$) 
	\item[(ii)] if $A\in \mathcal{A}$ then $A^{c} \in \mathcal{A}$.
	\item[(iii)] if $A_1,\ldots ,A_{n},\ldots \in \mathcal{A}$, disjoint or not, then $\bigcup^{\infty }_{n=1}A_{n}\in \mathcal{A}$  
\end{itemize}
\end{defi}
\vspace{2ex}
\begin{ex}
	An example of a $\sigma $-algebra is $\mathcal{A}=\{ \phi ,X \}$.
\end{ex}
\vspace{2ex}
\begin{defi}
 If $\mathcal{F}\subset \mathcal{P}(X)$ then there is the smallest $\sigma $-algebra $\mathcal{D}$ which contains $\mathcal{F}$. In this case, $\mathcal{D}$ is called the $\sigma $-algebra generated by $\mathcal{F}$. $\mathcal{D}$ is the intersection of all $\sigma $-algebras which contain $\mathcal{F}$. 
\end{defi}
\vspace{2ex}
\begin{ex}
Let $\mathcal{F}=\{A,B,C\}\subset \mathcal{P}(X)$. In the case that they don't intersect, $n(\mathcal{D})=2^{4}$ and if they all intersect, $n(\mathcal{D})=2^{8}$. In full generality, for $\mathcal{F}=\{A_1,\ldots ,A_{n}\}$, we have the following inequality
\[2^{n+1}\leq n(\mathcal{D})\leq 2^{2^{n}}\]
From this we learn that there is no $\sigma $-algebra which has countably many elements.
\end{ex}
\vspace{2ex}
\begin{ex}
Let $X={\bm R}$ and $\mathcal{F}=\mathrm{the\; set\;of\;all\;finite\;subsets\;of\;}{\bm R}$. In this case $\{1,2\}$ would be in $\mathcal{F}$, $[0,1]$ would not be and neither would ${\bm N}$. What would be the $\sigma $-algebra generated by $\mathcal{F}$?
\[\mathcal{D}=\{E\subset {\bm R} \;|\; \mathrm{either}\;E\;\mathrm{or}\;E^{c}\mathrm{\;is\;countable} \}\]
\end{ex}
\vspace{2ex}
\begin{defi}
$\mathcal{B} $ or $\mathcal{B} ({\bm R})$ is the called the Borel $\sigma $-algebra on ${\bm R}$ and is defined as the $\sigma $-algebra generated by all open subsets of ${\bm R}$. The elements $B\in \mathcal{B} $ is called a Borel set (however, this set has too many holes to do measure theory on). The elements of the Borel set are open sets, closed sets, $G_{\delta }$ sets (a countable intersection of open sets), and $F_{\sigma }$ sets (countable union of closed sets). An example of a $G_{\delta }$ set would be
\[\bigcap^{\infty }_{n=1}\Big(1-\dfrac{1}{n},5+\dfrac{1}{n}\Big)=[1,5]\]
and an example of a $F_{\sigma }$ set would be
\[\bigcup^{\infty }_{n=1}\Big[1+\dfrac{1}{n},5-\dfrac{1}{n}\Big]=(1,5)\]
We can extend this and create countable unions and countable intersections of these sets to create sets such as $G_{\delta \sigma }$ and $F_{\sigma \delta }$. We note that
\[n(\mathcal{B} )=2^{\aleph_{0}}=c\]
and is a miserably small set.
\end{defi}
\vspace{2ex}
\begin{recall}
We previously saw limit superior and inferiors, 
\[\limsup a_{n}=\lim _{k\rightarrow \infty }\sup_{n\geq k} \{a_{n}\}\]
then we define a limit superior and inferiors of sequences of sets (say, $A_1\ldots A_{n}$ is a sequence of sets)
\[\limsup A_{n}=\bigcap ^{\infty }_{k=1}\bigcup_{n\geq k} A_{n} \hspace{5ex}\liminf A_{n}=\bigcup ^{\infty }_{k=1}\bigcap_{n\geq k} A_{n} \]
If an element belongs in the limit superior of $\{A_{n}\}$ this implies that $x\in A_{n}$ for infinitely many $n$'s. If it belongs in the limit inferior, if means that $x\in A_{n}$ for all but finitely many $n$'s.
\end{recall}
\vspace{2ex}
\textbf{Chapter 2}\hspace{2ex}Lebesgue Measure on ${\bm R}$
\\
\begin{rmk}
	We ask the following question: does there exist $m(E)$ for $E\subset {\bm R}$, or in other words, $m:\mathcal{P}({\bm R})\rightarrow [0,\infty ]$ that satisfies the following?
\begin{itemize}
	\item[(i)] $m(I)=l(I)$ if $I$ is an open interval.
	\item[(ii)] $m(E+x)=m(E)$ (translation invariant)
	\item[(iii)] If $A_1\ldots A_{n}\ldots $ are mutually disjoint, then $m\Big(\bigcup^{\infty }_{n=1}A_{n}\Big)=\sum ^{\infty }_{n=1}m(A_{n})$
\end{itemize}
However, there is no $m$ that satisfies these three criteria. In particular, by axiom of choice, we can express $[0,1)=\bigcap^{\infty }_{n=1}A_{n}$ where $A_{n}$ are mutually disjoint with same size.
\end{rmk}
\vspace{2ex}
\begin{defi}
	The outer measure $m^{*}$ on $\mathcal{P}({\bm R})$ is defined as a function $m^{*}:\mathcal{P}({\bm R})\rightarrow [0,\infty ]$ where 
	\[m^{*}(E)=\inf\Big\{\sum ^{\infty }_{n=1}l(I_{n})\;\Big|\;E\subset \bigcup^{\infty }_{n=1}I_{n},\;I_{n}\mathrm{\;is\;an\;open\;interval}\Big\}\]
then
\begin{itemize}
	\item[(i)] $m^{*}(I)=l(I)$ if $I$ is an open interval.
	\item[(ii)] $m^{*}(I+x)=m^{*}(I)$
	\item[(iii)] If $A_1,\ldots ,A_{n},\ldots $ are mutually disjoint subsets of ${\bm R}$, $m^{*}\Big(\bigcup^{\infty }_{n=1}A_{n}\Big)\leq \sum ^{\infty }_{n=1}m^{*}(A_{n})$ 
\end{itemize}
\end{defi}
\vspace{2ex}


