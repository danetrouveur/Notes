\section{Lecture 12 (April 10th)}
\begin{recall}
We denote $f\in L^{1}(E)$ if $f$ is measurable on $E$ with $\int _{E}|f|\;d m<\infty $. For $f\in L^{1}(E)$ we define
\[\int _{E}f\;d m=\int _{E}f^{+}\;d m-\int _{E}f^{-}\;d m\]
with $f^{+},f^{-}\in L^{+}(E)\cap L^{-}(E)$.
\end{recall}
\vspace{2ex}
\begin{thm}
For $f,g\in L^{1}(E)$, $a,b\in {\bm R}$, $af+bg\in L^{1}(E)$ with
\[\int _{E}(af+bg)\;d m=a\int _{E}f\;d m+b\int _{E}g\;d m\]
\end{thm}
\vspace{2ex}
\begin{prop}
For measurable $f$ and $\int _{E}|f|dm<\infty $, 
\[\Big|\int _{E}f\;d m\Big|\leq \int _{E}|f|\;dm\]
\end{prop}
\vspace{2ex}
\begin{thm}
($\ast$ LDCT, Lebesgue Dominated Convergence Theorem) Let $\{f_{n}\}$ be a sequence in $L^{1}(E)$ such that $f_{n}\rightarrow f$ pointwise almost everywhere on $E$. If there is $g\in L^{1}(E)$ such that $|f_{n}(x)|\leq g(x)$ for each $x\in E$, then $f\in L^{1}(E)$ and 
\[\int _{E}f\;d m=\lim _{n\rightarrow \infty }\int _{E}f_{n}\;d m\]
This is the most used theorem for exchanging limits and integrals. We will later learn that this is an extremely strong condition.
\end{thm}
\vspace{2ex}
\begin{proof}
$2g-|f_{n}-f|\in L^{1}(E)$ and $2g-|f_{n}-f|\rightarrow 2g$ almost everywhere on $E$. Applying Fatou's lemma, we have
\begin{align*}
\int _{E}2g\;d m\leq& \liminf \int _{E}(2g-|f_{n}-f|)\;d m\\
=&\liminf \Big[\int _{E}2g\;d m-\int _{E}|f_{n}-f|\;d m\Big]\\
=&-\limsup \int _{E}|f_{n}-f|\;d m
\end{align*}
where $\int _{E}2g\;d m\in {\bm R}$, so that $\limsup\int _{E}|f_{n}-f|\;d m\leq 0$. That is,
\[\lim _{n\rightarrow \infty }\int |f_{n}-f|\;d m=0\]
Then,
\[\Big|\int _{E}f_{n}\;d m-\int _{E}f\;d m\Big|\leq \int _{E}|f_{n}-f|\;d m\rightarrow 0\]
as $n\rightarrow \infty $. 
\end{proof}
\vspace{2ex}
\begin{cor}
(BCT, Bounded convergence theorem) Suppose that $f_{n}\in L^{1}(E)$ and there is $M>0$ such that $|f_{n}(x)|\leq M$ for all $n\in {\bm N}$ and $x\in E$. If $m(E)<\infty $ and $f_{n}\rightarrow f$ pointwise on $E$, then
\[\int _{E}f\;d m=\lim _{n\rightarrow \infty }\int _{E}f_{n}\;d m\]
\end{cor}
\vspace{2ex}
\begin{proof}
Put $g(x)=M\in L^{1}(E)$ since $\int _{E}g\;d m=M\times m(E)<\infty $. Use LDCT.
\end{proof}
\vspace{2ex}
\begin{thm}
If $f_{n}\in L^{1}(E)$ with $f_{n}\rightarrow f$ uniformly on $E$ and $m(E)<\infty $, then $f\in L^{1}(E)$ and 
\[\int _{E}f\;d m=\lim _{n\rightarrow \infty }f_{n}\;d m\]
\end{thm}
\vspace{2ex}
\begin{proof}
Take any $\varepsilon >0$ there is $N$ such that if $n\geq N$ then $|f_{n}(x)-f(x)|<\varepsilon /m(E)$ for all $x\in E$.
\begin{align*}
\int _{E}|f|\;d m\leq \int _{E}|f-f_{N}|\;d m+\int _{E}|f_{N}|\;d m\leq \varepsilon +\int _{E}|f_{N}|\;d m<\infty 
\end{align*}
which implies that $f\in L^{1}(E)$. If $n\geq N$,
\[\Big|\int _{E}f\;d m-\int _{E}f\;d m\Big|=\Big|\int _{E}(f-f_{n})\;d m\Big|\leq \int _{E}|f-f_{n}|\;d m<m(E)\cdot \varepsilon /m(E)=\varepsilon \]
\end{proof}
\vspace{2ex}
\begin{rmk}
Let's compare the Riemann integral and the Lebesgue integral. The upper sum can be considered as the integral of the step function which is a simple function.
\[U(f,P)=\sum ^{n}_{k=1}M_{k}\Delta x_{k}\]
Previously, the simple function we've considered for $f\in L^{+}(E)$ was
\[\phi _{n}=n\mathcal{X}_{C_{n}}+\sum ^{n2^{n}}_{k=1}\dfrac{k-1}{2^{n}}\mathcal{X}_{D_{n,k}}\]
where $C_{n}=\{x\in E \;|\; f(x)\geq n\}$a nd $D_{n,k}=\{x \in E \;|\; k-1/2^{n}\leq f(x)<k/2^{n}\}$ where $k=1,2,\ldots ,n2^{n}$. $\phi _{n}\rightarrow f$ increasingly and if $f$ is bounded then $\phi _{n}\rightarrow f$ uniformly. We can choose $\psi _{n}$ with a compact support $\phi _{n}\rightarrow f$ increasingly (or $\phi _{n}\rightarrow f$) uniformly for bounded $f$. Notce that $\psi _{n}=\phi _{n}\mathcal{X}_{[-n,n]}$ is a simple function with a compact support satisfying $\psi _{n}\rightarrow f$ increasingly (uniformly if $f$ is bounded). 
\end{rmk}
\vspace{2ex}
\begin{thm}
If $f$ is a bounded measurable function on $E$ then there exists sequences of simple functions $\{\phi _{n}\}$, $\{\psi _{n}\}$ such that $\phi _{n}\rightarrow f$ uniformly and increasingly or $\psi _{n}\rightarrow f$ uniformly and decreasingly. Here, by compact support, we mean $\{x\in E \;|\; f(x)\ne 0\}$ that is compact. 
\end{thm}
\vspace{2ex}
\begin{thm}
Let $f$ be bounded and $m(E)<\infty $. Then $f\in L^{1}(E)$ and
\[\int _{E}f\;d m=\sup_{\phi \leq}\int _{E}\phi \;d m=\inf_{f\leq \psi }\int _{E}\psi \;d m\]
for simple $\phi$ and $\psi $.
\end{thm}
\vspace{2ex}
\begin{rmk}
If $f$ is Riemann integrable, then 
\[\overline{\int} ^{b}_{a}f\;dx=\underline{\int} ^{b}_{a}f\;dx\]
where 
\[\underline{\int}_{a}^{b}f\;dx=\sup_{\phi \leq f}\int _{a}^{b}\phi \;dx=\sup_{\phi \leq f}\int _{[a,b]}\phi\;d m \]
where $\phi $ is a step function. Also,
\[\overline{\int }^{b}_{a}f\;dx=\inf_{f\leq \phi }\int _{[a,b]}\phi \;d m\]
and $\phi $ is again a step function. The step function is a simple function so that
\[\underline{\int }^{b}_{a}f\;dx=\sup_{\phi \leq f}\int _{[a,b]}\phi\;dx\leq \sup_{\phi \leq f}\int _{[a,b]}\phi \;d m\leq \inf_{f\leq \phi }\int _{[a,b]}\psi \;d m\leq \inf_{f\leq \phi }\int _{[a,b]}\psi\;d m =\overline{\int }^{b}_{a}f\;dx \]
where on the far left and right, $\phi $ is a step function and the on the middle two, $\phi $ is a simple  function. By BCT, the middle two are equal to $\int _{E}f\;d m$. If $f$ is Riemann integrable on $[a,b]$, then 
\[\int ^{b}_{a}f\;dx=\int _{[a,b]}f\;d m\]
\end{rmk}
\vspace{2ex}
\begin{ex}
Find
\[\lim _{R\rightarrow \infty }\int ^{\pi }_{0}e^{-R\sin t}\;dt=\int ^{\pi }_{0}\lim _{R\rightarrow \infty }e^{-R\sin t}\;dt=\int ^{\pi }_{0}0\;dt=0\]
\end{ex}
\vspace{2ex}
\begin{thm}
If $f\in L^{+}({\bm R})$ and $\int ^{\infty }_{-\infty }f(x)\;dx$ exists in Riemann improper integral, then
\[\int ^{\infty }_{-\infty }f(x)\;dx=\int _{{\bm R}}f\;d m\]
\end{thm}
\vspace{2ex}
\begin{ex}
Find
\[\lim _{n\rightarrow \infty }\int ^{n}_{0}\dfrac{1}{\sqrt[n]{x}\Big(1+\dfrac{x}{n}\Big)^{n}}\]
\end{ex}
\vspace{2ex}
\begin{rmk}
Notice that for $E=(0,1)$, $f(x)=1/x^{p}$ is $f\in L^{1}(E)$ if $0<p<1$. Also, for $E=[1,\infty )$, $f(x)=1/x^{p}$ is $f\in L^{1}(E)$ if $p>1$. 
\end{rmk}
\vspace{2ex}
\begin{proof}
For $|f_{n}|\leq g\in L^{1}$ and $f_{n}\rightarrow f$ then $\int f=\lim \int f_{n}$. So, for
\[f_{n}=\dfrac{1}{\sqrt[n]{x}\Big(1+\dfrac{x}{n}\Big)^{n}}\mathcal{X}_{[0,n]}\]
we have
\[0\leq f_{n}\leq g(x)=\dfrac{1}{\sqrt{x}}\mathcal{X}_{(0,1)}+\dfrac{4}{x^2}\mathcal{X}_{[1,\infty )}\in L^{1}((0,\infty ))\]
and $f_{n}\rightarrow e^{-x}\mathcal{X}_{(0,\infty )}$ which when integrated is $1$.  
\end{proof}
\vspace{2ex}

