\section{Lecture 19 (May 22nd)}
\begin{recall}
We have seen how integral transforms of the form
\[T(f)=\int _{E}fg\,d m\]
are bounded linear operators with a norm of $||T||=||g||_{q}$. We see that, suprisingly, all bounded linear operators can be realised to be of this form.
\end{recall}
\vspace{2ex}
\begin{thm}
(Riesz representation theorem) If $1\leq p <\infty $ and $T$ is a bounded linear functional on $L^{p}(\mu )$, there is a unique $g\in L^{q}(\mu )$ (where $q$ is the conjugate exponent of $p$) such that 
\[T(f)=\int fg\,d \mu \]
for all $f\in L^{p}(\mu )$ and $||T||=||g||_{q}$. In particular, if $\mu $ is a $\sigma $-finite measure on $X$ and $T$ is a bounded linear functional on $L^{1}(\mu )$ then there is a unique $g\in L^{\infty }(\mu )$ such that 
\[T(f)=\int _{X}fg\,d \mu \]
for all $f\in L^{1}(\mu )$ and $||T||=||g||_{\infty }$.
\end{thm}
\vspace{2ex}
\begin{defi}
If $\mu $ is a counting measure on ${\bm N}$, we define three vector spaces. 
\begin{itemize}
\item [(i)] $l^{p}({\bm N})$ the set of sequence ${\bf x}=(x_1,x_2,\ldots )$ with the norm 
\[||{\bf x}||_{p}=\Big(\sum ^{\infty }_{n=1}|x_{n}|^{p}\Big)^{1/p}\]
for $p\geq 1$. 
\item [(ii)] $l^{\infty }({\bm N})$ is the set of all bounded sequences with the supremum norm.
\item[(iii)] $C_{0}({\bm N})$ the sequence ${\bf x}$ which converges to zero with the supremum norm.
\end{itemize}
\end{defi}
\vspace{2ex}
\begin{defi}
A linear map $T:l^{1}({\bm N})\rightarrow {\bm R}$ is defined by
\[T({\bf x})=\sum ^{\infty }_{n=1}x_{n}y_{n}\]
where ${\bf y}=(y_1,y_2,\ldots )\in l^{\infty }({\bm N})$. We now try to show that the norm of ${\bf y}$ is the norm of the linear operator. Observe that
\[|T({\bf x})|\leq ||y||_{\infty }\sum ^{\infty }_{n=1}|x_{n}|=||{\bf x}||_{1}||{\bf y}||_{\infty }\]
so that $||T||\leq ||{\bf y}||_{\infty }$. To show that $||T||=||{\bf y}||_{\infty }$, suppose $y=\{y_{n}  \}$. Then we put 
\[{\bf x}=\dfrac{|y_{k}|}{y_{k}}{\bf e}_{k}\quad\mathrm{for}\quad y_{k}\ne 0\quad \mathrm{implying}\quad ||{\bf x}||_{1}=1\]
and
\[T({\bf x})= \sum ^{\infty }_{n=1}x_{n}y_{n}=|y_{k}|\]
This tells us that 
\[|y_k|\leq ||T||\]
for every $k\in {\bm N}$ and $||T||\geq ||{\bf y}||_{\infty }$.
\end{defi}
\vspace{2ex}
\begin{thm}
(Riesz representation theorem for $l^{1}({\bm N})$) If $T$ is a bounded linear functional on $l^{1}({\bm N})$ then we'll show ${\bf y}=(y_1,y_2,\ldots )\in l^{\infty }({\bm N})$ such that
\[T({\bf x})=\sum ^{\infty }_{n=1}x_{n}y_{n}\]
with $||T||=||{\bf y}||_{\infty }$.
\end{thm}
\vspace{2ex}
\begin{proof}
To do that just define $T({\bf e}_{k})=y_{k}$. Then by continuity and linearity of $T$,
\[T({\bf x})=T\Big(\sum ^{\infty }_{k=1}x_{k}{\bf e}_{k}\Big)=\sum ^{\infty }_{k=1}T(x_{k}{\bf e}_{k})=\sum ^{\infty }_{k=1}T(x_{k}{\bf e}_{k})=\sum ^{\infty }_{k=1}x_{k}T({\bf e}_{k})=\sum ^{\infty }_{k=1}x_{k}y_{k}\]
To prove that uniqueness, do this again, and we'll find
\[T({\bf x})=\sum_{n=1}^{\infty } x_{n}y_{n}=\sum _{n=1}^{\infty }x_{n}z_{n}\]
and by putting ${\bf x}={\bf e}_{k}$ we find $y_{k}=z_{k}$.
\end{proof}
\vspace{2ex}
\begin{rmk}
We have previously shown that, in the limited case of $p=1$, 
\[l^{p}({\bm N})^{*}=l^{q}({\bm N})\] 
for $1\leq p<\infty $ and $1/p+1/q=1$. Also, for the Lesbegue measure,
\[L^{p}(E)^{*}=L^{q}(E)\]
for $1/p+1/q=1$ with $1\leq p<\infty $. We now show that
\[C_{0}({\bm N})^{*}=l^{1}({\bm N})\]
and also that
\[C_{0}(E)^{*}=M(E)\]
Notice that, importantly, the dual of $C_{0}({\bm N})$ is $l^{1}({\bm N})$ while the dual of $l^{1}({\bm N})$ is $l^{\infty }({\bm N})$. The dual of a dual is not itself!
\end{rmk} 
\vspace{2ex}
\begin{thm}
Recall that $C_{0}({\bm N})$ is defined as the set of infinite sequences that converge to $0$ provided the supremum norm. If $T:C_{0}({\bm N})\rightarrow {\bm R}$ is defined by
\[T({\bf x})=\sum ^{\infty }_{n=1}x_{n}y_{n}\]
for some ${\bf y}=(y_1,\ldots ,y_{n},\ldots )\in l^{1}({\bm N})$, then $T$ is continuous and linear with $||T||=||{\bf y}||_{1}$. Conversely, if $T$ is a bounded linear functional, then there is a unique ${\bf y}\in l^{1}({\bf N})$ such that 
\[T({\bf x})=\sum ^{\infty }_{n=1}x_{n}y_{n}\]
with $||T||=||{\bf y}||_{1}$. The latter converse follows from just defining $T(e_{k})=y_{k}$. 
\end{thm}
\vspace{2ex}
\begin{proof}
$T$ is obviously linear. 
\[|T(x)|=\Big|\sum ^{\infty }_{n=1}x_{n}y_{n}\Big|\leq\sum ^{\infty }_{n=1}|x_{n}|\,|y_{n}|\leq ||{\bf x}||_{\infty }||{\bf y}||_{1}\]
This implies that $||T||\leq ||y||_{1}$. Conversely, for each $n$, we define ${\bf x}_{n}\in C_{0}({\bm N})$ as ${\bf x}_{n}=\{x_{n,k}\}_{k=1}^{\infty }$ where 
\[x_{n,k}=\begin{cases}
\dfrac{|y_{k}|}{y_{k}}&\mathrm{if}\quad 1\leq k\leq n\quad \mathrm{and}\quad y_{k}\ne 0\\
0&\mathrm{if}\quad k>n\quad \mathrm{and}\quad y_{k}= 0
\end{cases}\]
Notice how $\lim _{k\rightarrow \infty }x_{n,k}=0$. Then $||{\bf x}_{n}||_{\infty }=1$ for each $n$, and 
\[T({\bf x}_{n})=\sum ^{\infty }_{k=1}x_{n,k}y_{k}=\sum ^{n}_{k=1}|y_{k}|\]
Thus, $\sum _{k=1}^{n}|y_{k}|\leq ||T||$ for every $n\in {\bm N}$. Therefore, $||T||\geq ||y||_{1}$. For the converse, we simply define $T({\bf e}_{k})=y_{k}$. Then by the continuity and linearity, $T({\bf x})=\sum ^{\infty }_{n=1}x_{n}y_{n}$.
\end{proof}
\vspace{2ex}
\begin{ex}
We see another example where which a sequence that is bounded fails to have a subsequence that converges. Define $f_{n}(x)=\sin nx$ for $x\in [0,2\pi ]$. This implies that $|f_{n}(x)|\leq 1$ for all $x\in [0,2\pi ]$ and for all $n\in {\bm N}$. Suppose $\{f_{n}\}$ has a subsequence $\{f_{n_{k}}\}$ which converges pointwise on $[0,2\pi ]$. Then, $\lim _{k\rightarrow \infty }\sin n_{k}x=\lim _{k\rightarrow \infty }\sin n_{k+1}x$, that is,
\[\lim _{k\rightarrow \infty }(\sin n_{k+1}x-\sin n_{k}x)=0\]
for all $x\in [0,1]$. Or, equivalently, $\lim _{k\rightarrow \infty }(\sin n_{k+1}x-\sin n_{k}x)^2=0$. By LDCT, 
\[\lim _{k\leftarrow \infty }\int ^{2\pi }_{0}(\sin n_{k+1}x-\sin n_{k}x)^2=0\]
When this is actually computed, we have $2\pi $ for all $k$. To elaborate further, observe that the above is equal to 
\begin{align*}
\int ^{2\pi }_{0}(\sin ^{2}n_{k+1}x+\sin ^2n_{k}x-2\sin n_{k+1}x\sin n_{kx})\,dx
\end{align*}
and that the first two terms become $\pi $ each and the last term vanishes. 
\end{ex}
\vspace{2ex}
\begin{defi}
Simply put, weak$^{*}$ convergence is pointwise convergence (for a sequence of bounded linear operators). For $T_{n}\in X^{*}$ and $T\in X^{*}$, $T_{n}\rightarrow T$ weak$^{*}$ in $X^{*}$ provided that $\lim _{n\rightarrow \infty }T_{n}(x)=T(x)$ for every $x\in X$.
\end{defi}
\vspace{2ex}
\begin{thm}
(Arzela-Ascoli theorem) (Important!) Let $X$ be a separable normed vector space. Then every bounded sequence in $X^{*}$ has a weak$^{*}$ convergent subsequence.  
\end{thm}
\vspace{2ex}
\begin{cor}
The following is an application for the above theorem in $L^{p}(E)=L^{q}(E)^{*}$. For $1\leq p<\infty $, let $f_{n}\in L^{p}(E)$ with $||f_{n}||_{p}\leq M$ for all $n\in {\bm N}$. Then $\{f_{n}\}$ is bounded in $L^{q}(E)^{*}$. Then there is $f\in L^{p}(X)$ and a subsequence $\{f_{n_{k}}\}$ such that $f_{n_{k}}\rightarrow f$ weak$^{*}$. Notice that by the Riesz-representation theorem, there is a bounded linear functional\[T(g)=\int _{E}gf\,dm\]
for $g\in L^{q}(E)$ so that
\[\lim _{k\rightarrow \infty }\int _{E}f_{n_{k}}g\,d m=\lim _{k\rightarrow \infty }T_{n_{k}}(g)=T(g)=\int _{E}fg\,d m\]
for all $g\in L^{q}(E)$. 
\end{cor}
\vspace{2ex}

