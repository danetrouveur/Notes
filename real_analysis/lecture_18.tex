\section{Lecture 18 (May 20th)}
\begin{thm}
If $m(E)<\infty $ and $1<p<q<\infty $ then $L^{\infty }(E)\subset L^{q}(E)\subset L^{p}(E)\subset L^{1}(E)$. 
\end{thm}
\vspace{2ex}
\begin{proof}
If $||f||_{\infty }<\infty $ then $\int _{E}|f|^{p}\,d m\leq \int _{E}||f||^{p}_{\infty }\,d m=||f||^{p}_{\infty }\cdot m(E)<\infty $ such that $f\in L^{p}(E)$. Meanwhile, if $p<q$ and $f\in L^{q}(E)$, we notice that 
\[\dfrac{q}{p}\quad\mathrm{and}\quad c=\dfrac{1}{1-\frac{q}{p}}\]
are conjugate exponents. Subsequently,
\begin{align*}
\int _{E}|f|^{p}\,d m=\int _{E}|f|^{p}\cdot 1\,d m\leq &\Big[\int _{E}(|f|^{p})^{q/p}\Big]^{p/q}\Big[\int _{E}1\,d m\Big]^{c}\\
=&||f||^{p}_{q}\cdot m(E)^{c}<\infty
\end{align*}
by taking $p=q/p$ and using H\"{o}lder's inequality.
\end{proof}
\vspace{2ex}
\begin{thm}
If $m(E)<\infty $ and $f\in L^{\infty }(E)$ then $\lim _{p\rightarrow \infty }||f||_{p}=||f||_{\infty }$
\end{thm}
\vspace{2ex}
\begin{proof}
\[\int _{E}|f|^{p}\,d m\leq \int _{E}||f||^{p}_{\infty }\,d m=||f||^{p}_{\infty }\cdot m(E)\]
so that 
\[\Big(\int _{E}|f|^{p}\,d m\Big)^{1/p}\leq ||f||_{\infty }\cdot m(E)^{1/p}\quad \mathrm{and}\quad \limsup_{p\rightarrow \infty }||f||_{p}\leq ||f||_{\infty } \]
where the measure goes to $1$ as $p\rightarrow \infty $. Take any $\varepsilon >0$, we'll show that 
\[\liminf_{p\rightarrow \infty }||f||_{p}\geq ||f||_{\infty }-\varepsilon \]
Define
\[E_{\varepsilon }=\{x\in E \,|\, |f(x)|>||f||_{\infty }-\varepsilon \}\]
Then, $0<m(E_{\varepsilon })\leq m(E)<\infty $ and
\begin{align*}
||f||_{p}=\Big(\int _{E}|f|^{p}\,d m\Big)^{1/p}=\Big(\int _{E_{\varepsilon }}|f|^{p}\,d m\Big)^{1/p}\geq& \Big(\int _{E_{\varepsilon }}(||f||_{\infty }-\varepsilon )^{p}\Big)^{1/p}\\
=&\Big[[||f||_{\infty }-\varepsilon ]^{p}m(E_{\varepsilon })\Big]^{1/p}=(||f||_{\infty }-\varepsilon )m(E_{\varepsilon })^{1/p}
\end{align*}
where $m(E_{\varepsilon })^{1/p}$ again goes to 1 as $p\rightarrow \infty $. Thus,
\[\liminf_{p\rightarrow \infty }||f||_{p}\geq ||f||_{\infty }-\varepsilon \]
\end{proof}
\vspace{2ex}
\begin{thm}
Let $1<p<q<r<\infty $. If $f\in L^{p}(E)$ and $f\in L^{r}(E)$, then $f\in L^{q}(E)$. Furthermore, $||f||_{q}<\max (||f||_{p},||f||_{r})$. 
\end{thm}
\vspace{2ex}
\begin{proof}
Let $q=\lambda p+(1-\lambda )r$ for some $\lambda \in (0,1)$. Then,
\begin{align*}
\int _{E}|f|^{q}\,d m=&\int _{E}|f|^{\lambda p}\cdot |f|^{(1-\lambda )r}\,d m\leq \Big(\int _{E}\Big[|f|^{\lambda p}\Big]^{1/\lambda }\,d m\Big)^{\lambda }\Big(\int _{E}\Big[|f|^{(1-\lambda )r}\Big]^{1/(1-\lambda )}\,d m\Big)^{1-\lambda }
\\=&||f||^{\lambda p}_{p}||f||_{r}^{(1-\lambda )r}\leq [\max(||f||_{p},||f||_{r})]^{\lambda p+(1-\lambda )r}=[\max(||f||_{p},||f||_{r})]^{p}
\end{align*}
\end{proof}
\vspace{2ex}
\begin{rmk}
For $p>0$, 
\[
\int ^{1}_{0}\dfrac{1}{x^{p}}\,dx<\infty \quad\mathrm{iff}\quad0<p<1\quad \mathrm{and}\quad
\int ^{\infty }_{1}\dfrac{1}{x^{p}}<\infty \quad \mathrm{iff}\quad1<p
\]
\end{rmk}
\vspace{2ex}
\begin{ex}
Given $a>0$, find $f$ on $(0,\infty )=E$ such that $f\in L^{p}((0,\infty ))$ if and only if $p\in (a,b)$. 
\end{ex}
\vspace{2ex}
\begin{proof}
Simply take
\[f(x)=\begin{cases}
x^{-1/b}&x\in (0,1)\\
x^{-1/a}&x\in (1,\infty )
\end{cases}=x^{-1/b}{\bf 1}_{(0,1)}+x^{-1/a}{\bf 1}_{(1,\infty )}\]
\end{proof}
\vspace{2ex}
\begin{ex}
Note how
\[\int ^{\infty }_{1}\dfrac{1}{[x(1+\ln x)^2]^{p}}\,d x=\int ^{\infty }_{0}\dfrac{e^{t(1-p)}}{(1+t)^{2p}}\,d t<\infty \]
if and only if $p\geq 1$, where we substituted $t=\ln x$. Also,
\[\int ^{1}_{0}\dfrac{1}{[x(1-\ln x)^2]^{p}}\,dx =\int ^{0}_{\infty }\dfrac{e^{t(p-1)}}{[1+t]^{2p}}\,(-dt)=\int ^{\infty }_{0}\dfrac{e^{t(p-1)}}{(1+t)^{2p}}\,d t<\infty \]
if and only if $0<p\leq 1 $,with the substitution being $t=-\ln x$. 
\end{ex}
\vspace{2ex}
\begin{ex}
Given $a>0$, find $f$ on $(0,\infty )=E$ such that $f\in L^{p}((0,\infty ))$ if and only if $p\in [a,b]$ ($0<a<b$). 
\end{ex}
\vspace{2ex}
\begin{proof}
Simply take this time
\[f(x)=\Big[x(1-\ln x)^2\Big]^{-1/b}{\bf 1}_{(0,1)}+\Big[x(1+\ln x)^2\Big]^{-1/a}{\bf 1}_{[0,\infty )}\]
\end{proof}
\vspace{2ex}
\begin{ex}
For $\alpha \in (0,\infty )$, find $f$ so that $f\in L^{p}((0,\infty ))$ if and only if $p=\alpha $. The $f$ is given by the expression above with $\alpha =a=b$. We now find that there is a function that is $L^{p}((0,\infty ))$ only when $p$ is a given number. There is no subset relation when the measure of the space is infinite!
\end{ex}
\vspace{2ex}
\begin{thm}
For $1<p<\infty $ and $1/p+1/q=1$, let $T:L^{p}(E)\rightarrow {\bm R}$ be defined by
\[T(f)=\int _{E}fg\,d m\]
for some $g\in L^{q}(E)$. Then, $T$ is a bounded linear functional on $L^{p}(E)$ with $||T||=||g||_{p}$. 
\end{thm}
\vspace{2ex}
\begin{rmk}
First of all, 
\[T(\alpha_1f_1+\alpha_2f_2)=\int _{E}(\alpha_1f_1+\alpha_2f_2)g\,d m=\alpha_1\int _{E}f_1g\,d m+\alpha_2\int _{E}f_2g\,d m=\alpha_1T(f_1)+\alpha_2T(f_2)\]
and by Holder's inequality, 
\[|T(f)|=\Big|\int _{E}fg\,d m\Big|\leq ||f||_{p}||g||_{q}\]
so that $||T||\leq ||g||_{q}$. We now want to prove that $||T||\geq ||g||_{q}$. Let
\[f=\dfrac{|g|^{q}}{g}\]
we can easily see that $f\in L^{p}(E)$ (given $g\ne 0$ and if $g=0$ we define $f=0$). Now see that $fg=|g|^{q}$ and 
\[T(f)=\int _{E}fg\, dm=\int _{E}|g|^{q}\,d m=||g||_{q}^{q}=||g||_{q}||g||_{q}^{q-1}=||f||_{p}||g||_{q}\]
given the fact that
\[||g||^{q-1}_{q}=\Big(\int _{E}|g|^{q}\,d m\Big)^{(q-1)/q}=\Big(\int _{E}|f|^{p}\,d m\Big)^{1/p}=||f||_{p}\]
We now see how
\[\Big|T\Big(\dfrac{f}{||f||_{p}}\Big)\Big|\geq ||g||_{q}\]
As the norm of an operator is defined by the supremum of which the left hand side is an element of,
\[||T||\geq \Big|T\Big(\dfrac{f}{||f||_{p}}\Big)\Big| \geq ||g||_{q}\]
Finally, due to the previous remark, $||T||=||g||_{q}$. 
\end{rmk}
\vspace{2ex}

\begin{cor}
Let $T:L^{1}(E)\rightarrow {\bm R}$ be defined by
\[T(f)=\int _{E}fg\,d m\]
for some $g\in L^{\infty }(E)$. $T$ is a bounded linear functional on $L^{1}(E)$ with $||T||=||g||_{\infty }$.
\end{cor}
\vspace{2ex}

