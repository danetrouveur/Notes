\section{Lecture 14 (April 29th)}
\begin{defi}
(Convergence in measure) For measurable functions on $E$, we say $f_{n}\rightarrow f$ in measure on $E$ provided that for every $\varepsilon >0$, 
\[\lim _{n\rightarrow \infty }m(\{x\in E\;|\;|f_{n}(x)-f(x)|>\varepsilon \})=0\]
\end{defi}
\vspace{2ex}
\begin{defi}
($L^{1}$ convergence) For $f_{n}\in L^{1}(E)$ and a measurable $f$ on $E$, we say that $f_{n}\rightarrow f$ in $L^{1}(E)$ provided that 
\[\lim _{n\rightarrow \infty }\int _{E}|f_{n}-f|\,dm=0\]
\end{defi}
\vspace{2ex}
\begin{thm}
If $f_{n}\rightarrow f$ and $f_{n}\rightarrow g$ in measure (or in $L^{1}(E)$) in $E$, then $f=g$ almost everywhere on $E$. 
\end{thm}
\vspace{2ex}
\begin{proof}
Suppose that $f_{n}\rightarrow f$ and $f_{n}\rightarrow g$ in measure. First note that 
\[\{x\in E \,|\, |f(x)-g(x)|>0\}=\bigcup ^{\infty }_{n=1}\Big\{x\in E \,\Big|\, |f(x)-g(x)|\geq \dfrac{1}{n}\Big\}\]
Meanwhile, for each $n$, due the triangle inequality, 
\[|f(x)-f_{k}(x)|+|f_{k}(x)-g(x)|\geq |f(x)-g(x)|\geq  \dfrac{1}{n}\]
and one of the two terms in the left must be greater than $1/2n$. We now see the following inclusion for $x\in E$:
\begin{align*}
\Big\{x \,\Big|\,|f(x)-g(x)|\geq \dfrac{1}{n}\Big\}\subset &\Big\{x\,\Big|\, |f(x)-f_{k}(x)|\geq \dfrac{1}{2n}\Big\}\cup \{x\,\Big|\,|g(x)-f_k(x)|\geq \dfrac{1}{2n} \Big\}
\end{align*}
See how for each term, we can choose $k_1$ and $k_2$ such that they are measure zero, and we see that by taking $\max\{k_1,k_2\}$, their infinite union is measure zero too.
\end{proof}
\vspace{2ex}
\begin{ex}
To this point we have learned the following convergences:
\begin{itemize}
\item[(i)] $f_{n}\rightarrow f$ uniformly
\item[(ii)] $f_{n}\rightarrow f$ pointwise almost everywhere on $E$
\item[(iii)] $f_{n}\rightarrow f$ almost uniformly on $E$
\item[(iv)] $f_{n}\rightarrow f$ in measure
\item[(v)] $f_{n}\rightarrow f$ in $L^{1}(E)$
\end{itemize}
Consider the following sequences.
\begin{itemize}
\item[(i)] $f_{n}={\bf 1}_{(n,\infty )}$ (uniform, pointwise almost everywhere)
\item[(ii)] $f_{n}=n{\bf 1}_{(0,1/n)}$ or $f_{n}={\bf 1}_{(0,n)}/n$ (pointwise almost everywhere, almost uniformly, in measure) (uniformly, almost everywhere, almost uniformly, and in measure)
\item[(iii)] $f_{n}={\bf 1}_{I_{n}}$ where 
\[I_1=[0,1/2]\quad I_2=[1/2,1]\quad I_3=[0,1/3]\quad I_4=[2/3,1]\quad I_5=[0,1/4]\]
and etcetera. (uniformly, almost everywhere, almost uniformly, and in measure)
\end{itemize}
\end{ex}
\vspace{2ex}
\begin{thm}
If $f_{n}\rightarrow f$ in measure on $E$, then there is a subsequence $\{f_{n_{k}}\}$ which converges pointwise almost everywhere to $f$. 
\end{thm}
\vspace{2ex}
\begin{proof}
If $f_{n}\rightarrow f$ in measure the following are both true.
\begin{itemize}
\item[(i)] For every $\varepsilon >0$ there is $N$ such that if $n>N$ then
\[m(\{x\in E\,|\,|f_{n}(x)-f(x)|>\varepsilon \})<\varepsilon \]
\item[(ii)] For every $\varepsilon >0$ there is $N$ such that if $n>N$ then
\[m(\{x\in E\,|\,|f_{n}(x)-f(x)|>\varepsilon \})<\varepsilon ^2\]
\end{itemize}
For all $k\in {\bm N}$ there exists $n_{k}$'s such that $n_{k+1}>n_{k}$ and \[m(\{x\in E\,|\,|f_{n_{k}}(x)-f(x)|> 1/k\})< 1/k^2\]
Let $E_{k}=\{x\in E\,|\,|f_{n_{k}}(x)-f(x)|> 1/k \}$. We will now show that $f_{n_{k}}\rightarrow f$ almost everywhere on $E$. Since $\sum ^{\infty }_{n=1}m(E_{k})\leq \sum _{k=1}^{\infty }1/k^2<\infty $, by the Borel-Cantelli lemma, almost all $x\in E$ belongs to at most finitely many $E_{k}$'s. So, $f_{n_{k}}$ almost everywhere on $E$. 
\end{proof}
\vspace{2ex}
\begin{thm}
\begin{itemize}
\item[(i)] If $f_{n}\rightarrow f$ almost uniformly on $E$, then $f_{n}\rightarrow f$ in measure.
\item[(ii)] If $f_{n}\rightarrow f$ in $L^{1}(E)$ then $f_{n}\rightarrow f$ in measure. 
\end{itemize}
\end{thm}
\vspace{2ex}
\begin{proof}
We prove (ii) first. By Chebychev, for each $\varepsilon $,
\[m(\{x\in E \,|\, |f_{n}(x)-f(x)|\geq \varepsilon \})\leq \dfrac{1}{\varepsilon } \int _{E}|f_{n}-f|\,dm\]
where the right hand side approaches $0$ as $n\rightarrow \infty $.
\\\\

Now for (i), suppose $f_{n}$ doesn't converge to $f$ in measure. Then there exists $\varepsilon ,\delta >0$ such that 
\[m(\{x\in E \,|\, |f_{n}(x)-f(x)|>\varepsilon \})\geq \delta \]
for infinitely many $n$'s so that there is a subsequence $\{f_{n_{k}}\}$ such that
\[m(\{x\in E \,|\,|f_{n_{k}}(x)-f(x)|>\varepsilon  \})\geq \delta \]
for all $k\in {\bm N}$. There is no $A\subset E$ such that $m(A)<\delta /2$ and $f_{n_{k}}\rightarrow f$ uniformly on $A^{c}=E\,\backslash\,A$. Hence $f_{n}$ does not converge to $f$ almost uniformly on $E$.
\end{proof}
\vspace{2ex}
