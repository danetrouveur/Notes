\section{Lecture 6 (March 20th)}
\begin{defi}
The Cantor set $C$ is defined as
\[C=\Big\{x\in [0,1] \;|\; x=\sum _{n=1}^{\infty }\dfrac{a_{n}}{3^{n}}\Big\}=[0,1]\;\backslash\; \bigcup ^{\infty }_{n=0}\bigcup_{k=0}^{3^{n}-1}\Big(\dfrac{3k+1}{3^{n+1}},\dfrac{3k+2}{3^{n+1}}\Big)\]
where $a_{n}\in \{0,2\}$. The expression for $x$ is called the expansion with base 3 not containing 1. Therefore, the Cantor set can be seen as all numbers between $[0,1]$ expressible as a base 3 expansion without a one.
\[\begin{cases}
	x=0.13\;[3]\iff x=\dfrac{1}{3}+\dfrac{2}{3^2}\\
	x=0.1\;[3]\iff x=0.0222\;[3]
\end{cases}\]
By constantly taking out the middle third of the interval $[0,1]$, you obtain $C_{n}$ which would be the union of $2^{n}$ disjoint closed intervals with each a length of $1/3^{n}$. It is a closed set with a measure of $(2/3)^{n}$. Then, $C=\bigcap ^{\infty }_{n=1}C_{n}$ which is a decreasing intersection. The length sum of all removed intervals is
\[\dfrac{1}{3}+2\cdot \dfrac{1}{3^2}+2^2\cdot \dfrac{1}{3^3}+\ldots =\dfrac{\frac{1}{3}}{1-\frac{2}{3}}=1\]
and the measure of the Cantor set is zero $m(C)=0$. The Cantor set is therefore a totally disconnected (it contains no interval) perfect set (it is nowhere dense). 
\end{defi}
\vspace{2ex}
\begin{defi}
The function $f:C\rightarrow [0,1]$ is defined by
\[f\Big(\sum _{n=1}^{\infty }\dfrac{2b_{n}}{3^{n}}\Big)=\sum ^{\infty }_{n=1}\dfrac{b_{n}}{2^{n}}\]
where $b_{n}=0$ or $1$. In other words,
\[f (0.a_1a_2a_3\ldots[3] )=0.\dfrac{a_1}{2}\dfrac{a_2}{2}\dfrac{a_3}{2}\ldots [2]\]
We note that $f (C)=[0,1]$ meaning that $f$ is onto. This because for each $x\in [0,1]$ can be expressed as $x=0.b_1b_2\ldots [2]$ (a binary expansion) where $b_{n}=0$ or 1. That is, for any $x\in [0,1]$,
\[\exists \;y=\sum ^{\infty }_{n=1}\dfrac{2b_{n}}{3^{n}}\in C\hspace{2ex}\mathrm{such\ that}\hspace{2ex}f (y)=x\]
This precisely means that $C$ is uncountable. In the field, $C$ is indeed the most popular uncountable measure zero set. 
\\\\
Meanwhile, note that endpoints of a curtailment are always in the Cantor set. The function values for the first two endpoints are 
\begin{align*}
	f \Big(\dfrac{1}{3}\Big)=&f (0.0222\ldots [3])=0.0111\ldots [2]=0.1\;[2]=\dfrac{1}{2}\\
	f \Big(\dfrac{2}{3}\Big)=&f(0.2\;[3])=0.1\;[2]=\dfrac{1}{2}
\end{align*}
If $I_{k}$ is one of the many removed open intervals with $I_{k}=(a_{k},b_{k})$ then $a_{k},b_{k}\in C$ and $f(a_{k})=f(b_{k})$. We define the Cantor function $\phi :[0,1]\rightarrow [0,1]$ as
\[\phi (x)=\begin{cases}
f (x)\hspace{5ex}\mathrm{if}\;x\in C\\
f (a_{k})\hspace{5ex}\mathrm{if}\;x\in I_{k}=(a_{k},b_{k})
\end{cases}\]
\end{defi}
\vspace{2ex}
\begin{ex}
\begin{align*}
	f  (0.202020\ldots [3])=&0.101010\ldots [2]\\
	=&\dfrac{1}{2}+\dfrac{1}{2^3}+\dfrac{1}{2^{5}}+\ldots =\dfrac{\dfrac{1}{2}}{1-\dfrac{1}{4}}=\dfrac{2}{3}\in [0,1]
\end{align*}
\end{ex}
\vspace{2ex}
\begin{ex}
Consider the following
\begin{spacing}{2}
\[\begin{cases}
\phi \Big(\dfrac{1}{2}\Big)=f \Big(\dfrac{1}{3}\Big)=\dfrac{1}{2}\\
\phi \Big(\dfrac{1}{3}\Big)=f\Big(\dfrac{1}{3}\Big)=\dfrac{1}{2}\\
\phi\Big(\dfrac{1}{5}\Big)=f \Big(\dfrac{1}{9}\Big)=f \Big(\dfrac{2}{9}\Big)=f (0.00222\ldots [3])=0.00111[2]=0.01[2]=\dfrac{1}{4}
\end{cases}\]
\end{spacing}
$\phi (1/5)$ would then be equal to $\phi (1/6)=\phi (1/9)$. What about $\phi (1/4)$? As
\[\dfrac{1}{4}=0.020202\ldots [3]\in C\]
and
\[\phi \Big(\dfrac{1}{4}\Big)=f\Big(\dfrac{1}{4}\Big)=0.010101\;[2]=\dfrac{1}{3}\]
Notice how you can let all numbers after 1 to be 0 and set all front 2's to 1 and expand it in binary. For example, if
\[\phi (0.22121021\;[3])=0.111\;[2]=\dfrac{1}{2}+\dfrac{1}{2^2}+\dfrac{1}{2^3}\]
Another example would be
\[\phi (0.2021012\;[3])=0.1011\;[2]=\dfrac{1}{2}+\dfrac{1}{2^3}+\dfrac{1}{2^{4}}\]
\end{ex}
\vspace{2ex}
\begin{rmk}
Let $I=C\cup G$ where $G$ is a union of removed open intervals (countable). We have previously established that $f (C)\subset [0,1]$, and as the number of outputs are countable, $m(\phi  (G))=0$. As we know that the function is monotone increasing (non-decreasing) on $[0,1]$, the fact that $\phi ([0,1])=[0,1]$ (surjective to an interval) tells us that $f$ is continuous.
\end{rmk}
\vspace{2ex}

