\section{Lecture 11 (April 8th)}
\begin{thm}
(MCT) Let $f\in L^{+}$ be a non-negative measurable function. For $f_{n}\in L^{+}(E)$ and $f_{n}\uparrow f$ (increasingly), we have
\[\int _{E}f\;dm=\lim _{n\rightarrow \infty }\int _{E}f_{n}\;dm\]
\end{thm}
\vspace{2ex}
\begin{thm}
If $f,g\in L^{+}(E)$ and for $c>0$, then
\begin{align*}
\int _{E}(f+g)d m=&\int _{E}f\;d m+\int _{E}g\;d m\\
\int _{E}cf\;d m=&c\int _{E}f\;d m
\end{align*}
\end{thm}
\vspace{2ex}
\begin{proof}
If $\phi _{n},\psi _{n}\in L^{+}(E)$ are simple such that $\phi _{n}\uparrow f$ and $\psi _{n}\uparrow g$, then $\phi _{n}+\psi _{n}\uparrow f+g$ and
\[\int _{E}(\phi  _{n}+\psi _{n})\;d m=\int _{E}\phi _{n}\;d m+\int _{E}\psi _{n}\;d m\]
By MCT,
\begin{align*}
\int _{E}(f+g)\;d m=&\lim \int _{E}(\phi _{n}+\psi _{n})\;d m=\lim \Big(\int _{E}\phi _{n}\;d m+\int _{E}\psi _{n}\;d m\Big)\\=&\lim \int _{E}\phi _{n}+\lim \int _{E}\psi _{n}=\int _{E}f+\int _{E}g
\end{align*}
\end{proof}
\vspace{2ex}
\begin{thm}
If $f,g\in L^{+}(E)$ and $g(x)\leq f(x)$ for all $x\in E$ and $\int _{E}g\;d m<\infty $, then
\[\int _{E}(f-g)\;d m=\int _{E}f\;d m-\int _{E}g\;d m\]
\end{thm}
\vspace{2ex}
\begin{proof}
Since $f=(f-g)+g$ we have $\int _{E}f=\int _{E}(f-g)+\int _{E}g$. Especially, if $0\leq g(x)\leq f(x)$ on $E$ and $\int _{E}g\;d m<\infty $ then $\int _{E}(f-g)\;d m=\int _{E}f\;d m-\int _{E}g\;d m$. 
\end{proof}
\vspace{2ex}
\begin{lem}
(Fatou's Lemma) If $f_{n}\in L^{+}(E)$ then 
\[\int _{E}\liminf_{n\rightarrow \infty } f_{n}\;dm\leq \liminf_{n\rightarrow \infty }\int _{E}f_{n}\;d m\]
Especially, if $f_{n}\in L^{+}$ and $f_{n}\rightarrow f$ pointwise (almost everywhere) on $E$, then
\[\int _{E}f\;d m\leq \liminf_{n\rightarrow \infty } \int _{E}f_{n}\;d m\]
\end{lem}
\vspace{2ex}
\begin{proof}
Define $g_{n}(x)=\inf_{k\geq n}\{f_{k}(x)\}$ then $g_{n}\in L^{+}(E)$ and $g_{n}\uparrow\liminf_{n\rightarrow \infty } \int _{E}f_{n}$. Also, if $k \leq n$ then $g_{k}\leq f_{n}$. Applying the MCT on $g_{n}$, we have
\[\lim_{n\rightarrow \infty } \int _{E}g_{n}\;d m=\int _{E}\liminf_{n\rightarrow \infty } f\;d m\]
and
\[\int _{E}g_{n}\;d m\leq \liminf_{n\rightarrow \infty } \int _{E}f_{n}\;d m\]
comparing the two, we arrive at
\[\int _{E}\liminf_{n\rightarrow \infty } f_{n}\;d m\leq \liminf_{n\rightarrow \infty } \int _{E}f\;d m\]
The integral of the limit is less than the limit of the integral!
\end{proof}
\vspace{2ex}
\begin{ex}
Suppose $f_{n}$ is negative, for example $f_{n}=-{\bf 1}_{(0,n)}/n$. Then the above theorem doesn't work.
\end{ex}
\vspace{2ex}
\begin{rmk}
Notice how MCT and Fatou's lemma or equivalent. We show that Fatou's lemma implies MCT.
\end{rmk}
\vspace{2ex}
\begin{proof}
If $0\leq f_{n}\leq f$ and $f_{n}\rightarrow f$ pointwise (almost everywhere) on $E$, then
\[\int _{E}f_{n}\;d m\leq \int _{E}f\;d m\]
implies that
\[\limsup_{n\rightarrow \infty }\int _{E}f_{n}\;d m\leq \int _{E}f\;d m\]
By Fatou's lemma, 
\[\int _{E}f\;d m\leq \liminf_{n\rightarrow \infty }\int _{E}f_{n}\;d m\]
Together,
\[\int _{E}f\;d m=\lim_{n\rightarrow \infty } \int _{E}f_{n}\;d m\]
\end{proof}
\vspace{2ex}
\begin{ex}
If $f_{n}={\bf 1}_{(n,\infty )}$ then $f_{n}\rightarrow 0$ decreasingly,
\[\liminf_{n\rightarrow \infty }\int _{{\bm R}}f\;d m=\infty \]
but
\[\int _{{\bm R}}f\;d m=\int _{{\bm R}}0\;d m=0\]
\end{ex}
\vspace{2ex}
\begin{cor}
If $f_{n}\in L^{+}$ and $f_{n}\uparrow f$ satisfies $\int _{E}f\;d m\ne \lim \int _{E}f_{n}\;d m$ then
\[\int_{E}f_{m}\;d m=\infty \]
for all $m\in {\bm N}$. 
\end{cor}
\vspace{2ex}
\begin{proof}
Suppose $\int _{E}f_{k}\;d m<\infty $ then $f_{k}-f_{n}\rightarrow f_{k}-f$ as $n\rightarrow \infty $ which implies due to MCT
\[\int _{E}(f_{k}-f)\;d m=\lim _{n\rightarrow \infty }\int _{E}(f_{k}-f_{n})\;d m\]
which implies 
\[\int _{E}f_{k}\;d m-\int _{E}f\;d m=\int _{E}f_{k}\;d m-\lim _{n\rightarrow \infty }\int _{E}f_{n}\;d m\]
Notice how if even one integral is convergent, we can prove that the limit of the integral is equal to the integral of the limit (a contradiction). We thus complete our proof. 
\end{proof}
\vspace{2ex}
\begin{ex}
Consider $E=[0,2]$ and 
\[f_{n}=\begin{cases}
{\bf 1}_{[0,1)}\hspace{3ex}n=2m-1\\
{\bf 1}_{[1,2]}\hspace{3ex}n=2m
\end{cases}\]
then $\int _{E}f_{n}\;d m=1$ on $[0,2]$ and $\liminf f_{n}(x)=0$ on $[0,2]$. 
\[\int _{E}\limsup f_{n}\;d m=2\hspace{5ex}\int _{E}\liminf f_{n}\;d m=0\]
\end{ex}
\vspace{2ex}
\begin{thm}
(Chebychev) Let $f\in L^{+}(E)$ and $\lambda >0$. Then,
\[m(\{x\in E \;|\; f(x)\geq \lambda \})\leq\dfrac{1}{\lambda }\int _{E}f\;d m\]
\end{thm}
\vspace{2ex}
\begin{proof}
Let $E_{\lambda }=\{x\in E \;|\; f(x)\geq \lambda \}$. Then $f\geq \lambda {\bf 1}_{E_{\lambda }}$ so that
\[\int _{E}f\;d m\geq \int _{E}\lambda {\bf 1}_{E_{\lambda }}\;d m=\lambda m(\{x\in E \;|\; f(x)\geq \lambda \})\]
\end{proof}
\vspace{2ex}
\begin{thm}
For $f\in L^{+}(E)$, $\int _{E}f\;d m=0$ if and only if $m(\{x\in E \;|\; f(x)>0\})=0$. 
\end{thm}
\vspace{2ex}
\begin{proof}
First prove that if $m(A)=0$ then $\int _{A}f\;d m=0$. Let $A=\{x\in E \;|\; f(x)>0\}$. Then
\[\int _{E}f\;d m=\int _{A}f\;d m+\int _{E\;\backslash\;A}f\;d m\]
If $\int _{E}f\;d m=0$, then by Chevychev, for every $n\in {\bm N}$,
\[0=\int _{E}f\;dm\geq \dfrac{1}{n}m\Big(\Big\{x\in E \;\Big|\; f(x)\geq \dfrac{1}{n}\Big\}\Big)\]
so that $m(\{x\in E \;|\; f(x)\geq {1}/{n}\})=0$ for every $n$. Note that
\[\{x\in E \;|\; f(x)>0\}=\bigcup ^{\infty }_{n=1}\Big\{x\in E \;|\; f(x)\geq \dfrac{1}{n}\Big\}\]
is measure zero.
\end{proof}
\vspace{2ex}
\begin{cor}
If $f\in L^{+}(E)$, then $\int _{E}f\;d m=0$ if and only if $f=0$ almost everywhere on $E$. 
\end{cor}
\vspace{2ex}
\begin{thm}
If $f_{n}\in L^{+}(E)$ then 
\[\int _{E}\sum ^{\infty }_{n=1}f_{n}\;d m=\sum ^{\infty }_{n=1}\int _{E}f_{n}\;d m\]
\end{thm}
\vspace{2ex}
\begin{proof}
We know that
\[\int _{E}\sum ^{n}_{k=1}f_{k}\;d m=\sum ^{n}_{k=1}\int _{E}f_{k}\;d m\]
if $g_{n}=\sum ^{n}_{k=1}f_{k}$ then $g_{n}\rightarrow \sum ^{\infty }_{n=1}f_{n}$ increasingly. By MCT,
\[\int _{E}\sum ^{\infty }_{n=1}f_{n}\;d m=\lim _{n\rightarrow \infty }\int _{E}g_{n}\;d m=\lim _{n\rightarrow \infty }\sum ^{n}_{k=1}\int _{E}f_{k}\;d m
=\sum ^{\infty }_{k=1}\int _{E}f_{k}\;d m
\]
\end{proof}
\vspace{2ex}
\begin{thm}
For $f\in L^{+}(E)$, if $\mu :\mathfrak{M}\rightarrow [0,\infty ]$ is defined by $\mu (E)=\int _{E}f\; d m$. Then $\mu $ is a measure on $\mathfrak{M}$ and $m(A)=0$ implies $\mu (A)=0$ by the previous theorem.
\end{thm}
\vspace{2ex}
\begin{proof}
Let $E_1,\ldots,E_{n}\in \mathfrak{M}$ be mutually disjoint and $E=\bigcup ^{\infty }_{n=1}E_{n}$. We have to show that $\mu (E)=\sum _{n=1}^{\infty }\mu (E_{n})$. We know that ${\bf 1}_{A\cup B}={\bf 1}_{A}+{\bf 1}_{B}$ if $A\cap B=\emptyset$. Defining the function $g_{n}$ like the following: 
\[g_{n}=f{\bf 1}_{\bigcup ^{n}_{k=1}E_{k}}=f{\bf 1}_{E_{1}}+\ldots +f{\bf 1}_{E_{n}}\rightarrow f{\bf 1}_{E}\]
The convergence above is monotone, and we can apply MCT. Then,
\begin{align*}
\mu (E)=&\int _{E}f\; dm=\int _{{\bm R}}f{\bf 1}_{E}\; d m=\lim_{n\rightarrow \infty }\int _{{\bm R}}g_{n}\; dm\\
=&\lim _{n\rightarrow \infty }\sum ^{n}_{k=1}\int _{{\bm R}}f{\bf 1}_{E_{k}}\;d m=\lim _{n\rightarrow \infty }\sum ^{n}_{k=1}\int _{E_{k}}f\;d m\\
=&\lim_{n\rightarrow \infty}\sum ^{n }_{k=1}\mu (E_{k})=\sum ^{\infty }_{k=1}\mu (E_{k})
\end{align*}
\end{proof}
\vspace{2ex}
\begin{defi}
For $f\in L^{+}(E)$, if $\int _{E}f\; dm<\infty $, then we denote $f\in L^{1}(E)$ and $f$ is called integrable on $E$. 
\end{defi}
\vspace{2ex}
\begin{thm}
If $f\in L^{+}(E)$ satisfies $\int _{E}f\; dm<\infty $, then for every $\varepsilon >0$ there is $\delta >0$ such that if $A\subset E$ satisfies $m(A)<\delta $, then $\int _{A}f\;d m<\varepsilon $. 
\end{thm}
\vspace{2ex}
\begin{proof}
Define $f_{n}(x)=\min\{f(x), n\}$ then $f_{n}\in L^{+}(E)\cap L^{1}(E)$ with $f_{n}\rightarrow f$ increasingly as $n\rightarrow \infty $. By MCT,
\[\lim _{n\rightarrow \infty }\int _{E}f_{n}\;d m=\int _{E}f\;d m\]
Thus, there exists an $N\in {\bm N}$ such that
\[\Big|\int _{E}f\;d m-\int _{E}f_{N}\;d m\Big|<\dfrac{\varepsilon }{2}\]
Note that since $\int _{E}f\;d m<\infty $,
\[\int _{E}(f-f_{N})\;dm<\dfrac{\varepsilon }{2}\]
Take $\delta =\varepsilon /2N$. If $m(A)<\delta $, then
\begin{align*}
\int _{A}f\;d m=&\int _{A}(f-f_{N})\;d m+\int _{A}f_{N}\;d m\\
\leq&\int _{E}(f-f_{N})\;d m+Nm(A)<\dfrac{\varepsilon }{2}+N\dfrac{\varepsilon }{2N}=\varepsilon 
\end{align*}
\end{proof}
\vspace{2ex}
\begin{thm}
(Generalised Lebesgue theorem) If $f$ is measurable on $E$, then we define 
\[\int _{E}f\;d m=\int _{E}f^{+}\;d m-\int _{E}f^{-}\;d m\]
provided that $f^{+},f^{-}\in L^{1}(E)$, that is, $|f|=f^{+}-f^{-}\in L^{1}(E)$. Here, $f=f^{+}-f^{-}$ where $f^{+},f^{-}\in L^{+}(E)$. 
\end{thm}
\vspace{2ex}
\begin{ex}
\[\int ^{\infty }_{0}\dfrac{\sin x}{x}\;dx\]
converges in terms of the Riemann integral. However,
\[\int ^{\infty }_{0}\Big|\dfrac{\sin x}{x}\Big|\;dx=\infty \]
so that the integral is not defined as a Lebesgue integral. Again, if 
\[\int _{E}|f|\;d m=\infty \]
then we say $f$ is not integrable. 
\end{ex}
\vspace{2ex}
\begin{defi}
We define $f\in L^{1}(E)$ or $f$ is integrable on $E$ if $|f|\in L^{1}(E)$ and 
\[\int _{E}f\;d m=\int _{E}f^{+}\;d m-\int _{E}f^{-}\;d m\]
\end{defi}
\vspace{2ex}
 
