\section{Lecture 25 (June 12th)}
\begin{defi}
An orthonormal set $\{u_{\alpha } \,|\, \alpha \in I\}$ in a Hilbert space $H$ is said to be complete (a complete orthonormal set, orthonormal basis, or maximal orthonormal set) provided that if $x\in H$ satisfies 
\[\langle x,u_{\alpha }\rangle =0\]
for all $\alpha \in I$, then $x={\bf 0}$. 
\end{defi}
\vspace{2ex}
\begin{thm}
(Axiom of choice) If $\{u_{\alpha }\}$ is an orthonormal set in $H$, then there is a complete orthonormal set $E$ which contains $\{u_{\alpha }\}$. As a result, all Hilbert spaces have a countable orthonormal basis. 
\end{thm}
\vspace{2ex}
\begin{thm}
If $E=\{u_{\alpha }|\alpha \in I\}$ is an orthonormal set in a separable Hilbert space, then $E$ is countable.
\end{thm}
\vspace{2ex}
\begin{proof}
Let $D$ be a countable dense subset of $H$. Let $i:E\rightarrow D$ satisfy 
\[||u-i(u)||<\dfrac{1}{2}\]
If $i(u)=i(v)$ then 
\[||u-v||=||u-i(u)+i(v)-v||\leq ||u-i(u)||+||v+i(v)||<1\]
so that $u=v$ as $||u-v||=\sqrt{2}$ if they are orthonormal. Since $i$ is one-to-one and $D$ is countable, $E$ is also countable. 
\end{proof}
\vspace{2ex}
\begin{thm}
Let $E=\{u_{n} \,|\, n\in {\bm N}\}$  be an orthonormal set in a separable Hilbert space $H$. 
\begin{itemize}
\item[(i)] (Bessel inequality) For every $x\in H$, we have 
\[\sum ^{\infty }_{n=1}\langle x,u_{n}\rangle ^2\leq ||x||^2\]
\item[(ii)] For ${\bm \alpha } =(\alpha_1,\alpha_2,\ldots ,\alpha _{n},\ldots )\in l^{2}({\bm N})$, 
\[\sum ^{\infty }_{n=1}\alpha _{n}u_{n}\in H\]
\item[(iii)] If $x\in H$ and 
\[y=\sum ^{\infty }_{n=1}\langle x,u_{n}\rangle u_{n}\]
then $\langle x,u_{n}\rangle =\langle y,u_{n}\rangle $ for all $n\in {\bm N}$
\end{itemize}
\end{thm}
\vspace{2ex}
\begin{proof}
({\it STEP} 1) We proved that
\[\sum ^{n}_{k=1}\langle x,u_{k}\rangle ^2\leq ||x||^2\]
for every $n$ so that the inequality works for infinity.
\\\newline
({\it STEP} 2) Let 
\[x_{n}=\sum ^{n}_{k=1}\alpha _{k}u_{k}\]
Then we'll show that $x_{n}$ is a Cauchy sequence in $H$. Note that if $n>m$ then 
\[||x_{n}-x_{m}||^2=\langle \sum ^{n}_{k=m+1}\alpha _{k}u_{k},\sum _{j=m+1}^{n}\alpha _{j}u_{j}\rangle =\sum ^{n}_{k=m+1}\alpha _{k}^2\]
Notice that this can be made arbitarily small by setting $m$ to be large. 
\\\newline
({\it STEP} 3) By (i), $\{\langle x,u_{n}\rangle \}\in l^{^2}({\bm N})$ and by (ii) $y\in H$. Thus for every $m\in {\bm N}$
\[\langle y,u_{m}\rangle =\langle \sum ^{\infty }_{n=1}\langle x,u_{n}\rangle u_{n},u_{m}\rangle =\lim _{k\rightarrow \infty }\sum ^{k}_{n=1}\langle \langle x,u_{n}\rangle u_{n},u_{m}\rangle =\langle x,u_{m}\rangle  \]
\end{proof}
\vspace{2ex}
\begin{thm}
Let $\{u_{n} \,|\, n\in {\bm N}\}$ be a complete orthonormal set in a separable Hilbert space $H$.
\begin{itemize}
\item[(i)] For every $x\in H$, we have 
\[x=\sum ^{\infty }_{n=1}\langle x,u_{n}\rangle u_{n}\]
\item[(ii)] (Parseval identity) If $x,y\in H$ then 
\[\langle x,y\rangle =\sum ^{\infty }_{n=1}\langle x,u_{n}\rangle \langle y,u_{m}\rangle \]
especially,
\[||x||^2=\sum ^{\infty }_{n=1}\langle x,u_{n}\rangle ^2\]
\end{itemize}
\end{thm}
\vspace{2ex}
\begin{proof}
Let 
\[y=\sum ^{\infty }_{n=1}\langle x,u_{n}\rangle u_{n}\]
then $y\in H$ and $\langle y,u_{n}\rangle =\langle x,u_{n}\rangle $ for all $n\in {\bm N}$ so that $\langle x-y,u_{n}\rangle =0$ for all $n\in {\bm N}$. Since $\{u_{n}\}$ is complete, we have $x=y$. Therefore,
\[x=\sum ^{\infty }_{n=1}\langle x,u_{n}\rangle u_{n}\quad \Big(=\sum ^{\infty }_{n=1}\hat{x}(n)u_{n}\Big)\]
We can take the above to be, for example, the Fourier series expansion of $x$ in terms of $\{u_{n}\}$. 
\end{proof}
\vspace{2ex}
\begin{cor}
It is known that $\{u_{n}\}^{\infty }_{n=-\infty }$ defined by
\[u_{n}(t)=e^{2\pi int}\]
is a complete orthonormal set in $L^2([0,1])$. Here, $L^2([0,1])$ is the sapce of functions $f:[0,1]\rightarrow {\bm C}$ with
\[\int ^{1}_{0}|f(t)|^2\,dt<\infty\]
and the norm defined as
\[\langle f,g\rangle =\int ^{1}_{0}f(t)\overline{g(t)}\,dt\]
Then by application of the above theorem, we have for all $f\in L^{2}([0,1])$
\[f(x)=\sum ^{\infty }_{n=-\infty }\Big(\int ^{1}_{0}f(t)e^{-2\pi in t}\,dt\Big)e^{2\pi inx} \]
in terms of $L^2([0,1])$. 
\end{cor}
\vspace{2ex}
\begin{lem}
The inner product $\langle ,\rangle :H\times H\rightarrow {\bm R}$ is continuous. 
\end{lem}
\vspace{2ex}
\begin{proof}
Take any $(a,b)\in H\times H$ and $\varepsilon >0$. We have to find $\delta >0$ so that if $||(x,y)-(a,b)||<\delta $ then 
\[|\langle x,y\rangle -\langle a,b\rangle |<\varepsilon \]
We see that the right hand side satisfies
\[|\langle x,y\rangle -\langle a,b\rangle |=|\langle x,y\rangle -\langle a,y\rangle +\langle a,y\rangle -\langle a,b\rangle |\leq |\langle x-a,y\rangle |+|\langle a,y-b\rangle |\]
such that 
\[|\langle x,y\rangle -\langle a,b\rangle |\leq ||x-a||\,||y||+||a||\,||y-b||\leq \delta (||b||+1)+||a||\delta \]
as if $\delta <1$ $||x||\leq ||a||+1$ and $||y||\leq ||b||+1$. 
\end{proof}
\vspace{2ex}
\begin{proof}
Observe that
\begin{align*}
\langle x,y\rangle =&\langle \sum ^{\infty }_{j=1}\langle x,u_{j}\rangle u_{j},\sum ^{\infty }_{k=1}\langle y,u_{k}\rangle u_{k}\rangle \\
=&\lim _{n\rightarrow \infty }\langle \sum ^{n}_{j=1}\langle x,u_{j}\rangle u_{j},\sum ^{n}_{k=1}\langle y,u_{k}\rangle u_{k}\rangle =\sum ^{\infty }_{n=1}\langle x,u_{n}\rangle \langle y,u_{n}\rangle 
\end{align*}
\end{proof}
\vspace{2ex}

