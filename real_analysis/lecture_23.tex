\section{Lecture 23 (June 10th)}
\begin{thm}
(Riesz-representation theorem) If $T$ is a bounded linear functional on a Hilbert space $H$, then there is a unique element $y\in H$ such that $T(x)=\langle x,y\rangle $ for all $x\in H$. 
\end{thm}
\vspace{2ex}
\begin{proof}
If $T(x)=0$ for all $x\in H$, then simply put $y=0$. If $T(x)\ne 0$ for some $x\in H$, define
\[M=\{x\in H \,|\, T(x)=0\}\]
which would be a closed subspace of $H$ so that there is $z\in M^{\perp}$ with $||z||=1$. We will show that $y=T(z)z$ satifies $T(x)=\langle x,y\rangle $ for all $x\in H$. Take any $x\in H$ and define $u=T(x)z-T(z)x$. Then,
\[T(u)=T(x)T(z)-T(z)T(x)=0\]
implies that $u\in M$. Therefore, $\langle u,z\rangle =0$ which means that 
\[0=\langle u,z\rangle =\langle T(x)z-T(z)x,z\rangle =T(x)-\langle x,T(z)z\rangle \]
That is, $T(x)=\langle x,y\rangle $. In addition to this, this $y$ is unique, as if there is $w\in H$ such that $T(x)=\langle x,w\rangle $ for all $x\in H$,
\[\langle y,x\rangle =\langle w,x\rangle \]
for all $x\in H$ and 
\[\langle y-w,x\rangle =0\]
By substituting $x=y-w$, we find that $y=w$. Note how $|T(x)|\leq ||x||\,||y||$ and how $T(y)=||y||^2$ such that $||T||=||y||$.
\end{proof}
\vspace{2ex}
\begin{thm}
Let $T\in \mathcal{L}(H)=\mathcal{L}(H,H)$, a linear operator between two Hilbert spaces. If we define $L:H\rightarrow {\bm R}$ by 
\[L(x)=\langle T(x),y\rangle \] 
$L$ is linear and 
\[|L(x)|\leq ||T(x)||\,||y||\leq ||T||\,||y||\,||x||\]
so that $L$ is a bounded linear functional on $H$. By the Riez representation theorem, there is a unique $u\in H$ such that 
\[L(x)=\langle x,u\rangle \]
That is, for every $x,y\in H$, there is $u\in H$ such that 
\[\langle T(x),y\rangle =\langle x,u\rangle \]
We define $T^{*}:H\rightarrow H$ by $T^{*}(y)=u$ (implying $\langle T(x),y\rangle =\langle x,T^{*}(y)$), then $T^{*}$ is linear and
\[||T^{*}(y)||^2=\langle T^{*}(y),T^{*}(y)\rangle =\langle T(T^{*}(y)),y\rangle \leq ||T||\,||T^{*}(y)||\,||y||\]
which implies that 
\[||T^{*}(y)||\leq ||T||\,||y||\]
and 
\[||T^{*}||\leq ||T||\]
as $T^{**}=T$, the equality works vice-versa, and $||T^{*}||=||T||$.
\end{thm}
\vspace{2ex}
\begin{rmk}
If $T\in \mathcal{L}(H)$, then $N(T^{*})=R(T)^{\perp}$ and $N(T)=R(T^{*})^{\perp}$ as if $T^{*}(y)=0$, $\langle x,T^{*}(y)\rangle =0$ for all $x\in H$ and $\langle T(x),y\rangle =0$ for all $x\in H$ and $y\in R(T)^{\perp}$. 
\end{rmk}
\vspace{2ex}
\begin{defi}
(Orthonormal set in $H$) If $\{u_1,\ldots ,u_{n}\}$ is an orthonormal set in $H$, then it is linearly independent. The linear span of $\{u_{\alpha } \,|\, \alpha \in I\}$ which are orthornomal is the set of all finite lienar combinations of $\{u_{\alpha }\}$ which is the smallest subspace of $H$ which contains $\{u_{\alpha }\}$.
\end{defi}
\vspace{2ex}
\begin{defi}
If $\{u_{\alpha } \,|\, \alpha \in I\}$ is an orthonormal set in $H$, then for $x\in H$ we define $\hat{x}(\alpha )=\langle x,u_{a}\rangle $ the Fourier coefficient of $x$ with respect to the orthonormal set $\{u_{\alpha }\}$.
\end{defi}
\vspace{2ex}
\begin{thm}
Let $\{u_{\alpha } \,|\, \alpha \in I\}$ be an orthonormal set and let $F$ be a finite subset of $I$. $M_{F}$ is the linear span of $\{u_{k} \,|\,k\in F\}$.
\begin{itemize}
\item[(i)] If $y\in M_{F}$ satisfies $y=\sum _{k\in F}a_{k}u_{k}$ then $a_{k}=\hat{y}(k)=\langle y,u_{k}\rangle $ for all $k\in F$ and
\[||y||^2=\sum _{k\in F}|\hat{y}(k)|^2\]
\item[(ii)] (Best approximation property) If $x\in H$ and $s_{F}(x)= \sum _{k\in F}\hat{x}(k)u_{k}$ then for every $s \in M_{F}$ except $s=s_{F}$, 
\[||x-s_{F}||<||x-s||\]
\end{itemize}
\end{thm}
\vspace{2ex}
\begin{proof}
({\it STEP} 1) If $y=\sum _{k \in F}a_{k}u_{k}$, then for $m\in F$, 
\[\hat{y}(m)=\langle \sum _{k\in F}a_{k}u_{k},u_{m}\rangle =a_{m}\]
({\it STEP} 2) The key idea is that 
\[||x-s||^2=||\underbrace{x-s_{F}}_{\in M_{F}^{\perp}}+\underbrace{s_{F}-s}_{\in M_{F}}||^2=||x-s_{F}||^2+||s_{F}-s||^2\]
which is true as for $m\in F$,
\[\hat{s}_{F}(m)=\langle \sum _{k\in F}\hat{x}(k)u_{k},u_{m}\rangle =\hat{x}(m)\]
and $\langle x-s_{F},u_{m}\rangle =0$ for all $m\in F$. This implies that
\[||x-s||^2\leq ||x-s_{F}||^2+||s_{F}-s||^2\]
and by putting $s=0$ we have
\[||s_{F}||^2\leq ||x||^2\]
which implies the Bessel inequality
\[\sum _{k\in F}|\hat{x}(k)|^2\leq ||x||^2\]

\end{proof}
\vspace{2ex}

