\section{Lecture 15 (May 8th)}
\begin{defi}
(Norm) A norm is a function $||\cdot ||:X\rightarrow [0,\infty )$ from a vector space of ${\bm R}$ that satisfies the following properties:
\begin{itemize}
\item[(i)] (Nonnegativity) $||x||\geq 0$ for all $x\in X$ and $||x||=0$ if and only if $x={\bf 0}$ (that is, $x$ is the zero vector)
\item[(ii)] (Positive homogenity) $||\alpha x||=|\alpha |\,||x||$ for all $\alpha \in {\bm R}$ and $x\in X$
\item[(iii)] (The triangle inequality) $||x+y||\leq ||x||+||y||$ for all $x$ and $y$ in $X$
\end{itemize}
Note that $d(x,y)=||x-y||$ is a metric on $X$. Therefore, all normed vector spaces are a metric space.
\end{defi}
\vspace{2ex}
\begin{defi}
(Banach space) A metric space is called complete if every Cauchy sequence in $X$ converges (in $X$) with respect to the metric. A normed vector space that is complete with respect to its norm that acts as a metric is called a Banach space.
\end{defi}
\vspace{2ex}
\begin{ex}
(Examples of normed vector spaces) 
\begin{itemize}
\item[(i)] ($C([0,1])$ with uniform norm) $C([0,1])=\{f\,|\,f\mathrm{\ is\ continuous\ on\ }[0,1]\}$ maybe given the norm $||f||=\max_{[0,1]}|f(x)|$. With respect to this norm, $C([0,1])$ is also a Banach space. This is because 
\[|f_{n}(x)-f_{m}(x)|\leq ||f_{n}(x)-f_{m}(x)||\]
and all Cauchy sequences converge uniformly to some $f$ that is continuous. 
\item[(ii)] ($C([0,1])$ with $L^1$ norm) $C([0,1])$ may also given the norm $||f||=\int ^{1}_{0}|f(x)|\,dx$. With respect to this norm, $C([0,1])$ is not Banach space. This is because Cauchy sequences such as  $\{f_{n}\}=\{x_{n}\}$ do not converge to a continuous function.
\item[(iii)] ($C^1([0,1])$ with uniform norm) $C^{1}([0,1])$ is a normed vector space with respect to the uniform norm but it is not complete. For example, take $f_{n}(x)=\sqrt{x+1/n}$. $f_{n}\rightarrow f(x)=\sqrt{x}$ uniformly on $[0,1]$ which implies that it is a Cauchy sequence, but it is not in $C^{1}([0,1])$.
\end{itemize}
\end{ex}
\vspace{2ex}
\begin{defi}
($L^{1}(E)$) For a measure space $(E,X,m)$, the space $L^{1}(E)$ is the vector space of measurable functions endowed with the function $||f||_{1}=\int _{E}|f|\,dm$. Notice that 
\begin{align*}
\int _{E}|f|\,d m=0\quad \mathrm{then}&\quad f=0\ \mathrm{\ almost\ everywhere\ on\ }E\\
\int _{E}|\alpha f|\,dm=|\alpha |\int _{E}|f|\,d m\quad\mathrm{for}&\quad\alpha \in R,f\in L^{1}(E)\\
\int _{E}|f+g|\,d m\leq \int _{E}|f|\,d m+\int _{E}|g|\,d m\quad\mathrm{for}&\quad f,g\in L^{1}(E)
\end{align*}
Given this knowledge, if $f,g:E\rightarrow {\bm R}$ satisfies $f,g\in L^{1}(E)$ and $f=g$ almost everywhere on $E$, define $f=g$ as an element of $L^{1}(E)$. Then, $L^{1}(E)$ is a normed vector space with respect to the norm $||f||_{1}=\int _{E}|f|\,d m$.
\end{defi}
\vspace{2ex}
\begin{defi}
($L^{p}(E)$) Let $1< p<\infty $. For a measure space $(E,X,m)$ , we define $L^{p}(E)$ as the set of measurable functions of a measure space satisfying 
\[\int _{E}|f|^{p}\,dm<\infty \]
If $f,g\in L^{p}(E)$ and $f=g$ almost everywhere on $E$, then define $f$ and $g$  as the same element of $L^{p}(E)$. To satisfy $||\alpha f||_{p}=|\alpha |\,||f||_{p}$, we define
\[||f||_{p}=\Big(\int _{E}|f|^{p}\, dm\Big)^{1/p}\]
We will prove later on that $L^{p}(E)$ is a Banach space for $1\leq p\leq \infty $.
\end{defi}
\vspace{2ex}
\begin{defi}
If $X$ and $Y$ are normed vector spaces, $T:X\rightarrow Y$ is called a linear operator provided that $T(\alpha x+\beta y)=\alpha T(x)+\beta T(y)$ for $\alpha ,\beta \in {\bm R}$ and $x,y\in X$. 
\end{defi}
\vspace{2ex}
\begin{thm}
If $T:X\rightarrow Y$ is a linear operator, then the following are equivalent. 
\begin{itemize}
\item[(i)] $T$ is continuous on $X$
\item[(ii)] $T$ is uniformly continuous on $X$
\item[(iii)] $T$ is continuous at ${\bf 0}$
\end{itemize}
\end{thm}
\vspace{2ex}
\begin{defi}
$T:X\rightarrow Y$ is called a bounded linear operator provided that $T$ is linear and there is $M>0$ such that 
\[||T(x)||_{Y}\leq M||x||_{X}\]
for all $x\in X$. Intuitively, bounded linear operators are operators that do not ``blow up" small imputs.
\end{defi}
\vspace{2ex}
\begin{prop}
If $T$ is continuous at ${\bf 0}$, then $T$ is bounded linear.
\end{prop}
\vspace{2ex}
\begin{proof}
By continuity, we see that for every $\varepsilon >0$, there is $\delta >0$ such that if $||x||<\delta $ then $||T(x)||<\varepsilon $. Take any $w\in X\,\backslash\,\{{\bf 0}\}$ and we see that the vector $\delta w/2||w||\in X$ has a norm $\delta /2$. This implies that
\[||T\Big(\dfrac{\delta w}{2||w||}\Big)||<\varepsilon \quad \mathrm{and}\quad T(w)\leq \dfrac{2\varepsilon }{\delta }||w||\]
for all $w\ne {\bf 0}$. 
\end{proof}
\vspace{2ex}

