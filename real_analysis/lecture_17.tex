\section{Lecture 17 (May 15th)}
\begin{thm}
Take $1\leq p<\infty $. Then $L^{p}(E)$ is complete.
\end{thm}
\vspace{2ex}
\begin{proof}
Let $\{f_{n}\}$ be a Cauchy sequence in $L^{p}(E)$. Then there is a subsequence $\{f_{n_{k}}\}$ such that 
\[||f_{n_{k+1}}-f_{n_{k}}||_{p}<\dfrac{1}{k^2}\]
If so, we'll show that 
\begin{itemize}
\item[(i)] $f_{n_{k}}$ converges pointwise almost everywhere on $E$
\item[(ii)] If $f(x)=\lim _{n\rightarrow \infty }f_{n_{k}}(x)$ almost everywhere then $f_{n}\rightarrow f$ in $L^{p}(E)$
\end{itemize}
\newline
({\it STEP} 1) We know that $f\in L^{+}(E)$ and $\int _{E}f\,dm<\infty $ then $m(\{x\in E \,|\, f(x)=\infty \})=0$. Notice how
\[\sum ^{k-1}_{i=1}(f_{n_{i+1}}(x)-f_{n_{i}}(x))=f_{n_{k}}(x)-f_{n_1}(x)\]
Now define $g_{n}(x)=\sum ^{n}_{k=1}|f_{n_{k+1}}(x)-f_{n_{k}}(x)|$ and $g(x)=\sum _{k=1}^{\infty }|f_{n_{k+1}}(x)-f_{n_{k}}(x)|$. Then by the Minkowski inequality, 
\[||g_{n}||_{p}\leq \sum ^{n }_{k=1}\dfrac{1}{k^2}\leq \sum ^{\infty }_{k=1}\dfrac{1}{k^2}=\dfrac{\pi ^2}{6}<2\]
As $g_{n}\nearrow g$, we see that $g_{n}^{p}\nearrow g^{p}$, and due to the monotone convergence theorem, 
\[\int _{E}|g|^{p}\,d m=\lim _{n\rightarrow \infty }\int _{E}|g_{n}|^{p}\,d m<2^{p}\]
In otherwords, $m(\{x\in E \,|\, g(x)=\infty \})=0$ and it converges almost everywhere on $E$ which implies that, by defintiion, 
\[\sum ^{\infty }_{k=1}(f_{n_{k+1}}(x)-f_{n_{k}}(x))\]
converges almost everywhere on $E$. Since 
\[f_{n_{k}}(x)=f_{n_{1}}(x)+\sum ^{k-1}_{i=1}(f_{n_{i+1}}(x)-f_{n_{i}}(x))\]
We see that $\lim _{k\rightarrow \infty }f_{n_{k}}(x)$ converges pointwise almost everywhere on $E$. 
\\\\
({\it STEP} 2) Now define $f(x)=\lim _{k\rightarrow \infty }f_{n_{k}}(x)$ where it converges and 0 elsewhere. We'll show that $f\in L^{p}(E)$ and $\lim _{n\rightarrow \infty }||f_{n}-f||_{p}=0$. Notice how for some $N$, if $n\geq N$ then 
\[\int _{E}|f-f_{n}|^{p}\,d m\leq \liminf_{k\rightarrow \infty }\int _{E}|f_{n_{k}}-f_{n}|^{p}\,d m<\varepsilon ^{p}\]
or equivalently, if $n\geq N$ then $||f-f_{n}||_{p}<\varepsilon $. Additionally, for $n\geq N$, $f$ can be expressed as  
\[f=\underbrace{f-f_{n}}_{\in L^{p}}+\underbrace{f_{n}}_{\in L^{p}}\]
proving that $f\in L^{p}$. 
\end{proof}
\vspace{2ex}
\begin{defi}
$f\in L^{\infty }(E)$ provided that there is $M\geq 0$ such that $|f(x)|\leq M$ almost everywhere on $E$. The norm $||f||_{\infty }$ is defined as the infimum of such uppoer bouneds$ M$. If $f\in L^{\infty }(E)$, $f$ is called essentially bounded and measurable on $E$ and $||f||_{\infty }$ is called the essential supremum of $f$ on $E$. \end{defi}
\vspace{2ex}
\begin{rmk}
\begin{itemize}
\item[(i)] $|f(x)|\leq ||f||_{\infty }$ almost everywhere on $E$
\item[(ii)] If the zero vector of $L^{\infty }(E)$ is defined by zero function almost everywhere then $L^{\infty }(E)$ is a Banach space
\end{itemize}
\end{rmk}
\vspace{2ex}
\begin{defi}
$l^{p}({\bm N})$ is the space $L^{p}({\bm N})$ with respect to the counting measure $\lambda $. We show that elements in $l^{p}({\bm N})$  are bounded sequences. For $f\in l^{1}({\bm N},\lambda )$ define 
\[f_{n}(k)=\begin{cases}
|f(k)|&1\leq k\leq n\\
0&n<k
\end{cases}\]
such that $f_{n}(k)$ is a simple function in $L^{+}({\bm N})$ with $f_{n}\nearrow |f|$. By MCT, 
\[\int _{{\bm N}}|f|\,d \lambda =\lim _{n\rightarrow \infty }\int f_{n}(k)\,d \lambda =\lim _{n\rightarrow \infty }\sum ^{n }_{k=1}|f(k)|=\sum ^{\infty }_{k=1}|f(k)|\]
A sequence $\{a_{n}\}$ ($a_{n}=f(n)$) therefore belongs to $l^{p}({\bm N})$ if 
\[
||\{a_{n}\}||_{p}=\Big(\sum _{n=1}^{\infty }|a_{n}|^{p}\Big)^{1/p}<\infty \]
and $|a_{n}|^{p}$ converges absolutely. As an extension, $\{a_{n}\}\in l^{\infty }({\bm N})$ if 
\[||\{a_{n}\}||_{\infty }=\sup_{n\in {\bm N}}|a_{n}|<\infty \]
and is a bounded sequence.
\end{defi}
\vspace{2ex}
\begin{defi}
Let ${\bf x}=\{a_{n}\}$ be a sequence. ${\bf x}\in l^{2}({\bm N})$ if $||{\bf x}||^2=\sum ^{\infty }_{k=1}|a_{n}|^{2}<\infty $. Consider the sequence $\{e_{n}\}\subset l^2({\bm N})$ (an orthonormal basis for $l^2({\bm N})$) then
\[\begin{cases}
||e_{n}||=1\\
||e_{j}-e_{k}||=\sqrt{2}&j\ne k
\end{cases}\]
this implies that $\{e_{n}\}$ is a bounded sequence in $l^2({\bm N})$ but that there is no subsequence that is a Cauchy sequence. To summerise, $\{e_{n}\}$ is a 
\begin{itemize}
\item[(i)] Bounded sequence with no convergence subsequence in $l^2({\bm N})$
\item[(ii)] Bounded infinite set with no limit point
\item[(iii)] Closed and bounded but not compact 
\end{itemize}
\end{defi}
\vspace{2ex}
\begin{ex}
(Unbounded linear operators) Consider $X:C^{1}([0,1]))$ and $Y:C([0,\pi ])$ both with the uniform norm 
\[||f||=\max_{[0,1]}|f(x)|\]
Let $T:X\rightarrow Y$ be defined by $Tf=f'$ which is linear. Take
\[f_{n}(x)=x^{n}\quad \mathrm{and}\quad g_{n}(x)=\sin (nx)\]
Then, $f_{n},g_{n}\in X$ with $||f_{n}||=1$ and $||g_{n}||=1$ for all $n$. Note that $T(f_{n})(x)=nx^{n-1}$ so that $||Tf_{n}||=n$ and $T(g_{n})(x)=n\cos nx$ so that $T(g_{n})(0)=n$. Therefore, there is no $M>0$ so that 
\[n=||Tf_{n}||\leq M||f_{n}||=M\] nor
\[n=||Tg_{n}||\leq M||g_{n}||=M\]
for all $n\in {\bm N}$.
\end{ex}
\vspace{2ex}
\begin{defi}
(Separable) A normed vector space $X$ is called separable if $X$ has a countable dense subset. 
\end{defi}
\vspace{2ex}
\begin{rmk}
For $1\leq p<\infty $, $L^{p}(E)$ is separable but $L^{\infty }(E)$ is not separable.
\end{rmk}
\vspace{2ex}
\begin{proof}
Suppose that $\{f_{n}\}$ is a countable dense subset of $L^{\infty }([a,b])$. Then, for every $a<x<b$ there is $f_{n(x)}$ such that 
\[||{\bf 1}_{[a,x]}-f_{n(x)}||_{\infty }<\dfrac{1}{2}\]
Suppose $a<x<y<b$ satisfies $f_{n(x)}=f_{n(y)}$. Then, 
\begin{align*}
1=&||{\bf 1}_{[a,x]}-{\bf 1}_{[a,y]}||_{\infty }\\
=&||{\bf 1}_{[a,x]}-f_{n(x)}+f_{n(y)}-{\bf 1}_{[a,y]}||_{\infty }\\
\leq &||{\bf 1}_{[a,x]}-f_{n(x)}||_{\infty }+||f_{n(y)}-{\bf 1}_{[a,y]}||_{\infty }\\
<& \dfrac{1}{2}+\dfrac{1}{2}<1
\end{align*}
which is a contradiction. We thus found that all dense subsets of $L^{\infty }(E)$ are uncountable. 
\end{proof}
\vspace{2ex}
\begin{defi}
We define $C^{\infty }_{c}(E)$ as the space of smooth functions with a compact support inside $E$. By support we mean the closure of the set where $f(x)\ne 0$.
\[\mathop{\mathrm{supp}}(f)=\overline{\{x\in E \,|\, f(x)\ne 0 \}}\]
We remark that $C^{\infty }_{c}(E)$ is dense in $L^{p}(E)$ for $1\leq p< \infty $.
\end{defi}
\vspace{2ex}

