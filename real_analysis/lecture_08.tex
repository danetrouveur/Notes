\section{Lecture 8 (March 27th)}
\begin{recall}
A function $f:E\rightarrow {\bm R}$ (or $[-\infty ,\infty ]$) is called measurable if $E\in \mathfrak{M}$ and $\{x\in E \;|\; f(x)>c\}\in \mathfrak{M}$ for all $c\in {\bm R}$. This is equivalent to saying that the following sets are in $\mathfrak{M}$.
\[\bigcap ^{\infty }_{n=1}\{x\in E \;|\; f(x)>c-\dfrac{1}{n}\}\hspace{5ex}\{x\in E \;|\; f(x)<c\}\]
\end{recall}
\vspace{2ex}
\begin{thm}
If $f,g:E\rightarrow {\bm R}$ are measurable, then $f+g$ are measurable.
\end{thm}
\vspace{2ex}
\begin{proof}
Consider the following representation
\[\{x\in E \;|\; f(x)+g(x)<c\}=\bigcup _{q\in {\bm Q}}\Big[\{x\in E \;|\; f(x)<q\}\cap \{x\in E \;|\; g(x)<c-q\}\Big]\]
If $x_0$ is in the first set, then $f(x_0)+g(x_0)<c$, i.e., $f(x_0)<c-g(x_0)$ thus there is $q_0\in {\bm Q}$ such that $f(x_0)<q_0<c-g(x_0)$, that is, $f(x_0)<q_0$ and $g(x_0)<c-q_0$.
\end{proof}
\vspace{2ex}
\begin{rmk}
Let $E\in \mathfrak{M}$. If $\phi :E\rightarrow {\bm R}$ has a finite range ($\phi (E)=\{a_1,a_2,\ldots ,a_{n}\}$) and is measurable, then $\phi $ is called a simple function. The function
\[\phi (x)=\sum ^{n}_{k=1}a_{k}{\bf 1} _{E_{k}}\]
where $E_{k}=\{x\in E \;|\; \phi (x)=a_{k}\}$ ($E_{k}\cap E_{j}=\emptyset$ and $\bigcup ^{n}_{k=1}E_{k}=E$) is called a canonical (or standard) representation of $\phi $. 
\end{rmk}
\vspace{2ex}
\begin{thm}
(Simple approximation theorem) The simple approximation theorem states that a measurable function can be approximated to a simple function. In detail, if $f:E\rightarrow [0,\infty ]$ is measurable, then there is a sequence of non-negative measurable simple functions $\{\psi _{n}\}$ such that $\psi _{n}\rightarrow f$ increasingly and pointwise on $E$. If $f$ is bounded, the convergence can be uniform.
\end{thm}
\vspace{2ex}
\begin{proof}
We first show that there exists a pointwise increasingly converging sequence of simple functions for any measurable function. Consider for each $n$, 
\[A_{n}=\{x\in E \;|\; f(x)\geq n\}\] 
and
\[B_{n,k}=\{x\in E \;|\;\dfrac{k-1}{n}\leq f(x)<\dfrac{k}{n} \}\]
for $k=1,2,\ldots ,n^2$. Note that for each $n$, $B_{n,k}\cap B_{n,j}=\emptyset$. For each $n$,
\[E=A_{n}\cup \Big[\bigcup ^{n^2}_{k=1}B_{n,k}\Big]\]
Now define
\[\phi _{n}(x)=n{\bf 1} _{A_{n}}+\sum ^{n^2}_{k=1}\dfrac{k-1}{n}{\bf 1} _{B_{n,k}}(x)\]
then $0\leq \phi _{n}(x)\leq f(x)$. Note that
\begin{itemize}
	\item[(i)] $\phi _{n}\rightarrow f$ pointwise on $E$
	\item[(ii)] if $f$ is bounded, then there is $M$ such that $0\leq f(x)\leq M$
\end{itemize}
For each $\varepsilon >0$, we can take $N$ such that $f(x)\leq N$ and $1/N<\varepsilon $. If $n\geq N$ then
\[|\phi _{n}(x)-f(x)|<\dfrac{1}{n}<\dfrac{1}{N}<\varepsilon \]
for all $x\in E$. Thus if $f$ is bounded then $\phi _{n}\rightarrow f$ uniformly.	
\\
We can create multiple such functions. Now consider the set
\[C_{n}=\{x\in E \;|\; f(x)\geq n\}\]
and
\[D_{n,k}=\{x\in E \;|\; \dfrac{k-1}{2^{n}}\leq f(x)<\dfrac{k}{2^{n}}\}\]
with $D_{n,k}\cap D_{n,j}=\emptyset$. For each $n$, notice that again,
\[C_{n}\cup \Big(\bigcup _{k=1}^{n 2^{n}}D_{n,k}\Big)=E\]
Define, this time, 
\[\psi _{n}(x)=n{\bf 1} _{C_{n}}(x)+\sum ^{n2^{n}}_{k=1}\dfrac{k-1}{2^{n}}{\bf 1} _{D_{n,k}}(x)\]
then $0\leq \psi _{n}\leq \psi _{n+1}\leq f$ so that $\lim _{n\rightarrow \infty }\psi _{n}(x)\rightarrow f(x)$ increasingly for each $x\in E$. If $f$ is a bounded non-negative function on $E$ then $\psi _{n}\rightarrow f$ increasingly and uniformly.
\end{proof}
\vspace{2ex}

