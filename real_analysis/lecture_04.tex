\section{Lecture 4 (March 13th)}
\begin{recall}
Last class, we established two lemmas. First, we've shown that $\mathfrak{M}$ is closed under finite unions. Second, if $E_1,\ldots ,E_{n}\in \mathfrak{M}$ are mutually disjoint, then for every $A\subset {\bm R}$ we had
\[m^{*}\Big(A\cap [\bigcup ^{n}_{k=1}E_{k}]\Big)=\sum ^{n}_{k=1}m^{*}(A\cap E_{k})\]
and $m^{*}$ satisfies {\it finite additivity}. We'll now show that $m^{*}$ satisfies {\it countable additivity}.
\end{recall}
\vspace{2ex}
\begin{rmk}
If $\{E_{n}\}^{\infty }_{n=1}$ is a sequence in $\mathfrak{M}$ then define $F_{1}=E_1$, $F_2=E_2\;\backslash\;A_1$, $F_3=E_3\;\backslash\; \bigcup _{k=1}^{2}E_{k}$ and
\[F_{n}=E_{n}\;\backslash\;\bigcup ^{n-1}_{k=1}E_{k}\]
Then, $F_{n}\in \mathfrak{M}$ for each $n$, $F_{n}\subset E_{n}$, and $\{F_{n}\}$ are mutually disjoint. 
\end{rmk}
\vspace{2ex}
\begin{thm}
$\mathfrak{M}$ is a $\sigma $-algebra on ${\bm R}$ and satisfies countable additivity.
\end{thm}
\vspace{2ex}
\begin{proof}
Let $\{E_{n}\}_{n=1}^{\infty }$ be any sequence in $\mathfrak{M}$. We have to show that $\bigcup ^{\infty }_{n=1}E_{n}\in \mathfrak{M}$. As the two are equivalent by the preceding remark, we'll show that $\bigcup ^{\infty }_{n=1}E_{n}\in \mathfrak{M}$ for a mutually disjoint sequence $\{E_{n}\}$ in $\mathfrak{M}$. Define 
\[F_{n}=\bigcup ^{n}_{k=1}E_{k}\in \mathfrak{M}\]
Let $E=\bigcup ^{\infty }_{n=1}E_{n}$ and note that $F_{n}\subset E$ implies $F_{n}^{c}\supset E^{c}$. For every $A\subset {\bm R}$ we have
\begin{align*}
m^{*}(A)=m^{*}(A\cap F_{n})+m^{*}(A\cap F_{n}^{c})\geq & m^{*}(A\cap F_{n})+m^{*}(A\cap E^{c})\\
=&\sum ^{n}_{k=1}m^{*}(A\cap E_{k})+m^{*}(A\cap E^{c})
\end{align*}
due to the theorem from last class. We now showed how for every $n\in {\bm R}$, $A\subset {\bm R}$,
\[m^{*}(A)\geq \sum ^{n}_{k=1}m^{*}(A\cap E_{k})+m^{*}(A\cap E^{c})\]
now take $n\rightarrow \infty $ we get
\begin{align*}
	m^{*}(A)\geq& \sum ^{\infty }_{n=1}m^{*}(A\cap E_{k})+m^{*}(A\cap E^{c})\\
	\geq&m^{*}(A\cap E)+m^{*}(A\cap E^{c})
\end{align*}
by subadditivity. Therefore,
\[E=\bigcup ^{\infty }_{n=1}E_{n}\]
To prove countable additivity, let $\{A_{n}\}$ be a mutually disjoint sequence in $\mathfrak{M}$. We showed that
\[m^{*}\Big(\bigcup_{k=1}^{n}E_{k}\Big)=\sum ^{n}_{k=1}m^{*}(E_{k})\]
Since
\[\bigcup ^{n}_{k=1}E_{k}\subset \bigcup ^{\infty }_{n=1}E_{k}\in \mathfrak{M}\]
we have
\[m^{*}\Big(\bigcup _{k=1}^{\infty }E_{k}\Big)\geq m^{*}\Big(\bigcup ^{n}_{k=1}E_{k}\Big)=\sum ^{n}_{k=1}m^{*}(E_{k})\]
again taking $n\rightarrow \infty $, we have 
\[m^{*}\Big(\bigcup ^{\infty }_{k=1}E_{k}\Big)\geq \sum ^{\infty }_{k=1}m^{*}(E_{k})\]
\end{proof}
\vspace{2ex}
So far, we have shown that 
\begin{itemize}
	\item[(i)] If $m^{*}(A)=0$ then $A\in \mathfrak{M}$
	\item[(ii)] $\mathfrak{M}$ is a $\sigma $-algebra on ${\bm R}$
	\item[(iii)] $m^{*}$ satisfies countable additivity on $\mathfrak{M}$
\end{itemize}
We now show that all intervals are measurable 
\vspace{4ex}
\begin{thm}
For every $a\in {\bm R}$, $(a,\infty )\in\mathfrak{M}$.
\end{thm}
\vspace{2ex}
\begin{proof}
We have to show that for every $A\subset {\bm R}$ that 
\begin{align*}
	m^{*}(A)\geq& m^{*}(A\cap (a,\infty ))+m^{*}(A\cap (-\infty ,a])\\
	=&m^{*}(A\cap (a,\infty ))+m^{*}(A\cap (-\infty ,a)) 
\end{align*}
We denote $A_1=A\cap (a,\infty )$ and $A_2=A\cap (-\infty ,a)$. Let $\{I_{n}\}^{\infty }_{n=1}$ be any sequence of open intervals satisfying $A\subset \bigcup ^{\infty }_{n=1}I_{n}$. We have to show that 
\[m^{*}(A_1)+m^{*}(A_2)\leq \sum ^{\infty }_{n=1}l(I_{n})\]
For each $n\in {\bm N}$, let $I_{n}'=(a,\infty )\cap I_{n}$ and $I_{n}''=(-\infty ,a)\cap I_{n}$. Then,
\[A_1\subset \bigcup ^{\infty }_{n=1}I'_{n}\hspace{5ex}A_{2}\subset \bigcup ^{\infty }_{n=1}I_{n}''\hspace{5ex}l(I_{n})=l(I'_{n})+l(I_{n}'')\]
So that  $m^{*}(A_1)\leq \sum ^{\infty }_{n=1}l(I_{n}')$ and $m^{*}(A_2)\leq \sum ^{\infty }_{n=1}l(I''_{n})$. Thus
\[m^{*}(A_1)+m^{*}(A_2)\leq \sum ^{\infty }_{n=1}l(I_{n}')+\sum ^{\infty }_{n=1}l(I''_{n})=\sum ^{\infty }_{n=1}(l(I_{n}')+l(I_{n}''))=\sum _{n=1}^{\infty }l(I_{n})\]
\end{proof}
\vspace{2ex}
\begin{thm}
If $B$ is a Borel set, then $B\in \mathfrak{M}$. 
\end{thm}
\vspace{2ex}
\begin{rmk}
If $E\in \mathfrak{M}$, then we say $E$ is measurable. 
\end{rmk}
\vspace{2ex}
\begin{thm}
The following are equivalent
\begin{itemize}
	\item[(i)] $E$ is measurable
	\item[(ii)] For every $\varepsilon >0$, there is an open set $\mathcal{O}\supset E$ such that $m^{*}(\mathcal{O}\;\backslash\;E)<\varepsilon $
	\item[(iii)] There is a $G_{\delta }$ set $G\supset E$ such that $m^{*}(G\;\backslash\; E)=0$
\end{itemize}
\end{thm}
\vspace{2ex}
\begin{rmk}
Last class, we have shown how for every $A\subset {\bm R}$, there is a $G_{\delta }$ set $G\supset A$ such that $m^{*}(A)=m^{*}(G)$. Recall that if $A\in \mathfrak{M}$ with $m^{*}(A)<\infty $ and $A\subset B$, then 
\begin{align*}
	m^{*}(B)=&m^{*}(B\cap A)+m^{*}(B\cap A^{c})\\
	=&m^{*}(A)+m^{*}(B\;\backslash\; A)\\
 m^{*}(B\;\backslash\; A)=&m^{*}(B)-m^{*}(A)
\end{align*}
\end{rmk}
\vspace{2ex}
\begin{proof}
We prove their equivalence in a cyclic chain of implications. We start with (i) to (ii). There are two cases, depending on the measure of $E$.
\begin{itemize}
\item[(1)] Let $m^{*}(E)<\infty $. For every $\varepsilon >0$ there is a sequence of open intervals $\{I_{n}\}$ such that $E\subset \bigcup ^{\infty }_{n=1}I_{n}$ and $\sum ^{\infty }_{n=1}l(I_{n})<m^{*}(E)+\varepsilon $. If we define $\mathcal{O}=\bigcup ^{\infty }_{n=1}I_{n}$ then $\mathcal{O}\in \mathfrak{M}$, $E\subset \mathcal{O}$ and $m^{*}(\mathcal{O})<m^{*}(E)+\varepsilon $. Since $E\in \mathfrak{M}$ and $m^{*}(E)<\infty $ we also have $m^{*}(\mathcal{O}\backslash E)=m^{*}(\mathcal{O})-m^{*}(E)$ which completes the proof.
\item[(2)] Consider $m^{*}(E)=\infty $. Then, we can express $E$ as $E=\bigcup ^{\infty }_{n=1}E_{n}$ where $E_{n}$ are mutually disjoint with $m^{*}(E_{n})<\infty $. Then for every $n\in {\bm N}$, there is an open set $\mathcal{O}_{n}\supset E_{n}$ such that $m^{*}(\mathcal{O}_{n}\;\backslash\; E_{n})<\varepsilon /2^{n}$. Define $\mathcal{O}=\bigcup ^{\infty }_{n=1}\mathcal{O}_{n}$ then $m^{*}(\mathcal{O}\;\backslash\; E)<\varepsilon $.
\end{itemize}
We now try (ii) to (iii). Continuing from the previous remark, for every $n\in {\bm N}$, there is $\mathcal{O}_{n}\supset E_{n}$ such that $m^{*}(\mathcal{O}_{n}\;\backslash\; E)<1/k$. Define $G=\bigcap ^{\infty }_{n=1}\mathcal{O}_{n}$ then $G\supset E$ and $G\;\backslash\; E\subset \mathcal{O}_{n}\;\backslash\; E$.
	\[m^{*}(G\;\backslash\;E)\leq m^{*}(\mathcal{O}_{k}\;\backslash\;E)<1/k\]

What about (iii) to (i)? From set theory, we know $E=G\;\backslash\; (G\;\backslash\;E)$ which concludes the proof.
\end{proof}
\vspace{2ex}


