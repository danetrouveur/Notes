\section{Lecture 22 (June 5th)}
\begin{thm}
(Parallelogram law) If $x,y\in H$, then
\[||x+y||^2+||x-y||^2=2(||x||^2+||y||^2)\]
\end{thm}
\vspace{2ex}
\begin{prop}
If $M$ is a subspace of $H$, then $\bar{M}$ is a closed subspace of $H$. 
\end{prop}
\vspace{2ex}
\begin{defi}
(Convex subset) A convex subset of $H$ is a subset where if $x,y\in E$ and $\lambda \in (0,1)$, we have $\lambda x+(1-\lambda )y\in E$. 
\end{defi}
\vspace{2ex}
\begin{prop}
If $E$ is a convex subset of $H$, then $x+E=\{x+y \,|\,y\in E \}$ is also convex. 
\end{prop}
\vspace{2ex}
\begin{defi}
($x^{\perp}$) For $x\in H$, we define $x^{\perp}=\{y\in H \,|\,\langle x,y\rangle =0 \}$. This is a subspace of $H$. If we define $T:H\rightarrow {\bm R}$ by $T(y)=\langle x,y\rangle $, by the Cauchy Schwarz inequality, $|T(y)|\leq ||x||\,||y||$. That is, $T$ is continuous and $x^{\perp}=\{y\in H \,|\,T(y)=0 \}$ is a closed subspace of $H$. 
\end{defi}
\vspace{2ex}
\begin{defi}
($M^{\perp}$) If $M$ is a subspace of $H$, we define 
\[M^{\perp}=\{y\in H \,|\, \langle x,y\rangle =0\}=\bigcap _{x\in M}x^{\perp}\]
which implies that $M^{\perp}$ is a closed subspace of $H$.
\end{defi}
\vspace{2ex}
\begin{thm}
If $E$ is a non-empty closed convex subset of a Hilbert space $H$, then $E$ has a unique element of smallest norm.
\end{thm}
\vspace{2ex}
\begin{proof}
Let $\delta =\inf\{||x|| \,|\, x\in E\}$. We'll show that there is a unique $x_0\in E$ with $||x_0||=\delta $. Let $x,y\in E$. Then, $x/2,y/2\in H$. Apply the parallelogram law and we have
\[\Big|\Big|\dfrac{x}{2}+\dfrac{y}{2}\Big|\Big|^2+\Big|\Big|\dfrac{x}{2}-\dfrac{y}{2}\Big|\Big|^2=2\Big(\Big|\Big|\dfrac{x}{2}\Big|\Big|^2+\Big|\Big|\dfrac{y}{2}\Big|\Big|^2\Big)\]
so that 
\[||x-y||^2=2||x||^2+2||y||^2-4\Big|\Big|\dfrac{x}{2}+\dfrac{y}{2}\Big|\Big|^2\]
for $x,y\in E$. By the convexity of $E$, the last term is in $E$. We then know that
\[||x-y||^2\leq 2(||x||^2+||y||^2)-4\delta ^2\]
If $||x||=||y||=\delta $, then $||x-y||^2\leq 0$ such that $x=y$. This tells us that $x_0$ is unique if it exists.
\newline
\newline
By definition of $\delta $, there is a sequence $y_{n}\in E$ such that $\lim _{n\rightarrow \infty }||y_{n}||=\delta $. Put $y_{n},y_{m}$ in the inequality above and we have
\[||y_{n}-y_{m}||^2\leq 2(||y_{n}||^2+||y_{m}||^2)-4\delta ^2\]
so that $\{y_{n}\}$ is a Cauchy sequence in $E\subset H$. Since $H$ is complete, there is $x_0\in H$ such that $\lim _{n\rightarrow \infty }||y_{n}-x_0||=0$. Since $E$ is closed, $x_0\in E$. Finally,
\[\lim _{n\rightarrow \infty }||x_0||=\lim _{n\rightarrow \infty }||y_{n}||=\delta \]
since $||\cdot ||$ is continuous.
\end{proof}
\vspace{2ex}
\begin{ex}
In $L^{1}([0,1])$, define
\[E=\Big\{f\in L^{1}([0,1]) \,\Big|\,\int ^{1}_{0}f(x)\,dx=1 \Big\}\]
Notice that the set is convex as the integral of $g=\lambda f+(1-\lambda )h$ is $1$. In addition to this, $E$ is closed. To see this, we ask whether for $f_{n}\in E$ and $\lim _{n\rightarrow \infty }||f_{n}-f||_{1}=0$, $\int ^{1}_{0}f\,dx=1$. this is true, as 
\[\Big|\int ^{1}_{0}f_{n}\,dx-\int ^{1}_{0}f\,dx\Big|\leq \int ^{1}_{0}|f_{n}-f|\,dx\rightarrow 0\]
Together, we now know that the set is a closed convex subset of $L^{1}([0,1])$ with infinitely many elements with the smallest norm. 
\end{ex} 
\vspace{2ex}
\begin{ex}
Consider $f_{n}\in C([0,1])$ with the uniform norm. Define
\[E=\Big\{f\in C([0,1]) \,\Big|\,\int ^{1/2}_{0}f(x)\,dx-\int ^{1}_{1/2}f(x)\,dx=1 \Big\}\]
Obviously, this is a convex set. By LDCT, $E$ is closed. The infinum of the norm is $1$, while there is no single function that has this norm. 
\end{ex}
\vspace{2ex}
\begin{thm}
(Orthogonal decomposition) (Big theorem) If $M$ is a closed subspace of a Hilbert space $H$ then every $x\in H$ can be uniquely expressed as $x=P_{x}+Q_{x}$ where $P_{x}\in M$ and $Q_{x}\in M^{\perp}$. Indeed, $P\in \mathcal{L}(H,M)$ and $Q\in \mathcal{L}(H,M^{\perp})$ are norm $1$ linear operators with
\[||x||^2=||P_{x}||^2+||Q_{x}||^2\]
The key idea behind this theorem is that $x+M$ is a closed convex subset of $H$. 
\end{thm}
\vspace{2ex}
\begin{proof}
$x+M=\{x+y \,|\, y\in M\}$ is a non-empty closed convex subset of $H$. Define $Q_{x}$ as the unique element of $x+M$ with the smallest norm. Then define $P_{x}=x-Q_{x}$, and by definition, $P_{x}\in M$. 
\newline
\newline
We have to show that $Q_{x}\in M^{\perp}$ and that the decomposition is unique. The latter part is simple, as if we take $x=p+q=p'+q'$, $x-x'=y'-y\in M\cap M^{\perp}=\{0\}$. To show that $Q_{x}\in M^{\perp}$, we need to show that $\langle Q_{x},y\rangle =0$ for all $y\in M$ or that $\langle Q_{x},y\rangle =0$ for all $y\in M$ with $||y||=1$. 
\newline
\newline
Take any $y\in M$ with $||y||=1$. Then for every $\alpha \in {\bm R}$, $Q_{x}-\alpha y\in x+M$ so that
\[||Q_{x}||^2\leq ||Q_{x}-\alpha y||^2=||Q_{x}||^2-2\alpha \langle Q_{x},y\rangle +|\alpha |^2\] 
for all $\alpha \in {\bm R}$. Put $\alpha =\langle Q_{x},y\rangle $ to get
\[||Q_{x}||^2\leq ||Q_{x}||^2-\langle Q_{x},y\rangle ^2\]
or that $\langle Q_{x},y\rangle ^2\leq 0$ and $\langle Q_{x},y\rangle =0$.
\end{proof}
\vspace{2ex}
\begin{cor}
If $M$ is a closed subspace of $H$ with $M\ne H$, then there is $x\in M^{\perp}$ with $||x||=1$.
\end{cor}
\vspace{2ex}
\begin{thm}
(Riesz-representation theorem) If $T$ is a bouned linear functional on a Hilbert space $H$, then there is a unique element $y\in H$ such that $T(x)=\langle x,y\rangle $ for all $x\in H$. 
\end{thm}
\vspace{2ex}
\begin{proof}
If $T(x)=0$ for all $x\in H$ then put $y=0$. If $T(x)\ne 0$ for some $x\in H$,
\[M=\{x\in H \,|\, T(x)=0\}\]
is a closed subspace of $H$ so that there is $z\in M^{\perp}$ with $||z||=1$. Put $y=T(z)z$ and the proof is over. Take any $x\in H$ and put $u=T(x)z-T(z)x$ and $T(u)=0$ so that $u\in M$. Hence $\langle u,y\rangle =0$ as 
\[\langle T(x)z-T(z)x,T(z)z\rangle= T(x)-\langle x,T(z)z\rangle \]
\end{proof}
\vspace{2ex}

