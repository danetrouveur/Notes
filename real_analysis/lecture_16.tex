\section{Lecture 16 (May 13th)}
\begin{defi}
If $X$ and $Y$ are normed vector spaces, we define $\mathcal{L}(X,Y)$ as the set of all bounded linear operators from $X$ to $Y$. We call $X^{*}=\mathcal{L}(X,{\bm R})$ the dual of $X$. The elements of $X^{*}=\mathcal{L}(X,{\bm R})$ are called bounded linear functionals. 
\end{defi}
\vspace{2ex}
\begin{thm}
For $T\in \mathcal{L}(X,Y)$, let norm of $T$ be defined as $||T||=\sup_{||x||=1}||T(x)|| $. If $Y$ is a Banach space, then $\mathcal{L}(X,Y)$ is a Banach space. In particular, $X^{*}=\mathcal{L}(X,{\bm R})$ is a Banach space.
\end{thm}
\vspace{2ex}
\begin{rmk}
For $x\ne 0$, observe that 
\[\Big|\Big|T\Big(\dfrac{x}{||x||}\Big)\Big|\Big|\leq ||T||\implies ||T(x)||\leq ||T||\, ||x||\]
As $||T(x)||\leq M\,||x||$ we also have that 
\[\Big|\Big|T\Big(\dfrac{x}{||x||}\Big)\Big|\Big|\leq M\implies ||T||\leq M\]
This tells us the following equivalence of defintions, 
\[||T||=\inf(\{M\geq 0 \,|\,||T(x)||\leq M\,||x|| \}) \]
for every $x\in X$. This tells us that the norm measures how much a vector can be scaled and stretched. 
\end{rmk}
\vspace{2ex}
\begin{thm}
The proof that $\mathcal{L}(X,Y)$ is a normed vector space with the aforementioned norm and will be for now will be skipped. If $Y$ is a Banach space, we'll show that $\mathcal{L}(X,Y)$ is a Banach space. Given a Cauchy sequence $\{T_{n}\}$, we have to show that there is $T\in \mathcal{L}(X,Y)$ such that $\lim _{n\rightarrow \infty }||T_{n}-T||=0$. 
\end{thm}
\vspace{2ex}
\begin{proof}
We define the target linear operator pointwise. For each $x\in X$, 
\[||T_{n}(x)-T_{m}(x)||=||(T_{m}-T_{n})(x)||\leq ||T_{n}-T_{m}||\, ||x||\] 
and we see that $\{T_{n}(x)\}$ is a Cauchy sequence in $Y$. Since $Y$ is complete, there is $y\in Y$ such that $\lim _{n\rightarrow \infty }T_{n}(x)=y$. Define $y=T(x)$, that is, $T(x)=\lim _{n\rightarrow \infty }T_{n}(x)$ in $Y$. Then 
\begin{align*}
T(\alpha x)=&\lim _{n\rightarrow \infty }T_{n}(\alpha x)=\alpha\lim _{n\rightarrow \infty } T_{n}(x)=\alpha T(x)\\
T(x+y)=&\lim _{n\rightarrow \infty }T_{n}(x+y)=T(x)+T(y)
\end{align*}
and therefore $T$ is linear. To show $T\in \mathcal{L}(X,Y)$ we need to still prove that $T$ is bounded. Take any $\varepsilon >0$, there is $N\in {\bm N}$ such that if $n,m\geq N$ then $||T_{n}-T_{m}||<\varepsilon $. This implies that, for such $N$, if $n\geq N$ then $||T_{n}||\leq ||T_{N}||+\varepsilon $. Thus for each $x\in X$,
\[||T(x)||=\lim _{n\rightarrow \infty }||T_{n}(x)||\leq \lim _{n\rightarrow \infty }\Big[||T_{n}||\,||x||\Big]\leq \Big[||T_{N}||+\varepsilon \Big]||x||\]
Therefore, the norm of the target is bounded with
\[||T||\leq ||T_{N}||+\varepsilon \]
Lastly, we prove that $T$ is the limit of the Cauchy sequence. Take any $\varepsilon >0$. Then, there exists an $N$ such that if $n>N$, 
\[||T_{n}(x)-T(x)||=\lim _{m\rightarrow \infty }||T_{n}(x)-T_{m}(x)||\leq \lim _{m\rightarrow \infty }||T_{n}-T_{m}||\,||x||\leq \varepsilon ||x||\]
and $||T_{n}-T||\leq \varepsilon $. In sum, we have followed process of, for an arbitrary Cauchy sequence, finding a target, proving that the target is linear and bounded, and lastly proving that the target is indeed the limit of the Cauchy sequence. 
\end{proof}
\vspace{2ex}
\begin{prop}
$\mathrm{Lip}_{\alpha }([0,1])$ is a Banach space for $0<\alpha <1$ where $\mathrm{Lip}_{\alpha }([0,1])$ is the set of all $f\in C([0,1])$ such that
\[M_{f}=\sup_{s\ne t}\dfrac{|f(s)-f(t)|}{|s-t|^{\alpha }}<\infty \]
with the norm $||f||=|f(0)|+M_{f}$. 
\end{prop}
\vspace{2ex}
\begin{cor}
If $X$ is a normed vector space, then
\[X^{*}=\mathcal{L}(X,{\bm R})\]
is a Banach space with the operator norm. 
\end{cor}
\vspace{2ex}
\begin{cor}
Let $L^{p}(E)$ denote the set of all measurable functions on $E$ with 
\[||f||_{p}=\Big(\int _{E}|f|^{p}\,dm\Big)^{1/p}<\infty \]
were we define $f,g\in L^{p}(E)$ to be equivalent if $f=g$ almost everywhere on $E$. 
\end{cor}
\vspace{2ex}
\begin{proof}
To show that $L^{p}(E)$ is a normed vector space for $1<p<\infty $ we simply need to prove that the function $||\cdot ||_{p}$ is a norm. Note how
\begin{itemize}
\item[(i)] If $||f||_{p}=0$ then $f=0$ almost everywhere on $E$ so that $f$ is a zero vector in $L^{p}(E)$.
\item[(ii)] If $f\in L^{p}(E)$ and $\alpha \in {\bm R}$, then 
\[||\alpha f||_{p}=\Big(\int _{E}|\alpha f|^{p}\,dm\Big)^{1/p}=|\alpha |\Big(\int _{E}|f|^{p}\,dm\Big)^{1/p}=|\alpha |\,||f||_{p}\]
\item[(iii)] If $f,g\in L^{p}(E)$ then $||f+g||_{p}\leq ||f||_{p}+||g||_{p}$ which is called the Minkowski inequality. Due to its difficulty, we omit the proof for now.
\end{itemize}
\end{proof}
\vspace{2ex}
\begin{lem}
For $a,b\geq 0$ and $0<\lambda <1$ the equality $a^{\lambda }b^{1-\lambda }\leq \lambda a+(1-\lambda )b$ holds.
\end{lem}
\vspace{2ex}
\begin{proof}
For $t\geq 0$, define $\phi (t)=\lambda t-t^{\lambda }$. The derivative is given as $\phi '(t)=\lambda -\lambda t^{\lambda -1}=\lambda (1-t^{\lambda -1})$. With a minimum at $t=1$ we know that $\phi (t)\geq \phi (1)$ for all $t>0$. Therefore,
\[\lambda t-t^{\lambda }\geq \lambda -1\quad\mathrm{so\ that}\quad t^{\lambda }\leq \lambda t+(1-\lambda )\]
for all $t\geq 0$. Notice that the equality holds when $t=1$. Now put $t=a/b$ to get 
\[\Big(\dfrac{a}{b}\Big)^{\lambda  }\leq \lambda\Big(\dfrac{a}{b}\Big)+(1-\lambda )\]
so that multiplying $b$ on both sides yields the sought inequality.
\end{proof}
\vspace{2ex}
\begin{cor}
Let $1<p<\infty $ satisfy $1/p+1/q=1$. Put $\lambda =1/p$, $a=A^{p}$, and $b=B^{p}$ for $A,B\geq 0$. Then, 
\[a^{\lambda }b^{1-\lambda }\leq \lambda a+(1-\lambda )b\]
becomes
\[AB\leq \dfrac{A^{p}}{p}+\dfrac{B^{q}}{q}\]
where the equality holds when $A^{p}=B^{q}$.
\end{cor}
\vspace{2ex}
\begin{thm}
(H\"{o}lder's inequality) Let $1\leq p<\infty $. If $f\in L^{p}(E)$ and $g\in L^{q}(E)$ with $1/p+1/q=1$, then $fg\in L^{1}(E)$ and $||fg||_{1}\leq ||f||_{p}||g||_{q}$. That is,
\[\int _{E}|fg|\,d m\leq \Big(\int _{E}|f|^{p}\,d m\Big)^{1/p}\Big(\int _{E}|g|^{q}\,d m\Big)^{1/q}\]
\end{thm}
\vspace{2ex}
\begin{proof}
From $AB\leq A^{p}/p+B^{q}/q$ we put $A=|f(x)|/||f||_{p}$ and $B=|g(x)|/||g||_{q}$ for $x\in E$ to get
\[\dfrac{|f(x)g(x)|}{||f||_{p}||g||_{q}}\leq \dfrac{1}{p}\dfrac{|f(x)|^{p}}{||f||^{p}_{p}}+\dfrac{1}{q}\dfrac{|g(x)|^{q}}{||g||^{q}_{q}}\]
then we integrate.
\begin{align*}
\dfrac{1}{||f||_{p}||g||_{q}}\int _{E}|f(x)g(x)|\,d m(x)\leq& \dfrac{1}{p}\dfrac{1}{||f||^{p}_{p}}\int _{E}|f(x)|^{p}\,d m(x)+\dfrac{1}{q}\dfrac{1}{||g||^{q}_{q}}\int _{E}|g(x)|^{q}\,d m(x)\\=&\dfrac{1}{p}+\dfrac{1}{q}=1
\end{align*}
This completes the proof.
\end{proof}
\vspace{2ex}
\begin{thm}
(Minkowski inequality) Let $1\leq p<\infty $. If $f,g\in L^{p}(E)$ then $||f+g||_{p}\leq ||f||_{p}+||g||_{p}$. 
\end{thm}
\vspace{2ex}
\begin{proof}
If $f,g\in L^{p}(E)$ then $|f(x)+g(x)|\leq 2 \max(|f(x)|,|g(x)|)$ which implies that
\[\int _{E}|f+g|^{p}\,d m\leq 2^{p}\int _{E}(|f|^{p}+|g|^{p})\,d m<\infty \]
Then
\begin{align*}
\int _{E}|f+g|^{p}\,d m=&\int _{E}|f+g|\,|f+g|^{p-1}\,d m\leq\int _{E}(|f|+|g|)|f+g|^{p-1}\,d m\\
=&\int _{E}|f|\,|f+g|^{p-1}\,d m+\int _{E}|g|\,|f+g|^{p-1}\,d m
\end{align*}
where, from H\"{o}lder's inequality,
\begin{align*}
\int _{E}|f|\,|f+g|^{p-1}\,d m\leq ||f||_{p}\Big[\int _{E}|f+g|^{(p-1)q}\,d m\Big]^{1/q}=&||f||_{p}\Big[\int _{E}|f+g|^{p}\,d m\Big]^{1/q}\\=&||f||_{p}||f+g||_{p}^{p/q}
\end{align*}
For the justification of the use of the inequality, we must show that each is integrand are elements of $L^{p}$ and $L^{q}$. Note that
\[\int _{E}\Big[|f+g|^{p-1}\Big]^{q}\,d m=\int _{E}|f+g|^{pq-q}\,d m=\int _{E}|f+g|^{p}\,d m<\infty \]
We now see that, in parallel,
\[\int _{E}|g|\,|f+g|^{p-1}\,d m\leq ||g||_{p}||f+g||^{p/q}_{p}\]
and therefore
\[||f+g||^{p}_{p}\leq \Big[||f||_{p}+||g||_{p}\Big]\,||f+g||^{p/q}_{p}\]
Dividing both sides by $||f+g||^{p/q}_{p}$,
\[||f+g||_{q}^{p-p/q}\leq ||f||_{p}+||g||_{p}\quad\mathrm{and}\quad  ||f+g||_{p}\leq ||f||_{p}+||g||_{p}\]
now we seen how for $p\geq 1$, $L^{p}(E)$ is a normed vector space. Next class, we show that, in addition to this, $L^{p}(E)$ is complete.
\end{proof}
\vspace{2ex}


