\section{Lecture 3 (March 11th)}
\begin{recall}
	Last class we have introduced the Lebesgue outer measure $m^{*}:\mathcal{P}({\bm R})\rightarrow [0,\infty ]$  which is a function which, when $\bigcup I_{n}\supset  E$, satisfies
\[m^{*}(E)=\inf \Big\{\sum ^{\infty }_{n=1}l(I_{n})\Big\}\] \end{recall}
\vspace{2ex}
\begin{rmk}
The outer measure satisfies the following properties.
\begin{itemize}
	\item[(i)] $m^{*}(I)=l(I)$ where $I$ is an interval
	\item[(ii)] $m^{*}(E+x)=m^{*}(E)$
	\item[(iii)] (subadditivity) For $E_{n}\subset {\bm R}$, disjoint or not,

		\[m^{*}\Big(\bigcup^{\infty }_{n=1}E_{n}\Big)\leq\sum ^{\infty }_{n=1}m^{*}(E_{n})\]
\end{itemize}
\end{rmk}
\vspace{2ex}
\begin{proof}
For subadditivity, we suppose the contrary and that there exists a set of subsets of ${\bm R}$ $\{E_{n}\}_{n=1}^{\infty }$ such that 
\[m^{*}\Big(\bigcup_{n=1}^{\infty }E_{n}\Big)-\sum ^{\infty }_{n=1}m^{*}(E_{n})=L>0\]
On the other hand, due to the definition of infinum, for a single set $E_{n}$, we note that for every $n\in {\bm N}$ there exists $\{I_{n,k}\}_{k=1}^{\infty }$ where $I$ is an open interval satisfying $E_{n}\subset \bigcup^{\infty }_{k=1}I_{n,k}$ such that
\[\sum ^{\infty }_{k=1}l(I_{n,k})<m^{*}(E_{n})+\dfrac{L}{2^{n+1}}\]
Since $\bigcup^{\infty }_{n,k=1}I_{n}\subset \bigcup^{\infty }_{n=1}E_{n}$ we have
\[m^{*}\Big(\bigcup^{\infty }_{n=1}E_{n}\Big)\leq \sum ^{\infty }_{k,n=1}l(I_{n,k})=\sum ^{\infty }_{n=1}\Big[\sum ^{\infty }_{k=1}l(I_{n,k})\Big]\leq \sum _{n=1}^{\infty }\Big(m^{*}(E_{n})+\dfrac{L}{2^{n+1}}\Big)=\sum ^{\infty }_{n=1}m^{*}(E_{n})+\dfrac{L}{2}\]
which is a contradiction.
\end{proof}
\vspace{2ex}
\begin{rmk}
(Properties of outer measure) 
\begin{itemize}
	\item[(i)] $m^{*}(\emptyset)=0$
	\item[(ii)] $m^{*}(E)=0$ if $E$ is countable (last class)
	\item[(iii)] If $m^{*}(B)=0$ then $m^{*}(A)=m^{*}(A)+m^{*}(B)\geq m^{*}(A\cup B)$ for every $A\subset {\bm R}$
\end{itemize}
\end{rmk}
\vspace{2ex}
\begin{proof}
The last works since $m^{*}(A)\leq m^{*}(A\cup B)\leq m^{*}(A)+m^{*}(B)=m^{*}(A)$.
\end{proof}
\vspace{2ex}
\begin{thm}
	(Important)
\begin{itemize}

	\item[(i)] For every $A\subset {\bm R}$ and $\varepsilon >0$ there is an open set $G\subset {\bm R}$ such that $A\subset G$ and $m^{*}(G)\leq m^{*}(A)+\varepsilon $. 
	\item[(ii)] For every $A\subset {\bm R}$ there is a $G_{\delta }$ set (countable intersection of open sets) $D$ such that $A\subset D$ and $m^{*}(A)=m^{*}(D)$.
\end{itemize}
\end{thm}
\vspace{2ex}
\begin{proof}
Notice that $[0,1)=\bigcup_{n=1}^{\infty }E_{n}$ is a disjoint union with the same size.
\begin{itemize}
	\item[(i)] By definition, for every $\varepsilon >0$, there is a sequence of open intervals $\{I\}_{n=1}^{\infty }$ such that $A\subset \bigcup_{n=1}^{\infty }I_{n}$ and $\sum _{n=1}^{\infty }l(I_{n})\leq m^{*}(A)+\varepsilon $. Define $G=\bigcup^{\infty }_{n=1}I_{n}$ then $G\supset A$ is open and
		\[m^{*}(G)\leq \sum ^{\infty }_{n=1}l(I_{n})\leq m^{*}(A)+\varepsilon \]
	\item[(ii)] By (1), for every $n$ there is an open set $G_{n}\supset A$ such that $m^{*}(G_{n})\leq m^{*}(A)+1/n$. Define $D=\bigcap^{\infty }_{n=1}G_{n}$. Then, $D\supset A$ and $m^{*}(D)=m^{*}(A)$. 
\end{itemize}
\end{proof}
\vspace{2ex}
\begin{rmk}
We aim to do the following
\begin{itemize}
	\item[(i)] Define $\mathfrak{M}$ the set of all measurable subsets (countable additivity) of ${\bm R}$ 
	\item[(ii)] $\mathfrak{M}$ is a $\sigma $-algebra which contains every open set.
	\item[(iii)] $\mathfrak{M}$ is complete (if $m^{*}(E)=0$ then $E\in \mathfrak{M}$)
\end{itemize}
Note that $\mathfrak{M}\subset \mathcal{B} ({\bm R})$, but the two are almost equivalent.
\end{rmk}
\vspace{2ex}
\begin{defi}
(Carathéodory's criterion) For $\mathfrak{M}\subset \mathcal{P}({\bm R})$, we say $E\in \mathfrak{M}$ (or $E$ is Lebesgue measurable) provided that for every $A\subset {\bm R}$ we have
\[m^{*}(A)=m^{*}(A\cap E)+m^{*}(A\cap E^{c})\]
\end{defi}
\vspace{2ex}
\begin{rmk}
We now show that the strict inequality 
\[m^{*}(A\cup B)<m^{*}(A)+m^{*}(B)\]
for disjoint $A$ and $B$ cannot occur when when one of the sets are measurable.
\end{rmk}
\vspace{2ex}
\begin{proof}
Consider the measure of $m^{*}(A\cup E)$. If $E\in \mathfrak{M}$, 
\begin{align*}
	m^{*}(A\cup E)=&m^{*}([A\cup E]\cap E)+m^{*}([A\cup E]\cap E^{c})\\=& m^{*}(E)+m^{*}(A)
\end{align*}
\end{proof}
\vspace{2ex}
Also note how for overlapping $E$ and $A$, $E\in \mathfrak{M}$ and $A\supset E$ and then 
\begin{align*}
	m^{*}(A)=&m^{*}(A\cap E)+m^{*}(A\cap E^{c})\\=&m^{*}(E)+m^{*}(A\;\backslash\;E)
\end{align*}
Especially, if $A\supset E$ and $m^{*}(E)<\infty $, then $m^{*}(A\;\backslash\;E)=m^{*}(A)-m^{*}(E)$. This property is extremely important.
\\\\
We lastly note that due to the finite subadditive property of the outer measure, the measure of $A=[A\cap E]\cup [A\cap E^{c}]$ allows us to see that 
\[m^{*}(A)\leq m^{*}(A\cap E)+m^{*}(A\cap E^{c})\]
\begin{thm}
If $m^{*}(E)=0$ then $E$ is measurable. 
\end{thm}
\vspace{2ex}
\begin{proof}
Note that
\[m^{*}(A)\leq m^{*}(A\cap E)+m^{*}(A\cap E^{c})\leq m^{*}(E)+m^{*}(A)=m^{*}(A)\]
\end{proof}
\vspace{2ex}
\begin{cor}
If $B\subset E$ and $m^{*}(E)=0$ then $B\in \mathfrak{M}$. Also that that $n({\bm R})=2^{\aleph_{0}}=c$ and $n(\mathfrak{M})=2^{\aleph_{0}}=c$ since there is an uncountable number of measure zero sets.
\end{cor}
\vspace{2ex}
\begin{thm}
$\mathfrak{M}$ is a $\sigma $-algebra and $m^{*}$ satisfies countable additivity on $\mathfrak{M}$.
\end{thm}
\vspace{2ex}
\begin{lem}
The symmetry of the definition implies that $E\in \mathfrak{M}$ if and only if $E^{c}\in \mathfrak{M}$. It suffices to prove that a countable union of measurable sets is a measurable set. We first prove that a finite union is measurable. If $E_1,E_2\in \mathfrak{M}$, then $E_1\cup E_2\in \mathfrak{M}$. We have to show for every $A\in {\bm R}$ that $m^{*}(A)\geq m^{*}(A\cap(E_1\cup E_{2}))+m^{*}(A\cap (E_1\cup E_{2})^{c})$. For every $A\subset {\bm R}$, 
\begin{align*}
	m^{*}(A)=&m^{*}(A\cap E_1)+m^{*}(A\cap E_{1}^{c})\\
	=&m^{*}(A\cap E_1)+m^{*}(A\cap E_{1}^{c}\cap E_{2})+m^{*}(A\cap E_{1}^{c}\cap E_{2}^{c})\\
	\geq& m^{*}(A\cap [E_1\cup E_2])+m^{*}(A\cap [E_1\cup  E_2]^{c})
\end{align*}
Since $E_{2}\in \mathfrak{M}$ and $(A\cap E_1)\cup (A\cap E_1^{c}\cap E_{2})=A\cap (E_1\cup E_2)$. The last term also becomes $m^{*}(A\cap [E_{1}\cup E_{2}]^{c})$. The last equality finally shows that $E_1\cup E_2\in \mathfrak{M}$. This tells us that if $E_1\ldots E_{n}\in \mathfrak{M}$, then $\bigcup_{k=1}^{n}E_{k}\in \mathfrak{M}$.
\end{lem}
\vspace{2ex}
\begin{lem}
We now try the establish the finite additivity of the outer measure on $\mathfrak{M}$ (the Lebesgue measure). Let's say $E_1\ldots E_{n}\in \mathfrak{M}$ are mutually disjoint. Then for every $A\subset {\bm R}$,
\[m^{*}(A\cap[\bigcup^{n}_{k=1}E_{k}])=\sum ^{n}_{k=1}m^{*}(A\cap E_{k})\]
especially when $A={\bm R}$, we have
\[m^{*}\Big(\bigcup^{n}_{k=1}E_{k}\Big)=\sum ^{n}_{k=1}m^{*}(E_{k})\]
\end{lem}
\vspace{2ex}
\begin{proof}
The proof is done by induction. The statement is true for $n=1$. Assume $m^{*}(A\cap [\bigcup_{k=1}^{n-1}E_{k}])=\sum ^{n-1}_{k=1}m^{*}(A\cap E_{k})$. Since $E_{n}\in \mathfrak{M}$,
\begin{align*}
m^{*}\Big(A\cap [\bigcup ^{n}_{k=1}E_{k}] \Big)=&m^{*}\Big(A\cap [\bigcup^{n}_{k=1}E_{k}]\cap  E_{n}\Big)+m^{*}\Big(A\cap [\bigcup^{n}_{k=1}E_{k}]\cap E_{n}^{c}\Big)\\
=&m^{*}(A\cap E_{n})+m^{*}(A\cap [\bigcup^{n-1}_{k=1}E_{k}])\\
=&m^{*}(A\cap E_{n})+\sum _{k=1}^{n-1}m^{*}(A\cap E_{k})\\
=&\sum ^{n}_{k=1}m^{*}(A\cap E_{k})
\end{align*}
\end{proof}
\vspace{2ex}

