\section{Lecture 9 (April 1st)}
\begin{defi}
A sequence of functions is said to converge $f_{n}\rightarrow f$ almost uniformly on $E$ provided that for every $\delta  >0$ there is $A\subset E$ such that $m(A)<\delta $ and $f_{n}\rightarrow f$ uniformly on $A^{c}=E\;\backslash\;A$.
\end{defi}
\vspace{2ex}
\begin{thm}
If $f_{n}\rightarrow f$ almost uniformly on $E$ then $f_{n}\rightarrow f$ almost everywhere on $E$.
\end{thm}
\vspace{2ex}
\begin{proof}
For each $n\in {\bm N}$ there is $A_{n}\subset E$ such that $m(A_{n})<1/n$ and $f_{n}\rightarrow f$ uniformly on $A_{n}^{c}$. Notice how $m\Big(\bigcap _{n=1}^{\infty }A_{n}\Big)=0$ and if $x_{0}\in \Big[\bigcap ^{\infty }_{n=1}A_{n}\Big]^{c}$ then there is $k\in {\bm N}$ such that $x_{0}\in A_{k}^{c}$ (where $f_{n}\rightarrow f$ uniformly) and $\lim _{n\rightarrow \infty }f_{n}(x_0)=f(x_0)$.
\end{proof}
\vspace{2ex}
\begin{ex}
The function $f_{n}={\bf 1}_{(n,\infty )}\rightarrow 0$ pointwise on ${\bm R}$ but $f_{n}$ does not converge to 0 almost uniformly on ${\bm R}$. The reason why this fails is precisely because $m(E)=\infty $.
\end{ex}
\vspace{2ex}
\begin{thm}
(Egorov's theorem) If $m(E)<\infty $ and $f_{n}\rightarrow f$ almost everywhere on $E$ then $f_{n}\rightarrow f$ almost uniformly on $E$. 
\end{thm}
\vspace{2ex}
\begin{lem}
Suppose $f_{n}\rightarrow f$ on $E$ where $m(E)<\infty $. Then for every $\varepsilon >0$ and $\delta >0$ there is $A\subset E$ such that $m(A)<\delta $ (note how $A$ depends on $\varepsilon >0$). If $n>N$ then $|f_{n}(x)-f(x)|<\varepsilon$ for all $x\in A^{c}=E\;\backslash\;A$. 
\end{lem}
\vspace{2ex}
\begin{proof}
For each $n$, define
\[E_{n}=\bigcup ^{\infty }_{k=n}\{x\in E \;|\; |f_{k}(x)-f(x)|\geq \varepsilon \}\]
Then $E_{n}\supset E_{n+1}$ and $\bigcap ^{\infty }_{n=1}E_{n}=\emptyset$. Since $m(E)<\infty $, $\lim_{n\rightarrow \infty }m(E_{n})=m\Big(\bigcap ^{\infty }_{n=1}E_{n}\Big)=0$. Thus there is $N$ such that $m(E_{N})<\delta $. Put $A=E_{N}$. Since $A^{c}=\bigcap ^{\infty }_{k=N}\{x\in E \;|\; |f_{k}(x)-f(x)|<\varepsilon \}$, 
\[x\in A^{c}\hspace{2ex}\iff\hspace{2ex}\mathrm{if\ }n\geq N\mathrm{\ then\ }|f_{n}(x)-f(x)|<\varepsilon  \]
\end{proof}
\vspace{2ex}
\begin{proof}
Let $E_1=\{x\in E \;|\; \lim _{n\rightarrow \infty }f_{n}(x)=f(x)\}$ then $E_{1}^{c}=E\;\backslash\;E_{1}$ is measure zero. Thus we can assume that $f_{n}\rightarrow f$ pointwise on $E$. Take any $\delta >0$. By the previous lemma, for each $k$ ($\varepsilon _{k}=1/k$, $\delta _{k}=\delta /2^{k}$) there is $A_{k}\subset E$ and $N_{k}\in {\bm N}$ such that $m(A_{k})<\delta /2^{k}$ and if $n\geq N_{k}$ then $|f_{n}(x)-f(x)|<1/k$ for $x\in A^{c}_{k}$. Let $A=\bigcup ^{\infty }_{k=1}A_{k}$ then $m(A)<\delta $ and we'll show that $f_{n}\rightarrow f$ uniformly on $A^{c}$. 
\\\\
Take any $\varepsilon >0$, there is $N$ such that $1/N<\varepsilon $. If $x\in A^{c}=\bigcap ^{\infty }_{k=1}A^{c}_{k}$ then $x\in A_{N}^{c}$ so that if $n>N_{n}$ then $|f_{n}(x)-f(x)|<1/N<\varepsilon $. 
\end{proof}
\vspace{2ex}
\begin{proof}
We may assume that $f_{n}\rightarrow f$ pointwise on $E$. For $k,n\in {\bm N}$ we define
\[E_{k,n}=\bigcup _{j=n}^{\infty }\{x\in E \;|\; |f_{j}(x)-f(x)|\geq 1/k\}\]
Take any $\delta >0$, then for each $k$, there is $n_{k}$ such that 
\[m(E_{k,n_{k}})<\dfrac{\delta }{2^{k}}\]
Then define $A=\bigcup ^{\infty }_{k=1}E_{k,n_{k}}$ then $m(A)<\delta $ and $f_{n}\rightarrow f$ uniformly on $A^{c}$.
\end{proof}
\vspace{2ex}
\begin{thm}
(Lusin's theorem) If $f$ is measurable on $[a,b]$, then for every $\varepsilon >0$ there is $g\in C([a,b])$ such that $m\Big(\{x\in [a,b]\;|\; f(x)\ne g(x)\}\Big)<\varepsilon $.
\end{thm}
\vspace{2ex}
\begin{defi}
(Integral of a non-negative simple function) If $\phi >0$, with 
\[\phi =\sum ^{n}_{k=1}a_{k}{\bf 1} _{A_{k}}\]
then we define
\[\int _{{\bm R}}\phi \; dm=\sum ^{n}_{k=1}a_{k}m(A_{k})\]
If $A_{k}=\{x\in {\bm R} \;|\; \phi (x)=a_{k}\}$, that is, if $\{A_{k}\}$ form a partition, then $\sum ^{n}_{k=1}a_{k}{\bf 1} _{A_{k}}$ is called a canonical expression of $\phi $. We denote the set of non-negative measurable functions on ${\bm R}$ as $L^{+}$. We then define
\[\int _{{\bm R}}f\;dm=\mathrm{sup}\;\Big\{\int _{{\bm R}}\phi \;dm \;|\; 0\leq \phi \leq f\Big\}\]
where $\phi $ is simple.
\end{defi}
\vspace{2ex}
\begin{defi}
(General measure theorem) Let $X$ be a set and $\mathrm{\Sigma} $ be a $\sigma  $-algebra on $X$. A function $\mu :\mathrm{\Sigma}  \rightarrow [0,\infty ]$ is called a measure on $(X,\mathrm{\Sigma} )$ if
\begin{itemize}
\item[(i)] $\mu (\emptyset)=0$
\item[(ii)] $\mu $ satisfies countable additivity on $\mathrm{\Sigma} $
\end{itemize} 
\end{defi}
\vspace{2ex}

