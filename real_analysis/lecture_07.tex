\section{Lecture 7 (March 25th)}
\begin{defi}
For the Cantor function $f:[0,1]\rightarrow [0,1]$, we define the continuous map
\[\psi (x)=x+f(x)\]
which maps $[0,1]$ onto $[0,2]$.
\end{defi}
\vspace{2ex}
\begin{thm}
The function $\psi $ provides us three important counter-examples.
\begin{itemize}
	\item[(i)] For $E\in \mathfrak{M}$, $f$ being continuous does not imply that $f^{-1}(E)\in \mathfrak{M}$
	\item[(ii)] There exists a non-Borel measurable set. 
	\item[(iii)] A composition of measurable functions may not be a non-measurable function
\end{itemize}
The last will be shown later on.
\end{thm}
\vspace{2ex}
\begin{proof}
We first show that the inverse of a measurable set isn't always measurable for a continuous function. First, note that 
\[\psi([0,1])=\psi (C)\cup \psi (G)\]
is a disjoint union as $[0,1]=C\cup G$. We also know that $m(\psi (C))=1$, as $m(G)=1$ and $\phi $ sends this set into an set in $[0,2]$ of equal measure. We now notice that there is a non-measurable set $B\subset \psi (C)$ according to the Vitali theorem. By defining $A=\psi ^{-1}(B)$ ($\psi (A)=B$), we know that $A\subset C$ and that $m(A)=0$, and that $A$ is measurable. Now,  
\[B=\psi (A)=(\psi ^{-1})^{-1}(A)\]
To summarise, we have found $A$, a non-measurable set whose inverse is measurable. 
\end{proof}
\vspace{2ex}
\begin{proof}
We now show the existence of a non-Borel measurable set. We know if $E$ is a Borel set and $f$ is continuous on ${\bm R}$ then $f^{-1}(E)$ is a Borel set. If $A$ is a Borel set, then $B=(\psi ^{-1})^{-1}(A)$ must be Borel, but it is not measurable and therefore not. Thus, $A$ is not a Borel set but is measurable.
\end{proof}
\vspace{2ex}
\begin{defi}
The generalised Cantor set (or the fat Cantor set) is defined like the following. Consider $F_{n}$, a disjoint union of $2^{n}$ closed intervals, and let $F=\bigcap ^{\infty }_{n=1}F_{n}$ which is a closed set with empty interior. The length sum of the removed open intervals (let's consider removing $1/4$ths of the lengths) would be
\[\dfrac{1}{4}+2\cdot \dfrac{1}{4^2}+2^2\cdot \dfrac{1}{4^3}+\ldots =\dfrac{\frac{1}{4}}{1-\frac{2}{4}}=\dfrac{1}{2}\]
and we have $m(F)=1/2$. There are two ways to construct a fat Cantor set.
\begin{itemize}
	\item[(i)] Take any $0<\alpha <1/3$ and keep removing the middle $\alpha $-lengthed intervals. The length of the removed open intervals is 
\[m(F)=1-\dfrac{\alpha }{1-2\alpha }=\dfrac{1-3\alpha }{1-2\alpha }\]
\item[(ii)] Take any $0<\varepsilon <1$. Keep removing middle intervals with length $\varepsilon /3^{n}$ and intersect them to obtain $F=\bigcap F_{n}$. The length sum of the removed opened intervals is
\[\varepsilon \Big(\dfrac{1}{3}+2\cdot \dfrac{1}{3^2}+2^2\cdot \dfrac{1}{3^3}+\ldots \Big)=\varepsilon \]
and the measure would be $m(F)=1-\varepsilon$. 
\end{itemize}
\end{defi}
\vspace{2ex}
\begin{ex}
We construct a Borel set $B\subset {\bm R}$ satisfying $0<m(B)<1$ and $0<m(B\cap I)<m(I)$ for all intervals $I\subset {\bm R}$. Thus, $B$ is a Borel set that touches every interval but never fully contains it. First, notice that
\[\mathcal{F}=\{[a,b] \;|\; a,b\in {\bm Q}\}\]
contains countably many intervals. Let
\[\mathcal{F}=\{B_{n} \;|\; B_{n}\mathrm{\ is\ a\ closed\ interval\ with\ rational\ endpoints\ and\ }m(B_{n})\leq 1\}\]
Define $A_1=B_1$ and let $E_1$ be a fat Cantor set in $A_1$ with $m(E_1)=m(A_1)/3$. Then $B_2\;\backslash\;E_1$ contains a closed interval $A_2$. Let $E_2$ be a fat Cantor set in $A_2$ with 
\[
m(E_2)=\mathrm{min}\{m(A_1),m(A_2)\}/3^2\]
Then $B_2\;\backslash\;E_1\cup E_2$ contains a closed interval $A_3$. Let $E_3$ be a fat Cantor set in $A_3$ with
\[m(E_3)=\dfrac{1}{3^3}\mathrm{min}\{m(A_1),m(A_2),m(A_3)\}\]
In general, $B_{n}\cup \ldots \cup E_{n-1}$ contain a closed interval $A_{n}$ with 
\[m(E_{n})=\dfrac{1}{3^{n}}\mathrm{min}\{m(A_1),\ldots ,m(A_{n})\}\]
We remark that 
\[E_{n}\subset A_{n}\subset B_{n}\hspace{5ex}A_{n}\cap \Big[\bigcup ^{n-1}_{k=1}E_{k}\Big]=\emptyset\]
Lastly, define
\[E=\bigcup ^{\infty }_{n=1}E_{n}\]
In conclusion, for any interval, $I\supset B_{m}$ for some $m$ and
\[m(E\cap I)\geq m(E\cap B_{m})\geq m(E_{m})>0\]
and as $m(E_1\;\backslash\;E_2)\geq m(E_1)-m(E_2)$ if $m(E_1)<\infty $,
\begin{align*}
	m(I\;\backslash\;E)=&m\Big(B_{m}\;\backslash\;\bigcup ^{\infty }_{n=1}E_{n}\Big)\geq m\Big(A_{m}\;\backslash\;\bigcup ^{\infty }_{n=1}E_{n}\Big)\\
=&m\Big(A_{m}\;\backslash\;\bigcup ^{\infty }_{n=m}E_{n}\Big)
\geq m(A_{m})-m\Big(\bigcup ^{\infty }_{n=m}E_{n}\Big)\\
\geq & m(A_{m})-\sum ^{\infty }_{n=m}m(E_{n})
\geq m(A_{m})-\sum ^{\infty }_{n=m}\dfrac{m(A_{m})}{3^{n}}\\
=&m(A_{m})\Big[1-\sum ^{\infty }_{m}\dfrac{1}{3^{n}}\Big]
\geq \dfrac{1}{2}m(A_{m})>0
\end{align*}
\end{ex}
\vspace{2ex}
\begin{defi}
	A function $f:E\rightarrow {\bm R}$ or $f:E\rightarrow [-\infty ,\infty ]$ is called measurable (function) provided that
\begin{itemize}
	\item[(i)] $E\in \mathfrak{M}$
	\item[(ii)] $\{x\in E \;|\; f(x)>c\}\in \mathfrak{M}$ for every $c\in {\bm R}$
\end{itemize}
Succinctly, a function is measurable if its domain is measurable and the domain of slices of the function are also measurable.
\end{defi}
\vspace{2ex}
\begin{ex}
The characteristic function is defined as 
\[{\bf 1}_{E}(x)=\begin{cases}
1\hspace{5ex}x\in E\\
0\hspace{5ex}x\ne E
\end{cases}\]
Let $A\notin \mathfrak{M}$ then $\{x\in {\bm R} \;|\; {\bf 1}_{A}(x)>1/2\}=A\ne \mathfrak{M}$. In this way, $f={\bf 1}_{A}$ for $A\notin \mathfrak{M}$ is not a measurable function. We conclude that ${\bf 1}_{A}$ is measurable if and only if $A$ is measurable.
\end{ex}
\vspace{2ex}
\begin{ex}
If $f$ is continuous on $E$ then $f$ is measurable since $\{x\in E \;|\; f(x)>c\}=E\cap G$ for some open set $G$. 
\end{ex}
\vspace{2ex}
\begin{ex}
Define $h={\bf 1}_{A}\circ \psi ^{-1}$ where $\psi (x)=x+f(x)$. Let $A=\psi ^{-1}(B)$ where $B$ is non-measurable subset of $\psi (C)$. $m(A)=0$ implies that ${\bf 1}_{A}$ is measurable. However,
\[\Big\{x\in {\bm R} \;|\; h(x)>\dfrac{1}{2}\Big\}=({\bf 1}_{A}\circ \psi ^{-1})^{-1}\Big(\dfrac{1}{2},\infty \Big)=\psi ({\bf 1}^{-1}_{A}\Big(\dfrac{1}{2},\infty \Big))=\psi (A)=B\ne \mathfrak{M}\]
even though $\psi ^{-1}$ and $X_{A}$ are measurable functions. This is the aforementioned counterexample.
\end{ex}
\vspace{2ex}
\begin{recall}
$x\in f^{-1}(B)$ means that $f(x)\in B$. For a bijection $f$, 
\[\begin{cases}
f^{-1}(A\cup B)=f^{-1}(A)\cup f^{-1}(B)\hspace{5ex}f^{-1}(A\cap B)=f^{-1}(A)\cap f^{-1}(B)\\
f^{-1}(A^{c})=[f^{-1}(A)]^{c}\hspace{16ex}(f\circ g)^{-1}(A)=g^{-1}(f^{-1}(A))
\end{cases}\]
\end{recall}
\vspace{2ex}
\begin{defi}
A function $f:E\rightarrow {\bm R}$ is called Borel-measurable provided that
\begin{itemize}
	\item[(i)] $E\in \mathcal{B}$
	\item[(ii)] $\{x\in E \;|\;f(x)>c \}\in \mathcal{B}$ for every $c\in {\bm R}$ 
\end{itemize}
where $\mathcal{B}$ is the Borel $\sigma $-algebra.
\end{defi}
\vspace{2ex}
\begin{thm}
If $f$ is measurable on $E$ and $f(x)=g(x)$ almost everywhere on $E$, then $g$ is measurable.
\end{thm}
\vspace{2ex}
\begin{proof}
We require that the set
\[A=\{x \;|\; f(x)\ne g(x)\}\subset E\]
is measure zero. 
\[\{x\in E \;|\; g(x)>c\}=\{x\in E\;\backslash\;A \;|\; g(x)=f(x)>c\}\cup \{x\in A \;|\; g(x)>c\}\]
Is a union of two measurable sets and thus $g(x)$ is measurable. 
\end{proof}
\vspace{2ex}


