\section{Lecture 10 (April 3rd)}
\begin{ex}
We defined the measure space as a triple $(X,\Sigma ,\mu )$. Examples include the following:
\begin{itemize}
\item[(i)] The Lebesgue measure space $({\bm R},\mathfrak{M},m)$ ($\sigma $-finite and complete)
\item[(ii)] The Borel measure space $(X,\mathcal{P}(X),\mu )$ $({\bm R},\mathcal{B},m)$ ($\sigma $-finite but not complete)
\item[(iii)] The counting measure on $X$ where $\mu (E)$ is the number of elements of $E$ (complete but if $X$ is uncountable then the measure space is not $\sigma $-finite)
\item[(iv)] The Dirac measure $(X,\mathcal{P}(X),\delta _{x_0})$ where for $x_0\in X$ we define
\[\delta _{x_0}(x)=
\begin{cases}
0\hspace{5ex}x_0\notin E\\
1\hspace{5ex}x_0\in E
\end{cases}
\] (complete and $\sigma $-finite)
\item[(v)] $\mathrm{\Sigma} =\{E\subset {\bm R} \;|\; E\mathrm{\ or \ }E^{c}\mathrm{\ is\ countable}\}$ and
\[\mu (E)=
\begin{cases}
0\hspace{5ex}E\mathrm{\ countable}\\
1\hspace{5ex}E^{c}={\bm R}\;\backslash\;E\mathrm{\ countable}
\end{cases}
\] (complete and $\sigma $-finite)
\item[(vi)] $X=\{a,b\}$, $\mathrm{\Sigma} =\{\emptyset,X\}$, $\mu (\emptyset)=\mu (X)=0$ ($\sigma $-finite but not complete)
\end{itemize}
\end{ex}
\vspace{2ex}
\begin{defi}
For a measure space $(X,\mathrm{\Sigma} ,\mu )$,
\begin{itemize}
\item[(i)] $\mu $ is complete provided that if $E\subset A$ and $\mu (A)=0$, then $E\in \mathrm{\Sigma} $. That is, $\mathrm{\Sigma} $ contains all subsets of measure zero sets. 
\item[(ii)] $\mu $ is called $\sigma $-finite provided that $X=\bigcup ^{\infty }_{n=1}E_{n}$ for which $\mu (E_{n})<\infty $ (the set $X$ is expressible as an infinite union of finite-measure sets)
\end{itemize}
\end{defi}
\vspace{2ex}
\begin{defi}
We previously defined the integral of a simple function. If $\phi =\sum ^{n}_{k=1}a_{k}{\bf 1}_{A_{k}}$ then the following is well-defined.
\[\int _{{\bm R}}\phi \;dm=\sum ^{n}_{k=1}a_{k}m(A_{k})\]
If $\phi $ and $\psi $ are non-negative simple functions, then for $c>0$,
\begin{align*}
\int _{{\bm R}}c\phi \;dm&=c\int _{{\bm R}}\phi \;dm\\
\int _{{\bm R}}(\phi +\psi )\;dm=&\int _{{\bm R}}\phi \;dm+\int _{{\bm R}}\psi  \;dm
\end{align*}
In other words, integration is a linear operator.
\end{defi}
\vspace{2ex}
\begin{lem}
Let $\{E_{i}\}$ be a finite disjoint collection of measurable subsets of a set of finite measure $E$. If
\[\phi =\sum ^{n}_{i=1}a_{i}{\bf 1}_{E_{i}}\]
then
\[\int _{E}\phi =\sum ^{n}_{i=1}a_{i}m(E_{i})\]
\end{lem}
\vspace{2ex}
\begin{proof}
Notice that
\begin{align*}
\int _{E}(\alpha \phi +\beta \psi )=&\sum ^{n}_{i=1}(\alpha a_{i}+\beta b_{i})m(E_{i})\\
=&\alpha \sum ^{n}_{i=1}a_{i} m(E_{i})+\beta \sum ^{n}_{i=1}b_{i} m(E_{i})\\
=&\alpha \int _{E}\phi +\beta \int _{E}\psi 
\end{align*}
\end{proof}
\vspace{2ex}
\begin{defi}
For $E\in \mathfrak{M}$ we define
\[\int _{E}\phi \;dm=\int _{{\bm R}}\phi {\bf 1}_{E}\;dm=\sum ^{n}_{k=1}a_{k}m(A_{k}\cap E)\]
Note that ${\bf 1} _{A}{\bf 1} _{E}={\bf 1}_{A\cap E}$.
\end{defi}
\vspace{2ex}
\begin{thm}
For a non-negative simple function $\phi $ if we define
\[\mu (E)=\int _{E}\phi \; dm\]
then $\mu $ is a measure on $({\bm R},\mathfrak{M})$. 
\end{thm}
\vspace{2ex}
\begin{proof}
Let $\phi =\sum ^{n}_{k=1}b_{k}{\bf 1}_{B_{k}}$ be the canonical representation. Let $A=\bigcup ^{\infty }_{n=1}A_{n}$ where $A_{n}$'s are mutually disjoint and measurable. Then
\begin{align*}
\mu (A)=\int _{A}\phi \; dm=\int _{{\bm R}}\phi {\bf 1} _{A}\; dm=\sum ^{n}_{k=1}b_{k}m(A\cap B_{k})=&\sum ^{n}_{k=1}b_{k}\sum ^{\infty }_{j=1}m(A_{j}\cap B_{k})\\
=&\sum ^{\infty }_{j=1}\sum ^{n}_{k=1}b_{k}m(A_{j}\cap B_{k})\\
=&\sum ^{\infty }_{j=1}\mu (A_{j})
\end{align*}
\end{proof}
\vspace{2ex}
\begin{thm}
The measure $\mu $ on $(X,\mathrm{\Sigma} )$ is continuous, that is, if $E_1\subset E_2\subset \ldots $ then
\[\int _{\bigcup E_{n}}\phi\; dm=\lim \int _{E_{n}}\phi \; dm\]
\end{thm}
\vspace{2ex}
\begin{proof}
\begin{itemize}
\item[(i)] If $\{A_{n}\}$ is an increasing sequence in $\mathrm{\Sigma} $, then $\mu \Big(\bigcup ^{\infty }_{n=1}A_{n}\Big)=\lim _{n\rightarrow \infty }\mu (A_{n})$
\item[(ii)] If $\mu(A_{1})<\infty $ and $\{A_{n}\}$ is decreasing in $\mathrm{\Sigma} $ then $\mu \Big(\bigcap ^{\infty }_{n=1}A_{n}\Big)=\lim _{n\rightarrow \infty }\mu (A_{n})$
\end{itemize}
\end{proof}
\vspace{2ex}
\begin{defi}
For a non-negative measurable function $f$ on ${\bm R}$ ($f\in L^{+}({\bm R})$), we define
\[\int _{{\bm R}}f\;dm=\mathrm{sup}\Big\{\int _{E}\phi \;dm \;|\; 0\leq \phi \leq f\mathrm{\ where\ }\phi \mathrm{\ is\ simple}\Big\}\]
\end{defi}
\vspace{2ex}
\begin{recall}
The simple approximation theorem stated that there is a sequence of simple functions $\{\phi _{n}\}$ such that $\lim _{n\rightarrow \infty }\phi _{n}(x)\rightarrow f(x)$ increasingly.
\end{recall}
\vspace{2ex}
\begin{thm}
(Monotone convergence theorem I) If $\{f_{n}\}$ is a sequence in $L^{+}({\bm R})$ such that $f_{n}\uparrow f$ (increasingly) pointwise on ${\bm R}$ (that is, $f_{n}(x)\leq  f_{n+1}(x)\ \forall x\in {\bm R}$), then 
\[\int _{{\bm R}}f\; dm=\lim _{n\rightarrow \infty }\int _{{\bm R}}f_{n}\; dm\]
\end{thm}
\vspace{2ex}
\begin{thm}
(Monotone convergence theorem II) If $\{f_{n}\}\subset L^{+}(E)$ such that $f_{n}\uparrow f$ increasingly on $E$, then
\[\int _{E}f\;dm=\lim _{n\rightarrow \infty }\int _{E}f_{n}\;dm\]
\end{thm}
\vspace{2ex}
\begin{lem}
If $f,g\in L^{+}(E)$ and $f(x)\leq g(x)$ on $E$, then
\[\int _{E}f\;dm\leq \int _{E}g\;d m\]
Also, if $E\subset F\in \mathfrak{M}$ and $f\in L^{+}({\bm R})$ then
\[\int _{E}f\;d m\leq \int _{F}f\;d m\]
since $f{\bf 1} _{E}\leq f{\bf 1}_{F}$. Lastly, if $m(E)=0$ then $\int _{E}f\;d m=0 $  $\forall f\in L^{+}(E)$ and
\[\int _{E}\phi \;d m=\sum ^{n}_{k=1}a_{k}m(A_{k}\cap E)=0\]
\end{lem}
\vspace{2ex}
\begin{proof}
Since $\int _{E}f_{n}\;d m$ is monotone increasing in $[0,\infty ]$, $\lim \int _{E}f_{n}\;d m$ exists and $\lim \int _{E}f_{n}\;d m\leq \int _{E}f\;d m$. It suffices to show that $\int _{E}f\;d m\leq \lim \int _{E}f_{n}\;d m$. By definition of the supremum, we only need to prove that, for any simple function $0\leq \phi \leq f$,
\[\int _{E}\phi \;d m\leq \lim_{n\rightarrow \infty } \int _{E}f_{n}\;d m\]
This is equivalent to saying that for a given $\phi $ and every $0<\alpha <1$ it suffices to show $\alpha \int _{E}\phi\;d m\leq \lim _{n\rightarrow \infty }\int _{E}f_{n}\;d m$. Define $A_{n}=\{x\in E \;|\; \alpha \phi (x)<f_{n}(x)\}\in \mathfrak{M}$. Note that $A_{n}\subset A_{n+1}$ and $\bigcup _{n=1}^{\infty }A_{n}=E$. We therefore have the inequality
\[\int _{A_{n}}\alpha \phi \;d m\leq \int _{A_{n}}f_{n}\;d m\leq \int _{E}f_{n}\;d m\]
Here, we apply the continuity of the measure $\mu (A)=\int _{A}\phi \;d m$ and we have
\[\mu (E)=\int _{E}\phi \;d m=\lim _{n\rightarrow \infty }\int _{A_{n}}\phi \;d m\]
Taking $n\rightarrow \infty $ for the most left term in the inequality above,
\[\lim _{n\rightarrow \infty }\int _{A_{n}}\alpha \phi \;d m=\alpha \lim _{n\rightarrow \infty }\int _{A_{n}}\phi \;d m=\alpha \int _{E}\phi \;d m\]
We arrive at 
\[\alpha \int _{E}\phi \;d m\leq \lim_{n\rightarrow \infty } \int _{E}f_{n}\;d m\]
which concludes the proof.
\end{proof}
\vspace{2ex}

