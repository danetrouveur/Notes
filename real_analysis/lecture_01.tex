\section{Lecture 1 (Mar. 4th)}
\begin{recall}
	The Riemann integral is defined as the following limit given that the function $f$ is bounded on $[a,b]$. Take a partition $P=\{x_0,x_1,\ldots ,x_{n}\}$ of $[a,b]$ where $a=x_0<x_1<\ldots <x_{n}=b$. Let $I_{k}=[x_{k-1},x_{k}]$ denote the $k$-th subinterval of $[a,b]$. Then $\Delta x_{k}=x_{k}-x_{k-1}$ is called the length of $I_{k}$ and we can define for $k=1,\ldots  ,n$ the following
\[\begin{cases}
U(f,P)=\sum_{k=1}^{n}M_{k}\Delta x_{k}\\
L(f,P)=\sum _{k=1}^{n}m_{k}\Delta x_{k}
\end{cases}\]
called the upper sum and lower sum of $f$ with respect to $P$ respectively. Let $P_1$ be the refinement of $P_2$. Then 
\[U(f,P_1)\leq U(f,P_2)\hspace{3ex}L(f,P_1)\geq L(f,P_2)\]
and for any two arbitrary partitions,
\[L(f,P_1)\leq U(f,P_2)\]
Consider the following inequality
\[\overline{\int_{a}^{b}}f\;dx=\mathrm{inf}_{P}\;U(f,P)\geq \mathrm{sup}_{P}\;L(f,P)=\underline{\int_{a}^{b}} f\;dx\]
when the two limits become equal, the function $f$ is called Riemann-integrable and we denote the value of the limit as $\int _{a}^{b}f\;dx$. Note that if $f$ is continuous for all of $[a,b]$ but countablely many points, then $f$ is Riemann integrable.

However, we must note that there exists functions (similar to the Dirichlet function) such as the following
\[f(x)=\begin{cases}
	1\hspace{2ex}x\in [a,b]\cap{\bm Q}\\
	-1\hspace{2ex}x\in [a,b]\,\backslash\,{\bm Q}
\end{cases}\]
which is obviously not Riemann integrable. An idea would be to consider the integral as the following
\[\int_{0}^{1}f\;dx=1\cdot m({\bm Q}\cap[0,1])+(-1)\cdot m([0,1]\,\backslash\,{\bm Q})=1\times 0+(-1)\times 1=-1\]
\end{recall}
\vspace{2ex}
\begin{thm}
	(Lebesgue theorem) Let $f$ be bounded on $[a,b]$. $f$ is Riemann integrable on $[a,b]$ if and only if the points of discontinuity $f$ on $[0,1]$ denoted as $D_{f}[0,1]$ has measure zero.
	\[m(D_{f}[0,1])=0\]
	where, again, $D_{f}[0,1]=\{x_0\in [0,1] \;|\;f\mathrm{\;is\;not\;continuous\;at\;}x_0 \}$. On this note, $E \subset {\bm R}$ is called measure zero provided that for every $\varepsilon >0$ there is a collection of open intervals $\{I_{n}\}$ satisfying $E\subset \bigcup^{\infty }_{n=1}I_{n}$ and $\sum _{n=1}^{\infty }l(I_{n})<\varepsilon $ (recall that a set $\mathcal{O}$ is open in ${\bm R}$ if and only if $\mathcal{O}$ is a countable union of disjoint open intervals). 
\end{thm}
\vspace{2ex}
\begin{ex}
$E$ is countable, then $E$ has a measure of zero.
\end{ex}
\vspace{2ex}
\begin{proof}
Let $E=\{x_1,x_2,x_3,\ldots \}$ be an enumeration. Take any $\varepsilon >0$ and define
\[I_{n}=\Big(x_{n}-\dfrac{\varepsilon }{2^{n+2}},x_{n}+\dfrac{\varepsilon }{2^{n+2}}\Big)\]
then $l(I_{n})=\varepsilon /2^{n+1}$. It can observed that $E\subset \bigcup^{\infty }_{n=1}I_{n}$ and 
\[\sum ^{\infty }_{n=1}l(I_{n})=\sum ^{\infty }_{n=1}\dfrac{\varepsilon }{2^{n+1}}=\dfrac{\varepsilon }{2}<\varepsilon \]
\end{proof}
\vspace{2ex}


