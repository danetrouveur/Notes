\section{Lecture 20 (May 27th)}
\begin{defi}
A separable metric space $X$ is a space that has a countable dense subset.
\end{defi}
\vspace{2ex}
\begin{defi}
For $f_{n}:X\rightarrow {\bm R}$, $\{f_{n}\}$ is pointwise bounded on $X$ provided that for every $x\in X$ there is $M_{x}>0$ such that $|f_{n}(x)|\leq M_{x}$ for all $n\in {\bm N}$.
\end{defi}
\vspace{2ex}
\begin{ex}
Let $X,Y$ be normed vector spaces. Let $T_{n}:X\rightarrow Y$ be a bounded linear map such that $||T_{n}||\leq M$ for all $n\in {\bm N}$. Then for every $x\in X$, we have
\[||T_{n}(x)||\leq ||T_{n}||\,||x||\leq M||x||\]
Thus $\{T_{n}\}$ is pointwise bounded on $X$.
\end{ex}
\vspace{2ex}
\begin{defi}
$\{f_{n}\}$ is equicontinuous on $X$ provided that for every $\varepsilon >0$, there is $\delta >0$ such that if $x,y\in X$ satisfies $d(x,y)<\delta $, then $|f_{n}(x)-f_{n}(y)|<\varepsilon $ for all $n\in {\bm N}$.
\end{defi}
\vspace{2ex}
\begin{ex}
Consider $f_{n}:{\bm R}\rightarrow {\bm R}$ where $f_{n}=nx$. As
\[|f_{n}(x)-f_{n}(y)|=n|x-y|\]
and each $f_{n}$ is uniformly continuous on ${\bm R}$. However, $\{f_{n}\}$ is not equicontinuous on ${\bm R}$.
\end{ex}
\vspace{2ex}
\begin{ex}
The function $f_{n}(x)=\sin nx$ is not equicontinuous on $[0,2\pi ]$.
\end{ex}
\vspace{2ex}
\begin{ex}
Let $T_{n}:X\rightarrow Y$ be norm bounded, that is, $||T_{n}||\leq M$ for all $n\in {\bm N}$. This implies that 
\[||T_{n}(x)-T_{n}(y)||=||T_{n}(x-y)||\leq M||x-y||\]
so that $\{T_{n}\}$ is equicontinuous on $X$. 
\end{ex}
\vspace{2ex}
\begin{thm}
(Arzela-Ascoli theorem) Consider a sequence in $X^{*}$, that is, $f_{n}:X\rightarrow {\bm R}$ for $n\in {\bm N}$ where $X$ is a separable metric space. Take $\{f_{n}\}$ to be pointwise bounded and equicontinuous on $X$. Then, $\{f_{n}\}$ has a subsequence $\{f_{n_{k}}\}$ which converges uniformly on all compact subsets of $X$. 
\end{thm}
\vspace{2ex}
\begin{proof}
 Let $E=\{x_1,x_2,\ldots  \}$ be a countable dense subset of $X$.
\begin{align*}
&f_{1,1},\quad f_{2,1}\quad f_{3,1},\quad \ldots\\
&f_{1,2},\quad f_{2,2}\quad f_{3,2},\quad \ldots \\
&\qquad\qquad \vdots\\
&f_{1,k},\quad f_{2,k}\quad f_{3,k},\quad \ldots
\end{align*}
({\it STEP} 1) We construct a subsequence of $\{f_{n}\}$ which converges at every $x_{k}\in E$. This requires only pointwise boundedness. Note that $\{f_{n}(x_1)\}$ is a bounded sequence on ${\bm R}$, so that there is a subsequence $\{f_{n,1}\}$ of $\{f_{n}\}$ which converges at $x_1$. Likewise, $\{f_{n,1}(x_2)\}$ is a bounded sequence in ${\bm R}$ so that there is a subsequence $\{f_{n,2}\}$ of $\{f_{n,1}\}$ converges at $x_2$ (it also converges at $x_1$). Continuing, we find $\{f_{n,k}\}$ which converges at $\{x_{k+1}\}$ and also $x_1,x_2,\ldots, x_{k}$. Now, take the diagonal sequence $\{f_{n,n}\}^{\infty }_{n=1}$. Then we'll show that $\lim _{n\rightarrow \infty }f_{n,n}(x_{k})$ converges at every $x_{k}\in E$. Take any $x_{k}\in E$ then $\{f_{n,n}\}^{\infty }_{n=k}$ is a subsetquence of $\{f_{n,k}\}^{\infty }_{n=k}$.
\newline
\newline
Define $g_{n}=f_{n,n}$. Let $K$ be any compact subset of $X$. We'll show that $\{g_{n}\}$ converges uniformly on $K$. Equivalently, we'll show that the sequence $\{g_{n}\}$ is uniformly Cauchy on $K$. Take any $\varepsilon >0$, we will find $N\in {\bm N}$ such that if $n,m>N$ then $|g_{m}(x)-g_{n}(x)|<\varepsilon $ for all $x\in K$.
\newline
\newline
({\it STEP} 2) We show that the above subsequence converges uniformly on all compact subsets of $X$. This requires only equicontinuity. By equicontinuity of $\{g_{n}\}$, there is $\delta>0$ such that if $d(p,q)<\delta $, then $|g_{n}(p)-g_{n}(q)|<\varepsilon/3 $ for all $n\in {\bm N}$. Notice that
\[\mathcal{F}=\Big\{B\Big(x,\dfrac{\delta }{2}\Big) \,\Big|\,x\in K \Big\}\]
is an open cover of $K$ so that is a finite subcover $\{B_1,B_2,\ldots ,B_{M}\}$ all with radius $\delta /2$. Recall that $E$ was a countable dense subset in $X$ such that there is a subsequence of $\{f_{n}\}$ that converges. Since $E=\{x_{n}\}$ is dense in $X$, for every $1\leq k\leq M$, there is $x_{k}\in E\cap B_{k}$. Then there is $N\in {\bm N}$ such that if $n,m>N$ then
\[|g_{n}(x_{k})-g_{m}(x_{k})|<\dfrac{\varepsilon }{3}\]
for $1\leq k\leq M$. In sum, if $n,m>N$ for some $N$ and $x\in K$, then $x\in B_{j}$ for some $1\leq j\leq M$ so that $d(x,y)<\delta $ if $y\in B_{j}$. We finally see that, 
\begin{align*}
|g_{n}(x)-g_{m}(x)|=&|g_{n}(x)-g_{n}(p_{j})+g_{n}(p_{j})-g_{m}(p_{j})+g_{m}(p_{j})-g_{m}(x)|\\
\leq &|g_{n}(x)-g_{n}(p_{j})|+|g_{n}(p_{j})-g_{m}(p_{j})|+|g_{m}(p_{j})-g_{m}(x)|\\<&\dfrac{\varepsilon }{3}+\dfrac{\varepsilon }{3}+\dfrac{\varepsilon }{3}=\varepsilon 
\end{align*}
\end{proof}
\vspace{2ex}
\begin{rmk}
Let $X$ be a normed vector space, and let $X^{*}$ be the dual space $\mathcal{L}(X,{\bm R})$. There are three types of convergences in this book.
\begin{itemize}
\item[(i)] Norm convergence, both in $X$ and $X^{*}$
\item[(ii)] Weak convergence in $X$
\item[(iii)] Weak$^{*}$ convergence in $X^{*}$
\end{itemize}
We now learn weak convergence in $X$. 
\end{rmk}
\vspace{2ex}
\begin{defi}
Let $x_{n}\in X$ and $x\in X$. $x_{n}\rightarrow x$ weakly in $X$ if 
\[\lim _{n\rightarrow \infty }T(x_{n})=T(x)\] 
for every $T\in X^{*}$. On the other hand, $T_{n}\rightarrow T$ weak$^{*}$ if 
\[\lim _{n\rightarrow \infty }T_{n}(x)=T(x)\]
for every $x\in X$. The latter is simply pointwise convergence in $X$.
\end{defi}
\vspace{2ex}
\begin{rmk}
For $1\leq p<\infty $ and $1/p+1/q=1$, by the Riesz representation theorem, $(L^{p})^{*}=L^{q}$ and $(L^{q})^{*}=L^{p}$. Let $f_{n}\in L^{p}(E)$. $f_{n}\rightarrow f$ weakly in $L^{p}(E)$ means that
\[\lim _{n\rightarrow \infty }\int _{E}f_{n}g\,d m=\int _{E}fg\,d m\]
for every $g\in L^{q}(E)$. Notice how this is the same as $f_{n}\rightarrow f$ weak$^{*}$ in $(L^{q}(E))^{*}$. 
\[\mathrm{weak\ convergence\ in\ }L^{p}(E)=\mathrm{weak}^{*}\ \mathrm{convergence\ in\ }(L^{q}(E))^{*}!\]
\end{rmk}
\vspace{2ex}
\begin{cor}
Let $X$ be a separable normed vector space. Every bounded sequence in $X^{*}$ has a weak$^{*}$ convergent subsequence.
\end{cor}
\vspace{2ex}
\begin{proof}
If $\{T_{n}\}$ is a bounded sequence in $X^{*}$ ($T_{n}:X\rightarrow {\bm R}$, $||T_{n}||\leq M$) then $\{T_{n}\}$ is a pointwise bounded and equicontinuous. Then, as every singleton set is compact, for every $x\in X$, there exists a subsequence $T_{n_{k}}(x)$ that converges. Define $T:X\rightarrow {\bm R}$ as $T(x)=\lim _{k\rightarrow \infty }T_{n_{k}}(x)$. Then, $T\in X^{*}$ and $\{T_{n}\}$ converges weak$^{*}$.
\end{proof}
\vspace{2ex}
\begin{cor}
Assume $1\leq p<\infty $ and $||f_{n}||_{p}<\infty $. A bounded linear operator $T_{n}:L^{q}(E)\rightarrow {\bm R}$ is defined as
\[T_{n}(g)=\int _{E}f_{n}g\,d m\]
where $||T_{n}||=||f_{n}||_{p}$. By the previous corollary, $T_{n}$ has a weak$^{*}$ convergent subsequence where
\[\lim _{n\rightarrow \infty }T_{n_{k}}(g)=T(g)\]
for all $g\in L^{q}$. Then,
\[\lim _{k\rightarrow \infty }\int _{E}f_{n_{k}}g\,d m=\int _{E}fg\,d m\]
for some $f\in L^{p}(E)$. Therefore, if there is a sequence of $f_{n}\in L^{p}(E)$, there exists a subsequence $f_{n_{k}}$ and $f\in L^{p}(E)$ such that
\[\lim _{k\rightarrow \infty }\int _{E}f_{n_{k}}g\,d m=\int _{E}fg\,d m\]
for all $g\in L^{q}(E)$. 
\end{cor}
\vspace{2ex}

