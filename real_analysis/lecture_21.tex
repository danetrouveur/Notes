\section{Lecture 21 (May 29th)}
\begin{thm}
Take $X$ to be a separable Banach space. Let $T_{n}\in X^{*}$ with $||T_{n}||\leq M$ for all $n$. There is $\{T_{n_{k}}\}$ such that $\lim _{k\rightarrow \infty }T_{n_{k}}(x)$ converges for every $x\in X$. If we define $T:X\rightarrow {\bm R}$ by $T(x)=\lim _{k\rightarrow \infty }T_{n_{k}}(x)$ then obviously, $T$ is linear in $X$ and for $||x||=1$,
\[|T(x)|=\lim _{k\rightarrow \infty }|T_{n_{k}}(x)|\leq M\]
so that $||T||\leq M$. Therefore, $T\in X^{*}$.
\end{thm}
\vspace{2ex}
\begin{thm}
(Fubini theorem) Let $f(x,y)$ be measurable on $E\times F$. If 
\[\int _{F}\int _{E}|f(x,y)|\,dm(x)\,dm(y)<\infty \]
or 
\[\int _{E}\int _{F}|f(x,y)|\,d m(y)\,d m(x)<\infty \]
(meaning that $f\in L^{1}(E\times F,m\times m)$), then
\[\int _{E}\int _{F}f(x,y)\,dm(y)\,d m(x)=\int _{F}\int _{E}f(x,y)\,d m(x)\,dm(y)\]
If $f(x,y)\geq 0$, the result satisfies also.
\end{thm}
\vspace{2ex}
\begin{thm}
Take $f\in L^{p}(E)$, $1\leq p\leq\infty $. An integral operator is defined as
\[Tf(x)=\int _{E}K(x,y)f(y)\,d m(y)\]
If there is $c>0$ such that 
\[\sup_{x\in E}\int _{E}|K(x,y)|\,d m(y)\leq c\quad \mathrm{and}\quad \sup_{y\in E}\int |K(x,y)|\,d m(x)<c\]
then $||Tf||_{p}\leq c||f||_{p}$ for all $f\in L^{p}(E)$. 
\end{thm}
\vspace{2ex}
\begin{proof}
For $p=1$, 
\begin{align*}
||Tf||_{1}=&\int _{E}\Big|\int _{E}K(x,y)f(y)\,d m(y)\Big|\,d m(x)\\
\leq &\int _{E}\int _{E}|K(x,y)|\,|f(y)|\,d m(y)\,dm(x)\\
\leq &c\int_E |f(y)|\,d m(y)
\end{align*}
The case is identical for $p=\infty $. For $1<p<\infty $,
\begin{align*}
|Tf(x)|\leq \int _{E}|K(x,y)|\,|f(y)|\,d m(y)=&\int _{E}|K(x,y)|^{1/p+1/q}|f(y)|\,d m(y)\\
\leq &\Big[\int _{E}|K(x,y)|\,d m(y)\Big]^{1/q}\Big[\int _{E}|K(x,y)|\,|f(y)|^{p}\,d m(y)\Big]^{1/p}\\
\leq& c^{1/q}\Big[\int _{E}|K(x,y)|\,|f(y)|^{p}\,d m(y)\Big]^{1/p}
\end{align*}
So that
\begin{align*}
\int _{E}|Tf(x)|^{p}\,d m(x)\leq & c^{p/q}\int _{E}\int _{E}|K(x,y)|\,|f(y)|^{p}\,d m(y)\,d m(x)\\
=&c^{p/q}\int _{E}|f(y)|^{p}\,\int _{E}|K(x,y)|\,d m(x)\,d m(y)\\
\leq & c^{(p+q)/q}\int _{E}|f(y)|^{p}\,d m(y)
\end{align*}
this implies that
\[||Tf||_{p}\leq c||f||_{p}\]
\end{proof}
\vspace{2ex}
\begin{defi}
Let $f,g$ be measurable functions on ${\bm R}$. We define $f*g$ (convolution) by 
\[(f*g)(x)=\int ^{\infty }_{-\infty }f(x-y)g(y)\,d y=\int ^{\infty }_{-\infty }f(y)g(x-y)\,d y=(g*f)(x)\]
if it exists (note how it is commutative). 
\end{defi}
\vspace{2ex}
\begin{thm}
If $f\in L^{1}({\bm R})$ and $g\in L^{p}({\bm R})$ for $1\leq p\leq \infty $, then $||f*g||_{p}\leq ||f||_{1}||g||_{p}$.
\end{thm}
\vspace{2ex}
\begin{proof}
Define $K(x-y)=f(x-y)$ in the previous theorem. 
\end{proof}
\vspace{2ex}
\begin{rmk}
Consider
\[g(t)=\begin{cases}
\exp \Big(-\dfrac{1}{1-t^2}\Big)&-1<t<1\\
0&|t|\geq 1
\end{cases}\]
which is $C^{\infty }$ with a support $[-1,1]$. Define
\[\phi (x)=\dfrac{g(x)}{\int ^{1}_{-1}g(t)\,dt }\]
then $\phi \in C^{\infty }({\bm R})$ with support $[-1,1]$ with $\int ^{\infty }_{-\infty }\phi (x)\,d x=1$. Then
\[\phi _{\varepsilon }(x)=\dfrac{1}{\varepsilon }\phi \Big(\dfrac{x}{\varepsilon }\Big)\]
for $\varepsilon >0$ supported $[-\varepsilon ,\varepsilon ]$ with
\[\int ^{\infty }_{-\infty }\phi _{\varepsilon }(x)\,dx=1\]
If $f\in L^{p}({\bm R})$ with compact support then $f*\phi _{\varepsilon }\rightarrow f$ in $L^{p} $ for $1<p<\infty $ as $\varepsilon \rightarrow 0$. The convolution $f*\phi _{\varepsilon }\in C^{\infty }_{c}({\bm R})$ and $C^{\infty }_{c}({\bm R})$ is dense in $L^{p}({\bm R})$ ($1\leq p<\infty $).
\end{rmk}
\vspace{2ex}
\begin{thm}
(Schur) If there is a non-negative measurable function $h$ on $E$ such that 
\[\int _{E}|K(x,y)|\,h(y)^{q}\,d m(y)\leq c_1h(x)^{q}\]
and
\[\int _{E}|K(x,y)|\,h(y)^{p}\,d m(x)\leq c_2h(y)^{p}\]
then $||Tf||_{p}\leq c_1^{1/q}c_2^{1/p}||f||_{p}$ for $1<p<\infty $. 
\end{thm}
\vspace{2ex}
\begin{proof}
We provide a hint. 
\[|Tf(x)|\leq \int _{E}|K(x,y)|\,h(y)h(y)^{-1}|f(y)|\,d m(y)\]
\end{proof}
\vspace{2ex}
\begin{defi}
Let $H$ be a vector space over ${\bm R}$. If there is a function $\langle \cdot ,\cdot \rangle :H\times H\rightarrow {\bm R}$ satifying
\begin{itemize}
\item[(i)] $\langle x,y\rangle =\langle y,x\rangle $ for all $x,y\in H$
\item[(ii)] $\langle \alpha x,y\rangle =\alpha \langle x,y\rangle $ for $x,y\in H$, $\alpha \in {\bm R}$
\item[(iii)] $\langle x+z,y\rangle =\langle x,y\rangle +\langle z,y\rangle $ for $x,y,z\in H$
\item[(iv)] $\langle x,x\rangle \geq 0$ for all $x\in H$ and $\langle x,x\rangle =0$ if and only if $x={\bf 0}$.
\end{itemize}
Then, $H$ is called an inner product space over ${\bm R}$. If we define $||x||=\langle x,x\rangle ^{1/2}$, we can show that $||x||$ is a norm. 
\end{defi}
\vspace{2ex}
\begin{ex}
\begin{itemize}
\item[(i)] ${\bm R}^{n}$ with the inner product $\langle x,y\rangle =\sum _{k=1}^{n}x_{k}y_{k}$
\item[(ii)] $L^{2}(\mu )$ with the inner product 
\[\langle f,g\rangle =\int _{X}fg\,d \mu \]
\item[(iii)] $C([0,1])$ with the inner product
\[\langle f,g\rangle =\int ^{1}_{0}f(x)g(x)\,d x\]
\end{itemize}
\end{ex}
\vspace{2ex}
\begin{thm}
(Cauchy-Schwarz Inequality) For $x,y\in H$, $|\langle x,y\rangle |\leq ||x||\,||y||$ 
\end{thm}
\vspace{2ex}
\begin{proof}
For every $t\in {\bm R}$, $||x-ty||^2\geq 0$ and 
\[||x-ty||^2=\langle x-ty,x-ty\rangle =||x||^2-2\langle x,y\rangle t+||y||^2t^2\]
If $y\ne 0$, then $D/4\leq 0$ where $D/4=\langle x,y\rangle ^2-||x||^2\,||y||^2$. 
\end{proof}
\vspace{2ex}
\begin{cor}
$||x+y||\leq ||x||+||y||$ for $x,y\in {\bm R}$. 
\end{cor}
\vspace{2ex}
\begin{proof}
\begin{align*}
||x+y||^2=\langle x+y,x+y\rangle =&||x||^2+2\langle x,y\rangle +||y||^2\\
\leq &||x||^2+2||x||\,||y||+||y||^2\\
=&(||x||+||y||)^2
\end{align*}
We now have shown that $\langle x,x\rangle ^{1/2}$ indeed satisfies a norm.
\end{proof}
\vspace{2ex}
\begin{defi}
An inner product spcae $H$ is called a Hilbert space if it is complete with respect to the norm which is induced by the inner product. 
\end{defi}
\vspace{2ex}
\begin{thm}
An Hilbert space is a Banach space. 
\end{thm}
\vspace{2ex}
\begin{ex}
$C([0,1])$ is a Banach space and an inner product space but is not an Hilbert space with respect to the norm
\[\langle f,g\rangle =\int ^{1}_{0}f(t)g(t)\,dt\]
Technically speaking, an Hilbert space is a Banach space, not with the inner product, but satisfying the paralleogram law.  
\end{ex}
\vspace{2ex}

