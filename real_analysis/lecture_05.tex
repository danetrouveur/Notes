\section{Lecture 5 (March 18th)}
\begin{recall}
	$\mathfrak{M}$ is a $\sigma $-algebra that contains the measurable subsets of ${\bm R}$. The Lebesgue measure $m:\mathfrak{M}\rightarrow [0,\infty ]$ is defined as $m(E)=m^{*}(E)$ for $E\in \mathfrak{M}$. Note that, importantly, $m$ satisfies countable additivity for mutually disjoint measurable sets.
\end{recall}
\vspace{2ex}
\begin{thm}
(Continuity of measure) 
\begin{itemize}
	\item[(i)] If $E_{n}\in \mathfrak{M}$ satisfies $E_{n}\subset E_{n+1}$ for every $n$, then 
\[m\Big(\bigcup _{n=1}^{\infty }E_{n}\Big)=\lim _{n\rightarrow \infty }m(E_{n})\]  
	\item[(ii)] If $E_{n}\in \mathfrak{M}$ satisfies $E_{n}\supset E_{n+1}$ for every $n$ and $m(E_{1})<\infty $, then 
\[m\Big(\bigcap ^{\infty }_{n=1}E_{n}\Big)=\lim _{n\rightarrow \infty }m(E_{n})\]
\end{itemize}
\end{thm}
\vspace{2ex}
\begin{ex}
Notice how the second remark requires that the first set has a finite measure. Consider $E_{n}=(n,\infty )$, then $E_{n}\supset E_{n+1}$. We note that $m(E_{n})=\infty $ for all $n\in {\bm N}$. However, $\bigcap ^{\infty }_{n=1}E_{n}=\emptyset$ which implies that
\[\infty =\lim _{n\rightarrow \infty }m(E_{n})\ne \Big(\bigcap ^{\infty }_{n=1}E_{n}\Big)=0\]
\end{ex}
\vspace{2ex}
\begin{proof}
For (i), define $B_1=E_1$ and $B_{n}=E_{n}\backslash E_{n-1}$. Then, $\{B_{n}\}$ are mutually disjoint sets such that $\bigcup ^{\infty }_{n=1}B_{n}=\bigcup ^{\infty }_{n=1}E_{n}$ and $\bigcup^{n}_{k=1}B_{k}=E_{n}$. Hence 
\[m\Big(\bigcup ^{\infty }_{n=1}E_{n}\Big)=m\Big(\bigcup ^{\infty }_{n=1}B_{n}\Big)=\sum ^{\infty }_{k=1}m(B_{k})=\lim _{n\rightarrow \infty }\sum ^{n}_{k=1}m(B_{k})=\lim _{n\rightarrow \infty }m\Big(\bigcup ^{n}_{k=1}B_{k}\Big)=\lim _{n\rightarrow \infty }m(E_{n})\]
For (ii), if $A_{n}=E_1\;\backslash\;E_{n}$ then $A_{n}\subset A_{n+1}$ and $\bigcup ^{\infty }_{n=1}A_{n}=E_{1}\;\backslash\; \bigcap ^{\infty }_{n=1}E_{n}$. We can now apply (i) and obtain
\[m\Big(E_1\backslash \bigcap ^{\infty }_{n=1}E_{n}\Big)=\lim _{n\rightarrow \infty }m(E_1\backslash E_{n})=\lim _{n\rightarrow \infty }[m(E_1)-m(E_{n})]\]
as $m(E_{1})<\infty $. Additionally,
\[ m(E_1)-m\Big(\bigcap ^{\infty }_{n=1}E_{n}\Big)=m(E_1)-\lim _{n\rightarrow \infty }m(E_{n})\]
which finishes the proof.
\end{proof}
\vspace{2ex}
\begin{defi}
A set $A$ is almost everywhere (almost sure) in ${\bm R}$ or almost all $x\in {\bm R}$ belongs in $A$ if 
\[m(\{x\in {\bm R} \;|\; x\in A^{c}\})=0\]
In general, for a statement $P$ to hold almost every everywhere on $E$ means that the subset of $E$ on which $P$ does not hold is measure zero. 
\end{defi}
\vspace{2ex}
\begin{ex}
\[f(x)=\begin{cases}
	1\hspace{5ex}[0,1]\cap {\bm Q}\\
	0\hspace{5ex}[0,1]\;\backslash\;{\bm Q}
\end{cases}\]
Two questions are asked: (1) is $f$ continuous on $[0,1]$ almost everywhere? This is completely false.
(2) Is there a continuous function $g$ on $[0,1]$ such that $f=g$ almost everywhere on $[0,1]$? Definitely ($g=0$ on $[0,1]$).
\end{ex}
\vspace{2ex}
\begin{ex}
$f$ and $g$ is continuous on ${\bm R}$. Let $f=g$ almost everywhere on ${\bm R}$. Is $f=g$ on ${\bm R}$? Note that $f-g=h$ is continuous. Note how
\[E=\{x\in {\bm R} \;|\; f(x)\ne g(x)\}=[h^{-1}(\{0\})]^{c}\]
Note how the term in the brackets is closed and the set $E$ is open. As an open set cannot be measure zero, it is the empty set and $f=g$. 
\end{ex}
\vspace{2ex}
\begin{lem}
(Borel-Cantelli lemma) Let $\{E_{n}\}_{n=1}^{\infty }$ be a sequence of measurable sets satisfying $\sum ^{\infty }_{n=1}m(E_{n})<\infty $. Then almost all $x\in {\bm R}$ belongs to at most finitely many $E_{n}$'s. 
\end{lem}
\vspace{2ex}
\begin{recall}
Recall that $x\in \bigcap ^{\infty }_{n=1}\bigcup ^{\infty }_{k=n}A_{k}$ means that $x\in A_{k}$ for infinitely many $k$'s. 
\end{recall}
\vspace{2ex}
\begin{proof}
It suffices to show that $m\Big(\bigcap _{n=1}^{\infty }\bigcup ^{\infty }_{k=n}E_{k}\Big)=0$. Since $\sum ^{\infty }_{n=1}m(E_{n})<\infty $, we have $\lim _{n\rightarrow \infty }\sum ^{\infty }_{k=n}m(E_{k})=0$. Since $m\Big(\bigcup ^{\infty }_{k=n}E_{k}\Big)<\infty $, by continuity of the Lebesgue measure,
\[m\Big(\bigcap ^{\infty }_{n=1}\bigcup ^{\infty }_{k=n}E_{k}\Big)=\lim _{n\rightarrow \infty }m\Big(\bigcup ^{\infty }_{k=n}E_{k}\Big)\leq \lim _{n\rightarrow \infty  }\sum _{k=n}^{\infty }m(E_{k})=0\]
\end{proof}
\vspace{2ex}
\begin{thm}
	(Vitali theorem) If $E\subset {\bm R}$ satisfies $m^{*}(E)>0$ then there is a non-measurable subset of $E$. 
\end{thm}
\vspace{2ex}
