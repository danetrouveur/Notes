\section{Lecture 22 (May 27th)}
Last time, we have learned Gauss' lemma. Today, we will be learning irreducibility tests. 
\newline
\begin{recall}
(Gauss' lemma) Let $R$ be a UFD. For $f(x), g(x)\in R[x]$, 
\[\mathop{\mathrm{cont}}(f\cdot g)=\mathop{\mathrm{cont}}(f)\cdot \mathop{\mathrm{cont}}(g)\]
\end{recall}
\vspace{2ex}
\begin{cor}
(7.26) Let $R$ be a UFD, $Q=\mathop{\mathrm{Frac}}(R)$ and $f(x)\in R[x]$ be a primative polynomial. $f(x)$ is irreducible in $R[x]$ if and only if $f(x)$ is irreducible in $Q[x]$.
\end{cor}
\vspace{2ex}
\begin{proof}
We do the only if part first. Suppose that $f(x)=g(x)h(x)$ in $Q[x]$. Write for $f(x)=g(x)h(x)$ for $g(x),h(x)\in Q[x]$. Write $g(x)=(a/b)g_1(x)$ and $h(x)=(c/d)h_1(x)$ where $a,b,c,d\in R\,\backslash\,\{0\}$, and $g_1(x)$ and $h_1(x)$ are primatives in $R[x]$. Then $bd\cdot f(x)=ac\cdot g_1(x)h_1(x)$.
\end{proof}
\vspace{2ex}
\begin{cor}
(7.27) If $R$ is a UFD, so is $R[x_1,x_2,\ldots ,x_{n}]$. 
\end{cor}
\vspace{2ex}
\begin{ex}
\begin{itemize}
\item[(i)] ${\bm Z}[x_1,\ldots ,x_{n}]$ is a UFD.
\item[(ii)] If $K$ is a field, $K[x_1,\ldots ,x_{n}]$ is a UFD.
\end{itemize}
\end{ex}
\vspace{2ex}
\begin{thm}
(7.22) For a UFD $R$, let $f(x)\in R[x]$. If $f(x)=g(x)\cdot h(x)$ in $Q[x]$, there exists $a,b\in Q$ such that $f(x)=(a\cdot g(x))(b\cdot h(x))$ where $a\cdot g(x)$ and $b\cdot h(x)$ are in $R[x]$. By Gauss' lemma, $u\cdot bd=ac$ for some $u\in R^{\times }$. Then $f(x)=u\cdot g_1(x)\cdot h_1(x)$. Since $f(x)\in R[x]$ is irreducible, either $g_1(x)$ and $h_1(x)$ is in $R^{\times }$. Consequently, either 
\[g(x)=\dfrac{a}{b}g_1(x)\in Q^{\times }\quad\mathrm{or}\quad h(x)=\dfrac{c}{d}h_1(x)\in Q^{\times }\]
\end{thm}
\vspace{2ex}
\begin{ex}
$R={\bm Z}$ and $Q={\bm Q}$. Test the irreducibility of
\[f(x)=\dfrac{4}{3}-\dfrac{6}{5}x^{7}\]
\end{ex}
\vspace{2ex}
{\bf Chapter 7.4}\hspace{2ex}Irreducibility Tests in $R[x]$ Where $R$ is a UFD
\newline
\begin{prop}
(7.29) (The rational root test) Let $R$ be a UFD. Let $f(x)=a_0+a_1x+\ldots +a_{n}x^{n}\in R[x]$ be a polynomial of degree $n$. Let $c=p/q\in {\bm Q}$ be a zero of $f(x)$, where $p,q\in R$ with $(p,q)=1$. Then $p|a_0$ and $q|a_{n}$.  
\end{prop}
\vspace{2ex}
\begin{proof}
Notice that
\[0=f\Big(\dfrac{p}{q}\Big)=a_0+a_1\Big(\dfrac{p}{q}\Big)+\ldots +a_{n}\Big(\dfrac{p}{q}\Big)^{n}\]
and multiplying $q^{n}$, we have
\[0=a_0q^{d}+(a_1pq^{n-1}+\ldots +a_{n}p^{n})\]
carefully examining the expression, we find that $p$ divides $a_0$ and $q$ divides $a_{n}$.
\end{proof}
\vspace{2ex}
\begin{ex}
(7.30, 31, 32) 
\begin{itemize}
\item[(i)] $f(x)=4x^3+2x^2-5x+3$ is irreducible in ${\bm Q}[x]$. The possible roots are
\[\pm \Big(1,3,\dfrac{1}{2},\dfrac{3}{2},\dfrac{1}{4},\dfrac{3}{4}\Big)\]
however, none of these is a zero of $f(x)$. 
\item[(ii)] $g(x)=3-4x+2x^2+4x^3$ is a reducible polynomial. This is bceause $g(-3/2)=0$. 
\item[(iii)] $h(x)=1-3x^2+x^{4}$ has no rational roots. It does not follow that $h(x)$ is irreducible. In fact, $h(x)=(-1+x+x^2)(-1-x+x^2)$, so $h(x)$ is reducible in ${\bm Q}[x]$.  
\end{itemize}
\end{ex}
\vspace{2ex}
\begin{thm}
(7.35) (Eisenstein's criterion) Let $R$ be a UFD and let $f(x)=a_0+a_1x+\ldots +a_{n}x^{n}\in R[x]$ be a polynomial of degree $n$. Let $p\in R$ be an irreducible element. Suppose that
\begin{itemize}
\item[(i)] $p\nmid a_{n}$
\item[(ii)] $p|a_0,a_1,\ldots ,a_{n-1}$
\item[(iii)] $p^2\nmid a_0$
\end{itemize}
Then $f(x)$ cannot be written as a product of two polynomials of positive degrees, and hence it is irreducible in $Q[x]$.
\end{thm}
\vspace{2ex}
\begin{ex}
Let $f(x)=5+10x+x^{4}\in {\bm Z}[x]\subset {\bm Q}[x]$. Take $p=5$ which is a prime number in ${\bm Z}$. $5\nmid 1$, $5|5,10$ and $5^2=25\nmid 5$. Altogether, they imply that $f(x)$ is irreducible in ${\bm Q}[x]$.   
\end{ex}
\vspace{2ex}

