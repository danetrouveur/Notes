\section{Lecture 13 (April 15th)}
Last time we have learned
\begin{thm}
(5.37, 5.38) Let $\phi :R\rightarrow S$ be a ring homomorphism. Then: 
\begin{itemize}
\item[(i)] The induced map $\overline{\phi }:R\;/\;\mathrm{ker}\;\phi \rightarrow S$ is defined by $\overline{\phi }(\overline{r})=\phi (r)$ is a well-defined ring homomorphism
\item[(ii)] $\overline{\phi }$ is injective
\item[(iii)] $\mathrm{Im}\;\overline{\phi }=\mathrm{Im}\;\phi $
\end{itemize}
In particular, there is an isomorphism of rings $\overline{\phi }:R\;/\;\mathrm{ker}\;\phi \rightarrow \mathrm{Im}\;\phi $. 
\end{thm}
\vspace{2ex}
\begin{ex}
\begin{itemize}
\item[(i)] For each $r\in R$ there is an isomorphism
\[R[x]\;/\;(x-r)\rightarrow R\]
\item[(ii)] There is an isomorphism of rings
\[{\bm R}[x]\;/\;(x^2+1)\rightarrow {\bm C}\]
\end{itemize}
\end{ex}
\vspace{2ex}
{\bf Chapter 5.6}\hspace{2ex}The Chinese Remainder Theorem
\\\\
\begin{ex}
Let's examine whether we can solve the following system of congruences.
\[\begin{cases}
x\equiv_3 2\\
x\equiv_7 2\\
x\equiv_8 5
\end{cases}\]
We attempt to generalize this from ${\bm Z}$ to $R$.
\end{ex}
\vspace{2ex}
\begin{defi}
(5.43) Let $R$ be a ring and let $I$ and $J$ be ideals of $R$. The sum ($I+J$) of $I$ and $J$ is defined to be 
\[\{a+b \;|\; a\in I,b\in J\}\]
The product ($IJ$) of $I$ and $J$ is defined to be
\[\Big\{\sum _{i=1}^{n}a_ib_{i} \;|\;a_{i}\in I,b_{i}\in J \Big\}\]
\end{defi}
\vspace{2ex}
\begin{rmk}
\begin{itemize}
\item[(i)] $I+J$ and $IJ$ are ideals of $R$
\item[(ii)] $I+J$ is the smallest ideal containing both $I$ and $J$
\item[(iii)] $IJ\subset I\cap J\subset I\;\mathrm{or}\;J\subset I+J$
\item[(iv)] In $R[x]$, set $I=J=(x)$. Then $IJ=(x^2)\not\subset  (x)=I\cap J$
\item[(v)] In $R[x]$, let $I=(x,2)$. Then $x^2+4$ cannot be written as a product of two elements of $I$. Moreover, $I^2\not\subset I$
\end{itemize}
\end{rmk}
\vspace{2ex}
\begin{ex}
In ${\bm Z}$, 
\begin{itemize}
\item[(i)] $(a)+(b)=(\mathrm{gcd}(a,b)$)
\item[(ii)] $(a)\cap (b)=(\mathrm{lcd}(a,b))$
\item[(iii)] $(a)\cdot (b)=(ab)$
\end{itemize}
\end{ex}
\vspace{2ex}
\begin{thm}
(5.52) (Chinese Remainder Theorem) Let $R$ be a commutative ring, and let $I$ and $J$ be ideals of $R$. If $I+J=R$, then
\[R\;/\;IJ\cong R\;/\;I\times R\;/\;J\]
\end{thm}
\vspace{2ex}
\begin{cor}
(5.53) Let $n_1,\ldots ,n_{r}$ be pairwise relatively prime positive integers. Let $N=n_1n_2\ldots n_{r}$. Then ${\bm Z}/n{\bm Z}$ is isomorphic to ${\bm Z}/n_1{\bm Z}\times {\bm Z}/n_2{\bm Z}\times \ldots \times {\bm Z}/n_{r}{\bm Z}$. 
\end{cor}
\vspace{2ex}
\begin{rmk}
Going back to solving a system of congruences, $(3,7)=(7,8)=(3,8)=1$ implied that there existed a unique solution in modulo $3\cdot  7\cdot 8$. 
\[{\bm Z}/3\cdot 7\cdot 8\cong {\bm Z}/3{\bm Z}\times {\bm Z}/7{\bm Z}\times {\bm Z}/8{\bm Z}\]
\end{rmk}
\vspace{2ex}
\begin{prop}
(5.48) Let $R$ be a ring, let $I$ and $J$ be ideals of $R$. Assume that $I+J=R$. Then the homomorphism $\pi :R\rightarrow R/I\times R/J:r\mapsto (\bar{r},\bar{r})$ is surjective. 
\end{prop}
\vspace{2ex}
\begin{proof}
For a given $(\overline{a},\overline{b})\in R/I\times R/J$, we can find a $x\in R$ such that $\pi (x)=(\bar{a},\bar{b})\in R/I\times R/J$. Let $x-a=i\in I$ and $x-b=j\in J$. 
\end{proof}
\vspace{2ex}

