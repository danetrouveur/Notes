\section{Lecture 14 (April 19th)}
\begin{thm}
(5.57) (Third Isomorphism Theorem) Let $R$ be a ring and let $I$ be an ideal of $R$. Let $J$ be an ideal containing $I$ ($I\subset J$). Then $\bar{J}=J/I$ is an ideal of $R/I$, and all ideals of $R/I$ may be realised in this way. Furthermore, the natural map $R/I\rightarrow R/J$ (given by $\bar{r}\mapsto \bar{r}$) induces an isomorphism 
\[R/I\Big/J/I\,\cong\, R/J\]
\end{thm}
\vspace{2ex}
\begin{proof}
We start by proving that $\bar{J}$ is an ideal.
\begin{itemize} 
\item[(1)] If $J$ containing $I$ is an ideal of $R$, 
\[\bar{J}=J/I=\{\bar{j} \,|\, j\in J\}\triangleleft R/I\]
is an ideal of $R/I$ as it is the image of an ideal under a surjective homomorphism $\pi :R\rightarrow R/I$ .
\item[(2)] There is a bijection between the set of ideals of $R$ containing $I$ and the set of ideals of $R/I$. 
\item[(2-1)] Let $\bar{J}$ be an ideal of $R/I$, that is, $\bar{J}\triangleleft R/I$. This implies that $ I\subset \pi ^{-1}(\bar{J})\triangleleft R$. To show that $I\subset \pi ^{-1}(\bar{J})$, let $i\in I$. We want to verify that $\pi (i)\in J$. Note that $\pi (i)=\bar{i}=\bar{0}\in J$.
\item[(2-2)] We have seen that all ideals containing $I$ have images that are ideals of $R/I$ and the converse. Lastly notice that $J\mapsto \bar{J}$ and $\bar{J}\mapsto \pi ^{-1}(J)$ are inverses of each other. 
\item[(3)] Note that there is a well-defined ring homomorphism $p:R/I\rightarrow R/J:\bar{r}\mapsto \bar{r}$ ($\bar{r}=\bar{s}\in R/I \implies \bar{r}=\bar{s}\in R/J$).
\item[(3-1)] $p$ is surjective.
\item[(3-2)] $r\in \mathrm{ker}\,p\iff p(\bar{r})=\bar{r}=\bar{0}\in R/J\iff r-0=r\in J$. So, $\mathrm{ker}\,p=J/I$.
\item[(3-3)] Using the 1st isomorphism theorem, we see that
\[R/I\Big/ \mathrm{ker}\,p=R/I\Big/ J/I\cong R/J\]

\end{itemize}
\end{proof}
\vspace{2ex}
\begin{ex}
\begin{itemize}
\item[(i)] Consider the ideals of ${\bm Z}/6{\bm Z}$, which are also the ideals of ${\bm Z}=\{(n) \;|\; n\in {\bm Z}\}$ containing $6{\bm Z}=(6)$. They are $\{ (1),(2),(3),(6)\}$, and
\[{\bm Z}/6{\bm Z}=\{\overline{(1)},\overline{(2)},\overline{(3)},\overline{(6)}\}\]
\end{itemize}
\end{ex}
\vspace{2ex}
{\bf Chapter 6}\hspace{2ex}Integral Domains
\\
\begin{rmk}
There are some key properties of ${\bm Z}$, such as
\begin{itemize}
\item[(i)] (Division algorithm) $a=bq+r$ which generalises to Euclidean domains (EDs)
\item[(ii)] (Every ideal is principle) $(n)=I\triangleleft {\bm Z}$ which generalises to principle ideal domains (PIDs)
\item[(iii)] (There exists an unique factorisation) $n=p_1p_2\ldots p_{r}$ which generalises to qnique factorisation domains (UFDs)
\end{itemize}
\end{rmk}
\vspace{2ex}
{\bf Chapter 6.1}\hspace{2ex}Prime and Maximal Ideals
\\
\begin{defi}
(6.4, 6.9) Let $R$ be commutative ring, and $I$ be an ideal of $R$. 
\begin{itemize}
\item[(i)] A prime ideal $I$ satisfies $I\ne R$ ($I$ is proper) and if $a$ and $b$ are in $R$ such that $ab\in I$, then $a\in I$ or $b\in I$. 
\item[(ii)] A maximal ideal $I$ satisfies $I\ne R$ and if $J$ is an ideal containing $I$, then $J=I$ or $J=R$ 
\end{itemize}
\begin{center}
\begin{tikzpicture}
  \node (O) at (0,0) {$0$};
  \node (I) at (2,0) {$I$};
  \node (J) at (4,1) {$J$};
  \node (R) at (6,0) {$R$};

  \draw (O) -- (I);
  \draw (I) -- (J);
  \draw (I) -- (R);
  \draw (J) -- (R);
\end{tikzpicture}
\end{center}
\end{defi}
\vspace{2ex}
\begin{rmk}
\begin{itemize}
\item[(i)] If $p$ is a prime number, then $(p)\subset {\bm Z}$ is a prime ideal.
\item[(ii)] $0\subset {\bm Z}$ is a prime ideal.
\end{itemize}
\end{rmk}
\vspace{2ex}
\begin{prop}
(6.3) Every maximal ideal is a prime ideal.
\end{prop}
\vspace{2ex}
\begin{proof}
Let $I$ be a maximal ideal of $R$. Since $I$ is maximal, $I\ne R$. Suppose now that $ab\in I$ but $a$ is not in $I$. In this case, $I+(a)$ is an ideal properly containing $I$, so that $I+(a)=R$ by the maximality of $I$. In particular, $1=i+ca$ for some $c\in R$. Then $b=b\cdot 1=b(i+ca)=bi+cab\in I$, because $ab\in I$. 
\end{proof}
\vspace{2ex}
\begin{thm}
(6.1) Let $R$ be a commutative ring, and $I$ be an ideal of $R$. Then $I$ is a prime ideal if and only if $R/I$ is an integral domain.
\end{thm}
\vspace{2ex}
\begin{thm}
(6.2) Let $R$ be a commutative ring, and $I$ be an ideal of $R$. Then $I$ is a maximal ideal if and only if $R/I$ is a field.
\end{thm}
\vspace{2ex}


