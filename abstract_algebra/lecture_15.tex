\section{Lecture 15 (May 6th)}
{\bf Chapter 6.2}\hspace{2ex}Primes and Irreducibles
\\
\begin{defi}
(6.15) Let $R$ be a commutative ring. Let $a$ and $b$ be elements of $R$. We say that $b$ divides $a$ and write $b|a$ if $a\in (b)$.
\end{defi}
\vspace{2ex}
\begin{defi}
(6.18) Let $R$ be an integral domain. Then $p\in R$ is a prime element if it is nonzero, nonunit, and if $p|ab$, then $p|a$ or $p|b$. 
\end{defi}
\vspace{2ex}
\begin{ex}
In ${\bm Z}$, a prime number is a prime element. 
\end{ex}
\vspace{2ex}
\begin{lem}
Let $R$ be an integral domain and let $p$ be a nonzero element of $R$. Then $p$ is a prime element if and only if $(p)$ is a prime ideal of $R$. 
\end{lem}
\vspace{2ex}
\begin{defi}
(6.20) Let $R$ be an integral domain and let $q$ be an element of $R$. We say that $q$ is irreducible if it is nonunit and if $q=ab$, then $a$ or $b$ is a unit.
\end{defi}
\vspace{2ex}
\begin{rmk}
An irreducible element is nonzero. 
\end{rmk}
\vspace{2ex}
\begin{thm}
(6.23) Let $R$ be an integral domain. Then every prime element is irreducible. 
\end{thm}
\vspace{2ex}
\begin{proof}
Let $p\in R$ be a prime element. Note that $p$ is not a unit. Assume that $p=ab$. Then, $p|a$ or $p|b$ because $p|ab$. If $p$ divides $a$, then $a=pc$ for some $c\in R$. Therefore, $p=ab=pcb$. This implies that $b$ is a unit. 
\end{proof}
\vspace{2ex}
\begin{ex}
(6.21) In ${\bm C}[x,y]/(y^2-x^3)$, the element $\bar{x}$ is irreducible, but not prime. 
\end{ex}
\vspace{2ex}
{\bf Chapter 6.3}\hspace{2ex}Euclidean Domains and Principle Ideal Domains 
\\
\begin{defi}
(6.24) Let $R$ be an integral domain. We say that $R$ is a Euclidean domain if there exists a function
\[\nu :R\;\backslash\;\{0\}\rightarrow {\bm Z}^{\geq 0}\]
satisfying the following property: for every $a\in R$ and every nonzero $b\in R$, there exists $q$ and $r$ in $R$ such that
\[a=bq+r\]
where $r=0$ or $\nu (r)<\nu (b)$. The function $\nu $ is called a Euclidean valuation.
\end{defi}
\vspace{2ex}
\begin{ex}
\begin{itemize}
\item[(i)] Consider $({\bm Z},|\cdot |:{\bm Z}\;\backslash\;\{0\}\rightarrow {\bm Z}^{\geq 0})$. This implies that ${\bm Z}$ is an Euclidean domain.
\item[(ii)] Let $K$ be a field. Consider $(K[x],\mathrm{deg}:K[x]\;\backslash\;\{0\}\rightarrow {\bm Z}^{\geq 0})$. This implies that $K[x]$ is a Euclidean domain.
\end{itemize}
\end{ex}
\vspace{2ex}
\begin{defi}
(6.25) Let $R$ be an integral domain. We say that $R$ is a principle ideal domain or PID, if every ideal in $R$ is principle. That is, if $I$ is an ideal of $R$, then $I=(r)$ for some $r\in R$.
\end{defi}
\vspace{2ex}
\begin{thm}
(6.26) Every Euclidean domain is a PID.
\end{thm}
\vspace{2ex}
\begin{proof}
Let $R$ be a ED. Let $I$ be an ideal of $R$. If $I=0$, then $I=(0)$. Otherwise, consider the set $\{\nu (a):a\in I\,\backslash\,\{0\}\}$. Since this set is nonempty, it contains the smallest element by the well-ordering principle $b$ for ${\bm Z}^{\geq 0}$. We claim that $I=(b)$. Since $b\in I$, $(b)\subset I$. So it suffices to show that $I\subset (b)$. Let $a\in I$. Then $a=bq+r$, where $r=0$ or $\nu (r)<\nu (b)$. If $\nu (r)<\nu (b)$, then $r\in I\,\backslash\,\{0\}$. This contradicts to the fact that $b$ is the smallest element of the set $\{\nu (a) \,|\, a\in I\,\backslash\,\{0\}\}$. Therefore, $r=0$, and $a=bq\in (b)$. 
\end{proof}
\vspace{2ex}
\begin{ex}
(6.27, 28)
\begin{itemize}
\item[(i)] ${\bm Z}$ is a PID
\item[(ii)] Every field is a PID (the ideals of $K$ are $0=(0)$ and $K=(1)$)
\item[(iii)] If $K$ is a field, then $K[x]$ is a ED, hence  a PID
\item[(iv)] ${\bm Z}[x]$ is not a PID. The ideal $(2,x)$ is not principle. 
\end{itemize}
\end{ex}
\vspace{2ex}
\begin{prop}
(6.30) Let $R$ be a PID. Let $I$ be a nonzero prime ideal. Then $I$ is a maximal ideal. 
\end{prop}
\vspace{2ex}
\begin{proof}
Suppose we are given an ideal $J$ such that $I\subset J\subset R$. Since $R$ is a PID, we can set $I=(a)$ and $J=(b)$. Then $a=bc$ for some $c\in R$. Note that $a\ne 0$. Since $I=(a)$ is a prime ideal, $a$ is a prime element. Therefore, $a|b$ or $a|c$. If $a|b$, then $I=J$ and if $a|c$, $b=(1)=R$. 
\end{proof}
\vspace{2ex}
\begin{prop}
(6.32) Let $R$ be a PID. Then it satisfies the ascending chain condition on principle ideals: if $I_1\subset I_2\subset \ldots $ is a chain of ideals in $R$, then there is an index $m$ such that $I_{m}=I_{m+1}=I_{m+2}=\ldots $ and so on.
\end{prop}
\vspace{2ex}
\begin{proof}
Let $I$ be the union of the ideals $I_{k}$ (verify that $I$ is an ideal of $R$). Since $R$ is a PID, $I=(r)$ for some $r\in R$. In particular, $r\in I_{m}$ for some $m\geq 1$. This implies that $I_{m}=I_{m+i}$ for every $i\geq 0$. 
\[I_{m}\subset I=\bigcup I_{k}=(r)\subset I_{m}\]
\end{proof}
\vspace{2ex}
{\bf Chapter 6.4}\hspace{2ex}Principle Ideal Domains and Unique Factorisation Domains 
\\
\begin{defi}
(6.37) Let $R$ be an integral domain. Then $R$ is a unique factorization domain or UFD if it satisfies the following condition: suppose that $a\in R$ is nonzero and nonunit. Then there exists finitely many irreducible elements $q_1,q_2,\ldots ,q_{r}$ such that $a=q_1q_2\ldots q_{r}$. Moreover, this factorisation is unique in the senes that if 
\[a=q_1q_2\ldots q_{r}=p_1p_2\ldots p_{s}\]
where all $p_{i}$ and $q_{j}$ are irreducible, then $s=r$ and after reordering factors we have $(p_{i})=(q_{i})$ for all $i$. 
\end{defi}
\vspace{2ex}
\begin{prop}
(6.33) Let $R$ be a PID and let $q\in R$ be an irreducible element. Then $(q)$ is a maximal ideal.
\end{prop}
\vspace{2ex}
\begin{proof}
Since $q$ is nonunit, $(q)\ne R$. Let $I$ be an ideal of $R$ such that $(q)\subset I\subset R$. Since $R$ is a PID, $I=(a)$ for some $a\in R$. Then $q=ab$ for some $b\in R$. 
\end{proof}
\vspace{2ex}
\begin{lem}
Let $R$ be a PID. Let $a$ be a nonzero and nonunit element of $R$. If $(a)$ is maximal, then $a$ is irreducible. 
\end{lem}
\vspace{2ex}
\begin{proof}
Let $a=bc$. Then $(a)\subset (b)\subset R$. As $(a)$ is maximal, $(b)=(1)=R$ or $(b)=(a)$ and either $b$ or $c$ are units. 
\end{proof}
\vspace{2ex}
\begin{thm}
(6.35) Every PID is a UFD.
\end{thm}
\vspace{2ex}
\begin{proof}
Let $R$ be a PID. We make the following observation: let $a$ be a non-zero and non-unit element of $R$. If $a$ is not irreducible, then there exists an irreducible element $q\in R$ such that $q|a$. By lemma, $(a)$ is not maximal. That is, we can find $(a)\subsetneq (a_{1})\subsetneq R$. If $(a_1)$ is maximal, then $a_1$  is irreducible by lemma. Otherwise, we can find $(a_1)\subsetneq (a_2)\subsetneq R $. 
\\\\
For existence, let $a$ be a nonzero nonunit element of $R$. If $a$ is irreducible, then we're done. Otherwise, we can find an irreducible factor $q_1$ of a by observation, so that $a=q_1a_1$. Note that $a_1$ is nonzero and nonunit (in particular, $(a)\subsetneq (a_1)\subsetneq R$). If $a_1$ is irreducible, then we're done. Otherwise, we can find an irreducible factor $q_2$, so that
\[a_1=q_2a_2\]
Try uniqueness on your own.
\end{proof}
\vspace{2ex}
\begin{rmk}
(6.39) Not every UFD is a PID. For example, ${\bm Z}[x]$ and ${\bm C}[x,y]$.
\[\mathrm{ED}\subset \mathrm{PID}\subset \mathrm{UFD}\subset \mathrm{ID}\]
\end{rmk}
\vspace{2ex}

