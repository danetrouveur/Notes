\section{Lecture 18 (May 13th)}
\begin{rmk}
Last time, we learned about prime and maximal ideals. Today, we will be learning some basics on polynomials.
\end{rmk}
\vspace{2ex}
{\bf Chapter 7}\hspace{2ex}Polynomial Rings and Factorization
\\\\
{\bf Chapter 7.2}\hspace{2ex}The Polynomial Rings with Coefficients in a Field
\\
\begin{defi}
(7.1) Let $R$ be a ring. Let $f(x)=a_0+a_1x+a_2x^2+\ldots +a_{n}x^{n}$ be a nonzero polynomial with $a_{n}\ne 0$. Then the integer $n$ is called the degree of $f$; $a_{n}$ is the leading coefficient of $f$; $a_0$ is the constant term of $f$.  
\end{defi}
\vspace{2ex}
\begin{thm}
(7.2) Let $K$ be a field. Then $K[x]$ is a Euclidean domain. 
\end{thm}
\vspace{2ex}
\begin{prop}
(7.3) Let $R$ be a ring and let $f(x)$ and $g(x)$ be elements of $R[x]$	with $g(x)\ne 0$. Suppose that the leading coefficient of $g(x)$ is a unit. Then there exists $q(x)$ and $r(x)$ in $R[x]$ such that $f(x)=g(x)\cdot q(x)+r(x)$, where $r(x)=0$ or $\mathrm{deg}\,r(x)<\mathrm{deg}\,g(x)$.
\end{prop}
\vspace{2ex}
\begin{proof}
The proof is omitted for now. Loosely speaking, one we set the minimum order $n$ for $f(x)$ such that the division algorithm doesn't work, we find that it doesn't work for $n-1$ either, which is a contradiction.
\end{proof}
\vspace{2ex}
\begin{ex}
(7.5) $R$ is a ring, $f(x)\in R[x]$. Then $(x-a)$ is a factor of $f(x)$ if and only if $f(a)=0$. For the if direction, write $f(x)=(x-a)\cdot q(x)+r(x)$. Then $r(x)$ must be $0$. 
\end{ex}
\vspace{2ex}
\begin{defi}
(7.6) Let $R$ be a ring and $f(x)\in R[x]$ be a polynomial. We will say that $a\in R$ is a zero of $f(x)$ if $f(a)=0$.
\end{defi}
\vspace{2ex}
\begin{cor}
(7.7) If $K$ is a field, then $K[x]$ is a PID. 
\end{cor}
\vspace{2ex}
\begin{cor}
(7.8) If $K$ is a field, then $K[x]$ is a UFD. 
\end{cor}
\vspace{2ex}
\begin{prop}
(7.9) Let $R$ be an integral domain. Let $f(x)$ and $g(x)$ be polynomials over $R[x]$ with $g(x)\ne 0$. Suppose that the leading coefficient of $g(x)$ is a unit. Then the quotient $q(x)$ and the remainder $r(x)$ are unique.
\end{prop}
\vspace{2ex}
\begin{proof}
See how
\[g_1(x)-g_2(x)=(q_1(x)-q_2(x))f(x)+(r_1(x)-r_2(x))\]
If $(r_1(x)-r_2(x))\ne 0$, we find that neither of the left terms are zero and as $\mathop{\mathrm{deg}}f(x)>\mathop{\mathrm{deg}}r(x)$ the statement is a contradiction. 
\end{proof}
\vspace{2ex}
\begin{thm}
(7.11) Let $K$ be a field and let $f(x)$ be a nonzero polynomial of degree $n$. Then $K[x]\,/\,(f(x))$ is an $n$-dimensional vector space over $K$. 
\end{thm}
\vspace{2ex}
\begin{proof}
We briefly sketch a proof. First note how
\[a\in K,\ \overline{g(x)}\in K[x]\,/\,(f(x))\implies a\cdot \overline{g(x)}=\overline{a\cdot g(x)}\]
and that $K[x]\,/\,(f(x))$ is closed under scalar multiplication. We further note how the subset $\{\overline{x^{0}},\overline{x^{1}},\overline{x^2},\ldots ,\overline{x^{n-1}} \}\subset K[x]\,/\,(f(x))$ forms a basis for the $K$-vector space $K[x]\,/\,(f(x))$. See that for any $\overline{g(x)}\in K[x]\,/\,(f(x))$, 
\[\overline{g(x)}=\overline{f(x)}\cdot \overline{q(x)}+\overline{r(x)}=\overline{r(x)}=\overline{b_0}\cdot \overline{1}+\ldots +\overline{b_{n-1}}\cdot \overline{x^{n-1}}\]
and that a polynomial can be expressed in terms of the basis. 
\end{proof}
\vspace{2ex}

