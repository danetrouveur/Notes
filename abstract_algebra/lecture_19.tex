\section{Lecture 19 (May 15th)}
Last time we have learned some definitions about polynomials. Today, we will learn about irreducible polynomials.\\
\newline
{\bf Chapter 7.3}\hspace{2ex}Irreducibility in Polynomial Rings 
\newline
\begin{ex}
(7.13) If $K$ is a field and $a$ is nonzero, then $ax+b$ is irreducible. Note that, first of all, it is a non-zero non-unit element ($ax+b\notin (K[x])^{\times }=K^{\times }=K$). Secondly, if $ax+b=f(x)\cdot g(x)$, we observe that $f(x)$ or $g(x)$ is in $K$, which would make it a unit. 
\end{ex}
\vspace{2ex}
\begin{rmk}
We make two short remarks. Let $R$ be an integral domain. 
\begin{itemize}
\item[(i)] $f(x),g(x)\in R[x]$ implies that $\mathop{\mathrm{deg}}(f(x)\cdot g(x))=\mathop{\mathrm{deg}}f(x)+\mathop{\mathrm{deg}}g(x)$
\item[(ii)] $(R[x])^{\times }=R^{\times }$
\end{itemize}
\end{rmk}
\vspace{2ex}
\begin{rmk}
The condition that $K$ is a field is essential. Note how $2x\in {\bm Z}[x]$ is not irreducible (it is reducible) as it can be expressed as $2\cdot x$ with both $2$ and $x$ being non-unit non-zero elements.
\end{rmk}
\vspace{2ex}
\begin{prop}
(7.14) Let $K$ be a field and let $f(x)\in K[x]$ be a polynomial of degree at least 2 ($\mathop{\mathrm{deg}}f(x)\geq 2$). If $f(x)$ has a zero, then $f(x)$ is reducible. 
\end{prop}
\vspace{2ex}
\begin{proof}
Assume that $f(a)=0$ for some $a\in K$. We can write $f(x)=(x-a)g(x)$, where $\mathop{\mathrm{deg}}g(x)\geq 1$ ($\mathop{\mathrm{deg}}f(x)=\mathop{\mathrm{deg}}(x-a)+\mathop{\mathrm{deg}}g(x)$). $g(x)$ is not a unit and $(x-a)$ isn't either. Therefore, $f(x)$ is reducible. The converse isn't true, with examples such as $(x^2+1)^2\in {\bm R}[x]$.
\end{proof}
\vspace{2ex}
\begin{prop}
(7.15) Let $K$ be a field and let $f(x)\in K[x]$ be a polynomial of degree 2 or 3. Then $f(x)$ is irreducible if and only if $f(x)$ has no zeros in $K$. 
\end{prop}
\vspace{2ex}
\begin{proof}
Thanks to the above proposition, it suffices to prove that if $f(x)$ has no zeros, $f(x)$ is irreducible. We prove by contraposition, that if $f(x)$ is reducible (it is trivially non-zero and non-unit), $f(x)$ has zeros in $K$. As $f(x)$ is reducible
\[f(x)=g(x)\cdot h(x)\]
for non-unit $g(x)$ and $h(x)$. Then, the degree of $g(x)$ and $h(x)$ are atleast $1$, telling us that either one of them has to be 1 due to the fact that
\[\mathop{\mathrm{deg}}f(x)=\mathop{\mathrm{deg}}g(x)+\mathop{\mathrm{deg}}h(x)\]
Now, as $f(x)$ contains a linear factor $(ax+b)$ with $a\ne 0$, $f(x)$ has a zero. 
\end{proof}
\vspace{2ex}
\begin{rmk}
(7.16) 
\begin{itemize}
\item[(i)] The fact that the ring of choice is a field is critical. For example, $(2x-1)^2\in {\bm Z}[x]$ is reducible, but has no zero in ${\bm Z}$. 
\item[(ii)] We can't extend the argument to higher degrees. Take, for example, $x^{4}-4=(x^2+2)(x^2-2)\in {\bm Q}[x]$. The polynomial is reducible and has no zeros in ${\bm Q}$. 
\end{itemize}
\end{rmk}
\vspace{2ex}
\begin{thm}
(7.17) (Fundamental theorem of algebra) Every non-constant polynomial over ${\bm C}$ has a zero in ${\bm C}$. 
\end{thm}
\vspace{2ex}
\begin{cor}
A polynomial $f(x)\in {\bm C}[x]$ is irreducible if and only if its $\mathop{\mathrm{deg}}f(x)=1$. 
\end{cor}
\vspace{2ex}
\begin{prop}
(7.20) Let $f(x)\in {\bm R}[x]$ be a polynomial. Then $f(x)$ is irreducible if and only if 
\begin{itemize}
\item[(i)] $\mathop{\mathrm{deg}}f(x)=1$
\item[(ii)] $\mathop{\mathrm{deg}}f(x)=2$ and $f(x)=ax^2+bx+c$ with $b^2-4ac<0$
\end{itemize}
\end{prop}
\vspace{2ex}
\begin{proof}
The if part should be trivial. We prove the only if part. For degrees higher than 3, Let $f(x)=c(x-a_1)(x-a_2)\ldots (x-a_{n})\in {\bm C}[x]$ and $f(x)\in{\bm R}[x]\subset {\bm C}[x]$. From here, if $a_{i}\in {\bm C}$, $f(a_{i})=0$ implies that $f(\bar{a}_{i})=0$. With this knowledge, $(x-a_{i})(x-\bar{a}_{i})$ is a polynomial with real coefficients that divides $f(x)$.
\end{proof}
\vspace{2ex}

