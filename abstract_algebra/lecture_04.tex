\section{Lecture 4 (March 13th)}

Last class, we have learned some properties of ${\bm Z}/n{\bm Z}$. Today, we will learn about rings.
\\
\begin{prop}
(2.16) Let $n$ be an integer greater than 1. Then $\bar{a} \in {\bm Z}/n{\bm Z}$ has a multiplicative inverse if and only if $(a,n)=1$.
\end{prop}
\vspace{2ex}
\begin{proof}
We know the following
\[1=(a,n)=ax+ny\]
for some $x,y\in {\bm Z}$. 
\end{proof}
\vspace{2ex}
\begin{thm}
(2.18) (Fermat's little theorem) Let $p$ be a prime number and let $a$ be an integer. Then $\bar{a}^{p}=\bar{a}$. In fact, $\bar{a}^{p-1}=\bar{1}$ for $\bar{a}\ne \bar{0}$. The proof is up to you.
\end{thm}
\vspace{2ex}
{\bf Chapter 3}\hspace{2ex}Rings
\\\\
{\bf Chapter 3.1}\hspace{2ex}Definition \& Examples
\\
\begin{defi}
 (3.1) A ring is a set $R$ equipped with two binary operations (a function $R\times R\rightarrow R$), an addition $+$ and a multiplication $\cdot $ which satisfies the following.
 \begin{itemize}
	 \item[(i)] $(a+b)+c=a+(b+c)$
	 \item[(ii)] There exists an element $0\in R$ such that for every $a\in R$, $a+0=a=0+a$
	 \item[(iii)] For each $a$, there exists an $a'$ such that $a+a'=0=a'+a$
	 \item[(iv)] $a+b=b+a$
	 \item[(v)] $(ab)c=a(bc)$
	 \item[(vi)] There exists an element $1\in R$ such that for all $a\in R$, $a\cdot 1=a=1\cdot a$
	 \item[(vii)] $a(b+c)=a\cdot b+a\cdot c$
	 \item[(viii)] $(a+b)c=ac+bc$
 \end{itemize}
\end{defi}
\vspace{2ex}
\begin{ex}
Some examples of groups are
\begin{itemize}
	\item[(i)] ${\bm Z}, {\bm Q},{\bm R},{\bm C}$ are rings
	\item[(ii)] ${\bm Z}/n{\bm Z}$ is a ring
	\item[(iii)] $5{\bm Z}$ is not a ring as it has no multiplicative identity
	\item[(iv)] ${\bm Z}^{\geq 0}=\{m\in {\bm Z} \;|\; m\geq 0\}$ is not a ring
	\item[(v)] $(M_{n\times n}({\bm R}),+,\cdot )$
	\item[(vi)] ${\bm R}[x]$
\end{itemize}
\end{ex}
\vspace{2ex}
{\bf Chapter 3.2}\hspace{2ex} Basic Properties
\\
\begin{prop}
(3.14) The additive and multiplicative identities are unique.
\end{prop}
\vspace{2ex}
\begin{proof}
Suppose there $O$ and $O'$ are two additive identities. Then,
\[O=O+O'=O'\]
\end{proof}
\vspace{2ex}
\begin{prop}
 (3.15) The additive inverse is unique.
\end{prop}
\vspace{2ex}
\begin{proof}
Let $a$ be an element of $R$. Assume that both $b$ and $c$ are additive inverses of $a$.
\[c=O+c=(b+a)+c=b+O=b\]
\end{proof}
\vspace{2ex}
\begin{rmk}
(Notation)
\begin{itemize}
	\item[(i)] $a\cdot b=ab$
	\item[(ii)] $a+a+\ldots +a=na$ and $a\cdot a\cdot \ldots \cdot a=a^{n}$
	\item[(iii)] $a^{0}=1$ by convention
	\item[(iv)] For $n>0$, $(-n)a=(-a)+\ldots +(-a)$
\end{itemize}
\end{rmk}
\vspace{2ex}
\begin{prop}
(3.17) Let $R$ be a ring. If $a+c=b+c$, then $a=b$. 
\end{prop}
\vspace{2ex}

