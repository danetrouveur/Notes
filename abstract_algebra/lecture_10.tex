\section{Lecture 10 (April 3rd)}
Last time, we learned homomorphisms and kernals. Today, we will learn ideals and quotient rings.
\\
\begin{recall}
Let $\phi :R\rightarrow S$ be a ring homomorphism. 
\end{recall}
\vspace{2ex}
\begin{prop}
(5.1) $\mathrm{Im}\;\phi =\phi (R)\subset S$ is a subring.
\end{prop}
\vspace{2ex}
\begin{rmk}
(5.2) If $R'\subset R$ is a subring, then $\phi (R')\subset S$ is also a subring.
\end{rmk}
\vspace{2ex}
\begin{defi}
(5.3) $\mathrm{Ker}\;\phi =\phi ^{-1}(\{0\})=\{r\in R \;|\; \phi (r)=0\}$ is called the kernal of $\phi $. 
\end{defi}
\vspace{2ex}
\begin{prop}
(5.6) $(\mathrm{Ker}\;\phi ,+)$, closed under addition, satisfies the ring properties (i) through (iv). That is, it is a abelian group. 
\end{prop}
\vspace{2ex}
\begin{prop}
(5.17) For all $a\in \mathrm{Ker}\;\phi $ and all $r\in R$ both $ra$ and $ar$ belong to $\mathrm{Ker}\;\phi $. 
\end{prop}
\vspace{2ex}
\begin{defi}
(5.8) Let $R$ be a ring and $I$ be a subset of $R$. $I$ is an ideal if it is:
\begin{itemize}
\item[(i)] Closed under addition
\item[(ii)] The additive identity is in $I$ ($0\in I$)
\item[(iii)] (Absorption property) For all $a\in I$ and $r\in R$, $ar$ and $ra$ are in $I$
\end{itemize}
\end{defi}
\vspace{2ex}
\begin{rmk}
\begin{itemize}
\item[(i)] If $a\in I$, then $(-1)\cdot a=-a\in I$
\item[(ii)] If $I$ is nonempty, then the condition that $0\in I$ is redundant for $I$ to be an ideal
\end{itemize}
\end{rmk}
\vspace{2ex}
\begin{ex}
(5.10 -- 15)
\begin{itemize}
\item[(i)] $\mathrm{Ker}\;\phi $ is an ideal of $R$
\item[(ii)] $0$ and $R$ are ideals of $R$
\item[(iii)] ${\bm Z}\subset {\bm Q}$ is not an ideal
\item[(iv)] $m{\bm Z}\subset {\bm Z}$ is an ideal for all $m\in {\bm Z}$
\item[(v)] The set of all polynomials $f(x,y)$ in ${\bm C}[x,y]$ that have no constant term is an ideal. 
\end{itemize}
\end{ex}
\vspace{2ex}
\begin{prop}
(5.16) Let $R$ be a commutative ring and let $r\in R$ be an element. Then the subset
\[(a)=\{ra \;|\; r\in R\}\]
is an ideal of $R$. 
\end{prop}
\vspace{2ex}
\begin{defi}
(5.17) Let $R$ be a commutative ring and let $a\in R$ be an element. We say that $(a)$ is a principle ideal generated by $a$. 
\end{defi}
\vspace{2ex}
\begin{rmk}
Let $a_1,a_2,\ldots ,a_{n}$ be elements of a commutative ring $R$. Then the subset $(a_1,\ldots ,a_{r})=\{r_1a_1+r_2a_2+\ldots +r_{n}a_{n} \;|\;r_i\in R \}$ is an ideal and called the ideal generated by $a_1,\ldots ,a_{n}$.  
\end{rmk}
\vspace{2ex}
{\bf Chapter 5.3}\hspace{2ex}Quotient Rings
\\\\
The following diagram shows what we are trying to do.
\begin{align*}
{\bm Z}\hspace{3ex}&\longleftrightarrow \hspace{3ex}R\\
n{\bm Z}\hspace{3ex}&\longleftrightarrow\hspace{3ex} I\\
{\bm Z}/n{\bm Z}\hspace{3ex}&\longleftrightarrow\hspace{3ex} R/I
\end{align*}
Alike how we partitioned the integers using the relationship of multiples of integers, we are going to partition a ring using the relation that elements are in identical ideals.
\\
\begin{defi}
(5.19) Let $R$ be a ring and $I$ be an ideal of $R$. We define a relation $\sim_{I}$ on $R$ by declaring that $a\sim_{I}b$ if and only if $b-a\in I$. We say that $a$ is congruent to $b$ modulo the ideal $I$. 
\end{defi}
\vspace{2ex}
\begin{prop}
(5.20) The relation $\sim_{I}$ is an equivalence relation. 
\end{prop}
\vspace{2ex}
\begin{rmk}
Let $R={\bm Z}$ and $I=n{\bm Z}$ then 
\[a\sim_{I}b\hspace{3ex}\iff \hspace{3ex}a \equiv_{n}b\]
\end{rmk}
\vspace{2ex}
\begin{rmk}
For each $a\in R$, the equivalence class $a$ can be described as follows: 
\[[a]=\{r\in R\;|\; a\sim_{I}r\}=\{a+i \;|\; i\in I\}=a+I\]
For example,
\[\bar{2}=2+5{\bm Z}\]
\end{rmk}
\vspace{2ex}
\begin{defi}
(5.22) We call $\bar{a}$ the coset of $a$ modulo $I$. We will denote by $R/I$ the set of all cosets and call it the quotient of $R$ modulo $I$. 
\end{defi}
\vspace{2ex}

