\section{Lecture 16 (May 17th)}
{\bf Chapter 6.5}\hspace{2ex}The Field of Fractions of an Integral Domain
\\\\
\begin{rmk}
In this chapter, we try to generalise the construction of the field of integers ${\bm Q}$ of rational numbers from the ring ${\bm Z}$ of integers.
\end{rmk}
\vspace{2ex}
\begin{lem}
(6.42) Let $R$ be an integral domain. Let $\hat{F}$ be the set defined as follows: 
\[\{(n,d)\in R\times R\,|\,d\ne 0\}\]
We define the relation on $\hat{F}$ as follows:
\[(n_1,d_1)\sim (n_2,d_2)\iff n_1d_2=n_2d_1\]
The relation $\sim $ on $\hat{F}$ is an equivalence relation. 
\end{lem}
\vspace{2ex}
\begin{proof}
We prove reflexivity, symmetry, and transistivity.
\begin{itemize}
\item[(i)] (Reflexivity) $(r,s)\sim (r,s)$
\item[(ii)] (Symmetry) $(r_1,s_1)\sim (r_2,s_2)$ implies that $(r_2,s_2)\sim (r_1,s_1)$
\item[(iii)] (Transistivity) $(r_1,s_1)\sim (r_2,s_2)$ and $(r_2,s_2)\sim (r_3,s_3)\implies (r_1,s_2)\sim (r_3,s_3)$. For transistivity, we want to show that $r_1s_3=r_3s_1$. Note that
\begin{align*}
(r_1s_3)s_2=&r_1(s_3s_2)=r_1(s_2s_3)=r_1(s_2s_3)=(r_1s_2)s_3=(r_2s_1)s_3\\=&r_2(s_1s_3)=r_2(s_3s_1)=(r_2s_3)s_1=(r_3s_2)s_1=r_3(s_1s_2)=(r_3s_1)s_2
\end{align*}
Since $s_2$ is nonzero and $R$ is an integral domain, $r_1s_2=r_3s_1$.
\end{itemize}
\end{proof}
\vspace{2ex}
\begin{defi}
Let $R$ be an integral domain. The field of fractions of $R$ is the quotient $F=\hat{F}/\sim$ endowed with the following addition and multiplication. 
\[\overline{(r_1,s_1)}+\overline{(r_2,s_2)}=\overline{(r_1s_2+r_2s_1,s_1s_2)}\]
and
\[\overline{(r_1,s_1)}\cdot \overline{(r_2,s_2)}=\overline{(r_1r_2,s_1s_2)}\]
\end{defi}
\vspace{2ex}
\begin{rmk}
The addition and multiplication are well-defined! In other words,
$\overline{(r_1,s_1)}=\overline{(r_1',s_1')}$ and $\overline{(r_2,s_2)}=\overline{(r_2',s_2')}$ implies that $\overline{(r_1r_2,s_1s_2)}=\overline{(r_1'r_2',s_1's_2')}$. We first do multiplication. Suppose that $(r_1,s_1)\sim (r_1',s_1')$ and $(r_2,s_2)\sim (r_2',s_2')$. We wish to show that $(r_1r_2,s_1s_2)\sim (r_1'r_2',s_1's_2')$. We see that
\begin{align*}
(r_1r_2)(s_1's_2')=&(r_1s_1')(r_2s_2')\\
=&(r_1's_1)(r_2s_2')\\
=&(r_1's_1)(r_2's_2)\\
=&(r_1'r_2')(s_1s_2)
\end{align*}
\end{rmk}
\vspace{2ex}
\begin{prop}
$(F,+,\cdot )$ forms a field. For the proof, we need to verify the following:
\begin{itemize}
\item[(i)] $(F,+,\cdot )$ forms a ring
\item[(ii)] The multiplcation on $F$ is commutative
\item[(iii)] $(F,+,\cdot )$ is nontrivial
\item[(iv)] If $\overline{(r,s)}$ is nonzero in $F$, it has a multiplicative inverse.
\end{itemize}
\end{prop}
\vspace{2ex}
\begin{proof}
For (ii), we see that  
\[\overline{(r_1,s_1)}\cdot \overline{(r_2,s_2)}=\overline{(r_1r_2,s_1s_2)}=\overline{(r_2r_1,s_2s_1)}=\overline{(r_2,s_2)}\cdot \overline{(r_1,s_1)}\]
For (iii), we claim that $\overline{(0,1)}\ne \overline{(1,1)}$. Suppose not. Then $0\cdot 1=1\cdot 1$ which is a contradiction because $R$ is an integral domain.\\\\
For (iv), assume that $\overline{(r,s)}\ne \overline{(0,1)}$ in $F$. Then $r\ne 0$ and $\overline{(s,r)}$ is a multiplcative inverse of $\overline{(r,s)}$. Indeed, 
\[
\overline{(r,s)}\cdot \overline{(s,r)}=\overline{(s,r)}\cdot \overline{(r,s)}=\overline{(sr,rs)}=\overline{(rs,sr)}=\overline{(1,1)}\]
\end{proof}
\vspace{2ex}
\begin{rmk}
\begin{itemize}
\item[(i)] Just like there is an injective ring homomorphism ${\bm Z}\rightarrow {\bm Q}$ by sending $n$ to $n/1$, there exists an injective ring homomorphism 
\[i:R\rightarrow F:r\mapsto \overline{(r,1)}\] 
Where (1) $i$ is a ring homomorphism, and (2) we have
\begin{align*}
r\in \mathop{\mathrm{ker}}i\iff & \overline{(r,1)}=\overline{(0,1)}\\
\iff & r\cdot 1=0\cdot 1
\end{align*}
in $R$.
\end{itemize}
\end{rmk}
\vspace{2ex}
\begin{thm}
(6.45) Let $R$ be an integral domain. Let $F$ be the field of fractions of $R$ and let $i:R\rightarrow F$ denote the ring homomorphism given by $r \mapsto \overline{(r,1)}$. Let $K$ be a field and let $i_{K}:R\rightarrow K$ be an injective ring homomorphism. Then there exists a unique injective ring homomorphism $j:F\rightarrow K$ such that $i_{K}=j\circ i$.
\end{thm}
\vspace{2ex}
\begin{proof}
We claim that there exists a unique injective ring homomorphism $j$ such that $j\circ i=i_{K}$. For existence, define $j:F\rightarrow K$ to be defined as $j(\overline{(r,s)})=i_{K}(r)\cdot (i_{K}(s))^{-1}$. Then
\begin{align*}
j(\overline{(r,s)})=&j(\overline{(r,1)}\cdot \overline{(1,s)})\\
=&j(\overline{(r,1)})\cdot j(\overline{(1,s)})\\
=&j(i(r))\cdot j(\overline{(s,1)}^{-1})\\
=&j(i(r))\cdot j(i(s)^{-1})\\
=&(j\circ i)(r)\cdot ((j\circ i)(s))^{-1}\\
=&i_K(r)\cdot (i_K(s))^{-1}
\end{align*}
The remaining task is to verify that $j$ is well defined, that it is a ring homomorphism, $j$ is injective, and that $j\circ i=i_{K}$.
\\\\
For uniqueness, suppose that $i_{K}':F\rightarrow K$ is another injective ring homomorphism with $j'\circ=i_{K}$. We need to show that $j=j'$.
\end{proof}
\vspace{2ex}


