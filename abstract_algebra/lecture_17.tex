\section{Lecture 17 (May 8th)}
\begin{recall}
(5.57) (The 3rd Isomorphism Theorem) Let $R$ be a ring and $I\subset J$ be ideals of $R$. Then,
\[R/I\Big/ J/I\cong R/J\]
\end{recall}
\vspace{2ex}
\begin{ex}
(5.38) Observe that
\[{\bm Z}[x]\,/ \,(2,x)\cong {\bm Z}\,/\,2{\bm Z}\quad \mathrm{as}\quad{\bm Z}[x]\,/\,(2,x)\cong {\bm Z}[x]\,/\,(x)\Big/(2,x)\,/\,(x)\]
\end{ex}
\vspace{2ex}
\begin{ex}
Here are some points that are worth considering.
\begin{itemize}
\item[(1)] Let $R$ be a commutative ring. $R$ is an integral domain if and only if 0 is a prime ideal. 
\item[(2-1)] $0\ne (p)\subset {\bm Z}$ is a prime ideal if and only if $p$ is a prime number.
\item[(2-2)] $0\subset {\bm Z}$ is a prime ideal.
\item[(3-1)] Is $(x)\subset {\bm Z}[x]$ a prime ideal? Well, $(a)\ne {\bm Z}[x]$ and $(x)\,|\,f(x)\cdot g(x)$ implies that $(x)\,|\,f(x)$ or $(x)\,|\,g(x)$. An equivalent question would be whether ${\bm Z}[x]\,/\,(x)$ is an integral domain. By the 1st isomorphism theorem, it is isomorphic to ${\bm Z}$ and this is true.
\item[(3-2)] Since ${\bm Z}[x]\,/\,(x)$ is not a field, $(x)$ is not a maximal ideal of ${\bm Z}[x]$. Notice that there exists $(x,2)$.
\item[(4)] $(y-x^2)\subset {\bm C}[x,y]$ is prime. Notice that the following is an integral domain
\[{\bm C}[x,y]\,/\,(y-x^2)\cong {\bm C}[x]\]
\item[(5)] Let $I\subset {\bm Z}$ be an ideal. $I$ is maximal if and only if $I$ is a nonzero prime ideal.
\item[(6)] Let $K$ be a field. Then $(x)$ is a maximal ideal in $K[x]$. However, $(x)$ is not maximal in $K[x,y]$, as 
\[(x)\subsetneq (x,y)\subsetneq K[x,y] \]
Note that $(x,y)\subset K[x,y]$ is maximal.
\end{itemize}
\end{ex}
\vspace{2ex}

