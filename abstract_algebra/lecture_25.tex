\section*{Lecture 25 (June 10th)}
Last time, we went over some properties of groups. Today, we learn about the symmetric group $S_{n}$, normal subgroups $H\triangleleft G$, and cyclic groups. Next time, we will learn about the classification of finite abelian groups.
\newline
\begin{ex}
\begin{itemize}
\item[(i)] The general linear group over ${\bm R}$
\[\mathrm{GL}_{n}({\bm R})=\{M\in M_{n\times n}({\bm R}) \,|\,\det M\ne 0 \}\]
\item[(ii)] The special linear group over ${\bm R}$
\[\mathrm{SL}_{2}({\bm R})=\{M\in M_{2\times 2}({\bm R}) \,|\, \det M=1\}\]
\item[(iii)] The symmetric group
\[S_{n}=(\sigma  ,\circ)\]
where $\sigma : \{1,2,\ldots ,n\}\rightarrow \{1,2,\ldots ,n\} $ is a bijection.
\end{itemize}
\end{ex}
\vspace{2ex}
\begin{defi}
(11.32) (Dihedral group $D_{2n}$) 
\end{defi}
\vspace{2ex}
\begin{defi}
(11.41) For each $r\geq 2$, an $r$-cycle $(a_1,a_2,\ldots ,a_{r})$ is a permutation cyclically permuting $r$ elements, $a_1$ through $a_{r}$. That is, 
\[\begin{cases}
a_i\mapsto a_{i+1}&1\leq i\leq r-1 \\
a_{r}\mapsto a_{1}&i=r
\end{cases}\]
A transposition, meanwhile, is a $2$-cycle. Observe how $(123)=(13)(12)$. 
\end{defi}
\vspace{2ex}
\begin{lem}
(11.45) 
\[(a_1\ldots a_{r})=(a_1a_{r})(a_1a_{r-1})\ldots (a_1a_2)\]
\end{lem}
\vspace{2ex}
\begin{rmk}
Two cycles are disjoint if their nontrivial orbits are disjoints. 
\end{rmk}
\vspace{2ex}
\begin{prop}
(11.44) Every nontrivial permutation can be written as product of disjoint nontrivial cycles.
\end{prop}
\vspace{2ex}
\begin{thm}
Every permutation can be written as a product of transpositions. Furthermore, the partiy of the number of transpositions in each decomposition is the same.
\end{thm}
\vspace{2ex}
\begin{ex}
There is no unique decompostion of a permutation.
\begin{align*}
\sigma =&(123)(67)\\=&(13)(12)(67)\\=&(13)(12)(67)(52)(52)
\end{align*}
\end{ex}
\vspace{2ex}
\begin{defi}
The sign of a permutation is defined as
\[\mathop{\mathrm{sgn}}(\sigma)=
\begin{cases}
1&\mathrm{if\ even }\\
-1&\mathrm{if\ odd }
\end{cases}
\]
as in the number of transpositions that construct $\sigma $. We will call $\sigma $ even if $\mathop{\mathrm{sgn}}(\sigma)=1 $ and odd if $\mathop{\mathrm{sgn}}(\sigma) =-1$. This is called the permutation's parity.
\end{defi}
\vspace{2ex}
{\bf Chapter 11.4}
\newline
\begin{defi}
We can define an equivalence relation on $G$ by saying, for a subgroup $H$, 
\[a\sim b\iff a^{-1}b\in H\]
The following set is called a left coset
\[aH=\{ah \,|\, h\in H\}=\bar{a}\]
The set of equivalence classes can be denoted
\[G/H=\{aH \,|\, a\in G\}=\{\bar{a} \,|\, a\in H\}\]
We notice that the definition of an ideal of a ring is exactly the defintion of a coset where the ring addition is identified as the group operation.
\end{defi}
\vspace{2ex}
\begin{defi}
(11.52) We will say that $H$ is a normal subgroup if $aH=Ha$ for all $a\in G$. 
\end{defi}
\vspace{2ex}
\begin{rmk}
(11.53) $H$ is a normal in $G$ if and only if $aHa^{-1}\subset H$ for all $a\in G$. 
\end{rmk}
\vspace{2ex}

