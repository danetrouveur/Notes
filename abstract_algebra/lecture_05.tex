\section{Lecture 5 (March 18th)}
Last class, we have dealt with the basic properties of rings. Today, we learn more about rings.\\
\begin{recall}
Notation-wise, we have noted that
\begin{itemize}
	\item[(i)] $a\cdot b=ab$
	\item[(ii)] $a^{n}$ for $n>0$ and $a^{0}$ is defined as 1.
	\item[(iii)] $n a$ for $n\in {\bm Z}$
\end{itemize}
\end{recall}
\vspace{2ex}
\begin{prop}
 (3.14, 15, 16) The uniqueness of 0, 1, and $-a$. 
\end{prop}
\vspace{2ex}
\begin{prop}
 (3.17) Let $R$ be a ring. If $a+c=b+c$ then $a=b$.
\end{prop}
\vspace{2ex}
\begin{cor}
 (3.18) For every $a$ in a ring $R$,
 \[0a=0=a0\] 
\end{cor}
\vspace{2ex}
\begin{proof}
\[0+0a=(0+0)a=0a+0a\]
\end{proof}
\vspace{2ex}
\begin{rmk}
There is no cancellation law for multiplication ($ac=dc$ does not imply that $a=d$). 
\end{rmk}
\vspace{2ex}
{\bf Chapter 3.3}\hspace{2ex} Special Types of Rings
\\
\begin{ex}
 (3.19) Not every ring is commutative. For instance, consider $M_{2\times 2}({\bm R})$. 
\end{ex}
\vspace{2ex}
\begin{defi}
 (3.20) A ring is commutative if $ab=ba$ for all $a,b\in R$.  
\end{defi}
\vspace{2ex}
\begin{defi}
 (3.22) Let $a$ be an element of $R$. We say that $a$ is a zero divisor if there exists a non-zero $b\in R$ such that $ab=0$ or $ba=0$. 
\end{defi}
\vspace{2ex}
\begin{ex}
In $M_{2\times 2}({\bm R})$, 
\[\begin{bmatrix}
	1&0\\0&0
\end{bmatrix}\begin{bmatrix}
	0&0\\0&1
\end{bmatrix}=0\]
Observe that both matrices are zero-divisors.
\end{ex}
\vspace{2ex}
\begin{defi}
 (3.23) Let $R$ be a commutative ring. We will say that $R$ is an integral domain if $1\ne 0$ and $ab=0$ implies that $a=0$ or $b=0$.
\end{defi}
\vspace{2ex}
\begin{ex}
${\bm Z}$ or equivalently ${\bm Z}/0{\bm Z}$ are integral domains whereas ${\bm Z}/1{\bm Z}$ is not as its multiplicative and additive identities are equal to each other.
\end{ex}
\vspace{2ex}
\begin{ex}
 (3.26) Let $n>1$ be an integer. Then ${\bm Z}/n{\bm Z}$ is an integral domain if and only if $n$ is a prime.
\end{ex}
\vspace{2ex}
\begin{prop}
Let $a\in R$ be an element. If $a$ is not a zero-divisor, then the multiplicative cancellation holds for $a$. That is, if $ab=ac$ or $ba=ca$, then $b=c$. 
\end{prop}
\vspace{2ex}
\begin{proof}
 (3.26) If $ab=ac$, then $0=a(b-c)=a(b+(-c))$. 
\end{proof}
\vspace{2ex}
\begin{defi}
 (3.27) If $a$ has a multiplicative inverse (that is, there exists $b\in R$ such that $ab=1=ba$), then we say that $a$ is invertible or a unit.
\end{defi}
\vspace{2ex}
\begin{rmk}
We denote the set of units in $R$ by $R^{\times }$.
\end{rmk}
\vspace{2ex}
\begin{prop}
 (3.28) Let $n>0$ be an integer.
 \[({\bm Z}/n{\bm Z})^{\times }=\{\bar{a} \;|\; (a,n)=1\}\]
\end{prop}
\vspace{2ex}
\begin{defi}
 (3.27) We say that $R$ is a field if 
 \begin{itemize}
	 \item[(i)] $R$ is commutative
	 \item[(ii)] $1\ne 0$ in $R$ 
	 \item[(iii)] Every nonzero element is invertible
 \end{itemize}
\end{defi}
\vspace{2ex}
\begin{rmk}
If $R$ is a field, then $R^{\times }=R-\{0\}$.
\end{rmk}
\vspace{2ex}
\begin{ex}
 (3.30) 
 \begin{itemize}
	 \item[(i)] ${\bm Q}$, ${\bm R}$, ${\bm C}$
	 \item[(ii)] ${\bm Z}/p{\bm Z}$ where $p$ is a prime
 \end{itemize}
\end{ex}
\vspace{2ex}
\begin{prop}
 (3.31) Every field is an integral domain.
\end{prop}
\vspace{2ex}
\begin{proof}
Yours!
\end{proof}
\vspace{2ex}
\begin{rmk}
The conserve doesn't hold. 
\end{rmk}
\vspace{2ex}
 
