\section{Lecture 6 (March 20th)}
Last time, we have learned integral domains \& fields. Today, we will learn Cartesian products and subrings. 
\begin{recall}
\begin{itemize}
	\item[(i)] (3.22) $a\in R$ is a zero-divisor if there exists a $b\ne 0$ such that $ab=0$ or $ba=0$
	\item[(ii)] (3.23) $R$ is an integral domain if (1) $R$ is commutative, (2) $1\ne 0$, and (3) $R$ has no nonzero zero-divisors
	\item[(iii)] $R$ has no nonzero zero-divisors
	\item[(iv)] $a\in R$ is a unit for $ab=ba=1$ for some $b\in R$
	\item[(v)] $R$ is a field if (1), (2), and every nonzero element is a unit
\end{itemize}
\end{recall}
\vspace{2ex}
\begin{rmk}
The fact that $1=0$ in $R$ is equivalent to saying that $R$ is the trivial ring $\{0\}$
\end{rmk}
\vspace{2ex}
\begin{prop}
 (3.28) When $n>0$, 
 \[({\bm Z}/n{\bm Z})^{\times }=\{\bar{a}\in {\bm Z}/n{\bm Z} \;|\; (a,n)=1\}\]
The problem with this definition is that we don't know whether the condition $(a,n)=1$ works for the entirety of $\bar{a}$. However, we know from number theory that if $a\equiv_{n}b$, then $(a,n)=(b,n)$. 
\end{prop}
\vspace{2ex}
\begin{rmk}
We note that
\[\mathrm{Fields}\subset M_{2\times 2}({\bm R}),{\bm R}[x]\in \mathrm{Integral\;Domains}\subset {\bm Z}/4{\bm Z}\in  \mathrm{Rings}\]
\end{rmk}
\vspace{2ex}
\begin{prop}
 (3.33) Let $R$ be an integral domain having finitely many elements. Then $R$ is a field.
\end{prop}
\vspace{2ex}
\begin{proof}
The proof is similar to the proof of Fermat's little theorem. Set $R=\{a_1,a_2,\ldots ,a_{n}\}$. It suffices to prove that We now prove that this is a field. Let's fix $a_{i}\ne 0$. If suffices to prove that $a_{i}\in R^{\times }$. Consider the subset of $R$ $a_{i}R=\{a_{i}\cdot a_1,a_{i}\cdot a_2,\ldots ,a_{i}\cdot a_{n}\}$. Since $R$ is an integral domain, $a_{i}R=R$. Indeed, if $a_{i}\cdot a_{j}=a_{i}\cdot a_{k}$, $a_{j}=a_{k}$. Then, $1=a_{i}\cdot a_{j}$ for some $j$. Consequently, every nonzero element in $R$ has a multiplicative inverse.
\end{proof}
\vspace{2ex}
{\bf Chapter 4}\hspace{2ex}The Category of Rings\\\\
{\bf Chapter 4.1}\hspace{2ex}Cartesian Products
\\
\begin{defi}
(4.1) Lets $R$ and $S$ be rings. The catesian product of $R$ and $S$ is the set $R\times S$ equipped with component-wise addition and multiplication. 
\end{defi}
\vspace{2ex}
\begin{rmk}
In $R\times S$,
\[\begin{cases}
	(r_1,s_1)+(r_2,s_2)=(r_1+r_2,s_1+s_2)\\
(r_1,s_1)\cdot (r_2,s_2)=(r_1\cdot r_2,s_1\cdot s_2)
\end{cases}\]
forms a ring. If it exists, the inverse of an element would look like $(r^{-1},s^{-1})$. We remind ourselves that there exists projection functions ($pr_{1},pr_{2}$) from $X\times Y$ to $X$ and $Y$. 
\end{rmk}
\vspace{2ex}
\begin{ex}
 (4.2, 4.3)
\begin{itemize}
	\item[(i)] ${\bm Z}/2{\bm Z}\times {\bm Z}/2{\bm Z}\ne {\bm Z}/4{\bm Z}$
	\item[(ii)] ${\bm Z}/2{\bm Z}\times {\bm Z}/3{\bm Z}\cong {\bm Z}/6{\bm Z}$
\end{itemize}
\end{ex}
\vspace{2ex}
