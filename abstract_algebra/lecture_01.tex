\section{Lecture 1 (March 4th)}
{\bf Chapter 2}\; Modular Arithmetic
\\
\begin{rmk}
In abstract algebra, we learn algebraic structures such as rings (eg. $({\bm Z},+,\times )$) and groups (eg. $({\bm Z},+)$).
\end{rmk}
\vspace{2ex}
\begin{rmk}
By convention we are going to denote the set of integers equipped with addition and multiplication by the triple $({\bm Z},+,\times )$. 

\end{rmk}
\vspace{2ex}
{\bf Chapter 2.2}\; Congruence modulo $n$
\\
\begin{recall}
Fix an integer $n>0$, for example, $n=5$. We can group integers (create a partition) that have the same remainder when divided by $n=5$. This creates a congruence relation, denotable as ``$7=12$" (In number theory, we would say that $7$ and $12$ are congruent $\mathrm{mod}\; 5$). We let $\bar{a}$ denote the equivalence class of $a$ with respect to the congruence modulo $n$.
\end{recall}
\vspace{2ex}
\begin{rmk}
 Giving a partition on ${\bm Z}$ is equivalent to giving an equivalence relation on ${\bm Z}$. For example, we declare $6\equiv_{5} -4$. To summarize, $\bar{a}=\{b\;|\;b\equiv_{n}a\}$ or $[a]$. There is a mathematical reason why we prefer the former.
\end{rmk}
\vspace{2ex}
