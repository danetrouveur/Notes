\section{Lecture 23 (May 29th)}
Last time, we learned about irreducibility tests including Eisenstein's criterion. Today we learn about some applications of the criterion and group theory.
\newline
\begin{thm}
(7.35) (Eisenstein's criterion) Take $R$ to be a UFD. Let $f(x)=a_0+a_1x+\ldots +a_{n}x^{n}\in R[x]$ be a polynomial of degree $n$. Let $p\in R$ be irreducible. If
\begin{itemize}
\item[(i)] $p\nmid a_{n}$
\item[(ii)] $p\,|\,a_0,a_1,\ldots ,a_{n-1}$
\item[(iii)] $p^2\nmid a_0$ 
\end{itemize}
then $f(x)$ is irreducible in $Q[x]$. 
\end{thm}
\vspace{2ex}
\begin{ex}
\begin{itemize}
\item[(i)] $f(x)=x^{4}+10x+5\in {\bm Z}[x]$. By taking $p=5$, we notice that $f(x)$ is irreducible over ${\bm Q}[x]$. 
\item[(ii)] $f(x)=x^{4}+1\in {\bm Z}[x]$. The function is irreducible if and only if $f(x+1)\in {\bm Z}[x]$ is irreducible. That is,
\[f(x+1)=x^{4}+4x^{3}+6x^{2}+4x+2\]
is irreducible. Taking $p=2$, we notice that $f(x)$ is irreducible over ${\bm Q}[x]$.
\end{itemize}
\end{ex}
\vspace{2ex}
\begin{cor}
(7.38) There are irreducible polynomials in ${\bm Q}[x]$ of arbitrary high degree.
\end{cor}
\vspace{2ex}
\begin{proof}
For each $n\geq 1$, consider $x^{n}-2\in {\bm Q}[x]$ 
\end{proof}
\vspace{2ex}
\begin{ex}
(7.38) Let $p\in {\bm Z}$ be a prime number. Then the cyclotomic polynomial
\[\mathrm{\Phi} _{p}(x)=1+x+x^2+\ldots +x^{p-1}\]
is irreducible in ${\bm Q}[x]$. For this, we note that $\mathrm{\Phi} _{p}(x)$ is rreducible over ${\bm Q}$ if and only if $\mathrm{\Phi} _{p}(x+1)$ is irreducible over ${\bm Q}$. Since 
\[\mathrm{\Phi} _{p}(x)=\dfrac{x^{p}-1}{x-1}\]
we have
\[\mathrm{\Phi} _{p}(x+1)=\dfrac{(x+1)^{p}-1}{x}=\sum _{i=1}{}^{p}C_{i}x^{i-1}x^{i-1}={}^{p}C_{1}+{}^{p}C_{2}x^2+\ldots +{}^{p}C_{p}x^{p-1}\]
Using the fact that $p\;|\;{}^{p}C_{i}$ for $1\leq i\leq p-1$, we can apply Eisenstein's criterion to deduce that $\mathrm{\Phi} _{p}(x+1)\in {\bm Q}[x]$ is irreducible.
\end{ex}
\vspace{2ex}
\begin{rmk}
Read proposition (7.33) and example (7.34).
\end{rmk}
\vspace{2ex}
{\bf Chapter 11}\hspace{2ex}Groups-Preliminaries
\newline
\newline
{\bf Chapter 11.1}\hspace{2ex}Groups nad their Categories\
\newline
\begin{defi}
(11.1) A group consists of a set $G$ along with a binary operation on $G$ such that 
\begin{itemize}
\item[(i)] (Associativity) $(g_1*g_2)*g_3=g_1*(g_2*g_3)$ for all $g_1,g_2,g_3\in G$
\item[(ii)] (Identity element) There exists $e\in G$ such that $g*e=e*g=g$ for all $g\in G$
\item[(iii)] (Inverse element) For each $g_1\in G$, there exists $g_2\in G$ such that $g_1*g_2=e$ 
\end{itemize}
\end{defi}
\vspace{2ex}
\begin{defi}
For group $G$, we say that $G$ is abelian if
\[g_1*g_2=g_2*g_1\]
for all $g_1,g_2\in G$. 
\end{defi}
\vspace{2ex}
\begin{ex}
If $(R,+,\cdot )$ is a ring, then $(R,+)$ is an abelian group. 
\end{ex}
\vspace{2ex}
\begin{rmk}
$(R,\cdot )$ is not necessarily a group. 
\end{rmk}
\vspace{2ex}
\begin{rmk}
\begin{itemize}
\item[(i)] The identity $e$ is unique
\item[(ii)] For each $g\in G$, the inverse of $g$ is unique
\item[(iii)] $(g^{-1})^{-1}=g$
\item[(iv)] (Cancellation holds in groups) If $g_1*h=g_2*h$ or $h*g_1=h*g_2$, then $g_1=g_2$
\end{itemize}
\end{rmk}
\vspace{2ex}
\begin{rmk}
In the future, we will write
\[g*h \rightarrow gh\]
and
\[g*g*g\rightarrow g^3\]
\end{rmk}
\vspace{2ex}
