\section{Lecture 7 (March 27th)}
Last time, we have learned about isomorphisms. Today, we will learn about homomorphisms. 
\\
\begin{defi}
 (4.29) Let $R$ and $S$ be rings. A function $\phi :R\rightarrow S$ is said to be a isomorphism if
\begin{itemize}
	\item[(i)] $\phi $ is a bijection 
	\item[(ii)] $\phi (r_1,+r_2)=\phi (r_1)+\phi (r_2)$
	\item[(iii)] $\phi (r_1r_2)=\phi (r_1)\phi (r_2)$
\end{itemize}
We don't need conditions such as $\phi (0)=0$, $\phi (-r)=-\phi (r)$, and $\phi (1)=1$ as they are implied by the conditions above. 
\end{defi}
\vspace{2ex}
\begin{defi}
	(4.15) Let $R$ and $S$ be rings. A function $\phi :R\rightarrow S$ is called a (ring) homomorphism if
\begin{itemize}
	\item[(i)] $\phi (r_1+r_2)=\phi (r_1)+\phi (r_2)$
	\item[(ii)] $\phi (r_1r_2)=\phi (r_1)\phi (r_2)$
	\item[(iii)] $\phi (1)=1$
\end{itemize}
As the function isn't bijective, there doesn't need to be a mapping of $\phi (1)$, and we require the third condition.
\end{defi}
\vspace{2ex}
\begin{ex}
The function $f:{\bm Z}\rightarrow {\bm Z}\times {\bm Z}$ defined by $n\mapsto (n,0)$ is not a ring homomorphism. 
\end{ex}
\vspace{2ex}
\begin{defi}
 (4.32) We will say that two rings $R$ and $S$ are isomorphic if there exists an isomorphism $\phi :R\rightarrow S$.
\end{defi}
\vspace{2ex}
\begin{rmk}
 (4.33) The isomorphic relation is an equivalence relation. 
\end{rmk}
\vspace{2ex}
\begin{ex}
 (4.38 - 4.43)
\begin{itemize}
	\item[(i)] ${\bm Z}/4{\bm Z}$ is not isomorphic to ${\bm Z}/2{\bm Z}\times {\bm Z}/2{\bm Z}$
	\item[(ii)] ${\bm Z}/6{\bm Z}$ is isomorphic to ${\bm Z}/2{\bm Z}\times {\bm Z}/3{\bm Z}$
	\item[(iii)] The complex conjugation ${\bm C}\rightarrow {\bm C}:z\mapsto \bar{z}$ is an isomorphism
	\item[(iv)] The function
	\[{\bm C}\rightarrow \Big\{\begin{pmatrix}
		a&-b\\b&a
	\end{pmatrix}
	\;\Big|\;a,b\in {\bm R}\Big\}\]
	defined as
	\[a+bi\mapsto \begin{pmatrix}
		a&-b\\b&a
	\end{pmatrix}
	\]
	is an isomorphism. 
\item[(v)] $R\rightarrow \Delta _{R}=\{(r,r) \;|\; r\in R\}$ defined as $r\mapsto (r,r)$ is an isomorphism
\item[(vi)] $R[x,y]$ and $(R[x])[y]$ is isomorphic
\end{itemize}
\end{ex}
\vspace{2ex}

