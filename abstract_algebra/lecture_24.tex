\section{Lecture 24 (June 5th)}
Last time, we learned the basics of group theory. Today, we'll learn basic properties of groups.
\newline
\begin{rmk}
A group is called abelian if $g*h=h*g$. 
\begin{itemize}
\item[(i)] (Uniqueness of the identity element) $e$ is unique
\item[(ii)] (Uniqueness of inverses) $g^{-1}$ is unique
\item[(iii)] $(g^{-1})^{-1}=g$
\item[(iv)] (Cancellation law) $g_1*h=g_2*h$ implies that $g_1=g_2$
\item[(v)] We use multiplicative notation
\item[(vi)] We use additive notation for abelian groups
\end{itemize}
\end{rmk}
\vspace{2ex}
\begin{defi}
(11.2) Let $G$ and $H$ be groups. A function $\phi :G\rightarrow H$ is a group homomorphism if it satisfies
\[\phi (g_1*g_2)=\phi (g_1)*\phi (g_2)\]
for all $g_1$ and $g_2$ in $G$. 
\end{defi}
\vspace{2ex}
\begin{ex}
(The trivial group homomorphism) The function $G\rightarrow H:g\mapsto e$ is a group homomorphism, where $1$ is the identity element.
\end{ex}
\vspace{2ex}
\begin{prop}
(11.3)
\begin{itemize}
\item[(i)] The map $\mathrm{id}:G\rightarrow G$ is a group homomorphism
\item[(ii)] A composition of group homomorphisms is also a group homomorphism. That is, if $\phi :G\rightarrow H$ and $\psi :H\rightarrow K$ are group homomorphisms, then so is the composition $\psi \circ \phi :G\rightarrow K$ as
\[(\psi \circ\phi )(g_1*g_2)=\psi (\phi(g_1)*\phi (g_2))=\psi (\phi (g_1))*\psi (\phi (g_2))\]
\end{itemize}
\end{prop}
\vspace{2ex}
\begin{defi}
(11.5) Let $\phi :G\rightarrow H$ be a group homomorphism. We will say that $\phi $ is an isomorphism if there exists a group homomorphism $\psi :H\rightarrow G$ such that $\psi \circ\phi =\mathrm{id}_{G}$ and $\phi \circ\psi =\mathrm{id}_{H}$. 
\end{defi}
\vspace{2ex}
\begin{rmk}
\begin{itemize}
\item[(i)] $\phi $ is an isomorphism if and only if it is bijective
\item[(ii)] Being isomorphic is an equivalence relation
\item[(iii)] $\phi (e)=e$
\end{itemize}
\end{rmk}
\vspace{2ex}
\begin{defi}
Let $A$ be a nonempty set. Then the pair 
\[\Big(\mathrm{the\ set\ of\ all\ bijections\ on\ }A,\mathrm{\ their\ composition}\Big)\]
is called a symmetric group on $A$ and is denoted $S_{A}$. Notationwise, if $|A|=n$, we will denote $S_{A}$ by $S_{n}$.
\end{rmk}
\end{defi}
\vspace{2ex}
\begin{rmk}
If $|A|=|B|$, then $S_{A}$ is isomorphic to $S_{B}$. Thus, we justify our notation from above.
\end{rmk}
\vspace{2ex}
\begin{defi}
(11.6) Let $G$ be a group and $H$ be a subset of $G$. We say that $H$ is a subgroup of $G$ if
\begin{itemize}
\item[(i)] $H$ is closed under the group operation
\item[(ii)] $e$ is in $H$
\item[(iii)] For each $h\in H$, $h^{-1}\in H$
\end{itemize}
\end{defi}
\vspace{2ex}
\begin{ex}
(11.8, 11.9, 11.10)
\begin{itemize}
\item[(i)] If $H$ and $K$ are subgroups of $G$, then $H\cap K$ is also a subgroup of $G$ 
\item[(ii)] Let $\phi :G\rightarrow G'$ be a group homomorphism. If $H$ is a subgroup of $G$, then $\phi (H)$ is a subgroup of $G'$. In particular, $\mathop{\mathrm{im}}\phi =\phi (G)$ is a subgroup of $G'$. If $H'$ is a subgroup of $G'$, then $\phi ^{-1}(H')$ is a subgroup of $G$. 
\end{itemize}
\end{ex}
\vspace{2ex}
\begin{defi}
(11.49) Let $\phi :G\rightarrow H$ be a group homomorphism. Then the kernel of $\phi $, denoted by $\mathop{\mathrm{ker}}\phi $, is the subgroup of $\phi ^{-1}(\{1\})$. 
\end{defi}
\vspace{2ex}
\begin{rmk}
If $\phi :R\rightarrow S$ is a ring homomorphism, then
\[\mathop{\mathrm{ker}}(\phi :(R,+,\cdot )\rightarrow (S,+,\cdot ))=\mathop{\mathrm{ker}}(\phi :(R,+)\rightarrow (S,+))\]
\end{rmk}
\vspace{2ex}
\begin{ex}
(11.14, 11.15, 11.16)
\begin{itemize}
\item[(i)] A singleton set is a group
\item[(ii)] If $(R,+,\cdot )$ is a ring, then $(R,+)$ is an abelian group
\item[(iii)] If $V$ is a vector space, then $(V,+)$ is an abelian group
\item[(iv)] If $R$ is a ring, then the set $R^{\times }$ of units in $R$ is a group under multiplication. In particular, $GL_{n}({\bm R})=(M_{n\times n}({\bm R}))^{\times }$ is a group with matrix multiplication
\item[(v)] The set of $2\times 2$ real matrices whose determinants are $1$ forms a subgroup of $GL_{2}({\bm R})$
\[SL_{2}({\bm R})\]
\end{itemize}
\end{ex}
\vspace{2ex}

