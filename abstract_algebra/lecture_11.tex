\section{Lecture 11 (April 8th)}
Last class, we have learned kernals and ideals. Today, we will learn quotient rings and isomorphism theorems. 
\begin{defi}
(5.3) Let $\phi :R\rightarrow S$ be a ring homomorphism. 
\[\mathrm{ker}\;\phi =\{r\in R \;|\; \phi (r)=0\}=\phi ^{-1}(\{0\})\]
\end{defi}
\vspace{2ex}
\begin{defi}
(5.8) Let $I$ be a subset of a ring $R$. Then $I$ is said to be an ideal if
\begin{itemize}
\item[(i)] It is closed under $+$ and additive inverses
\item[(ii)] $0\in I$
\item[(iii)] (Absorbtion property) $ar=ra\in I$ for all $r\in R$ and all $a\in I$
\end{itemize}
\end{defi}
\vspace{2ex}
\begin{prop}
(5.16) If $R$ is commutative, then $\mathrm{ker}\;\phi $ is an ideal of $R$.
\end{prop}
\vspace{2ex}
\begin{defi}
(5.19) Let $R$ be a commutative ring and $I$ be an ideal of $R$. $a\sim_{I}b$ if and only if $b-a\in I$. Then, we say "$a$ is congruent to $b$ modulo $I$". 
\end{defi}
\vspace{2ex}
\begin{rmk}
\begin{itemize}
\item[(i)] Here, $\sim_{I}$ is an equivalence relation
\item[(ii)] The equivalence class of $a\in R$
\[[a]=\bar{a}+I=\{a+i \;|\; i\in I\}\subset R\]
is called the (left) coset of $a$ modulo $I$.
\end{itemize}
\end{rmk}
\vspace{2ex}
\begin{defi}
(5.22) $R/I$ (the set of all cosets $\{\bar{a} \;|\; a\in R\}$) is called the quotient of $R$ modulo $I$. 
\end{defi}
\vspace{2ex}
\begin{rmk}
We can give a ring structure to $R/I=\{\bar{a} \;|\; a\in R\}$by defining 
\[\begin{cases}
R/I\times R/I\rightarrow R/I:(\bar{a},\bar{b})\mapsto \overline{a+b}\\
R/I\times R/I\rightarrow R/I:(\bar{a},\bar{b})\mapsto \overline{ab}
\end{cases}\]
\end{rmk}
\vspace{2ex}
\begin{thm}
(5.26) Let $R$ be a (commutative) ring and let $I\subset R$ be an ideal. Then $(R/I,+,\cdot )$ is a ring.
\end{thm}
\vspace{2ex}
\begin{proof}
We first show well-definedness of $+$ and $\cdot $. Then, we can show the eight ring properties.
\end{proof}
\vspace{2ex}
\begin{ex}
(5.27 - 34)
\begin{itemize}
\item[(i)] ${\bm Z}/n{\bm Z}$
\item[(ii)] $R/R=\{\bar{a} \;|\; a\in R\}=\{R\}\ne R$
In this case, $\bar{a}=R$ for every $a\in R$. 
\item[(iii)] If $R$ is commutative, then $R/I$ is also commutative ($\bar{a}+\bar{b}=\overline{a+b}=\overline{b+a}=\bar{b}+\bar{a}$).
\item[(iv)] $R/I$ is not necessarily an integral domain, even if $R$ is an integral domain. For example, ${\bm Z}/4{\bm Z}$ is not an integral domain, since $\bar{2}\cdot \bar{2}=\bar{0}$.
\item[(v)] Consider $R[x]\;/\;(x)$. $\overline{f(x)}=\overline{1+2x+3x^2}=\overline{1}+\overline{2}\cdot \overline{x}+\overline{3}(\overline{x})^2$.
\item[(vi)] For a commutative ring $R$, $R[x]\;/\;(x-r)\cong R$ 
\item[(vii)] ${\bm R}[x]\;/\;(x^2+1)\cong {\bm C}$. For $f(x)\in {\bm R}[x]$, $f(x)=(x^2+1)q(x)+ax+b$ and $\overline{f(x)}=\overline{ax+b}$. The function would be $\overline{f(x)}\rightarrow ai+b$.
\end{itemize}
\end{ex}
\vspace{2ex}

