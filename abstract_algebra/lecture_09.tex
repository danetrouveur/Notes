\section{Lecture 9 (April 1st)}
Last class, we have learned about isomorphisms and homomorphisms. This class, we learn about some properties of homomorphisms.
\\
\begin{defi}
(4.29) A function $\phi :R\rightarrow S$ is called an isomorphism if it is a homomorphism and bijective. 
\end{defi}
\vspace{2ex}
\begin{prop}
(4.16) Let $\phi :R\rightarrow S$ be a ring homomorphism. Then $\phi(0)=0$ 
\end{prop}
\vspace{2ex}
\begin{proof}
$\phi (0+0)=\phi (0)+\phi (0)=\phi (0)=\phi (0)+0$
\end{proof}
\vspace{2ex}
\begin{ex}
(4.17 -- 26) 
\begin{itemize}
\item[(i)] The unique function $0:R\rightarrow 0$ is a homomorphism
\item[(ii)] $0\rightarrow R:0\mapsto 0$ is not a ring homomorphism if $R$ is nontrivial
\item[(iii)] $pr_{1}:R\times S\rightarrow R$ and $pr_{2}:R\times S\rightarrow S$ are ring homomorphisms
\item[(iv)] ${\bm Z}\rightarrow {\bm Z}/n{\bm Z}:m\mapsto \bar{m}$ is a ring homomorphism
\item[(v)] ${\bm Z}\rightarrow R:n\mapsto n\cdot 1$ is a ring homomorphism
\item[(vi)] ${\bm Z}/12{\bm Z}\rightarrow {\bm Z}/4{\bm Z}:\bar{n}\mapsto \bar{n}$ is a ring homomorphsim
\item[(vii)] Fix $r\in R$. $R[x]\rightarrow R:f(x)\mapsto f(r)$ is a ring homomorphism
\item[(viii)] ${\bm C}\rightarrow {\bm C}:a+bi\mapsto a-bi$ is a homomorphism
\item[(ix)] ${\bm Z}\rightarrow {\bm Z}:n\mapsto 2n$ is not a ring homomorphism
\item[(x)] ${\bm Z}/2{\bm Z}\rightarrow {\bm Z}$ there is no such ring homomorphism
\item[(xi)] $\mathrm{det}:M_{2\times 2}(R)\rightarrow R$ is not a ring homomorphism
\end{itemize}
We MUST check whether the function in (vi) is well defined. 
\end{ex}
\vspace{2ex}
\begin{cor}
(4.31) Let $\phi :R\rightarrow S$ be a ring homomorphism. Then $\phi $ is an isomorphism if and only if there exists a ring homomorphism $\psi :S\rightarrow R$ such that $\psi \circ\phi =\mathrm{id}_{R}$ and $\phi \circ \psi  =\mathrm{id}_{S}$. 
\end{cor}
\vspace{2ex}
{\bf Chapter 5}\hspace{2ex}Canonical Decomposition, Quotients, and Isomorphism Theorems\\\\
{\bf Chapter 5.1}\hspace{2ex}Rings: Canonical Decomposition I\\
\begin{rmk}
Any function can be written as a composition of a surjection and an injection.
\end{rmk}
\vspace{2ex}
\begin{prop}
(5.1) Let $\phi :R\rightarrow S$ be a homomorphism. Then the image of $\phi $ (denoted as $\mathrm{Im}\;\phi $) is a subring of $S$.
\end{prop}
\vspace{2ex}
\begin{rmk}
(5.2) If $R'$ is a subring of $R$, then $f(R')$ is a subring of $S$. 
\end{rmk}
\vspace{2ex}
{\bf Chapter 5.2}\hspace{2ex}Kernels and Ideals
\\
\begin{defi}
(5.3) Let $\phi :R\rightarrow S$ be a homomorphism. The kernal of $\phi $ is the subset $\{r\in R \;|\; \phi (r)=0\}\subset R$ and will be denoted by $\mathrm{Ker}\;\phi $. Note that $\mathrm{Ker}\;\phi =\phi ^{-1}(\{0\})$.
\end{defi}
\vspace{2ex}
\begin{ex}
(5.4, 5.5)
\begin{itemize}
\item[(i)] Let $n$ be a nonnegative integer. Then the kernel of the homomorphism ${\bm Z}\rightarrow {\bm Z}/n{\bm Z}$ (given by $m\mapsto \bar{m}$) is $n{\bm Z}$
\item[(ii)] Let $\mathrm{ev}_0:R[x]\rightarrow R$ be the homomorphism defined by the evaluation at 0. Then, $\mathrm{Ker}\;\mathrm{ev}_{0}$ is the set of all polynomials with no constant terms.
\end{itemize}
\end{ex}
\vspace{2ex}
\begin{prop}
(5.6) The set $(\mathrm{Ker}\;\phi ,+)$ satisfies the four ring axioms. 
\end{prop}
\vspace{2ex}

