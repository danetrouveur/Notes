\section{Lecture 7 (March 25th)}
Last class, we have learned the properties of rings and subrings. Today, we will learn about ring homomorphisms.
\\
\begin{prop}
 (3.33) A finite integral domain is a field
\end{prop}
\vspace{2ex}
\begin{defi}
 (4.1)
Let $R$ and $S$ be rings. The set
\[(R\times S,(r_1,s_1)+(r_2,s_2)=(r_1+r_2,s_1+s_2),(r_1,s_1)\cdot (r_2,s_2)=(r_1r_2,s_1s_2))\]
is called the cartesian product of $R$ and $S$.
\end{defi}
\vspace{2ex}
\begin{ex}
\begin{itemize}
	(4.2, 4.3)
	\item[(i)] Comparing ${\bm Z}/2{\bm Z}\times {\bm Z}/2{\bm Z}$ and ${\bm Z}/4{\bm Z}$ we find that these are very different sets. Adding identical elements in one results in the $0$ whereas this isn't always the case for the other.
	\item[(ii)] Comparing ${\bm Z}/2{\bm Z}\times {\bm Z}/3{\bm Z}$ and ${\bm Z}/6{\bm Z}$, we find that they are identical.
\end{itemize}
\end{ex}
\vspace{2ex}
\begin{defi}
 (4.5) Let $S$ be a subset of a ring $R$. We say that $S$ is a subring of $R$ if 
\begin{itemize}
	\item[(i)] $0,1\in S$
	\item[(ii)] $S$ is closed under $+$ and $\cdot $
	\item[(iii)] $(S,+,\cdot )$ is a ring
\end{itemize}
Note that a subring is not only a ring of its own but also manifests the algebraic structure of the original ring. 
\end{defi}
\vspace{2ex}
\begin{ex}
 (4.6, 4.7, 4.8, 4.9, 4.11, 4.12, 4.13)
\begin{itemize}
	\item[(i)] ${\bm Z}\subset {\bm Q}\subset {\bm R}\subset {\bm C}$
	\item[(ii)] ${\bm Z}/n{\bm Z}\subset {\bm Z}$
	\item[(iii)] $\Delta_{R}=\{(r,r) \;|\; r\in R\}\subset R\times R$
	\item[(iv)] $R\subset R[x]$
	\item[(v)] ${\bm Z}[i]=\{m+in \;|\; m,n\in {\bm Z}\}\subset {\bm C}$
	\item[(vi)] 
	\[\Big\{\begin{bmatrix}
			a&-b\\b&a
	\end{bmatrix}\Big|\;a,b\in R
	\Big\}\subset M_{2\times 2}({\bm R})\]
\end{itemize}
\end{ex}
\vspace{2ex}
\begin{prop}
 (4.14) $S$ is a subring of $R$ if and only if $\ldots $
\end{prop}
\vspace{2ex}
{\bf Chapter 4.3}\hspace{2ex}Ring Homomorphisms
\\\\
{\bf Chapter 4.4}\hspace{2ex}Isomorphisms of Rings
\\
\begin{defi}
 (cf. definition (4.29)) Let $R$ and $S$ be rings, and let $\phi :R\rightarrow S$ be a function. We will say that $\phi $ is a ring isomorphism if it satisfies
\begin{itemize}
	\item[(i)] $\phi$ is bijective 
	\item[(ii)] $\phi $ preserves the ring operations, or $\phi (r_1+r_2)=\phi (r_1)+\phi (r_2)$ and $\phi (r_1r_2)=\phi (r_1)\phi (r_2)$
\end{itemize}
\end{defi}
\vspace{2ex}
\begin{defi}
 (4.15) We will say that $\phi $ is a ring homomorphism if 
\begin{itemize}
	\item[(i)] $\phi (r_1+r_2)=\phi (r_1)+\phi (r_2)$ 
	\item[(ii)] $\phi (r_1r_2)=\phi(r_1)\phi(r_2)$
\end{itemize}
\end{defi}
\vspace{2ex}

