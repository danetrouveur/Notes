\section{Lecture 3 (March 11th)}
Last time, we dealt with the well-definedness of $+$ and $\cdot $ on ${\bm Z}\;/\;n{\bm Z}$. 
\begin{lem}
	(2.9) Let $n>0$ be an integer and let $a$, $b$, $c$, and $d$ be integers. If $a \equiv_{n} c$ and $b \equiv_{n} d$, then $a+b\equiv_{n} c+d$ and $a\cdot b\equiv_{n}c\cdot d$. A start of a proof would be by considering $(a+b)-(c+d)=(a-c)+(b-d)$.
\end{lem}
\vspace{2ex}
\begin{lem}
(2.13) Let $a$, $b$, and $c$ in ${\bm Z}$. Then,
\begin{itemize}
	\item[1.] $(\bar{a}+\bar{b})+\bar{c}=\bar{a}+(\bar{b}+\bar{c})$
	\item[2.] $\bar{a}+\bar{0}=\bar{a}=\bar{0}+\bar{a}$
	\item[3.] For each $\bar{a}\in {\bm Z}/n{\bm Z}$, there exists $\bar{b}\in {\bm Z}/n{\bm Z}$ such that $\bar{a}+\bar{b}=\bar{0}=\bar{b}+\bar{a}$
	\item[4.] $\bar{a}+\bar{b}=\bar{b}+\bar{a}$
	\item[5.] $\bar{a}(\bar{b}\bar{c})=(\bar{a}\bar{b})\bar{c}$
	\item[6.] $\bar{a}\bar{1}=\bar{a}=\bar{1}\bar{a}$
	\item[7.] $\bar{a}(\bar{b}+\bar{c})=\bar{a}\bar{b}+\bar{a}\bar{c}$
	\item[8.] $(\bar{a}+\bar{b})\bar{c}=\bar{a}\bar{c}+\bar{b}\bar{c}$
	\item[9.] $\bar{a}\bar{b}=\bar{b}\bar{a}$
\end{itemize}
The first three imply that ${\bm Z}$ is a group and four implies that it is abelian also. From five to eight, the properties tells us that group is a ring and the ninth tells us that it is a commutative one. 
\end{lem}
\vspace{2ex}
\begin{proof}
For the first property, we have
\begin{align*}
	(\bar{a}+\bar{b})+\bar{c}=&\overline{a+b}+\bar{c}\\
	=&\overline{(a+b)+c}\\
	=&\overline{a+(b+c)}\\
	=&\bar{a}+\overline{b+c}\\
	=&\bar{a}+(\bar{b}+\bar{c})
\end{align*}
\end{proof}
\vspace{2ex}
\begin{rmk}
Unlike in ${\bm Z}$, $\bar{2}\cdot \bar{3}=\bar{6}=\bar{0}$ in ${\bm Z}/6{\bm Z}$. Note that $\bar{2}\ne \bar{0}$ in ${\bm Z}/6{\bm Z}$. Like so, two non-zero numbers can multiply to become zero in ${\bm Z}/n{\bm Z}$. 
\end{rmk}
\vspace{2ex}
\begin{thm}
(2.15) Let $n$ be an integer greater than 1. Then the following are equivalent.
\begin{itemize}
	\item[1.] The integer $n$ is a prime number.
	\item[2.] Let $a$ and $b$ be in ${\bm Z}$. If $\bar{a}\bar{b}=\bar{0}$, then $\bar{a}=\bar{0}$ or $\bar{b}=\bar{0}$.
	\item[3.] For all $\bar{a}\ne \bar{0}$ in ${\bm Z}/n{\bm Z}$, $\bar{a}$ has a multiplicative inverse.
\end{itemize}
\end{thm}
\vspace{2ex}
\begin{proof}
We will prove that $2\implies 3$. Let $\bar{a}\ne \bar{0}$ be an element of ${\bm Z}/n{\bm Z}$. Consider the subset of ${\bm Z}/n{\bm Z}$ consisting $\{\bar{a}\bar{0},\bar{a}\bar{1},\ldots ,\bar{a}\overline{n-1}\}$. We claim that if $\bar{a}\bar{i}=\bar{a}\bar{j}$ for $0\leq i,j\leq n-1$, then $i=j$. Consequently, $\{\bar{a}\bar{0},\ldots ,\bar{a}\overline{n-1}\}={\bm Z}/n{\bm Z}$. In particular, $\bar{1}=\bar{a}\bar{b}$ for some $\bar{b}\in {\bm Z}/n{\bm Z}$.
\end{proof}
\vspace{2ex}

