\section{Lecture 12 (April 10th)}
Last time we have learned about quotient rings. Today, we learn about the first isomorphism theorems. \\\\
{\bf Chapter 5.3}\hspace{2ex}Rings: Canonical Decomposition II
\\
\begin{prop}
(5.35) Let $R$ be a ring and $I$ be an indeal of $R$. Then the natural projection $\pi :R\rightarrow R/I$ given by $r\mapsto \overline{r}$ is a surjective ring homomorphism with $\mathrm{ker}\;\pi =I$.
\end{prop}
\vspace{2ex}
\begin{proof}
Almost yours!
\begin{align*}
r\in \mathrm{ker}\;I \iff& \pi (r)=\overline{r}=\overline{0}\\
\iff& r=r-0\in I
\end{align*}
\end{proof}
\vspace{2ex}
\begin{thm}
(5.37 5.38) (The 1st isomorphism theorem) Let $\phi :R\rightarrow S$ be a ring homomorphism. Then
\begin{itemize}
\item[(i)] The function $\overline{\phi }:R\;/\;\mathrm{ker}\;\phi \rightarrow S$ given by the rule $\overline{\phi }(\overline{r})=\phi (r)$ is a well-defined ring homomorphism
\item[(ii)] $\overline{\phi }$ is an injective ring homomorphism
\item[(iii)] $\mathrm{Im}\;\overline{\phi }=\mathrm{Im}\;\phi $
\end{itemize}In particular, $\overline{\phi }$ induces an isomorphism $R\;/\;\mathrm{ker}\;\phi \rightarrow \mathrm{Im}\;\phi $. 
\end{thm}
\vspace{2ex}
\begin{rmk}
\begin{itemize}
\item[(i)] For the projection function $\pi :R\rightarrow R\;/\;I$, applying the above, we have $R\;/\;\mathrm{ker}\;\pi \cong \mathrm{Im}\;\pi $.
\item[(ii)] We recall that a function can be decomposed into a surjection and an injection. Likewise, a homomorphism can be decomposed into a projection, isomorphism, and surjection.
\[
\begin{tikzcd}[row sep=huge, column sep=huge, ampersand replacement=\&]
\huge R \arrow[r, "\huge \varphi"] \arrow[d, two heads, "\huge \pi"'] \& \huge S \\
\huge R/\ker\varphi \arrow[r, "\huge \sim"'] \& \huge \mathrm{Im}\,\varphi \arrow[u, hook]
\end{tikzcd}
\]
\item[(iii)] In linear algebra
\begin{align*}
L:V\rightarrow W\implies & V\;/\;\mathrm{ker}\;L \cong \mathrm{Im}\;L\\
\implies& \mathrm{dim}\;V-\mathrm{dim}\;\mathrm{ker}\;L=\mathrm{dim}\;V\;/\;\mathrm{ker}\;L=\mathrm{dim}\;\mathrm{Im}\;L=\mathrm{rank}\;L\\
\implies & \mathrm{dim}\;L=\mathrm{dim}\;\mathrm{ker}\;L+\mathrm{rank}\;L
\end{align*}
\end{itemize}
\end{rmk}
\vspace{2ex}
\begin{proof}
The steps are as follows.
\begin{itemize}
\item[($\ast$)] Well-definedness of $\overline{\phi }:R\;/\;\mathrm{ker}\;\phi \rightarrow S:\overline{r}\mapsto \phi (r)$. We claim that if $\overline{r_1}=\overline{r_2}$, then $\phi (r_1)=\phi (r_2)$. Note that $r_1-r_2\in \mathrm{ker}\;\phi $. Then,
\[0=\phi (r_1-r_2)=\phi (r_1)+\phi (-r_2)=\phi (r_1)-\phi (r_2)\]
\item[(i)] Let $\overline{\phi }$ be a ring homomorphism. Prove $+$ separately.
\[\overline{\phi }(\overline{a}\cdot \overline{b})=\overline{\phi }(\overline{ab})=\overline{\phi (ab)}=\overline{\phi (a)\phi (b)}=\overline{\phi (a)}\cdot \overline{\phi (b)}=\overline{\phi }(\overline{a})\cdot \overline{\phi }(\overline{b})\]
\item[(ii)] We now prove that $\overline{\phi }$ is injective. Suppose that $\overline{r}\in \mathrm{ker}\;\overline{\phi }$. We want to prove that $\overline{r}=\overline{0}$. By the assumption,
\[0=\overline{\phi }(\overline{r})=\phi (r)\]
that is, $r\in \mathrm{ker}\;\phi $, which completes the proof.
\item[(iii)] $\mathrm{Im}\;\overline{\phi }=\mathrm{Im}\;\phi $ 
\end{itemize}
\end{proof}
\vspace{2ex}
\begin{ex}
(5.40, 5.41) 
\begin{itemize}
\item[(i)] Let $R$ be a ring and $r\in R$. Consider the evaluation homomorphism 
\[\mathrm{ev}_{r}:R[x]\rightarrow R:f(x)\mapsto f(r)\]
Note that $\mathrm{ker}\;\mathrm{ev}_{r}=(x-r)$. Applying the 1st isomorphism theorem,
\[R[x]\;/\;(x-r)=R[x]\;/\;\mathrm{ker}\;\mathrm{ev}_{r}\rightarrow \mathrm{Im}\;\mathrm{ev}_{r}=R \]
\end{itemize}
\end{ex}
\vspace{2ex}

