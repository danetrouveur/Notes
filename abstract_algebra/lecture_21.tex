\section{Lecture 21 (May 22nd)}
Today, we will learn about Gauss' lemma.
\newline
\begin{rmk}
In a UFD $R$, if $p\in R$ is irreducible, $p$ is prime.
\end{rmk}
\vspace{2ex}
\begin{thm}
(Gauss' lemma) Let $R$ be a UFD, and let $f(x)$ and $g(x)$ be nonzero polynomials over $R$. Then
\[\mathop{\mathrm{cont}}(f g)=\mathop{\mathrm{cont}}(f)\cdot \mathop{\mathrm{cont}}(g)\]
In particular, if $f$ and $g$ are primitive, then so is $f\cdot g$.
\end{thm}
\vspace{2ex}
\begin{proof}
It will suffice to show that if $f(x)$ and $g(x)$ are primative, then so are $f(x)\cdot g(x)$. Then, the theorem follows automatically from the fact that 
\[\mathop{\mathrm{cont}}\Big(\mathop{\mathrm{cont}}(f)f_1\Big)=\mathop{\mathrm{cont}}(f)\mathop{\mathrm{cont}}(f_1)\]
Suppose that $f(x)g(x)$ is not primative. Then there exists an irreducible $p$ in $R$ such that the image of $f(x)g(x)$ under the ring homomorphism 
\[\phi :R[x]\rightarrow \Big(R/(p)\Big)[x]\]
is zero. That is,
\[\overline{f(x)\cdot {g}(x)}=0\]
This tells us that, as $R/(p)$ is an integral domain, one of the polynomials are zero. However, this implies that one of the polynomials has a coefficient that isn't $1$, that is, one of the polynomials are not primative. This is a contradiction.
\end{proof}
\vspace{2ex}
\begin{cor}
(7.26) Let $R$ be a UFD and $f(x)\in R[x]$ be primative. Then $f(x)$ is irreducible in $R[x]$ if and only if $f(x)$ is irreducible in $Q[x]$, where $Q=\mathop{\mathrm{Frac}}(R)$. 
\end{cor}
\vspace{2ex}
\begin{cor}
(7.27) If $R$ is a UFD, then $R[x]$ is also a UFD. 
\end{cor}
\vspace{2ex}
\begin{rmk}
If $R$ is a UFD, $R[x_1,\ldots ,x_{n}]$ is also a UFD.
\end{rmk}
\vspace{2ex}
\begin{rmk}
Let $f(x)\in Q[x]$ be a nonzero polynomial ($R$ is a UFD and $Q=\mathop{\mathrm{Frac}}(R)$). We can write 
\[f(x)=\dfrac{a}{b}f_1(x)\]
where $a$ and $b$ are nonzero elements in $R$ and $f_1(x)\in R[x]$ is primative polynomial over $R$. Remark that the irreducibility of $Q$ is equivalent to the irreducibility of $f_1(x)$ over $Q$. 
\end{rmk}
\vspace{2ex}

