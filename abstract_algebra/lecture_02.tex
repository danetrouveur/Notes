\section{Lecture 2 (March 6th)}
Last class, we have learnt ${\bm Z}/n{\bm Z}$ as a set. In this lecture, we will learn algebraic structures on ${\bm Z}/n{\bm Z}$.
\\
\begin{defi}
(2.1) Let $a$ and $b$ be integers. We say that $a$ is congruent to $b$ modulo $n$ if $a-b=nk$ for some $k\in {\bm Z}$. In this case, we write $a \equiv_{n} b $ (or $a\equiv b\;(\mathrm{mod}\;n)$).
\end{defi}
\vspace{2ex}
\begin{rmk}
	(2.2) $a\equiv_{n} b$ if and only if $a$ and $b$ have the same remainder after division by $n$.
\end{rmk}
\vspace{2ex}
\begin{rmk}
$\equiv_{n}$ is an equivalence relation on ${\bm Z}$. Accordingly, $\bar{a}=\{b\in {\bm Z} \;|\; b\equiv_{n}a\}$.
\end{rmk}
\begin{proof}
\begin{itemize}
	\item[(i)] $a\equiv_{n}a$
	\item[(ii)] $a\equiv_{n}b\implies b \equiv_{n}a$
	\item[(iii)] $a \equiv_{n} b$, $b \equiv_{n} c\implies a \equiv_{n} c$
\end{itemize}
\end{proof}
\vspace{2ex}
\begin{defi}
(2.5) We denote by ${\bm Z}/n{\bm Z}$ the set of congruence classes modulo $n$.
\end{defi}
\vspace{2ex}

\begin{rmk}
\begin{itemize}
\item[(i)] $\bar{a}=\bar{b}\in {\bm Z}/n{\bm Z}$ if and only if $a\equiv_{n}b$
\item[(ii)] If $\bar{a}\cap \bar{b}\ne \emptyset$, then $\bar{a}=\bar{b}$
\item[(iii)] ${\bm Z}=\bar{0}\coprod \bar{1}\coprod \ldots \coprod \overline{n-1}$
\end{itemize}
Lastly, we also use ${\bm Z}/n{\bm Z}$ instead of ${\bm Z}_{n}$.
\end{rmk}
\vspace{2ex}
\begin{rmk}
(2.9) 
\begin{itemize}
	\item[(i)] ${\bm Z}/n{\bm Z}$ is a finite set having exactly $n$ elements (how about $n=0$?)
	\item[(ii)] ${\bm Z}/0{\bm Z}={\bm Z}$
\end{itemize}
\end{rmk} 
\vspace{2ex}
{\bf Chapter 2.3}\hspace{2ex}Algebra in ${\bm Z}/n{\bm Z}$
\\\\
We want to define $+$ and $\cdot $ on ${\bm Z}/n{\bm Z}$. For example, $n=5$ and we have ${\bm Z}/5{\bm Z}=\{\bar{0},\bar{1},\bar{2},\bar{3},\bar{4} \}$. Can we simply define $\bar{2}+\bar{3}=\overline{5}$? However, in process of formulating addition, we bump into the problem that we can add different representatives every time. In other words, we don't know whether ``$+$" is well-defined! Let's phrase this differently. Let $\bar{a}=\bar{b}$ and $\bar{c}=\bar{d}$. Then we want $\overline{a+c}=\overline{b+d}$.
\\
\begin{lem}
(2.9) Let $a,b,c, d$ be in ${\bm Z}$. If $a \equiv_{n}b$ and $c\equiv_{n}d$ then $a+c\equiv_{n}b+d$ and $ac\equiv_{n}bd$.
\end{lem}
\vspace{2ex}

