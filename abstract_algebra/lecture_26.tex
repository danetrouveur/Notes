\section{Lecture 26 (June 12th)}
Last time, we learned about the symmetric group $S_{n}$ and normality. Today, we learn about cyclic groups and the classification of finite abelian groups.
\newline
\begin{defi}
(11.52) A subgroup $H$ of $G$ is normal if $g*H=H*g$ for all $g\in G$. 
\end{defi}
\vspace{2ex}
\begin{rmk}
(11.53) $H$ is normal in $G$ if and only if $a*H*a^{-1}\subset H$ for all $a\in G$.  
\end{rmk}
\vspace{2ex}

\begin{thm}
(11.64) $H$ is normal in $G$ if and only if $\overline{a}* \overline{b}=\overline{a*b}$ is well-defined. Moreover, $H$ is normal if and only if $(G/H,\cdot )$ forms a group. 
\end{thm}
\vspace{2ex}
\begin{defi}
(11.65) If $G$ is a group and $H\subset G$ is a normal subgroup, $(G/H,\cdot )$ is called the quotient group of $G$ modulo $H$.
\end{defi}
\vspace{2ex}
\begin{rmk}
If $G$ is abelian, every subgroup of $G$ is normal.
\end{rmk}
\vspace{2ex}
\begin{prop}
(11.58) If $\phi :G\rightarrow H$ is a group homomorphism, then
\[\mathop{\mathrm{ker}}\phi =\phi ^{-1}(\{e_{H}\})\]
is normal in $G$.
\end{prop}
\vspace{2ex}
\begin{ex}
(11.60) Let $G$ be a group. The center is defined as
\[Z(G)=\{z\in G \,|\, gz=zg\}\]
for all $g\in G$. This group is a subgroup and also normal in $G$.
\end{ex}
\vspace{2ex}
\begin{thm}
(Isomorphism theorems) These will be on you. 
\end{thm}
\vspace{2ex}
{\bf Chapter 10}\hspace{2ex}Abelian Groups
\newline
\begin{defi}
(11.11) Let $G$ be a group and $S\subset G$ be a subset. The subgroup of $G$ generated by $S$, denoted $\langle S\rangle $, is the smallest subgroup of $G$ containing $S$. 
\end{defi}
\vspace{2ex}
\begin{rmk}
$\langle S\rangle $ exists and is unique.
\end{rmk}
\vspace{2ex}
\begin{defi}
(10.8) Let $G$ be a group. We will say that $G$ is cyclic if there exists $g\in G$ such that $G=\langle g\rangle $.
\end{defi}
\vspace{2ex}
\begin{rmk}
\begin{itemize}
\item[(i)] $\langle g\rangle $ must contain $\{g^{n} \,|\, n\in {\bm Z}\}$ including negative powers. Notice that this is also a subgroup, and thus 
\[\langle g\rangle =\{g^{n} \,|\, n\in {\bm Z}\}\]
\item[(ii)] If $G$ is cyclic, any element can be expressed as $g^{n}$ for some $n\in {\bm Z}$. Notice that then 
\[g^{m}\cdot g^{n}=g^{m+n}=g^{n+m}=g^{n}\cdot g^{m}\]
and that $G$ must be abelian.
\end{itemize}
\end{rmk}
\vspace{2ex}
\begin{ex}
Consider the two central examples.
\begin{itemize}
\item[(i)] $({\bm Z},+)=\langle 1\rangle =\langle -1\rangle $
\item[(ii)] $({\bm Z}/n{\bm Z},+)=\langle \pm \overline{1}\rangle $
\end{itemize}
\end{ex}
\vspace{2ex}
\begin{prop}
Let $G$ be a cyclic group. Then
\[\begin{cases}
G \cong {\bm Z} &\quad\mathrm{if}\quad |G|=\infty \\
G\cong {\bm Z}/n{\bm Z}&\quad \mathrm{if}\quad |G|=n<\infty 
\end{cases}\]
\end{prop}
\vspace{2ex}
\begin{proof}
Choose a generator $g$ so that $G=\langle g\rangle $. Consider a group homomorphism 
\[\phi :{\bm Z}\rightarrow G:m\mapsto g^{n}\]
By the 1st isomorphism theorem, ${\bm Z}/\mathop{\mathrm{ker}}\phi $ is isomorphic to $\mathop{\mathrm{im}}\phi =G$. Note that $\mathop{\mathrm{ker}}\phi =k{\bm Z}$ for some $k\in {\bm Z}$. 
\[{\bm Z}/k{\bm Z}\cong G\]
This $k$ should be exactly equal to the cardinality of the group, therefore 
\[{\bm Z}/n{\bm Z}\cong G\]
\end{proof}
\vspace{2ex}
\begin{prop}
(10.10) 
\begin{itemize}
\item[(i)] Every subgroup of a cylic group is cylic
\item[(ii)] If $G$ is a cyclic group of order $n<\infty $, then $G$ has exactly one cyclic subgroup of order $m$ for each positive divisor $m$ of $n$
\end{itemize}
\end{prop}
\vspace{2ex}
{\bf Chapter 10.3}\hspace{2ex}The Classification Theorem
\newline
\begin{thm}
(10.18) (Classification of finite abelian groups) Let $G$ be a finite abelian group. Then there exists integers $1<d_{s}<d_{s-1}<\ldots <d_1$ with $d_{s}|d_{s-1}|\ldots |d_2|d_1$ such that $G$ is isomorphic to
\[{\bm Z}/d_{s}{\bm Z}\times {\bm Z}/d_{s-1}{\bm Z}\times \ldots \times {\bm Z}/d_1{\bm Z}\]
Furthermore, this decomposition is unique.
\end{thm}
\vspace{2ex}
\begin{rmk}
\begin{itemize}
\item[(i)] $d_{s}d_{s-1}\ldots d_{1}=n=|G|$
\item[(ii)] If $p$ is a prime divisor of $n$, then $p|d_1$
\end{itemize}
\end{rmk}
\vspace{2ex}

\begin{ex}
Let $n=180=2^{3}\cdot 3^{2}\cdot 5$.
\begin{table}[h!]
\centering
\begin{tabular}{|c|c|c|c|c|}
\hline
$G$ & $d_1$ & $d_2$ & $d_3$ \\
\hline
$\mathbb{Z}/30\mathbb{Z} \times \mathbb{Z}/6\mathbb{Z}$ & $2,3,5$ & $2$ & \\
\hline
$\mathbb{Z}/60\mathbb{Z} \times \mathbb{Z}/3\mathbb{Z}$ & $2^2,3,5$ & $3$ & \\
\hline
$\mathbb{Z}/90\mathbb{Z} \times \mathbb{Z}/2\mathbb{Z}$ & $2,3^2,5$ & $2,3$ & $3$ \\
\hline
$\mathbb{Z}/180\mathbb{Z}$ & $2^2,3^2,5$ & $2$ & \\
\hline
\end{tabular}
\caption{Decomposition of $\mathbb{Z}/180\mathbb{Z}$}
\end{table}
\begin{tabl\end{ex}
\vspace{2ex}

